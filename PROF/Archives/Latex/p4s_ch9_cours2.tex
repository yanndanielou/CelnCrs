  \documentclass[a4paper,11pt]{article}
\usepackage{amsmath,amsthm,amsfonts,amssymb,amscd,amstext,vmargin,graphics,graphicx,tabularx,multicol,yhmath} 
\usepackage[greek,francais]{babel}
\usepackage[utf8]{inputenc}  
\usepackage[T1]{fontenc} 
\usepackage{pstricks-add,tikz,tkz-tab,variations}
\usepackage[autolanguage,np]{numprint} 

\setmarginsrb{1cm}{0.5cm}{1cm}{0.5cm}{0cm}{0cm}{0cm}{0cm} %Gauche, haut, droite, haut

\renewcommand{\thesection}{\indent \Roman{section}}
\renewcommand{\thesubsection}{\indent\indent\arabic{subsection}.}
\renewcommand{\thesubsubsection}{\indent\indent\indent\alph{subsubsection})}

\theoremstyle{definition}
\newtheorem*{thm}{Théoreme}
\newtheorem*{prop}{Propriété}
\newtheorem*{cor}{Corollaire}
\newtheorem*{defi}{Définition}
\newtheorem*{propdefi}{Propriété / Définition}

\newtheorem*{ex}{Exemple}
\newtheorem*{rmq}{Remarque}

\newcounter{enumtabi}
\newcounter{enumtaba}
\newcommand{\q}{\stepcounter{enumtabi} \theenumtabi.  }
\newcommand{\qa}{\stepcounter{enumtaba} (\alph{enumtaba}) }
\newcommand{\initq}{\setcounter{enumtabi}{0}}
\newcommand{\initqa}{\setcounter{enumtaba}{0}}

\newcommand{\be}{\begin{enumerate}}
\newcommand{\ee}{\end{enumerate}}
\newcommand{\bi}{\begin{itemize}}
\newcommand{\ei}{\end{itemize}}
\newcommand{\bp}{\begin{pspicture*}}
\newcommand{\ep}{\end{pspicture*}}
\newcommand{\bt}{\begin{tabular}}
\newcommand{\et}{\end{tabular}}
\renewcommand{\tabularxcolumn}[1]{p{#1}} %(colonne m{} centrée, au lieu de p par défault) 
\newcommand{\tnl}{\tabularnewline}

\newcommand{\trait}{\noindent \rule{\linewidth}{0.2mm}}
\newcommand{\hs}[1]{\hspace{#1}}
\newcommand{\vs}[1]{\vspace{#1}}

\newcommand{\N}{\mathbb{N}}
\newcommand{\Z}{\mathbb{Z}}
\newcommand{\R}{\mathbb{R}}
\newcommand{\C}{\mathbb{C}}
\newcommand{\Dcal}{\mathcal{D}}
\newcommand{\Ccal}{\mathcal{C}}
\newcommand{\mc}{\mathcal}

\newcommand{\vect}[1]{\overrightarrow{#1}}
\newcommand{\ds}{\displaystyle}
\newcommand{\eq}{\quad \Leftrightarrow \quad}
\newcommand{\vecti}{\vect{\imath}}
\newcommand{\vectj}{\vect{\jmath}}
\newcommand{\Oij}{(O\,;\vec{\imath}\, , \vec{\jmath})}
\newcommand{\OIJ}{(O;I,J)}

%\newcommand*{\textgreek}[1]{%
%	\foreignlanguage{greek}{#1}%
%}
\newcommand*{\tg}[1]{\textgreek{#1}}

\newcommand{\titre}[5] 
% #1: titre #2: haut gauche #3: bas gauche #4: haut droite #5: bas droite
{
\noindent #2 \hfill #4 \\
#3 \hfill #5

\vspace{-1.8cm}

\begin{center}\rule{6cm}{0.5mm}\end{center}
\vspace{0.2cm}
\begin{center}{\large{\textbf{#1}}}\end{center}
\begin{center}\rule{6cm}{0.5mm}\end{center}
}



\begin{document}
\pagestyle{empty}

\setcounter{section}{1}

\section{Cosinus et Sinus}

\subsection{Généralités}

\noindent\begin{tabularx}{\linewidth}{cc}

\begin{minipage}{11.5cm}
\begin{defi}

 Soit $t$ un nombre réel, et $M$ son image sur le cercle trigonométrique : autrement dit $M$ est le point de $\mc C$ tel que l'angle orienté $\left(\vecti \,; \vect{OM}\right)=t \text{ (modulo $2\pi$)}$. Le cosinus de $t$ (noté $\cos(t)$ ou $\cos t$) est l'abscisse de $M$ et le sinus de $t$ (noté $\sin(t)$ ou $\sin t$) est l'ordonnée de $M$ 

\end{defi}

\begin{prop} Pour tout réel $t$ et tout entier relatif $k$ on a :

\begin{multicols}{2}
$\bullet \; -1 \leq \cos t \leq 1$

$\bullet \;-1 \leq \sin t \leq 1$

$\bullet \;\cos^2 t + \sin^2 t =1$  


$\bullet \;\cos(t+k \times 2\pi)=\cos t$

$\bullet \;\sin(t+k\times 2\pi)=\sin t$
\end{multicols}
\end{prop}
\end{minipage}
&
\begin{minipage}{6.3cm}
\psset{xunit=3cm , yunit=3cm,algebraic=true}
\begin{pspicture*}(-1.1,-1.1)(1.1,1.1)
\psgrid[subgriddiv=0,gridlabels=0,gridcolor=black,griddots=5,xunit=0.25,yunit=0.25](0,0)(-4,-4)(4,4)
\psaxes[subticks=1,labels=none,ticksize=0]{->}(0,0)(-1.1,-1.1)(1.1,1.1)
\psline{->}(0,0)(1,0)
\psline{->}(0,0)(0,1)
\uput[dl](0,0){$O$}
\uput[dl](1,0){$\vect{\imath}$}
\uput[dl](0,1){$\vect{\jmath}$}
\pscircle(0,0){3cm}
\end{pspicture*}
\end{minipage}
\end{tabularx}

\vs{4cm}

\noindent\begin{tabularx}{\linewidth}{cc}

\begin{minipage}{11.5cm}

\begin{prop}
On a les valeurs suivantes : \medskip

\renewcommand{\arraystretch}{3}
\begin{tabular}{|c|>{\centering}p{0.9cm}|>{\centering}p{0.9cm}|>{\centering}p{0.9cm}|>{\centering}p{0.9cm}|>{\centering}p{0.9cm}|>{\centering}p{0.9cm}|}
\hline
$t$ & 0 & $\dfrac{\pi}6$ & $\dfrac{\pi}4$ & $\dfrac{\pi}3$ & $\dfrac{\pi}2$ & $ \pi$ \tnl \hline
$\cos t $ & & & & & & \tnl \hline
$\sin t $ & & & & & & \tnl \hline
\end{tabular}

\end{prop}

\end{minipage}
&
\begin{minipage}{6.3cm}

\psset{xunit=6cm , yunit=6cm,algebraic=true}
\begin{pspicture*}(-0.1,-0.1)(1.1,1.1)
\psgrid[subgriddiv=0,gridlabels=0,gridcolor=black,griddots=5,xunit=0.25,yunit=0.25](0,0)(0,0)(4,4)
\psaxes[subticks=1,labels=none,ticksize=0]{->}(0,0)(0,0)(1.1,1.1)
\psline{->}(0,0)(1,0)
\psline{->}(0,0)(0,1)
\uput[dl](0,0){$O$}
\pscircle(0,0){6cm}
\end{pspicture*}

\end{minipage}
\end{tabularx}

\vfill

\begin{defi} \'Etant donnés deux vecteurs non nuls $\vect u$ et $\vect v$ on définit le cosinus et le sinus de l'angle orienté $\left(\vect u \,;\vect v\right)$ (notés $\cos \left(\vect u \,;\vect v\right)$ et $\sin \left(\vect u \,;\vect v\right)$) comme le cosinus et le sinus de leur mesure principale.

\end{defi}

\newpage

\subsection{Angles associés}

\begin{prop}
Pour tout $t\in\R$ on a les formules suivantes :

\noindent \begin{tabularx}{\linewidth}{X|X}

\begin{minipage}{9cm}

\begin{tabular}{cc}
\begin{minipage}{4.2cm}
$\cos (-t)=\cos t$

$\sin(-t)=-\sin t$
\end{minipage}
& 
\begin{minipage}{4.5cm}
\psset{xunit=2cm , yunit=2cm,algebraic=true}
\begin{pspicture*}(-1.1,-1.1)(1.1,1.1)
\psgrid[subgriddiv=0,gridlabels=0,gridcolor=black,griddots=5,xunit=0.5,yunit=0.5](0,0)(-2,-2)(2,2)
\psaxes[subticks=1,labels=none,ticksize=0]{->}(0,0)(-1.1,-1.1)(1.1,1.1)
\psline{->}(0,0)(1,0)
\psline{->}(0,0)(0,1)
\uput[dl](0,0){$O$}
\uput[dl](1,0){$\vect{\imath}$}
\uput[dl](0,1){$\vect{\jmath}$}
\psline(0,0)(0.866,0.5)
\pscircle(0,0){2cm}
\end{pspicture*}
\end{minipage}
\end{tabular}
\end{minipage}

&


\tnl


\begin{minipage}{9cm}

\begin{tabular}{cc}
\begin{minipage}{4.2cm}
$\cos (\pi+t)=$

$\sin(\pi+t)=$
\end{minipage}
& 
\begin{minipage}{4.5cm}
\psset{xunit=2cm , yunit=2cm,algebraic=true}
\begin{pspicture*}(-1.1,-1.1)(1.1,1.1)
\psgrid[subgriddiv=0,gridlabels=0,gridcolor=black,griddots=5,xunit=0.5,yunit=0.5](0,0)(-2,-2)(2,2)
\psaxes[subticks=1,labels=none,ticksize=0]{->}(0,0)(-1.1,-1.1)(1.1,1.1)
\psline{->}(0,0)(1,0)
\psline{->}(0,0)(0,1)
\uput[dl](0,0){$O$}
\uput[dl](1,0){$\vect{\imath}$}
\uput[dl](0,1){$\vect{\jmath}$}
\psline(0,0)(0.866,0.5)
\pscircle(0,0){2cm}
\end{pspicture*}
\end{minipage}
\end{tabular}
\end{minipage}
&

\begin{minipage}{9cm}

\begin{tabular}{cc}
\begin{minipage}{4.2cm}
$\cos (\pi-t)=$

$\sin(\pi-t)=$
\end{minipage}
& 
\begin{minipage}{4.5cm}
\psset{xunit=2cm , yunit=2cm,algebraic=true}
\begin{pspicture*}(-1.1,-1.1)(1.1,1.1)
\psgrid[subgriddiv=0,gridlabels=0,gridcolor=black,griddots=5,xunit=0.5,yunit=0.5](0,0)(-2,-2)(2,2)
\psaxes[subticks=1,labels=none,ticksize=0]{->}(0,0)(-1.1,-1.1)(1.1,1.1)
\psline{->}(0,0)(1,0)
\psline{->}(0,0)(0,1)
\uput[dl](0,0){$O$}
\uput[dl](1,0){$\vect{\imath}$}
\uput[dl](0,1){$\vect{\jmath}$}
\psline(0,0)(0.866,0.5)
\pscircle(0,0){2cm}
\end{pspicture*}
\end{minipage}
\end{tabular}
\end{minipage}

\tnl


\begin{minipage}{9cm}

\begin{tabular}{cc}
\begin{minipage}{4.2cm}
$\ds \cos \Big(\frac{\pi}2+t\Big)=$

\medskip

$\ds \sin\Big(\frac{\pi}2+t\Big)=$
\end{minipage}
& 
\begin{minipage}{4.5cm}
\psset{xunit=2cm , yunit=2cm,algebraic=true}
\begin{pspicture*}(-1.1,-1.1)(1.1,1.1)
\psgrid[subgriddiv=0,gridlabels=0,gridcolor=black,griddots=5,xunit=0.5,yunit=0.5](0,0)(-2,-2)(2,2)
\psaxes[subticks=1,labels=none,ticksize=0]{->}(0,0)(-1.1,-1.1)(1.1,1.1)
\psline{->}(0,0)(1,0)
\psline{->}(0,0)(0,1)
\uput[dl](0,0){$O$}
\uput[dl](1,0){$\vect{\imath}$}
\uput[dl](0,1){$\vect{\jmath}$}
\psline(0,0)(0.866,0.5)
\pscircle(0,0){2cm}
\end{pspicture*}
\end{minipage}
\end{tabular}
\end{minipage}
&

\begin{minipage}{9cm}

\begin{tabular}{cc}
\begin{minipage}{4.2cm}
$\ds \cos \Big(\frac{\pi}2-t\Big)=$

\medskip

$\ds \sin\Big(\frac{\pi}2-t\Big)=$
\end{minipage}
& 
\begin{minipage}{4.5cm}
\psset{xunit=2cm , yunit=2cm,algebraic=true}
\begin{pspicture*}(-1.1,-1.1)(1.1,1.1)
\psgrid[subgriddiv=0,gridlabels=0,gridcolor=black,griddots=5,xunit=0.5,yunit=0.5](0,0)(-2,-2)(2,2)
\psaxes[subticks=1,labels=none,ticksize=0]{->}(0,0)(-1.1,-1.1)(1.1,1.1)
\psline{->}(0,0)(1,0)
\psline{->}(0,0)(0,1)
\uput[dl](0,0){$O$}
\uput[dl](1,0){$\vect{\imath}$}
\uput[dl](0,1){$\vect{\jmath}$}
\psline(0,0)(0.866,0.5)
\pscircle(0,0){2cm}
\end{pspicture*}
\end{minipage}
\end{tabular}
\end{minipage}

\end{tabularx}

\end{prop}

\trait

\medskip

\noindent\textbf{Bilan.} Placer sur le cercle trigonométrique les cosinus et sinus  des angles de mesures $\dfrac{\pi}{6}$, $\dfrac{\pi}{4}$, $\dfrac{\pi}{3}$ et de leurs angles associés. (\emph{voir livre p 293}).

\begin{center}
\psset{xunit=5cm , yunit=5cm,algebraic=true}
\begin{pspicture*}(-1.1,-1.1)(1.1,1.1)
\psgrid[subgriddiv=0,gridlabels=0,gridcolor=black,griddots=5,xunit=0.5,yunit=0.5](0,0)(-2,-2)(2,2)
\psaxes[subticks=1,labels=none,ticksize=0]{->}(0,0)(-1.1,-1.1)(1.1,1.1)
\psline{->}(0,0)(1,0)
\psline{->}(0,0)(0,1)
\uput[dl](0,0){$O$}
\uput[dl](1,0){$\vect{\imath}$}
\uput[dl](0,1){$\vect{\jmath}$}
\pscircle(0,0){5cm}
\end{pspicture*}
\end{center}

\end{document}
