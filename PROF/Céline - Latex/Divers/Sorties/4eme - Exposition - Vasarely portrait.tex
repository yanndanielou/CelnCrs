\documentclass[a4paper,11pt]{article}
\usepackage{amsmath,amsthm,amsfonts,amssymb,amscd,amstext,vmargin,graphics,graphicx,tabularx,multicol} 
\usepackage[francais]{babel}
\usepackage[utf8]{inputenc}  
\usepackage[T1]{fontenc} 
\usepackage{pstricks-add,tikz,tkz-tab,variations}
\usepackage[autolanguage,np]{numprint} 
\usepackage{calc}
\usepackage{pifont}
\usepackage{lscape}

\setmarginsrb{1.5cm}{0.5cm}{1cm}{0.5cm}{0cm}{0cm}{0cm}{0cm} %Gauche, haut, droite, haut
\newcounter{numexo}
\newcommand{\exo}[1]{\stepcounter{numexo}\noindent{\bf Exercice~\thenumexo} : }
\reversemarginpar

\newcommand{\bmul}[1]{\begin{multicols}{#1}}
\newcommand{\emul}{\end{multicols}}

\newcounter{enumtabi}
\newcounter{enumtaba}
\newcommand{\q}{\stepcounter{enumtabi} \theenumtabi.  }
\newcommand{\qa}{\stepcounter{enumtaba} (\alph{enumtaba}) }
\newcommand{\initq}{\setcounter{enumtabi}{0}}
\newcommand{\initqa}{\setcounter{enumtaba}{0}}

\newcommand{\be}{\begin{enumerate}}
\newcommand{\ee}{\end{enumerate}}
\newcommand{\bi}{\begin{itemize}}
\newcommand{\ei}{\end{itemize}}
\newcommand{\bp}{\begin{pspicture*}}
\newcommand{\ep}{\end{pspicture*}}
\newcommand{\bt}{\begin{tabular}}
\newcommand{\et}{\end{tabular}}
\renewcommand{\tabularxcolumn}[1]{>{\centering}m{#1}} %(colonne m{} centrée, au lieu de p par défault) 
\newcommand{\tnl}{\tabularnewline}

\newcommand{\trait}{\noindent \rule{\linewidth}{0.2mm}}
\newcommand{\hs}[1]{\hspace{#1}}
\newcommand{\vs}[1]{\vspace{#1}}

\newcommand{\N}{\mathbb{N}}
\newcommand{\Z}{\mathbb{Z}}
\newcommand{\R}{\mathbb{R}}
\newcommand{\C}{\mathbb{C}}
\newcommand{\Dcal}{\mathcal{D}}
\newcommand{\Ccal}{\mathcal{C}}
\newcommand{\mc}{\mathcal}

\newcommand{\vect}[1]{\overrightarrow{#1}}
\newcommand{\ds}{\displaystyle}
\newcommand{\eq}{\quad \Leftrightarrow \quad}
\newcommand{\vecti}{\vec{\imath}}
\newcommand{\vectj}{\vec{\jmath}}
\newcommand{\Oij}{(O;\vec{\imath}, \vec{\jmath})}
\newcommand{\OIJ}{(O;I,J)}


\newcommand{\reponse}[1][1]{%
\multido{}{#1}{\makebox[\linewidth]{\rule[0pt]{0pt}{20pt}\dotfill}
}}

\newcommand{\titre}[5] 
% #1: titre #2: haut gauche #3: bas gauche #4: haut droite #5: bas droite
{
\noindent #2 \hfill #4 \\
#3 \hfill #5

\vspace{-1.6cm}

\begin{center}\rule{6cm}{0.5mm}\end{center}
\begin{center}{\large{\textbf{#1}}}\end{center}
\begin{center}\rule{6cm}{0.5mm}\end{center}
}



\begin{document}




\titre{{\large Vasarely - Le partage des formes}}{Nom :}{Prénom :}{4ème}{}

\vspace*{0.7cm}

\textbf{Lors de votre visite, prenez en photo les œuvres qui vous plaisent le plus à l'aide de votre téléphone portable \textit{(minimum 4 \oe{uvres})}.} \\
\textbf{En suivant le parcours de l'exposition, répondez aux questions suivantes.}

\vspace*{0.3cm}

\begin{center}
\textbf{\ding{47} \underline{LES AVANTS-GARDES EN HÉRITAGE}}
\end{center}

\q En combien de partie, le parcours de l'exposition est-il divisé ? Citez en une.\\
\reponse[2]\\

\q Dans quel domaine, Vasaraly a-t-il apporté son savoir artistique ?\\

\ding{111}  La télévision \hspace*{4.5cm} \ding{111} La publicité\\

\ding{111} Animalier   \hspace*{5cm} \ding{111} La médecine \\
 	
\begin{center}
\textbf{\ding{47} \underline{. . . . . . . . . . . . . . . . . . . . . . . . . . . . . . . . }}
\end{center}

\q Dans la suite de l'exposition, un morceau de plage se cache dans une \oe{uvre}, quelle est son titre ? Comment avez-vous reconnu la plage ?\\
\reponse[3]\\

\q Trouvez une \oe{uvre} qui a un point commun avec la précédente. Expliquez rapidement.\\
\reponse[2]\\


\noindent \reponse[2]\\

\q Parcourrez un peu les \oe{uvres} abstraites autour de vous. Choisissez-en une qui vous inspire : à quoi vous fait-elle penser ?\\
\reponse[3]\\

\q Cette \oe{uvre} pourrait s'appeler : \textit{\textbf{La géométrie en équilibre}}. De quelle \oe{uvre} s'agit-il ?\\
\reponse[1]\\

\newpage
\vspace*{0.4cm}

\begin{center}
\textbf{\ding{47} \underline{. . . . . . . . . . . . . . . . . . . . . . . . . . . . . . . .} }\\
\end{center}

\q Observer l'\oe{uvre} suivante : \textit{\textbf{Dobkö}} \textit{1957-1959, Collection particulière}. 	\\
Citez 3 formes de contraste :\\
\reponse[3]\\











\q Observer l'\oe{uvre} suivante : \textit{\textbf{Naissance}} \textit{1958, Collection particulière, Suisse}.
	Quelles sensations, effets, ressentez vous ?\\
\reponse[4]\\

\q Pour plusieurs de ses œuvres l'artiste n'a utilisé qu'une seule forme contrairement à ce que nous pouvons percevoir. Choisissez une \oe{uvres} en exemple, donnez son titre et faites en un rapide croquis : \\
\vspace*{-0.5cm}

\setlength{\fboxrule}{1.5pt}
\begin{flushleft}
\framebox{\begin{minipage}{\linewidth}

\underline{\textbf{Titre :}} 
\vspace*{9.5cm}

\end{minipage}}
\end{flushleft}

\vspace*{0.4cm}

\begin{center}
\textbf{\ding{47} \underline{. . . . . . . . . . . . . . . . . . . . . . . . . . . . . . . . }}\\
\end{center}

\q Après l'\oe{uvre}, \textit{\textbf{T.M}} \textit{1958, Collection particulière},  un autre élément entre en jeu dans le travail de l'artiste, lequel ?\\
\reponse[1]\\



\newpage



\q A l'aide de vos crayons de couleur réalisez votre prénom en vous appuyant sur l'alphabet de Vasarely.\\

\vspace*{-0.5cm}

\setlength{\fboxrule}{1.5pt}
\begin{flushleft}
\framebox{\begin{minipage}{\linewidth}

\underline{\textbf{Prénom :}} 
\vspace*{7cm}

\end{minipage}}
\end{flushleft}

\vspace*{0.25cm}

\q Citez une différence et un point commun entre ces deux \oe{uvres} :\\
\hspace*{4.7cm}	-  \textit{\textbf{Pyr Fekete}} \textit{1963, Collection particulière}\\
\hspace*{2cm} et	
\hspace*{2cm}	-  \textit{\textbf{Salgo Négatif}} \textit{1967, édition 2/10}\\
	\reponse[5]\\
	
\q Placez vous à 6 pas de l'\oe{uvre}  \textit{\textbf{V-Boglar}} \textit{1966, Musée National d'art moderne, Centre Pompidou, Paris, don de l'artiste en 1977}\\
	Avancez vers l'\oe{uvre} en fixant du regard le centre du tableau.\\
	Que remarquez vous ?\\
	\reponse[3]\\
Vous pouvez faire le même exercice avec l'\oe{uvre}  \textit{\textbf{Alom}} \textit{1966 Musée National d'art moderne, Centre Pompidou, Paris, don de l'artiste en 1977}.\\







\begin{center}
\textbf{\ding{47} \underline{ . . . . . . . . . . . . . . . . . . . . . . . . . . . . . . . .} }
\end{center}

\q Dans la salle Pop Abstraction, certaines \oe{uvres} / images, vous paraissent-elles familières ? Lesquelles ? A quoi vous font-elles penser ?\\
\reponse[4]\\


\begin{center}
\textbf{\ding{47} \underline{ . . . . . . . . . . . . . . . . . . . . . . . . . . . . . . . .} }
\end{center}

\q Dans cette nouvelle salle, choisissez une \oe{uvre} sur la table centrale et représentez la rapidement au dos.\\
\newpage

\setlength{\fboxrule}{1.5pt}
\begin{flushleft}
\framebox{\begin{minipage}{\linewidth}

\underline{\textbf{Titre :}} 
\vspace*{9.5cm}

\end{minipage}}
\end{flushleft}

\vspace*{0.4cm}




\q Citez une \oe{uvre} qui fait à la fois référence au bidimensionnel (2D) et au tridimensionnel (3D). Grâce à quel(s) procédé(s), Vasarely a-t-il réalisé cette illusion d'optique ? \\
\reponse[3]\\

\q  Allez jusqu'à la salle sombre et prenez le temps d'observer les différentes \oe{uvres}. Listez maintenant les différentes façons de créer des illusions d'optiques.\\
\reponse[4]\\

\q Recopier la citation écrite sur le dernier mur de l'exposition.\\
\vspace*{-0.75cm}

\setlength{\fboxrule}{1.5pt}
\begin{flushleft}
\framebox{\begin{minipage}{\linewidth}

\underline{\textbf{Citation :}} 
\vspace*{3cm}

\end{minipage}}
\end{flushleft}


Que comprenez-vous de cette citation  ?\\
\reponse[4]\\


\newpage
\end{document}
