\documentclass[a4paper,12pt]{article}
\usepackage{amsmath,amsthm,amsfonts,amssymb,amscd,amstext,vmargin,graphics,graphicx,tabularx,multicol} 
\usepackage[francais]{babel}
\usepackage[utf8]{inputenc}  
\usepackage[T1]{fontenc} 
\usepackage{pstricks-add,tikz,tkz-tab,variations}
\usepackage[autolanguage,np]{numprint} 


\setmarginsrb{2.5cm}{0.5cm}{2.5cm}{2cm}{0cm}{0cm}{0cm}{0cm} %Gauche, haut, droite, haut
\newcounter{numexo}
\newcommand{\exo}[1]{\stepcounter{numexo}\noindent{\bf Exercice~\thenumexo} : \marginpar{\hfill /#1}}
\reversemarginpar


\newcounter{enumtabi}
\newcounter{enumtaba}
\newcommand{\q}{\textbf{\stepcounter{enumtabi} \theenumtabi)}  }
\newcommand{\qa}{\textbf{\stepcounter{enumtaba} (\alph{enumtaba})} }
\newcommand{\initq}{\setcounter{enumtabi}{0}}
\newcommand{\initqa}{\setcounter{enumtaba}{0}}

\newcommand{\be}{\begin{enumerate}}
\newcommand{\ee}{\end{enumerate}}
\newcommand{\bi}{\begin{itemize}}
\newcommand{\ei}{\end{itemize}}
\newcommand{\bp}{\begin{pspicture*}}
\newcommand{\ep}{\end{pspicture*}}
\newcommand{\bt}{\begin{tabular}}
\newcommand{\et}{\end{tabular}}
\renewcommand{\tabularxcolumn}[1]{>{\centering}m{#1}} %(colonne m{} centrée, au lieu de p par défault) 
\newcommand{\tnl}{\tabularnewline}

\newcommand{\bmul}[1]{\begin{multicols}{#1}}
\newcommand{\emul}{\end{multicols}}

\newcommand{\trait}{\noindent \rule{\linewidth}{0.2mm}}
\newcommand{\hs}[1]{\hspace{#1}}
\newcommand{\vs}[1]{\vspace{#1}}

\newcommand{\N}{\mathbb{N}}
\newcommand{\Z}{\mathbb{Z}}
\newcommand{\R}{\mathbb{R}}
\newcommand{\C}{\mathbb{C}}
\newcommand{\Dcal}{\mathcal{D}}
\newcommand{\Ccal}{\mathcal{C}}
\newcommand{\mc}{\mathcal}

\newcommand{\vect}[1]{\overrightarrow{#1}}
\newcommand{\ds}{\displaystyle}
\newcommand{\eq}{\quad \Leftrightarrow \quad}
\newcommand{\vecti}{\vec{\imath}}
\newcommand{\vectj}{\vec{\jmath}}
\newcommand{\Oij}{(O;\vec{\imath}, \vec{\jmath})}
\newcommand{\OIJ}{(O;I,J)}


\newcommand{\reponse}[1][1]{%
\multido{}{#1}{\makebox[\linewidth]{\rule[0pt]{0pt}{20pt}\dotfill}
}}

\newcommand{\titre}[5] 
% #1: titre #2: haut gauche #3: bas gauche #4: haut droite #5: bas droite
{
\noindent #2 \hfill #4 \\
#3 \hfill #5

\vspace{-1.6cm}

\begin{center}\rule{6cm}{0.5mm}\end{center}
\vspace{0.2cm}
\begin{center}{\large{\textbf{#1}}}\end{center}
\begin{center}\rule{6cm}{0.5mm}\end{center}
}


\begin{document}
\pagestyle{empty}
\titre{Contrôle 3 : Limites de suites }{Nom :}{Prénom :}{\textbf{TCOM}}{Date:}

\vspace*{0.5cm}

\exo{7.5} Déterminer les limites des suites suivantes.\\

\initqa \qa  $u_n=-5\sqrt{n}-n^3$ \hspace*{1.5cm} \qa $v_n=\dfrac{3}{1-\dfrac{2}{n}}$ \hspace*{1.5cm} \qa $b_n=(e^n+9)(-7+e^{-n})$ \\

\qa  $w_n=4-0,25^n$ \hspace*{1.5cm} \qa $z_n=-2\times \left(\dfrac{3}{5} \right)^n$  \hspace*{1.5cm} \qa $a_n=-5 \times \sqrt{3}^n$\\

\vspace*{0.5cm}

\exo{4}Déterminer par comparaison, les limites des suites suivantes.\\

\initqa \qa $u_n=1-\dfrac{2\times (-1)^n}{n}$ \hspace*{1.5cm} \qa $v_n=n-3sin(n)$ \\

\vspace*{0.5cm}

\exo{5} 
Le 1er janvier 2005, une grande entreprise compte 1500 employés.\\
Une étude montre que lors de chaque année à venir, 10\% de l'effectif du 1er janvier partira à la retraite au cours de l'année. \\
Pour ajuster ses effectifs à ses besoins, l'entreprise embauche 100 jeunes dans l'année.\\
L'entreprise se demande comment va évoluer sur le long terme le nombre d'employés dans cette entreprise.\\
Pour tout entier $n$, on appelle $u_n$ le nombre d'employés le 1er janvier de l'année (2005 + $n$).\\

\initq \q Déterminer $u_1$.\\

\q Montrer que $u_{n+1} = 0,9u_n + 100$.\\

\q On pose $v_n = u_n - \np{1000}$.\\
\initqa \qa Montrer que $(v_n)$ est géométrique. En déduire $v_n$ en fonction de $n$.\\
\qa Exprimer $u_n$ en fonction de $n$.\\
\qa En déduire la limite de la suite $(u_n)$. Interpréter votre résultat.\\



\vspace*{0.5cm}

\exo{3.5} 
Soit $(w_n)$ la suite géométrique de raison $\dfrac{2}{3}$ et de premier terme $w_0=-2$. \\
On note $S_n$ la somme des $n$ premiers termes de la suite $(w_n)$.\\
\initq \q Déterminer l'expression explicite de la suite $(w_n)$.\\
\q Donner l'expressions de $S_n$ en fonction de $n$.\\
\q Déterminer la limite de la somme $S_n$ quand $n$ tend vers $+\infty$.\\

\end{document}
