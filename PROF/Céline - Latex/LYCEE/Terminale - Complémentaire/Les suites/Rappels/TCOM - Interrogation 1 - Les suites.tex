\documentclass[a4paper,11pt]{article}
\usepackage{amsmath,amsthm,amsfonts,amssymb,amscd,amstext,vmargin,graphics,graphicx,tabularx,multicol} 
\usepackage[francais]{babel}
\usepackage[utf8]{inputenc}  
\usepackage[T1]{fontenc} 
\usepackage{pstricks-add,tikz,tkz-tab,variations}
\usepackage[autolanguage,np]{numprint} 


\setmarginsrb{2.5cm}{0.5cm}{2.5cm}{2cm}{0cm}{0cm}{0cm}{0cm} %Gauche, haut, droite, haut
\newcounter{numexo}
\newcommand{\exo}[1]{\stepcounter{numexo}\noindent{\bf Exercice~\thenumexo} : \marginpar{\hfill /#1}}
\reversemarginpar


\newcounter{enumtabi}
\newcounter{enumtaba}
\newcommand{\q}{\textbf{\stepcounter{enumtabi} \theenumtabi)}  }
\newcommand{\qa}{\textbf{\stepcounter{enumtaba} (\alph{enumtaba})} }
\newcommand{\initq}{\setcounter{enumtabi}{0}}
\newcommand{\initqa}{\setcounter{enumtaba}{0}}

\newcommand{\be}{\begin{enumerate}}
\newcommand{\ee}{\end{enumerate}}
\newcommand{\bi}{\begin{itemize}}
\newcommand{\ei}{\end{itemize}}
\newcommand{\bp}{\begin{pspicture*}}
\newcommand{\ep}{\end{pspicture*}}
\newcommand{\bt}{\begin{tabular}}
\newcommand{\et}{\end{tabular}}
\renewcommand{\tabularxcolumn}[1]{>{\centering}m{#1}} %(colonne m{} centrée, au lieu de p par défault) 
\newcommand{\tnl}{\tabularnewline}

\newcommand{\bmul}[1]{\begin{multicols}{#1}}
\newcommand{\emul}{\end{multicols}}

\newcommand{\trait}{\noindent \rule{\linewidth}{0.2mm}}
\newcommand{\hs}[1]{\hspace{#1}}
\newcommand{\vs}[1]{\vspace{#1}}

\newcommand{\N}{\mathbb{N}}
\newcommand{\Z}{\mathbb{Z}}
\newcommand{\R}{\mathbb{R}}
\newcommand{\C}{\mathbb{C}}
\newcommand{\Dcal}{\mathcal{D}}
\newcommand{\Ccal}{\mathcal{C}}
\newcommand{\mc}{\mathcal}

\newcommand{\vect}[1]{\overrightarrow{#1}}
\newcommand{\ds}{\displaystyle}
\newcommand{\eq}{\quad \Leftrightarrow \quad}
\newcommand{\vecti}{\vec{\imath}}
\newcommand{\vectj}{\vec{\jmath}}
\newcommand{\Oij}{(O;\vec{\imath}, \vec{\jmath})}
\newcommand{\OIJ}{(O;I,J)}


\newcommand{\reponse}[1][1]{%
\multido{}{#1}{\makebox[\linewidth]{\rule[0pt]{0pt}{20pt}\dotfill}
}}

\newcommand{\titre}[5] 
% #1: titre #2: haut gauche #3: bas gauche #4: haut droite #5: bas droite
{
\noindent #2 \hfill #4 \\
#3 \hfill #5

\vspace{-1.6cm}

\begin{center}\rule{6cm}{0.5mm}\end{center}
\vspace{0.2cm}
\begin{center}{\large{\textbf{#1}}}\end{center}
\begin{center}\rule{6cm}{0.5mm}\end{center}
}


\begin{document}
\pagestyle{empty}
\titre{Interrogation 1 : Les suites}{Nom :}{Prénom :}{\textbf{TCOM}}{Date:}

\vspace*{1cm}
\exo{4} Etudier le sens de variation des suites suivantes.\\

\initqa \qa La suite arithmétique $(u_n)$ de premier terme -2 et de raison 11.\\

\qa La $(v_n)$ une suite géométrique définie par $v_0=-12$ et $v_{n+1} = 0,5vn$. \\

\qa La suite $(w_n)$ définie par $w_n=3+\dfrac{2}{n}$\\

\vspace*{0.75cm}

\exo{6} \\ 
Un fabricant de matériel électronique commercialise un nouveau smartphone depuis deux mois.\\
 Le premier mois 5000 exemplaires ont été vendus et le deuxième mois 5200 exemplaires.\\
  Le responsable des ventes envisage alors deux modèles d’évolution possibles des ventes dans les mois suivants.\\

\underline{\textbf{A. Premier modèle :}}\\
La variation des ventes reste constante.\\
On note $u_n$le nombre de smartphones vendus le $n$-ème mois. On a donc $u_ 1=5000$ et $u_2=5200$.\\
On suppose que chaque mois le nombre de smartphones vendus augmente de 200 exemplaires.\\

\noindent \initq \q Calculer le nombre de smartphones vendus le 3e mois.\\
\q Exprimer $u_{n+1}$ en fonction de $u_n$. Quelle est la nature de la suite $(u_n)$?\\
\q En déduire l'expression de $u_n$ en fonction de $n$ .\\



\underline{\textbf{B. Deuxième modèle :}}\\
Le taux variation des ventes reste constant.
On note $v_n$ le nombre de smartphones vendus le $n$-ème mois.\\
 On a donc $v_1=5000$ et $v_2=5200$.\\

\noindent \initq \q Justifier que le taux de variation des ventes du premier au deuxième mois est de +4\%.\\
\q En supposant que ce taux reste constant ensuite, exprimer $v_{n+1}$ en fonction de $v_n$. \\
Quelle est la nature de la suite $(v_n)$?\\
\q En déduire l'expression de $v_n$ en fonction de $n$ .\\

\underline{\textbf{BONUS :}} Déterminer le nombre total de smartphones vendus durant les douze premiers mois suivant les 2 modèles. Quel est le modèle le plus avantageux?


\end{document}
