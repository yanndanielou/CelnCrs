\documentclass[a4paper,12pt]{article}
\usepackage{amsmath,amsthm,amsfonts,amssymb,amscd,amstext,vmargin,graphics,graphicx,tabularx,multicol} 
\usepackage[francais]{babel}
\usepackage[utf8]{inputenc}  
\usepackage[T1]{fontenc} 
\usepackage{pstricks-add,tikz,tkz-tab,variations}
\usepackage[autolanguage,np]{numprint} 

\setmarginsrb{2cm}{1cm}{2cm}{0.5cm}{0cm}{0cm}{0cm}{0cm} %Gauche, haut, droite, haut
\newcounter{numexo}
\newcommand{\exo}[1]{\stepcounter{numexo}\noindent{\bf Exercice~\thenumexo} : \marginpar{\hfill /#1}}
\reversemarginpar


\newcounter{enumtabi}
\newcounter{enumtaba}
\newcommand{\q}{\textbf{\stepcounter{enumtabi} \theenumtabi)}  }
\newcommand{\qa}{\textbf{\stepcounter{enumtaba} (\alph{enumtaba})} }
\newcommand{\initq}{\setcounter{enumtabi}{0}}
\newcommand{\initqa}{\setcounter{enumtaba}{0}}

\newcommand{\be}{\begin{enumerate}}
\newcommand{\ee}{\end{enumerate}}
\newcommand{\bi}{\begin{itemize}}
\newcommand{\ei}{\end{itemize}}
\newcommand{\bp}{\begin{pspicture*}}
\newcommand{\ep}{\end{pspicture*}}
\newcommand{\bt}{\begin{tabular}}
\newcommand{\et}{\end{tabular}}
\renewcommand{\tabularxcolumn}[1]{>{\centering}m{#1}} %(colonne m{} centrée, au lieu de p par défault) 
\newcommand{\tnl}{\tabularnewline}

\newcommand{\bmul}[1]{\begin{multicols}{#1}}
\newcommand{\emul}{\end{multicols}}

\newcommand{\trait}{\noindent \rule{\linewidth}{0.2mm}}
\newcommand{\hs}[1]{\hspace{#1}}
\newcommand{\vs}[1]{\vspace{#1}}

\newcommand{\N}{\mathbb{N}}
\newcommand{\Z}{\mathbb{Z}}
\newcommand{\R}{\mathbb{R}}
\newcommand{\C}{\mathbb{C}}
\newcommand{\Dcal}{\mathcal{D}}
\newcommand{\Ccal}{\mathcal{C}}
\newcommand{\mc}{\mathcal}

\newcommand{\vect}[1]{\overrightarrow{#1}}
\newcommand{\ds}{\displaystyle}
\newcommand{\eq}{\quad \Leftrightarrow \quad}
\newcommand{\vecti}{\vec{\imath}}
\newcommand{\vectj}{\vec{\jmath}}
\newcommand{\Oij}{(O;\vec{\imath}, \vec{\jmath})}
\newcommand{\OIJ}{(O;I,J)}


\newcommand{\reponse}[1][1]{%
\multido{}{#1}{\makebox[\linewidth]{\rule[0pt]{0pt}{20pt}\dotfill}
}}

\newcommand{\titre}[5] 
% #1: titre #2: haut gauche #3: bas gauche #4: haut droite #5: bas droite
{
\noindent #2 \hfill #4 \\
#3 \hfill #5

\vspace{-1.6cm}

\begin{center}\rule{6cm}{0.5mm}\end{center}
\vspace{0.2cm}
\begin{center}{\large{\textbf{#1}}}\end{center}
\begin{center}\rule{6cm}{0.5mm}\end{center}
}



\begin{document}
\pagestyle{empty}
\titre{Contrôle 3 : Etude des fonctions affines }{Nom :}{Prénom :}{\textbf{2nd 8}}{Date:}





\vspace*{0.5cm}
\exo{4} Les fonctions suivantes sont-elles affines ? Si oui, donner leurs coefficients directeurs et leurs ordonnées à l'origine.\\

\initqa \qa $f(x)=2x^2+x$  \qa $g(x)=-5,5x-1$ \qa $k(x)=\dfrac{-2}{4+3x}$ \qa $h(x)=\dfrac{9-x}{7}$\\



\vspace*{0.75cm}

\exo{3} Déterminer le sens de variation des fonctions affines sur $\R$ définies par les expressions suivantes.\textit{Une justification est attendue.}\\

\initqa \qa $f(x)=4x+13$ \hspace*{0.5cm}\qa $g(x)=15-3,5x$\hspace*{0.5cm} \qa $h(x)=-11(5-x)$\\


\vspace*{0.75cm}

\exo{6} Construire le tableau de signe des fonctions affines sur $\R$ définies par les expressions suivantes. \textit{Une justification est attendue.}\\

\initqa \qa $f(x)=2,5x+7,5$ \hspace*{0.75cm}  \qa $h(x)=-2x+1$  \hspace*{0.75cm} \qa $h(x)=\dfrac{3-x}{9}$\\

\vspace*{0.75cm}

\exo{7} 
Soient $f$ et $g$ deux fonctions affines définies sur $\R$ par $f(x)=-2x+5$ et $g(x)=1,5x-2$\\

\initq \q Tracer chacune des fonctions dans un repère orthonormé avec la méthode de votre choix.\\

\q \initqa \qa Graphiquement, déterminer le plus précisément possible l'ensemble solution de l'équation $f(x)=g(x)$.\\
\qa Déterminter maintenant par le calcul l'ensemble solution de l'équation $f(x)=g(x)$.\\

\q \initqa \qa Résoudre graphiquement $f(x)\ge 0$ sur $\R$ .\\
\qa En déduire le tableau de signe de la fonction $f$.\\

\q \textbf{BONUS} Construire le tableau de signe de la fonction $g$, en justifiant votre réponse.\\




\end{document}
