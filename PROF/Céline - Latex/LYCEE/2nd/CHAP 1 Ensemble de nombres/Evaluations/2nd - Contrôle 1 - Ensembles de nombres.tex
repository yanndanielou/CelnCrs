\documentclass[a4paper,11pt]{article}
\usepackage{amsmath,amsthm,amsfonts,amssymb,amscd,amstext,vmargin,graphics,graphicx,tabularx,multicol} 
\usepackage[francais]{babel}
\usepackage[utf8]{inputenc}  
\usepackage[T1]{fontenc} 
\usepackage{pstricks-add,tikz,tkz-tab,variations}
\usepackage[autolanguage,np]{numprint} 

\setmarginsrb{1.5cm}{0.5cm}{1cm}{0.5cm}{0cm}{0cm}{0cm}{0cm} %Gauche, haut, droite, haut
\newcounter{numexo}
\newcommand{\exo}[1]{\stepcounter{numexo}\noindent{\bf Exercice~\thenumexo} : \marginpar{\hfill /#1}}
\reversemarginpar


\newcounter{enumtabi}
\newcounter{enumtaba}
\newcommand{\q}{\textbf{\stepcounter{enumtabi} \theenumtabi)}  }
\newcommand{\qa}{\textbf{\stepcounter{enumtaba} (\alph{enumtaba})} }
\newcommand{\initq}{\setcounter{enumtabi}{0}}
\newcommand{\initqa}{\setcounter{enumtaba}{0}}

\newcommand{\be}{\begin{enumerate}}
\newcommand{\ee}{\end{enumerate}}
\newcommand{\bi}{\begin{itemize}}
\newcommand{\ei}{\end{itemize}}
\newcommand{\bp}{\begin{pspicture*}}
\newcommand{\ep}{\end{pspicture*}}
\newcommand{\bt}{\begin{tabular}}
\newcommand{\et}{\end{tabular}}
\renewcommand{\tabularxcolumn}[1]{>{\centering}m{#1}} %(colonne m{} centrée, au lieu de p par défault) 
\newcommand{\tnl}{\tabularnewline}

\newcommand{\bmul}[1]{\begin{multicols}{#1}}
\newcommand{\emul}{\end{multicols}}

\newcommand{\trait}{\noindent \rule{\linewidth}{0.2mm}}
\newcommand{\hs}[1]{\hspace{#1}}
\newcommand{\vs}[1]{\vspace{#1}}

\newcommand{\N}{\mathbb{N}}
\newcommand{\Z}{\mathbb{Z}}
\newcommand{\R}{\mathbb{R}}
\newcommand{\C}{\mathbb{C}}
\newcommand{\Dcal}{\mathcal{D}}
\newcommand{\Ccal}{\mathcal{C}}
\newcommand{\mc}{\mathcal}

\newcommand{\vect}[1]{\overrightarrow{#1}}
\newcommand{\ds}{\displaystyle}
\newcommand{\eq}{\quad \Leftrightarrow \quad}
\newcommand{\vecti}{\vec{\imath}}
\newcommand{\vectj}{\vec{\jmath}}
\newcommand{\Oij}{(O;\vec{\imath}, \vec{\jmath})}
\newcommand{\OIJ}{(O;I,J)}


\newcommand{\reponse}[1][1]{%
\multido{}{#1}{\makebox[\linewidth]{\rule[0pt]{0pt}{20pt}\dotfill}
}}

\newcommand{\titre}[5] 
% #1: titre #2: haut gauche #3: bas gauche #4: haut droite #5: bas droite
{
\noindent #2 \hfill #4 \\
#3 \hfill #5

\vspace{-1.6cm}

\begin{center}\rule{6cm}{0.5mm}\end{center}
\vspace{0.2cm}
\begin{center}{\large{\textbf{#1}}}\end{center}
\begin{center}\rule{6cm}{0.5mm}\end{center}
}



\begin{document}
\pagestyle{empty}
\titre{Contrôle 1 : Ensembles de nombres }{Nom :}{Prénom :}{\textbf{2nd 8}}{Date:}



\exo{4}
Compléter en utilisant le symbole qui convient parmi $\in$, $\notin$, $\subset$ ou $\not\subset$ les phrases suivantes :\\

\hspace*{0.5cm}$\mathbb{D}$ . . . . $\Z$\hspace*{0.35cm}; \hspace*{0.35cm}  $\dfrac{2}{8}$ . . . . $\mathbb{D}$ \hspace*{0.35cm}; \hspace*{0.35cm} $\dfrac{7}{11}$ . . . . $\mathbb{Q}$ \hspace*{0.35cm}; \hspace*{0.35cm} $\sqrt{9}+\sqrt{4}$ . . . . $\mathbb{N}$ \\

\hspace*{0.5cm}$\{-1 ; 0;2;5\}$ . . . . $\Z^*$ \hspace*{0.35cm}; \hspace*{0.35cm}  $\{\dfrac{15}{3};\sqrt{64}\}$ . . . . $\N$ \hspace*{0.35cm}; \hspace*{0.35cm} $\pi$ . . . . $]3,14;3,15[$ \hspace*{0.35cm}; \hspace*{0.35cm} $\mathbb{Q}$ . . . . $\mathbb{N}$ \\

\vspace*{0.5cm}

\exo{4}Compléter le tableau suivant : \\

\begin{tabular}{|p{3cm}|p{6cm}|p{7cm}|}
\hline 

 \begin{center}
 \textbf{Intervalle}
 \end{center}
  & \begin{center}
 \textbf{Inégalité}
 \end{center} & \begin{center}
\textbf{Représentation}
\end{center} \\ 

\hline 
\begin{center}
[-5,5;2]
 \end{center} & • & • \\ 
 \hline 
\begin{center}
[$\pi$;$+\infty$[
 \end{center} & • & • \\ 
\hline 
 & \begin{center}
 $-3 \le x < 9$
 \end{center} & • \\ 
\hline 
  & \begin{center}
  $x\ge-1$
  \end{center} & • \\ 
\hline 


\end{tabular} 

\vspace*{0.75cm}

\exo{6} \textbf{Représenter} les intervalles I et J. Puis \textbf{déterminer} les ensembles $I\cap J$ et $I\cup J$ :\\

\initqa \qa	$I=[-6 ; 8]$   et   $J = [-2 ; 12]$	\hspace*{1cm}\qa	$I = ]1 ; 8[$   et   $J = [5 ; 9]$ \\

\qa	$I = ]0 ;\sqrt{2}  ]$   et   $J = [1 ; + \infty  [$ \hspace*{1cm} \qa $I = ]- \infty ; 3[$   et   $J = [3 ; +\infty[$ \\


\vspace*{0.5cm}

\exo{3}Indiquer si les propositions suivantes sont \textbf{vraies} ou \textbf{fausses}. \textit{Aucune justification n'est demandée.}\\

\initqa \qa $10^{-7} \notin ]-\infty;0[$ \hspace*{1cm} \qa  $-4 \in ]-\infty;4[$ \hspace*{1cm} \qa $\dfrac{1}{3} \notin [0;0,333[$\\

 \qa $]-\infty;-2] \cup ]-2;7[=]-\infty;7]$ \hspace*{0.75cm} \qa  $]-\infty;2[ \cap ]-1;15[ = ]-1;2]$  \hspace*{0.75cm} \qa $]-\infty;-2 ] \cap ]-2;7[=\{-2 \}$ \\


\vspace*{0.5cm}

\exo{3} Toutes les affirmations suivantes sont \textbf{fausses}. Pour chacune d'elle, donner \textbf{un contre-exemple}.\\

\initq \q   Si $x \in [0 ; 10]$, alors $x$ est un entier naturel.\\

\q Si $1 \le x \le 3$ alors $x \in ]1;3[$.\\

\q Pour tout entier $n$, si $n$ est divisible par 3, il est divisible par 6.\\


\exo{} \textbf{BONUS}\\
Soit $n + 1$ et $n$ deux entiers consécutifs.\\
Démontrer que la somme de ces deux entiers est égale à la différence de leurs carrés.\\

\end{document}
