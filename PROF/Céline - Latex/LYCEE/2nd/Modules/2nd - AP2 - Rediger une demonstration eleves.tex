\documentclass[a4paper,12pt]{article}
\usepackage{amsmath,amsthm,amsfonts,amssymb,amscd,amstext,vmargin,graphics,graphicx,tabularx,multicol} 
\usepackage[francais]{babel}
\usepackage[utf8]{inputenc}  
\usepackage[T1]{fontenc} 
\usepackage{pstricks-add,tikz,tkz-tab,variations}
\usepackage[autolanguage,np]{numprint} 
\usepackage{calc}
\usepackage{mathrsfs}

\setmarginsrb{1.5cm}{0.5cm}{1cm}{0.5cm}{0cm}{0cm}{0cm}{0cm} %Gauche, haut, droite, haut
\newcounter{numexo}
\newcommand{\exo}[1]{\stepcounter{numexo}\noindent{\bf Exercice~\thenumexo} : }
\reversemarginpar

\newcommand{\bmul}[1]{\begin{multicols}{#1}}
\newcommand{\emul}{\end{multicols}}

\renewcommand{\thesection}{\Roman{section}.}
	\renewcommand{\thesubsection}{\hspace{.5cm}\arabic{subsection}.}
	\renewcommand{\thesubsubsection}{\hspace{1cm}\alph{subsubsection})}

\newcounter{enumtabi}
\newcounter{enumtaba}
\newcommand{\q}{\stepcounter{enumtabi} \theenumtabi)  }
\newcommand{\qa}{\stepcounter{enumtaba} (\alph{enumtaba}) }
\newcommand{\initq}{\setcounter{enumtabi}{0}}
\newcommand{\initqa}{\setcounter{enumtaba}{0}}

\newcommand{\be}{\begin{enumerate}}
\newcommand{\ee}{\end{enumerate}}
\newcommand{\bi}{\begin{itemize}}
\newcommand{\ei}{\end{itemize}}
\newcommand{\bp}{\begin{pspicture*}}
\newcommand{\ep}{\end{pspicture*}}
\newcommand{\bt}{\begin{tabular}}
\newcommand{\et}{\end{tabular}}
\renewcommand{\tabularxcolumn}[1]{>{\centering}m{#1}} %(colonne m{} centrée, au lieu de p par défault) 
\newcommand{\tnl}{\tabularnewline}

\newcommand{\trait}{\noindent \rule{\linewidth}{0.2mm}}
\newcommand{\hs}[1]{\hspace{#1}}
\newcommand{\vs}[1]{\vspace{#1}}

\newcommand{\N}{\mathbb{N}}
\newcommand{\Z}{\mathbb{Z}}
\newcommand{\R}{\mathbb{R}}
\newcommand{\C}{\mathbb{C}}
\newcommand{\Dcal}{\mathcal{D}}
\newcommand{\Ccal}{\mathcal{C}}
\newcommand{\mc}{\mathcal}

\newcommand{\vect}[1]{\overrightarrow{#1}}
\newcommand{\ds}{\displaystyle}
\newcommand{\eq}{\quad \Leftrightarrow \quad}
\newcommand{\vecti}{\vec{\imath}}
\newcommand{\vectj}{\vec{\jmath}}
\newcommand{\Oij}{(O;\vec{\imath}, \vec{\jmath})}
\newcommand{\OIJ}{(O;I,J)}


\newcommand{\reponse}[1][1]{%
\multido{}{#1}{\makebox[\linewidth]{\rule[0pt]{0pt}{20pt}\dotfill}
}}

\newcommand{\titre}[5] 
% #1: titre #2: haut gauche #3: bas gauche #4: haut droite #5: bas droite
{
\noindent #2 \hfill #4 \\
#3 \hfill #5

\vspace{-1.6cm}

\begin{center}\rule{6cm}{0.5mm}\end{center}
\vspace{0.2cm}
\begin{center}{\Large{\textbf{#1}}}\end{center}
\begin{center}\rule{6cm}{0.5mm}\end{center}
}



\begin{document}
\pagestyle{empty}

\titre{Séance d'AP 2 : Rédiger une démonstration}{}{}{2nd}{}

\vspace*{0.2cm}

\section{Rappels d'arithmétique}



\textbf{\textcolor{violet}{\underline{MULTIPLE - DIVISEUR}}}\\
Soient $a$ et $b$ deux nombres entiers relatifs ($b$ non nul).\\
S'il existe un entier relatif $k$ tel que $a = bk$, on dit que $a$ est un \textbf{multiple} de $b$ et que $b$ est un \textbf{diviseur} de $a$.\\

\underline{Exemples :}
\bi
\item\textit{Montrer que 21 est un multiple de 7 : }\\
\reponse[2]\\

\item \textit{Montrer que tout multiple de 9 est multiple de 3 :} \\
\reponse[6]\\
\ei

\vspace*{0.5cm}

\textbf{\textcolor{violet}{\underline{PAIR - IMPAIR - PREMIER}}}\\
Soit $a$ un nombre entier relatif.
\bi 
\item $a$ est \textbf{pair} si 2 est un diviseur de $a$, c'est-à-dire s'il existe un nombre entier relatif $k$ tel que $a = 2k$.
\item $a$ est \textbf{impair} si 2 n’est pas un diviseur de $a$, c'est-à-dire s'il existe un nombre entier relatif $k$ tel que $a = 2k + 1$.
\item $a$ est \textbf{premier} s'il a exactement deux diviseurs positifs : 1 et lui-même.\\
\ei

\underline{Exemples :}
\bi
\item \textit{Montrer que 37 est un nombre impair :}\\
\reponse[2]\\
 
\item \textit{Montrer que la somme de deux nombres impairs est un nombre pair :}\\
 \reponse[6]\\
\ei



\newpage
\vspace*{0.5cm}

Sur votre cahier d'exercices rédiger les démonstrations suivantes :

\vspace*{0.5cm}

\textbf{Démonstration 1}\\
\textit{Propriété 1 : "La somme de deux multiples de $a$ est un multiple de $a$."}

\vspace*{0.5cm}


\textbf{Démonstration 2}\\
\textit{Propriété 2 : "Le carré d'un nombre impair est impair."}\\



\textbf{Démonstration 3}\\
\textit{Propriété 3 : "$\dfrac{1}{7}$ n'est pas un nombre décimal."}\\

\vspace*{0.5cm}
\vspace*{0.5cm}

Sur votre cahier d'exercices rédiger les démonstrations suivantes :
\vspace*{0.5cm}

\textbf{Démonstration 1}\\
\textit{Propriété 1 : "La somme de deux multiples de $a$ est un multiple de $a$."}

\vspace*{0.5cm}

\textbf{Démonstration 2}\\
\textit{Propriété 2 : "Le carré d'un nombre impair est impair."}\\



\textbf{Démonstration 3}\\
\textit{Propriété 3 : "$\dfrac{1}{7}$ n'est pas un nombre décimal."}\\
\vspace*{0.5cm}
\vspace*{0.5cm}
Sur votre cahier d'exercices rédiger les démonstrations suivantes :
\vspace*{0.5cm}


\textbf{Démonstration 1}\\
\textit{Propriété 1 : "La somme de deux multiples de $a$ est un multiple de $a$."}

\vspace*{0.5cm}


\textbf{Démonstration 2}\\
\textit{Propriété 2 : "Le carré d'un nombre impair est impair."}\\



\textbf{Démonstration 3}\\
\textit{Propriété 3 : "$\dfrac{1}{7}$ n'est pas un nombre décimal."}\\

\vspace*{0.5cm}
\vspace*{0.5cm}
Sur votre cahier d'exercices rédiger les démonstrations suivantes :
\vspace*{0.5cm}


\textbf{Démonstration 1}\\
\textit{Propriété 1 : "La somme de deux multiples de $a$ est un multiple de $a$."}

\vspace*{0.5cm}


\textbf{Démonstration 2}\\
\textit{Propriété 2 : "Le carré d'un nombre impair est impair."}\\



\textbf{Démonstration 3}\\
\textit{Propriété 3 : "$\dfrac{1}{7}$ n'est pas un nombre décimal."}\\

\vspace*{0.5cm}
\end{document}
