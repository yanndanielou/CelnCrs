\documentclass[a4paper,12pt]{article}
\usepackage{amsmath,amsthm,amsfonts,amssymb,amscd,amstext,vmargin,graphics,graphicx,tabularx,multicol} 
\usepackage[francais]{babel}
\usepackage[utf8]{inputenc}  
\usepackage[T1]{fontenc} 
\usepackage{pstricks-add,tikz,tkz-tab,variations}
\usepackage[autolanguage,np]{numprint} 
\usepackage{calc}
\usepackage{mathrsfs}

\setmarginsrb{1.5cm}{0.5cm}{1cm}{0.5cm}{0cm}{0cm}{0cm}{0cm} %Gauche, haut, droite, haut
\newcounter{numexo}
\newcommand{\exo}[1]{\stepcounter{numexo}\noindent{\bf Exercice~\thenumexo} : }
\reversemarginpar

\newcommand{\bmul}[1]{\begin{multicols}{#1}}
\newcommand{\emul}{\end{multicols}}

\renewcommand{\thesection}{\Roman{section}.}
	\renewcommand{\thesubsection}{\hspace{.5cm}\arabic{subsection}.}
	\renewcommand{\thesubsubsection}{\hspace{1cm}\alph{subsubsection})}

\newcounter{enumtabi}
\newcounter{enumtaba}
\newcommand{\q}{\stepcounter{enumtabi} \theenumtabi)  }
\newcommand{\qa}{\stepcounter{enumtaba} (\alph{enumtaba}) }
\newcommand{\initq}{\setcounter{enumtabi}{0}}
\newcommand{\initqa}{\setcounter{enumtaba}{0}}

\newcommand{\be}{\begin{enumerate}}
\newcommand{\ee}{\end{enumerate}}
\newcommand{\bi}{\begin{itemize}}
\newcommand{\ei}{\end{itemize}}
\newcommand{\bp}{\begin{pspicture*}}
\newcommand{\ep}{\end{pspicture*}}
\newcommand{\bt}{\begin{tabular}}
\newcommand{\et}{\end{tabular}}
\renewcommand{\tabularxcolumn}[1]{>{\centering}m{#1}} %(colonne m{} centrée, au lieu de p par défault) 
\newcommand{\tnl}{\tabularnewline}

\newcommand{\trait}{\noindent \rule{\linewidth}{0.2mm}}
\newcommand{\hs}[1]{\hspace{#1}}
\newcommand{\vs}[1]{\vspace{#1}}

\newcommand{\N}{\mathbb{N}}
\newcommand{\Z}{\mathbb{Z}}
\newcommand{\R}{\mathbb{R}}
\newcommand{\C}{\mathbb{C}}
\newcommand{\Dcal}{\mathcal{D}}
\newcommand{\Ccal}{\mathcal{C}}
\newcommand{\mc}{\mathcal}

\newcommand{\vect}[1]{\overrightarrow{#1}}
\newcommand{\ds}{\displaystyle}
\newcommand{\eq}{\quad \Leftrightarrow \quad}
\newcommand{\vecti}{\vec{\imath}}
\newcommand{\vectj}{\vec{\jmath}}
\newcommand{\Oij}{(O;\vec{\imath}, \vec{\jmath})}
\newcommand{\OIJ}{(O;I,J)}


\newcommand{\reponse}[1][1]{%
\multido{}{#1}{\makebox[\linewidth]{\rule[0pt]{0pt}{20pt}\dotfill}
}}

\newcommand{\titre}[5] 
% #1: titre #2: haut gauche #3: bas gauche #4: haut droite #5: bas droite
{
\noindent #2 \hfill #4 \\
#3 \hfill #5

\vspace{-1.6cm}

\begin{center}\rule{6cm}{0.5mm}\end{center}
\vspace{0.2cm}
\begin{center}{\Large{\textbf{#1}}}\end{center}
\begin{center}\rule{6cm}{0.5mm}\end{center}
}



\begin{document}
\pagestyle{empty}

\titre{Séance d'AP 4 : Développements}{}{}{2nd}{}
\vspace*{0.75cm}

\textbf{RAPPELS}\\

\textbf{\textcolor{green}{\underline{Développement simple :}}} Soient $a$, $b$ et $k$ trois réels, $k(a+b)=ka+kb$ et $k(a-b)=ka-kb$  \\

\underline{Exemples :}\\
\hspace*{3cm}  $3x(2-5x)$  \hfill    $10-(-9x-7)$  \hspace*{5cm}\\



\textbf{\textcolor{green}{\underline{Développement double :}}}
Soient $a$, $b$, $c$ et $d$ quatre réels, $(a+b)(c+d)=ac+ad+bc+bd$ \\

\underline{Exemples :}\\
\hspace*{3cm}  $(4x-2)(9-3x)$  \hfill    $(x-6)(3+x)-(2x-1)(3+x)$  \hspace*{5cm}\\

\textbf{\textcolor{green}{\underline{Identités remarquables :}}}
Soient $a$ et $b$ deux réels, $(a+b)^2=a^2+2ab+b^2$   \\
\hspace*{10.2cm} $(a-b)^2=a^2-2ab+b^2$ \\
\hspace*{10.2cm}  $(a+b)(a-b)=a^2-b^2$ \\

\underline{Exemples :}\\
\hspace*{3cm} $(2x-3)^2$  \hfill   $(10-7x)^2$  \hfill  $(8-3x)(8+3x)$ \hspace*{3cm}\\

\vspace*{1.5cm}
\exo\\ Développer les expressions littérales suivantes.
\bmul{3}
\initqa \qa  $-5\left( x-\dfrac{2}{3}\right)$\\

\qa $(7-2x)(3x+8)$

\columnbreak

\qa  $12x-\left( \dfrac{1}{9}x-\dfrac{3}{4}\right)$\\

\qa $(x+1)(2-x)$


\columnbreak

\qa  $3+x-2(1-x)$\\

\qa $3(6x-4)+(2x+1)(6-x)$



\emul

\vspace*{1cm}
\exo\\ Développer les expressions littérales suivantes.
\bmul{3}
\initqa \qa  $(3-x)^2$\\

\qa $(2x+3)(2x-3)$

\columnbreak

\qa    $(7x+2)^2$\\ 

\qa $-5x(8x-11)+(10x+4)^2$

\columnbreak

\qa  $(x-2)^2+(7x-1)^2$\\

\qa $(4+5x)^2+(6x-1)(6x+1)$



\emul




   
\end{document}
