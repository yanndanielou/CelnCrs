\documentclass[a4paper,11pt]{article}
\usepackage{amsmath,amsthm,amsfonts,amssymb,amscd,amstext,vmargin,graphics,graphicx,tabularx,multicol} \usepackage[french]{babel}
\usepackage[utf8]{inputenc}  
\usepackage[T1]{fontenc} 
\usepackage[T1]{fontenc}
\usepackage{amsmath,amssymb}
\usepackage{pstricks-add,tikz,tkz-tab,variations}
\usepackage[autolanguage,np]{numprint} 
\usepackage{color}
\usepackage{ulem}

\setmarginsrb{1.5cm}{0.5cm}{1cm}{0.5cm}{0cm}{0cm}{0cm}{0cm} %Gauche, haut, droite, haut
\newcounter{numexo}
\newcommand{\exo}[1]{\stepcounter{numexo}\noindent{\bf Exercice~\thenumexo} : \marginpar{\hfill /#1}}
\reversemarginpar


\newcounter{enumtabi}
\newcounter{enumtaba}
\newcommand{\q}{\stepcounter{enumtabi} \theenumtabi.  }
\newcommand{\qa}{\stepcounter{enumtaba} (\alph{enumtaba}) }
\newcommand{\initq}{\setcounter{enumtabi}{0}}
\newcommand{\initqa}{\setcounter{enumtaba}{0}}

\newcommand{\be}{\begin{enumerate}}
\newcommand{\ee}{\end{enumerate}}
\newcommand{\bi}{\begin{itemize}}
\newcommand{\ei}{\end{itemize}}
\newcommand{\bp}{\begin{pspicture*}}
\newcommand{\ep}{\end{pspicture*}}
\newcommand{\bt}{\begin{tabular}}
\newcommand{\et}{\end{tabular}}
\renewcommand{\tabularxcolumn}[1]{>{\centering}m{#1}} %(colonne m{} centrée, au lieu de p par défault) 
\newcommand{\tnl}{\tabularnewline}

\newcommand{\trait}{\noindent \rule{\linewidth}{0.2mm}}
\newcommand{\hs}[1]{\hspace{#1}}
\newcommand{\vs}[1]{\vspace{#1}}

\newcommand{\N}{\mathbb{N}}
\newcommand{\Z}{\mathbb{Z}}
\newcommand{\R}{\mathbb{R}}
\newcommand{\C}{\mathbb{C}}
\newcommand{\Dcal}{\mathcal{D}}
\newcommand{\Ccal}{\mathcal{C}}
\newcommand{\mc}{\mathcal}

\newcommand{\vect}[1]{\overrightarrow{#1}}
\newcommand{\ds}{\displaystyle}
\newcommand{\eq}{\quad \Leftrightarrow \quad}
\newcommand{\vecti}{\vec{\imath}}
\newcommand{\vectj}{\vec{\jmath}}
\newcommand{\Oij}{(O;\vec{\imath}, \vec{\jmath})}
\newcommand{\OIJ}{(O;I,J)}

\newcommand{\bmul}[1]{\begin{multicols}{#1}}
\newcommand{\emul}{\end{multicols}}


\newcommand{\reponse}[1][1]{%
\multido{}{#1}{\makebox[\linewidth]{\rule[0pt]{0pt}{20pt}\dotfill}
}}

\newcommand{\titre}[5] 
% #1: titre #2: haut gauche #3: bas gauche #4: haut droite #5: bas droite
{
\noindent #2 \hfill #4 \\
#3 \hfill #5

\vspace{-1.6cm}

\begin{center}\rule{6cm}{0.5mm}\end{center}
\vspace{0.2cm}
\begin{center}{\large{\textbf{#1}}}\end{center}
\begin{center}\rule{6cm}{0.5mm}\end{center}
}



\begin{document}
\pagestyle{empty}
\titre{Interrogation : Fonction et identités remarquables}{Nom}{Prénom}{Date}{Classe}


\exo{1} Compléter les identités remarquables suivantes :\\

\bi




\item $ (a-b)^{2}$ = . . . . . . . . . . . . . . . . . . . . . . . . \\

\item  . . . . . . . . . . . . . . . . . . . . . . . . =  $a^{2} - b^{2}$ \\

\ei


\exo{3} Développer et réduire les identités remarquables suivantes :

\bmul{3}

$L = (x + 1)^{2}$\\
\reponse[2]\\

\columnbreak

$E = (8x - 2)^{2}$\\
\reponse[2]\\

\columnbreak

$A = (44x - 100)(44x+ 100)$\\
\reponse[2]\\

\emul





\exo{3} Factoriser les identités remarquables suivantes :

\bmul{3}
$E = 25x^{2} + 40 x + 16$\\
\reponse[2]\\


\columnbreak


$R = 81x^{2} - 36 $\\
\reponse[2]\\
\columnbreak

$A = 49 - 42x + 9x^{2}$\\
\reponse[2]\\

\emul



\exo{3} Soit f la fonction suivante : $f(x)= (x+1){2} - (2x+9)(2x-9)$\\

\initq \q Développer f et montrer que $f(x) = -3x^{2} +2x + 81$\\
\reponse[6]\\

\q Calculer les images de -2 par la fonction f? de 0 et de 11 par la fonction f?\\
\reponse[6]\\

\q Trouver un antécédent de 82 par la fonction f\\
\reponse[2]\\







\end{document}
