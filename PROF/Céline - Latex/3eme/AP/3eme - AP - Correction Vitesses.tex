\documentclass[a4paper,11pt]{article}
\usepackage{amsmath,amsthm,amsfonts,amssymb,amscd,amstext,vmargin,graphics,graphicx,tabularx,multicol} 
\usepackage[francais]{babel}
\usepackage[utf8]{inputenc}  
\usepackage[T1]{fontenc} 
\usepackage{pstricks-add,tikz,tkz-tab,variations}
\usepackage[autolanguage,np]{numprint} 
\usepackage{calc}

\setmarginsrb{1.5cm}{0.5cm}{1cm}{0.5cm}{0cm}{0cm}{0cm}{0cm} %Gauche, haut, droite, haut
\newcounter{numexo}
\newcommand{\exo}[1]{\stepcounter{numexo}\noindent{\bf Exercice~\thenumexo} : }
\reversemarginpar

\newcommand{\bmul}[1]{\begin{multicols}{#1}}
\newcommand{\emul}{\end{multicols}}

\newcounter{enumtabi}
\newcounter{enumtaba}
\newcommand{\q}{\stepcounter{enumtabi} \theenumtabi.  }
\newcommand{\qa}{\stepcounter{enumtaba} (\alph{enumtaba}) }
\newcommand{\initq}{\setcounter{enumtabi}{0}}
\newcommand{\initqa}{\setcounter{enumtaba}{0}}

\newcommand{\be}{\begin{enumerate}}
\newcommand{\ee}{\end{enumerate}}
\newcommand{\bi}{\begin{itemize}}
\newcommand{\ei}{\end{itemize}}
\newcommand{\bp}{\begin{pspicture*}}
\newcommand{\ep}{\end{pspicture*}}
\newcommand{\bt}{\begin{tabular}}
\newcommand{\et}{\end{tabular}}
\renewcommand{\tabularxcolumn}[1]{>{\centering}m{#1}} %(colonne m{} centrée, au lieu de p par défault) 
\newcommand{\tnl}{\tabularnewline}

\newcommand{\trait}{\noindent \rule{\linewidth}{0.2mm}}
\newcommand{\hs}[1]{\hspace{#1}}
\newcommand{\vs}[1]{\vspace{#1}}

\newcommand{\N}{\mathbb{N}}
\newcommand{\Z}{\mathbb{Z}}
\newcommand{\R}{\mathbb{R}}
\newcommand{\C}{\mathbb{C}}
\newcommand{\Dcal}{\mathcal{D}}
\newcommand{\Ccal}{\mathcal{C}}
\newcommand{\mc}{\mathcal}

\newcommand{\vect}[1]{\overrightarrow{#1}}
\newcommand{\ds}{\displaystyle}
\newcommand{\eq}{\quad \Leftrightarrow \quad}
\newcommand{\vecti}{\vec{\imath}}
\newcommand{\vectj}{\vec{\jmath}}
\newcommand{\Oij}{(O;\vec{\imath}, \vec{\jmath})}
\newcommand{\OIJ}{(O;I,J)}


\newcommand{\reponse}[1][1]{%
\multido{}{#1}{\makebox[\linewidth]{\rule[0pt]{0pt}{20pt}\dotfill}
}}

\newcommand{\titre}[5] 
% #1: titre #2: haut gauche #3: bas gauche #4: haut droite #5: bas droite
{
\noindent #2 \hfill #4 \\
#3 \hfill #5

\vspace{-1.6cm}

\begin{center}\rule{6cm}{0.5mm}\end{center}
\vspace{0.2cm}
\begin{center}{\large{\textbf{#1}}}\end{center}
\begin{center}\rule{6cm}{0.5mm}\end{center}
}



\begin{document}
\pagestyle{empty}
\titre{Correction séance d'AP 4 : Notions de vitesse}{}{}{3ème}{}

\vspace*{0.2cm}


\exo \\

\q Un piéton met 2h pour parcourir 12,8 km. Donc v = 12,8 $\div$ 2 = 6,4 km/h.\\

\q Un camion roule pendant 3h à une vitesse moyenne de 85 km/h.\\
Donc  $d= v \times t$\\
 $d= 85 \times 3$\\ 
 $d = 255 km$.\\
 
 \q Une voiture roule à une vitesse moyenne de 75,5 km/h et parcourt 181,2 km.\\
 Donc $t= d \div v$\\
  $t= 181,2 \div 75,5$\\ 
 $t = 2,4 h$. \\
 
 \begin{tabular}{|c|c|}
 \hline 
 1h & 60 min \\ 
 \hline 
 0,4 h &  ? \\ 
 \hline 
 \end{tabular}  On effectue donc un produit en croix : 0,4 x 60 $\div$ 1 = 24 min.\\
 Donc 2,4 h ou 2h et 24 minutes.\\
 
 \exo \\
 
 Supposons que la hauteur du volcan (de la base jusqu'au sommet) soit de 2 500 m et que la nuée ardente dévale la pente à une vitesse de 4,58 km/min.\\

\initq  \q Calculons la longueur de la pente :\\

On sait que le triangle OAB est rectangle en O. \\
D'après le théorème de Pythagore, on a :\\
$AB^{2} = 0B^{2} + AO^{2}$\\
$AB^{2} = 2 500^{2} + 750^{2}$\\
$AB^{2} = 6812500$\\
$AB= \sqrt{6812500}$ or AB>0\\
$AB \simeq 2610 m$\\


\q La vitesse de la pente est de 4,58 km/min.

\qa Conversion en m/s :\\

4,58 km/min = 4 580 m/min, ce qui signifie que l'on parcourt 4 580 m en 1 minute.
Or 1 minute = 60 secondes. Donc on parcourt 4 580 m en 60 secondes.
Pour savoir combien de m on parcourt en 1 seconde, on va donc diviser par 60.\\
4 580 m/min $\simeq$ 76,3 m/s\\


\qa Conversion en km/h :\\

4,58 km/min signifie que l'on parcourt 4,58 km en 1 minute. Or 1 heure c'est 60 min, il suffit donc de multiplier par 60.\\
4,58 km/min = 274,8 km/h.\\

\q On cherche maintenant le temps que la nuée ardente va mettre pour dévaler la pente.\\

$t= d \div v $\\
$t= 2610 \div 76,3 $ Je choisi la vitesse en m/s car la distance est en m.\\
$t \simeq 34,2 s $\\

\newpage

\exo \\

8 jours et 22 heures = $8 \times 24 + 22 = 214 heures$\\

On cherche la vitesse moyenne en km/h de la navette :\\

$v = d \div t$\\
$v = 5,8 \times 10^{6} \div 214$\\
$v \simeq 27 102,8 km/h$\\
 
\exo \\

\initq \q 
\initqa \qa Convertir 138,89 m/s en m/h :\\

3 600 s = 1h Donc 138,89 m/s = 138,89 $\times$ 3 600 m/h = 500 004 m/h.\\

\qa Convertir 138,89 m/s en km/h :\\

138,89 m/s = 500 004 m/h = 500,004 km/h.\\

\q $t = d \div v$\\
$t = 1,3 \div 500,004$ Je choisi la vitesse en km/h car la distance est en km.\\
$t \simeq 0,0026 h \simeq 0,0026 \times 3600 \simeq 9,4 secondes$\\

La vague atteindra la maison en 9,4 secondes.\\

\q La vague parcourt 138,89 m en 1 seconde, 8 333,4 m en 1 minute, 375 003 m en 45 minutes.\\

\q On cherche une distance :\\
$d = v \times t$\\
$d = 8 333,4 \times 18$\\
$d = 150 001,2 m$\\

Le volcan serait donc à 150 000 m ou bien 150 km environ.\\

\exo \\

On cherche à savoir si Nina peut faire une dernière descente en 10 minutes pour être à l'heure au rendez-vous.\\

Le temps pour la montée :\\
$t_{M} = d \div v$\\
$t_{M} = 860 \div 3,33$\\
$t_{M} \simeq 258,3 s$ soit 4,305 minutes\\

Le temps pour la descente :\\
La piste fait 2km de long et Nina descend à une vitesse de 15km/h.\\
$t_{D} = 2 \div 15$\\
$t_{D} \simeq 0,13 h$ soit $0,13 \times 60 = 8$ minutes\\

Lorsque l'on additionne les 2 temps, on obtient environ 12 minutes, or Nina avait que 10 minutes pour faire un aller-retour donc elle n'a pas le temps de refaire une piste.\\



\end{document}
