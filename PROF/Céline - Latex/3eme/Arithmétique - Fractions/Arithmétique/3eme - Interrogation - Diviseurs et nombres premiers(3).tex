\documentclass[a4paper,11pt]{article}
\usepackage{amsmath,amsthm,amsfonts,amssymb,amscd,amstext,vmargin,graphics,graphicx,tabularx,multicol} 
\usepackage[francais]{babel}
\usepackage[utf8]{inputenc}  
\usepackage[T1]{fontenc} 
\usepackage{pstricks-add,tikz,tkz-tab,variations}
\usepackage[autolanguage,np]{numprint} 

\setmarginsrb{1.5cm}{0.5cm}{1cm}{0.5cm}{0cm}{0cm}{0cm}{0cm} %Gauche, haut, droite, haut
\newcounter{numexo}
\newcommand{\exo}[1]{\stepcounter{numexo}\noindent{\bf Exercice~\thenumexo} : \marginpar{\hfill /#1}}
\reversemarginpar


\newcounter{enumtabi}
\newcounter{enumtaba}
\newcommand{\q}{\stepcounter{enumtabi} \theenumtabi)  }
\newcommand{\qa}{\stepcounter{enumtaba} (\alph{enumtaba}) }
\newcommand{\initq}{\setcounter{enumtabi}{0}}
\newcommand{\initqa}{\setcounter{enumtaba}{0}}

\newcommand{\be}{\begin{enumerate}}
\newcommand{\ee}{\end{enumerate}}
\newcommand{\bi}{\begin{itemize}}
\newcommand{\ei}{\end{itemize}}
\newcommand{\bp}{\begin{pspicture*}}
\newcommand{\ep}{\end{pspicture*}}
\newcommand{\bt}{\begin{tabular}}
\newcommand{\et}{\end{tabular}}
\renewcommand{\tabularxcolumn}[1]{>{\centering}m{#1}} %(colonne m{} centrée, au lieu de p par défault) 
\newcommand{\tnl}{\tabularnewline}

\newcommand{\bmul}[1]{\begin{multicols}{#1}}
\newcommand{\emul}{\end{multicols}}

\newcommand{\trait}{\noindent \rule{\linewidth}{0.2mm}}
\newcommand{\hs}[1]{\hspace{#1}}
\newcommand{\vs}[1]{\vspace{#1}}

\newcommand{\N}{\mathbb{N}}
\newcommand{\Z}{\mathbb{Z}}
\newcommand{\R}{\mathbb{R}}
\newcommand{\C}{\mathbb{C}}
\newcommand{\Dcal}{\mathcal{D}}
\newcommand{\Ccal}{\mathcal{C}}
\newcommand{\mc}{\mathcal}

\newcommand{\vect}[1]{\overrightarrow{#1}}
\newcommand{\ds}{\displaystyle}
\newcommand{\eq}{\quad \Leftrightarrow \quad}
\newcommand{\vecti}{\vec{\imath}}
\newcommand{\vectj}{\vec{\jmath}}
\newcommand{\Oij}{(O;\vec{\imath}, \vec{\jmath})}
\newcommand{\OIJ}{(O;I,J)}


\newcommand{\reponse}[1][1]{%
\multido{}{#1}{\makebox[\linewidth]{\rule[0pt]{0pt}{20pt}\dotfill}
}}

\newcommand{\titre}[5] 
% #1: titre #2: haut gauche #3: bas gauche #4: haut droite #5: bas droite
{
\noindent #2 \hfill #4 \\
#3 \hfill #5

\vspace{-1.6cm}

\begin{center}\rule{6cm}{0.5mm}\end{center}
\vspace{0.2cm}
\begin{center}{\large{\textbf{#1}}}\end{center}
\begin{center}\rule{6cm}{0.5mm}\end{center}
}



\begin{document}
\pagestyle{empty}

\titre{Interrogation: Arithmétique (1)}{Nom :}{Prénom :}{Classe}{Date}

\vspace*{0.5cm}

\exo{3} Questions de cours


\q Donner la définition de deux nombres premiers et citer tous ceux inférieurs à 30.\\
\reponse[6]\\


\q On donne les nombres suivants :\\
48 \hfill ; \hfill 58 180 \hfill; \hfill 27 900 \hfill; \hfill 63 604 \hfill; \hfill 42 324 \hfill; \hfill 34 410.\\
Trouver ceux qui sont à la fois divisible par 3 et par 4. (Aucune justification n'est attendue.)\\
\reponse[4]\\


\q Trouver un nombre à quatre chiffres à la fois divisible par 2;  divisible par 3; divisible par 5 et non divisible par 9. (Aucune justification n'est attendue.)\\
\reponse[2]\\




\exo{2}

L'ensemble des écrits de Victor Hugo a été republié après sa mort en 153 volumes. La bibliothécaire classe ces volumes à raison de 7 volumes par étagères.\\

 \initq \q Combien d'étagères faut-il pour exposer toute l'\oe{uvre} de Victor Hugo? Justifier votre réponse avec une division euclidienne.\\
\reponse[7]\\

\q Combien de volumes l'étagère incomplète contiendra-t-elle? Justifier votre réponse.\\
\reponse[3]\\

\newpage


\exo{2.5}
\initq \q Citer tous les diviseurs de 48 et 72.\\
\reponse[10]\\

\q Quels sont tous \textbf{les diviseurs commun}s à 48 et 72?\\
\reponse[2]\\

\exo{2.5} Un snack vend des barquettes composées de nems et de samossas.\\
Le cuisinier a préparé 162 nems et 108 samossas.\\

Dans chaque barquette : \\
- le nombre de nems doit être le même. \\
- le nombre de samossas doit être le même.\\

Tous les nems et tous les samossas doivent être utilisés.\\

\initq \q Avec ces conditions, quel nombre de barquettes \textbf{au maximum} pourra-t-il réaliser?\\
\reponse[8]\\


\q Dans ce cas, combien y aura-t-il de nems et de samossas dans chaque barquette ?\\
\reponse[2]\\


\exo{} Bonus\\
Un nombre est \textbf{"gentil"} s'il est multiple des dix premiers nombres entiers.\\

\initq \noindent \q Expliquer pourquoi 10 080 est \textbf{"gentil"}. \\
\q Trouver le plus petit nombre \textbf{"gentil"}, en expliquant votre recherche.\\ 
\reponse[2]

\end{document}
