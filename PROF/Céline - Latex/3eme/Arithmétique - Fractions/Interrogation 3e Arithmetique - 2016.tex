\documentclass[a4paper,11pt]{article}
\usepackage{amsmath,amsthm,amsfonts,amssymb,amscd,amstext,vmargin,graphics,graphicx,tabularx,multicol} 
\usepackage[francais]{babel}
\usepackage[utf8]{inputenc}  
\usepackage[T1]{fontenc} 
\usepackage{pstricks-add,tikz,tkz-tab,variations}
\usepackage[autolanguage,np]{numprint} 

\setmarginsrb{1.5cm}{0.5cm}{1cm}{0.5cm}{0cm}{0cm}{0cm}{0cm} %Gauche, haut, droite, haut
\newcounter{numexo}
\newcommand{\exo}[1]{\stepcounter{numexo}\noindent{\bf Exercice~\thenumexo} : \marginpar{\hfill /#1}}
\reversemarginpar


\newcounter{enumtabi}
\newcounter{enumtaba}
\newcommand{\q}{\stepcounter{enumtabi} \theenumtabi.  }
\newcommand{\qa}{\stepcounter{enumtaba} (\alph{enumtaba}) }
\newcommand{\initq}{\setcounter{enumtabi}{0}}
\newcommand{\initqa}{\setcounter{enumtaba}{0}}

\newcommand{\be}{\begin{enumerate}}
\newcommand{\ee}{\end{enumerate}}
\newcommand{\bi}{\begin{itemize}}
\newcommand{\ei}{\end{itemize}}
\newcommand{\bp}{\begin{pspicture*}}
\newcommand{\ep}{\end{pspicture*}}
\newcommand{\bt}{\begin{tabular}}
\newcommand{\et}{\end{tabular}}
\renewcommand{\tabularxcolumn}[1]{>{\centering}m{#1}} %(colonne m{} centrée, au lieu de p par défault) 
\newcommand{\tnl}{\tabularnewline}

\newcommand{\bmul}[1]{\begin{multicols}{#1}}
\newcommand{\emul}{\end{multicols}}

\newcommand{\trait}{\noindent \rule{\linewidth}{0.2mm}}
\newcommand{\hs}[1]{\hspace{#1}}
\newcommand{\vs}[1]{\vspace{#1}}

\newcommand{\N}{\mathbb{N}}
\newcommand{\Z}{\mathbb{Z}}
\newcommand{\R}{\mathbb{R}}
\newcommand{\C}{\mathbb{C}}
\newcommand{\Dcal}{\mathcal{D}}
\newcommand{\Ccal}{\mathcal{C}}
\newcommand{\mc}{\mathcal}

\newcommand{\vect}[1]{\overrightarrow{#1}}
\newcommand{\ds}{\displaystyle}
\newcommand{\eq}{\quad \Leftrightarrow \quad}
\newcommand{\vecti}{\vec{\imath}}
\newcommand{\vectj}{\vec{\jmath}}
\newcommand{\Oij}{(O;\vec{\imath}, \vec{\jmath})}
\newcommand{\OIJ}{(O;I,J)}


\newcommand{\reponse}[1][1]{%
\multido{}{#1}{\makebox[\linewidth]{\rule[0pt]{0pt}{20pt}\dotfill}
}}

\newcommand{\titre}[5] 
% #1: titre #2: haut gauche #3: bas gauche #4: haut droite #5: bas droite
{
\noindent #2 \hfill #4 \\
#3 \hfill #5

\vspace{-1.6cm}

\begin{center}\rule{6cm}{0.5mm}\end{center}
\vspace{0.2cm}
\begin{center}{\large{\textbf{#1}}}\end{center}
\begin{center}\rule{6cm}{0.5mm}\end{center}
}



\begin{document}
\pagestyle{empty}
\titre{Interrogation: Arithmétiques }{Nom :}{Prénom :}{Classe}{Date}


\exo{2,5} 

\q Compléter les définitions suivantes :

- Un nombre est dit \textbf{premier} si \reponse[2]\\

- Deux nombres sont \textbf{premiers entre eux} si \reponse[2]\\

\q Parmi les 20 premiers nombres entiers naturel, quels sont ceux qui sont premiers ?\\
\reponse[1]\\

\vspace*{0.6cm}

\exo{2,5}

\initq \q  Écrire la liste des diviseurs de 75 et celle de 46.\\
\reponse[3]\\

\q  Montrer que 75 et 46 sont premiers entre eux.\\
\reponse[2]\\

\vspace*{0.6cm}

\exo{1}
Je suis un nombre entier naturel compris entre 1 509 et 1 534. Je suis divisible par 2 et par 3, mais pas par 4 ni par 9. Qui suis-je ?\\
\reponse[2]\\

\vspace*{0.5cm}

\exo{2} 

\initq \q Décomposer 630 et 924 en produit de facteurs premiers.\\
\reponse[2]\\

\q En déduire le plus grand diviseur commun de 630 et 924.\\
\reponse[1]\\

\q Écrire sous forme \textbf{irréductible} la fraction  $\frac{630}{924}$ en donnant le détail de vos calculs.\\
\reponse[3]\\

\vspace*{0.6cm}

\exo{3}

Un philatéliste possède 1 631 timbres français et 932 timbres étrangers. Il souhaite vendre toute sa collection en réalisant des lots identiques, c'est à dire comportant le même nombre de timbres et la même répartition de timbres français et étrangers.\\

\initq \q Calculer le nombre maximum de lots qu'il pourra réaliser. (Utiliser dans votre réponse la notion de décomposition en facteurs premiers.)\\
\reponse[3]\\

\q Combien y aura-t-il, dans ce cas, de timbres français et étrangers par lot ?\\
\reponse[2]\\



















\end{document}
