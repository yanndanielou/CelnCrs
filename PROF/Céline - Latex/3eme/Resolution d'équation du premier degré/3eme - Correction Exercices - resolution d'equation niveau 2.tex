\documentclass[a4paper,11pt]{article}
\usepackage{amsmath,amsthm,amsfonts,amssymb,amscd,amstext,vmargin,graphics,graphicx,tabularx,multicol} 
\usepackage[francais]{babel}
\usepackage[utf8]{inputenc}  
\usepackage[T1]{fontenc} 
\usepackage{pstricks-add,tikz,tkz-tab,variations}
\usepackage[autolanguage,np]{numprint} 
\usepackage{calc}
\usepackage{mathrsfs}

\usepackage{cancel}

\setmarginsrb{1.5cm}{0.5cm}{1cm}{0.5cm}{0cm}{0cm}{0cm}{0cm} %Gauche, haut, droite, haut
\newcounter{numexo}
\newcommand{\exo}[1]{\stepcounter{numexo}\noindent{\bf Exercice~\thenumexo} : }
\reversemarginpar

\newcommand{\bmul}[1]{\begin{multicols}{#1}}
\newcommand{\emul}{\end{multicols}}

\newcounter{enumtabi}
\newcounter{enumtaba}
\newcommand{\q}{\stepcounter{enumtabi} \theenumtabi.  }
\newcommand{\qa}{\stepcounter{enumtaba} (\alph{enumtaba}) }
\newcommand{\initq}{\setcounter{enumtabi}{0}}
\newcommand{\initqa}{\setcounter{enumtaba}{0}}

\newcommand{\be}{\begin{enumerate}}
\newcommand{\ee}{\end{enumerate}}
\newcommand{\bi}{\begin{itemize}}
\newcommand{\ei}{\end{itemize}}
\newcommand{\bp}{\begin{pspicture*}}
\newcommand{\ep}{\end{pspicture*}}
\newcommand{\bt}{\begin{tabular}}
\newcommand{\et}{\end{tabular}}
\renewcommand{\tabularxcolumn}[1]{>{\centering}m{#1}} %(colonne m{} centrée, au lieu de p par défault) 
\newcommand{\tnl}{\tabularnewline}

\newcommand{\trait}{\noindent \rule{\linewidth}{0.2mm}}
\newcommand{\hs}[1]{\hspace{#1}}
\newcommand{\vs}[1]{\vspace{#1}}

\newcommand{\N}{\mathbb{N}}
\newcommand{\Z}{\mathbb{Z}}
\newcommand{\R}{\mathbb{R}}
\newcommand{\C}{\mathbb{C}}
\newcommand{\Dcal}{\mathcal{D}}
\newcommand{\Ccal}{\mathcal{C}}
\newcommand{\mc}{\mathcal}

\newcommand{\vect}[1]{\overrightarrow{#1}}
\newcommand{\ds}{\displaystyle}
\newcommand{\eq}{\quad \Leftrightarrow \quad}
\newcommand{\vecti}{\vec{\imath}}
\newcommand{\vectj}{\vec{\jmath}}
\newcommand{\Oij}{(O;\vec{\imath}, \vec{\jmath})}
\newcommand{\OIJ}{(O;I,J)}


\newcommand{\reponse}[1][1]{%
\multido{}{#1}{\makebox[\linewidth]{\rule[0pt]{0pt}{20pt}\dotfill}
}}

\newcommand{\titre}[5] 
% #1: titre #2: haut gauche #3: bas gauche #4: haut droite #5: bas droite
{
\noindent #2 \hfill #4 \\
#3 \hfill #5

\vspace{-1.6cm}

\begin{center}\rule{6cm}{0.5mm}\end{center}
\vspace{0.2cm}
\begin{center}{\large{\textbf{#1}}}\end{center}
\begin{center}\rule{6cm}{0.5mm}\end{center}
}



\begin{document}
\pagestyle{empty}
\titre{Séance d'exercices: Résolution d'équation du premier degré}{}{}{3ème}{}

\vspace*{0.2cm}


\vspace*{0.25cm}

\exo \\


\initq \textbf{\qa} On considère l'équation suivante : \hspace*{0.5cm} $ 5x + 3(8-2x) = 15-(x-9)$\\
\textbf{4 est-il solution de l'équation ?}\\

\color{red}
\bmul{2}
D'une part, $ 5 \times 4 + 3 \times (8 - 2 \times 4)  = 20 + 3 \times (8-8)$\\

 \hspace*{5.65cm}$= 20 + 0$\\

 \hspace*{5.55cm} $=\underline{20}$\\

\columnbreak

D'autre part, $15 -(4-9) = 15 -(-5) $\\

 \hspace*{4.5cm}$=15 + 5$\\

\hspace*{4.5cm} $= \underline{20}$
\emul

L'égalité est donc  vérifiée pour $x=4$.\\
\color{black}

\qa On considère l'équation suivante : \hspace*{0.5cm} $(3x+2)^{2} = 9x^{2} + 6x + 4$\\
\textbf{-2 est-il solution de l'équation ?}\\

\color{red}
\bmul{2}
D'une part, $ (3 \times (-2)+2)^{2} = (-6 + 2 )^{2}$\\

 \hspace*{4.75cm}$= (-4)^{2}$\\

 \hspace*{4.65cm} $=\underline{16}$\\

\columnbreak

D'autre part, $9 \times (-2)^{2} + 6 \times (-2) +4 = 9 \times 4 -12 +4 $\\

 \hspace*{6.25cm}$=36-12+4$\\

\hspace*{6.05cm} $= \underline{28}$
\emul

L'égalité n'est donc pas vérifiée pour $x=4$.\\
\color{black}

\vspace{0.5cm}

\exo 

Résoudre les équations suivantes.


\bmul{3}
\initqa 

\textbf{a)} $$ 4x - 3 = 79$$
\color{red}
$$ 4x -3 +3 = 79 +3 $$

$$ 4x  =  82$$

$$ \dfrac{4x}{4}  =  \dfrac{82}{4}$$

$$\fbox{x =20,5}$$
\color{black}








\columnbreak



\textbf{b)} $$4x-7 = 3x+8$$

\color{red}
$$ 4x -7-3x = 3x+8 -3x$$

$$x-7  =  8$$

$$x-7+ 7 =  8+7$$

$$\fbox{x =15}$$

\color{black}


\columnbreak

\textbf{c)} $$6-8x=16x$$

\color{red}
$$ 6-8x-16x= 16x-16x $$

$$ 6-24x  =  0$$

$$ 6-24x -6 =  0-6$$

$$ -24x  =  -6$$

$$ \dfrac{-24x}{-24}  =  \dfrac{-6}{-24}$$

$$x = \dfrac{1}{4}$$



\color{black}

\emul

\newpage













\bmul{3}




\textbf{d)} $$-x+11 = \dfrac{3}{5} x+3$$

\color{red}
$$ -x+11-\dfrac{3}{5}x = \dfrac{3}{5} x+3 - \dfrac{3}{5}x $$

$$ \dfrac{-5}{5}x -\dfrac{3}{5}x +11  =  3$$

$$ \dfrac{-8}{5}x  +11  =  3$$

$$ \dfrac{-8}{5}x  +11 -11 =  3-11$$

$$ \dfrac{-8}{5}x   =  -8$$

$$ \dfrac{-\dfrac{8}{5}x }{-\dfrac{8}{5}} =  \dfrac{-8}{\dfrac{-8}{5}}$$

$$ x =  -8  \times  \dfrac{-5}{8}$$



$$\fbox{x =5}$$
\color{black}

\columnbreak


\textbf{e)} $50 = -2x + 35$


\color{red}
$$ 50-35= -2x +35 -35$$

$$15  =  -2x$$


$$ \dfrac{15}{-2}  =  \dfrac{-2x}{-2}$$

$$\fbox{-7,5 = x}$$

$$\fbox{x = -7,5}$$

\color{black}



\columnbreak

\textbf{f)} $$-2x+5+x+6=-8x+10 $$

\color{red}

$$-x+11=-8x+10 $$

$$-x+11+8x=10 $$

$$7x=10-11 $$

$$7x=-1 $$

$$x =  \dfrac{-1}{7} $$



\color{black}

\emul

\end{document}
