\documentclass[a4paper,11pt]{article}
\usepackage{amsmath,amsthm,amsfonts,amssymb,amscd,amstext,vmargin,graphics,graphicx,tabularx,multicol} 
\usepackage[francais]{babel}
\usepackage[utf8]{inputenc}  
\usepackage[T1]{fontenc} 
\usepackage{pstricks-add,tikz,tkz-tab,variations}
\usepackage[autolanguage,np]{numprint} 

\setmarginsrb{1.5cm}{0.5cm}{1cm}{0.5cm}{0cm}{0cm}{0cm}{0cm} %Gauche, haut, droite, haut
\newcounter{numexo}
\newcommand{\exo}[1]{\stepcounter{numexo}\noindent{\bf Exercice~\thenumexo} : \marginpar{\hfill /#1}}
\reversemarginpar


\newcounter{enumtabi}
\newcounter{enumtaba}
\newcommand{\q}{\stepcounter{enumtabi} \theenumtabi.  }
\newcommand{\qa}{\stepcounter{enumtaba} (\alph{enumtaba}) }
\newcommand{\initq}{\setcounter{enumtabi}{0}}
\newcommand{\initqa}{\setcounter{enumtaba}{0}}

\newcommand{\be}{\begin{enumerate}}
\newcommand{\ee}{\end{enumerate}}
\newcommand{\bi}{\begin{itemize}}
\newcommand{\ei}{\end{itemize}}
\newcommand{\bp}{\begin{pspicture*}}
\newcommand{\ep}{\end{pspicture*}}
\newcommand{\bt}{\begin{tabular}}
\newcommand{\et}{\end{tabular}}
\renewcommand{\tabularxcolumn}[1]{>{\centering}m{#1}} %(colonne m{} centrée, au lieu de p par défault) 
\newcommand{\tnl}{\tabularnewline}

\newcommand{\trait}{\noindent \rule{\linewidth}{0.2mm}}
\newcommand{\hs}[1]{\hspace{#1}}
\newcommand{\vs}[1]{\vspace{#1}}

\newcommand{\N}{\mathbb{N}}
\newcommand{\Z}{\mathbb{Z}}
\newcommand{\R}{\mathbb{R}}
\newcommand{\C}{\mathbb{C}}
\newcommand{\Dcal}{\mathcal{D}}
\newcommand{\Ccal}{\mathcal{C}}
\newcommand{\mc}{\mathcal}

\newcommand{\vect}[1]{\overrightarrow{#1}}
\newcommand{\ds}{\displaystyle}
\newcommand{\eq}{\quad \Leftrightarrow \quad}
\newcommand{\vecti}{\vec{\imath}}
\newcommand{\vectj}{\vec{\jmath}}
\newcommand{\Oij}{(O;\vec{\imath}, \vec{\jmath})}
\newcommand{\OIJ}{(O;I,J)}

\newcommand{\bmul}[1]{\begin{multicols}{#1}}
\newcommand{\emul}{\end{multicols}}

\newcommand{\reponse}[1][1]{%
\multido{}{#1}{\makebox[\linewidth]{\rule[0pt]{0pt}{20pt}\dotfill}
}}

\newcommand{\titre}[5] 
% #1: titre #2: haut gauche #3: bas gauche #4: haut droite #5: bas droite
{
\noindent #2 \hfill #4 \\
#3 \hfill #5

\vspace{-1.6cm}

\begin{center}\rule{6cm}{0.5mm}\end{center}
\vspace{0.2cm}
\begin{center}{\large{\textbf{#1}}}\end{center}
\begin{center}\rule{6cm}{0.5mm}\end{center}
}



\begin{document}
\pagestyle{empty}
\titre{Interrogation: Nombres relatifs}{Nom :}{Prénom :}{Classe}{Date}



\vspace*{0.5cm}

\begin{tabular}{|m{11cm}|c|c|c|}
\hline 
\textbf{Compétences} & \textbf{Acquis} & \textbf{En cours}  & \textbf{Insuffisant} \\ 
\hline 
- Savoir calculer la somme de deux nombres relatifs de même signe &  &  & \\
\hline
- Savoir calculer la somme de deux nombres relatifs de signe contraire &  &  &  \\ 
\hline 
- Savoir calculer une expression algébrique avec des nombres relatifs   &  &  &  \\ 
\hline 


\end{tabular} 



\vspace*{1cm}

\exo{1} Compléter les règles de calcul \\

Pour additionner deux nombres relatifs de signe contraire :\\

- \reponse[1]\\

- \reponse[1]\\

\vspace*{0.5cm}

\exo{3} Calculer :    
 
 \bmul{2} 
 
 (+ 4,6) + (- 12,1) =  \\

 (- 8,4) + (- 8,4) =\\
 
 (- 6) + (- 4) =\\
 
\columnbreak

 (- 56) + (+ 8) = \\

 (-4,9) + (+ 4,9) =     \\
 
(+ 21,4) + (- 10,07) =\\



 
\emul 


 \vspace*{1cm}


\exo{4.5} Calculer astucieusement les expressions suivantes \textbf{en détaillant vos étapes de calculs}:		

\bmul{2} 

$A = ( + 7 ) + ( - 12 ) + ( - 3 ) + ( + 12 ) $	\\
\reponse[5]\\


 \columnbreak
 

$J = (-5,1) + (+6,5) + (- 2,8) + (+7)+(-8,1)+(-4)$\\
\reponse[5]\\


\emul


$B = (- 10,9) + (+ 6) + (- 8,7) + (+ 10,9) + (+ 12) + (- 25,3) $\\
\reponse[5]\\


\newpage

\vspace*{0.5cm}

\exo{1.5}\\

Un professeur donne à ses élèves un QCM comportant 10 questions. Il note de la façon suivante :\\

 Réponse fausse \textbf{(F)} : \textbf{(- 2)} 	\hspace*{1cm}		 Sans réponse \textbf{(S)} : \textbf{(- 1)}	           \hspace*{1cm} Réponse bonne \textbf{(B)} : \textbf{(+ 4)}\\ 

$\rightarrow$ Calculer la note de Louis dont les résultats aux questions sont :\\

 \hspace*{2cm} \textbf{F ; F ; S ; B ; B ; F ; B ; F ; S ; B}\\
 
 
 \noindent \reponse[5]\\



\end{document}
