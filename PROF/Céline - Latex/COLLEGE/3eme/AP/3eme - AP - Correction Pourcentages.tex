\documentclass[a4paper,11pt]{article}
\usepackage{amsmath,amsthm,amsfonts,amssymb,amscd,amstext,vmargin,graphics,graphicx,tabularx,multicol} 
\usepackage[francais]{babel}
\usepackage[utf8]{inputenc}  
\usepackage[T1]{fontenc} 
\usepackage{pstricks-add,tikz,tkz-tab,variations}
\usepackage[autolanguage,np]{numprint} 
\usepackage{calc}

\setmarginsrb{1.5cm}{0.5cm}{1cm}{0.5cm}{0cm}{0cm}{0cm}{0cm} %Gauche, haut, droite, haut
\newcounter{numexo}
\newcommand{\exo}[1]{\stepcounter{numexo}\noindent{\bf Exercice~\thenumexo} : }
\reversemarginpar

\newcommand{\bmul}[1]{\begin{multicols}{#1}}
\newcommand{\emul}{\end{multicols}}

\newcounter{enumtabi}
\newcounter{enumtaba}
\newcommand{\q}{\stepcounter{enumtabi} \theenumtabi.  }
\newcommand{\qa}{\stepcounter{enumtaba} (\alph{enumtaba}) }
\newcommand{\initq}{\setcounter{enumtabi}{0}}
\newcommand{\initqa}{\setcounter{enumtaba}{0}}

\newcommand{\be}{\begin{enumerate}}
\newcommand{\ee}{\end{enumerate}}
\newcommand{\bi}{\begin{itemize}}
\newcommand{\ei}{\end{itemize}}
\newcommand{\bp}{\begin{pspicture*}}
\newcommand{\ep}{\end{pspicture*}}
\newcommand{\bt}{\begin{tabular}}
\newcommand{\et}{\end{tabular}}
\renewcommand{\tabularxcolumn}[1]{>{\centering}m{#1}} %(colonne m{} centrée, au lieu de p par défault) 
\newcommand{\tnl}{\tabularnewline}

\newcommand{\trait}{\noindent \rule{\linewidth}{0.2mm}}
\newcommand{\hs}[1]{\hspace{#1}}
\newcommand{\vs}[1]{\vspace{#1}}

\newcommand{\N}{\mathbb{N}}
\newcommand{\Z}{\mathbb{Z}}
\newcommand{\R}{\mathbb{R}}
\newcommand{\C}{\mathbb{C}}
\newcommand{\Dcal}{\mathcal{D}}
\newcommand{\Ccal}{\mathcal{C}}
\newcommand{\mc}{\mathcal}

\newcommand{\vect}[1]{\overrightarrow{#1}}
\newcommand{\ds}{\displaystyle}
\newcommand{\eq}{\quad \Leftrightarrow \quad}
\newcommand{\vecti}{\vec{\imath}}
\newcommand{\vectj}{\vec{\jmath}}
\newcommand{\Oij}{(O;\vec{\imath}, \vec{\jmath})}
\newcommand{\OIJ}{(O;I,J)}


\newcommand{\reponse}[1][1]{%
\multido{}{#1}{\makebox[\linewidth]{\rule[0pt]{0pt}{20pt}\dotfill}
}}

\newcommand{\titre}[5] 
% #1: titre #2: haut gauche #3: bas gauche #4: haut droite #5: bas droite
{
\noindent #2 \hfill #4 \\
#3 \hfill #5

\vspace{-1.6cm}

\begin{center}\rule{6cm}{0.5mm}\end{center}
\vspace{0.2cm}
\begin{center}{\large{\textbf{#1}}}\end{center}
\begin{center}\rule{6cm}{0.5mm}\end{center}
}



\begin{document}
\pagestyle{empty}
\titre{Correction de la séance d'AP 5 : Pourcentages}{}{}{3ème}{}

\vspace*{0.2cm}


\exo \\ Calculer

\bmul{2}
\initqa 
\qa 25 $\%$ de 32 km :\textbf{ 8 km }\\


\qa 65 $\%$ de 380 m : \textbf{247 m}


\columnbreak

\qa 14 $\%$ de 1500 euros : \textbf{210 euros }\\


\qa 125 $\%$ de 68 L : \textbf{85 L}



\emul

\vspace*{0.4cm}

\exo \\ Le volume d'eau sur Terre est d'environ 1 380 millions de $km^{3}$. 97,1 $\%$ de ce volume est composé d'eau salée. \\

Il y a 97,1 $\%$ d'eau salée, 100 - 97,1 = 2,9, donc il y a 2,9 $\%$ d'eau douce.\\

$\dfrac{2,9}{100} \times 1 380 \times 10^{6}= 40 020 000 = 4,002 \times 10^{7}$\\

\textbf{Donc il y a 40 020 000 $km^{3}$ d'eau douce.}\\

\vspace*{0.4cm}

\exo \\ Professions en 2005. On comptait environ 24 921 000 actifs en France.\\
\initq \q Sachant qu'il y avait 2,7 $\%$ d'agriculteurs en 2005, quel était leur nombre ?\\

$\dfrac{2,7}{100} \times 24 921 000= 672 867$. \textbf{Il y a donc en 2005, 672 867 agriculteurs.\\}

\q Sachant que le nombre d'ouvriers était environ de 5 972 000, calculer leur pourcentage par rapport au nombre d'actifs.\\

$\dfrac{5 972 000}{24 921 000} \times 100 = 23,96$. \textbf{Il y avait donc 23,95 $\%$ d'ouvriers.}
\vspace*{0.4cm}


\exo \\ Une usine est divisée en deux ateliers. 
L'atelier nord compte 1000 salariés dont 80$\%$ d'ouvriers.\\
L'atelier sud compte 540 salariés dont 70$\%$ d'ouvriers.
Calculer le pourcentage d'ouvriers dans cette usine.\\

\bmul{2}

\textbf{Atelier Nord :} \\
$\dfrac{80}{100} \times 1 000= 800$\\
Il y a donc 800 ouvriers dans l'atelier Nord.


\columnbreak

\textbf{Atelier Sud :} \\
$\dfrac{70}{100} \times 540= 378$\\
Il y a donc 378 ouvriers dans l'atelier Nord.


\emul

On compte au total, 800 + 378 = 1 178, 1 178 ouvriers dans l'usine qui compte 1 540 salariés.\\
$\dfrac{1178}{1540} \times 100 = 76,5$. \textbf{Il y a donc 76,5 $\%$ d'ouvriers dans toute l'usine.\\}

\vspace*{0.4cm}


\exo \\ Des prix.\\
\initqa \qa Julie obtient une réduction de 15 $\%$ sur un vélo valant 158 euros . Quel est le montant de la réduction obtenue par Julie ?\\

$\dfrac{15}{100} \times 158= 23,7$. \textbf{Elle obtient une réduction de 23,7 euros sur son vélo.}\\


\qa Patrick a obtenu une réduction de 27 euros sur une console de jeu qui valait 225 euros. Quel pourcentage de réduction a-t-il obtenu ?\\

$\dfrac{27}{225} \times 100 = 12$. \textbf{Il obtient une réduction de 12 $\%$}.\\

\newpage

\qa Paul a obtenu une baisse de 45 euros sur un appareil photo, soit une baisse de 30 $\%$ du prix initial. Quel était le prix initial de l'appareil photo ?\\

\begin{tabular}{|c|c|}
\hline 
45 &  ? \\ 
\hline 
30 & 100 \\ 
\hline 
\end{tabular}  $\dfrac{45 \times 100}{30} = 150$\\
\textbf{Le prix initial était de 150 euros.}\\
\vspace*{0.4cm}

\exo \\ Une montre coûtait 175 euros en 2006. Son prix est augmenté de 3 $\%$
en 2007, puis de 4 $\%$ en 2008.\\
\initq \q Calculer le prix de cette montre en 2007, puis en 2008.\\

En 2007 : $\dfrac{3}{100} \times 175 = 5,25 $ \textbf{Nouveau prix : 175 + 5,25 = 180,25 euros}\\

En 2008 : $\dfrac{4}{100} \times 180,25 = 7,21 $ \textbf{Nouveau prix : 180,25 + 7,21 = 187,46 euros}\\


\q Calculer le pourcentage d'augmentation sur les deux années.\\

$\dfrac{187,46}{175} \times 100 = 106,5$.\hspace*{1cm} 100 - 106,5 = 6,5.\\

\textbf{Il y a donc eu sur les 2 ans une augmentation de 6,5 $\%$}.



\end{document}
