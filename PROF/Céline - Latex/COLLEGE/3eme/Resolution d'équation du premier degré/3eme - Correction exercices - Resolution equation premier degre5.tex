\documentclass[a4paper,11pt]{article}
\usepackage{amsmath,amsthm,amsfonts,amssymb,amscd,amstext,vmargin,graphics,graphicx,tabularx,multicol} 
\usepackage[francais]{babel}
\usepackage[utf8]{inputenc}  
\usepackage[T1]{fontenc} 
\usepackage{pstricks-add,tikz,tkz-tab,variations}
\usepackage[autolanguage,np]{numprint} 
\usepackage{calc}
\usepackage{mathrsfs}

\usepackage{cancel}

\setmarginsrb{1.5cm}{0.5cm}{1cm}{0.5cm}{0cm}{0cm}{0cm}{0cm} %Gauche, haut, droite, haut
\newcounter{numexo}
\newcommand{\exo}[1]{\stepcounter{numexo}\noindent{\bf Exercice~\thenumexo} : }
\reversemarginpar

\newcommand{\bmul}[1]{\begin{multicols}{#1}}
\newcommand{\emul}{\end{multicols}}

\newcounter{enumtabi}
\newcounter{enumtaba}
\newcommand{\q}{\stepcounter{enumtabi} \theenumtabi.  }
\newcommand{\qa}{\stepcounter{enumtaba} (\alph{enumtaba}) }
\newcommand{\initq}{\setcounter{enumtabi}{0}}
\newcommand{\initqa}{\setcounter{enumtaba}{0}}

\newcommand{\be}{\begin{enumerate}}
\newcommand{\ee}{\end{enumerate}}
\newcommand{\bi}{\begin{itemize}}
\newcommand{\ei}{\end{itemize}}
\newcommand{\bp}{\begin{pspicture*}}
\newcommand{\ep}{\end{pspicture*}}
\newcommand{\bt}{\begin{tabular}}
\newcommand{\et}{\end{tabular}}
\renewcommand{\tabularxcolumn}[1]{>{\centering}m{#1}} %(colonne m{} centrée, au lieu de p par défault) 
\newcommand{\tnl}{\tabularnewline}

\newcommand{\trait}{\noindent \rule{\linewidth}{0.2mm}}
\newcommand{\hs}[1]{\hspace{#1}}
\newcommand{\vs}[1]{\vspace{#1}}

\newcommand{\N}{\mathbb{N}}
\newcommand{\Z}{\mathbb{Z}}
\newcommand{\R}{\mathbb{R}}
\newcommand{\C}{\mathbb{C}}
\newcommand{\Dcal}{\mathcal{D}}
\newcommand{\Ccal}{\mathcal{C}}
\newcommand{\mc}{\mathcal}

\newcommand{\vect}[1]{\overrightarrow{#1}}
\newcommand{\ds}{\displaystyle}
\newcommand{\eq}{\quad \Leftrightarrow \quad}
\newcommand{\vecti}{\vec{\imath}}
\newcommand{\vectj}{\vec{\jmath}}
\newcommand{\Oij}{(O;\vec{\imath}, \vec{\jmath})}
\newcommand{\OIJ}{(O;I,J)}


\newcommand{\reponse}[1][1]{%
\multido{}{#1}{\makebox[\linewidth]{\rule[0pt]{0pt}{20pt}\dotfill}
}}

\newcommand{\titre}[5] 
% #1: titre #2: haut gauche #3: bas gauche #4: haut droite #5: bas droite
{
\noindent #2 \hfill #4 \\
#3 \hfill #5

\vspace{-1.6cm}

\begin{center}\rule{6cm}{0.5mm}\end{center}
\vspace{0.2cm}
\begin{center}{\large{\textbf{#1}}}\end{center}
\begin{center}\rule{6cm}{0.5mm}\end{center}
}



\begin{document}
\pagestyle{empty}
\titre{Séance d'exercices: Résolution d'équation du premier degré}{}{}{3ème}{}

\vspace*{0.2cm}


 
 
\textbf{EXERCICE 4}\\
 
 Trouve un nombre sachant que son triple augmenté de 2 est égal à son double augmenté de 3.\\
 
 \color{red}
On appelle $x$ le nombre recherché.\\


L'équation est la suivante : \hspace*{1cm} $3x+2=2x+3$ 

$$3x+2-2x=2x+3-2x$$

$$x+2=3$$

$$x+2-2=3-2$$
 
 $$x=1$$
 
 Le nombre recherché est 1.\\
 
 \textbf{Vérification :} $ 3 \times 1 + 2 = 5$ et $2 \times 1 + 3 = 5$ \\
 
 \color{black}




\textbf{EXERCICE 6}\\

Deux frères, Marc et Jean, possèdent chacun un jardin. L'aire du jardin de Marc vaut les $\dfrac{3}{4}$ de l'aire du jardin de Jean. Les deux frères possèdent en tout 1 470 $ m^{2} $.\\

Quelles sont les aires des jardins de Marc et de Jean  ?\\

\color{red}


On appelle $x$, l'aire du terrain de Jean.\\

Pour trouver l'équation, il faut choisir une grandeur qui peut être exprimée de deux façons différentes.\\

Ici, il s'agit de l'aire du jardin qu'ils ont à eux 2.\hspace*{1cm} \textbf{Il y a 1 470 $m^{2}$ en tout}.\\

Mais cela peut aussi s'écrire : \textbf{Aire de Jean + Aire de Marc.}\\

On sait que :
\bi
\item \textcolor{green}{Aire de Jean = x}

\item l'aire du jardin de Marc vaut les $\dfrac{3}{4}$ de l'aire du jardin de Jean. , à savoir \textcolor{blue}{Aire de Marc = $\dfrac{3}{4}x$ }

\ei

L'équation est donc la suivante : \hspace*{1cm} $ \textcolor{green}{Aire de Jean } + \textcolor{blue}{Aire de Marc}  = 1 470$\\

$$ x + \dfrac{3}{4}x = 1470$$


$$ \dfrac{4}{4}x + \dfrac{3}{4}x = 1470$$

$$ \dfrac{7}{4}x = 1470$$

$$ \dfrac{\dfrac{7}{4}x}{\dfrac{7}{4}} = \dfrac{1470}{\dfrac{7}{4}}$$


$$x = \dfrac{1470}{\dfrac{7}{4}}$$


$$x =1470 \times \dfrac{4}{7}$$

$$x =840$$

Donc, l'aire du terrain de Jean vaut 840 $m^{2}$.\\

Pour Marc : $\dfrac{3}{4} \times 840 = 630 m^{2}$ \hspace*{1.5cm}\textbf{Vérification :} 840 + 630 = 1 470.\\



\color{black}


 


\end{document}
