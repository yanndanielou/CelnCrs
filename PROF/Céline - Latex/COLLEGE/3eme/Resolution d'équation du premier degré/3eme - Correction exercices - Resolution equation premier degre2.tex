\documentclass[a4paper,11pt]{article}
\usepackage{amsmath,amsthm,amsfonts,amssymb,amscd,amstext,vmargin,graphics,graphicx,tabularx,multicol} 
\usepackage[francais]{babel}
\usepackage[utf8]{inputenc}  
\usepackage[T1]{fontenc} 
\usepackage{pstricks-add,tikz,tkz-tab,variations}
\usepackage[autolanguage,np]{numprint} 
\usepackage{calc}
\usepackage{mathrsfs}

\usepackage{cancel}

\setmarginsrb{1.5cm}{0.5cm}{1cm}{0.5cm}{0cm}{0cm}{0cm}{0cm} %Gauche, haut, droite, haut
\newcounter{numexo}
\newcommand{\exo}[1]{\stepcounter{numexo}\noindent{\bf Exercice~\thenumexo} : }
\reversemarginpar

\newcommand{\bmul}[1]{\begin{multicols}{#1}}
\newcommand{\emul}{\end{multicols}}

\newcounter{enumtabi}
\newcounter{enumtaba}
\newcommand{\q}{\stepcounter{enumtabi} \theenumtabi.  }
\newcommand{\qa}{\stepcounter{enumtaba} (\alph{enumtaba}) }
\newcommand{\initq}{\setcounter{enumtabi}{0}}
\newcommand{\initqa}{\setcounter{enumtaba}{0}}

\newcommand{\be}{\begin{enumerate}}
\newcommand{\ee}{\end{enumerate}}
\newcommand{\bi}{\begin{itemize}}
\newcommand{\ei}{\end{itemize}}
\newcommand{\bp}{\begin{pspicture*}}
\newcommand{\ep}{\end{pspicture*}}
\newcommand{\bt}{\begin{tabular}}
\newcommand{\et}{\end{tabular}}
\renewcommand{\tabularxcolumn}[1]{>{\centering}m{#1}} %(colonne m{} centrée, au lieu de p par défault) 
\newcommand{\tnl}{\tabularnewline}

\newcommand{\trait}{\noindent \rule{\linewidth}{0.2mm}}
\newcommand{\hs}[1]{\hspace{#1}}
\newcommand{\vs}[1]{\vspace{#1}}

\newcommand{\N}{\mathbb{N}}
\newcommand{\Z}{\mathbb{Z}}
\newcommand{\R}{\mathbb{R}}
\newcommand{\C}{\mathbb{C}}
\newcommand{\Dcal}{\mathcal{D}}
\newcommand{\Ccal}{\mathcal{C}}
\newcommand{\mc}{\mathcal}

\newcommand{\vect}[1]{\overrightarrow{#1}}
\newcommand{\ds}{\displaystyle}
\newcommand{\eq}{\quad \Leftrightarrow \quad}
\newcommand{\vecti}{\vec{\imath}}
\newcommand{\vectj}{\vec{\jmath}}
\newcommand{\Oij}{(O;\vec{\imath}, \vec{\jmath})}
\newcommand{\OIJ}{(O;I,J)}


\newcommand{\reponse}[1][1]{%
\multido{}{#1}{\makebox[\linewidth]{\rule[0pt]{0pt}{20pt}\dotfill}
}}

\newcommand{\titre}[5] 
% #1: titre #2: haut gauche #3: bas gauche #4: haut droite #5: bas droite
{
\noindent #2 \hfill #4 \\
#3 \hfill #5

\vspace{-1.6cm}

\begin{center}\rule{6cm}{0.5mm}\end{center}
\vspace{0.2cm}
\begin{center}{\large{\textbf{#1}}}\end{center}
\begin{center}\rule{6cm}{0.5mm}\end{center}
}



\begin{document}
\pagestyle{empty}
\titre{Séance d'exercices: Résolution d'équation du premier degré}{}{}{3ème}{}

\vspace*{0.2cm}


{\large \textbf{\underline{PARTIE A :}}  Résolution d'équation}\\

\vspace*{0.25cm}



\textbf{Exercice 2 :}
Résoudre les équations suivantes.


\bmul{3}
\initqa 

\qa $$-2+x=11$$

\color{red}
$$-2 + x + 2 = 11 + 2$$

$$\fbox{x = 13}$$

\color{black} 

\qa $$\dfrac{3}{4}x=5$$

\color{red}

$$ \dfrac{\dfrac{3}{4}x}{\dfrac{3}{4}}= \dfrac{5}{\dfrac{3}{4}}$$

$$x= 5 \times \dfrac{4}{3}$$

$$x = \dfrac{20}{3} $$

\color{black} 

\columnbreak

\qa $$9+x=44$$


\color{red}

$$9+x-9 = 44-9$$

$$\fbox{x =35}$$
\color{black} 

\qa $3x=27$\\

\color{red}

$$ \dfrac{3x}{3}= \dfrac{27}{3}$$



$$\fbox{x =9}$$

\color{black} 


\columnbreak

\qa $$-6 +x =-41$$


\color{red}

$$-6 + x + 6 =-41 +6$$


$$\fbox{x =-35}$$

\color{black} 

\qa $-6x=-42$\\

\color{red}

$$ \dfrac{-6x}{-6}= \dfrac{-42}{-6}$$



$$\fbox{x =7}$$

\color{black} 

\emul


\textbf{Exercice 3 :}

Résoudre les équations suivantes.


\bmul{3}
\initqa 

\textbf{a)} $$ 4x - 3 = 79$$
\color{red}
$$ 4x -3 +3 = 79 +3 $$

$$ 4x  =  82$$

$$ \dfrac{4x}{4}  =  \dfrac{82}{4}$$

$$\fbox{x =20,5}$$
\color{black}







\columnbreak

\textbf{d)} $$6-8x=16x$$

\color{red}
$$ 6-8x-16x= 16x-16x $$

$$ 6-24x  =  0$$

$$ 6-24x -6 =  0-6$$

$$ -24x  =  -6$$

$$ \dfrac{-24x}{-24}  =  \dfrac{-6}{-24}$$

$$x = \dfrac{1}{4}$$

\color{black}








\columnbreak

\textbf{g)} $50 = -2x + 35$


\color{red}
$$ 50-35= -2x +35 -35$$

$$15  =  -2x$$


$$ \dfrac{15}{-2}  =  \dfrac{-2x}{-2}$$

$$\fbox{-7,5 = x}$$

$$\fbox{x = -7,5}$$

\color{black}





\emul





\end{document}
