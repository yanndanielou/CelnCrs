\documentclass[a4paper,11pt]{article}
\usepackage{amsmath,amsthm,amsfonts,amssymb,amscd,amstext,vmargin,graphics,graphicx,tabularx,multicol} 
\usepackage[francais]{babel}
\usepackage[utf8]{inputenc}  
\usepackage[T1]{fontenc} 
\usepackage{pstricks-add,tikz,tkz-tab,variations}
\usepackage[autolanguage,np]{numprint} 

\setmarginsrb{1.5cm}{0.5cm}{1cm}{0.5cm}{0cm}{0cm}{0cm}{0cm} %Gauche, haut, droite, haut
\newcounter{numexo}
\newcommand{\exo}[1]{\stepcounter{numexo}\noindent{\bf Exercice~\thenumexo} : \marginpar{\hfill /#1}}
\reversemarginpar


\newcounter{enumtabi}
\newcounter{enumtaba}
\newcommand{\q}{\stepcounter{enumtabi} \theenumtabi.  }
\newcommand{\qa}{\stepcounter{enumtaba} (\alph{enumtaba}) }
\newcommand{\initq}{\setcounter{enumtabi}{0}}
\newcommand{\initqa}{\setcounter{enumtaba}{0}}

\newcommand{\be}{\begin{enumerate}}
\newcommand{\ee}{\end{enumerate}}
\newcommand{\bi}{\begin{itemize}}
\newcommand{\ei}{\end{itemize}}
\newcommand{\bp}{\begin{pspicture*}}
\newcommand{\ep}{\end{pspicture*}}
\newcommand{\bt}{\begin{tabular}}
\newcommand{\et}{\end{tabular}}
\renewcommand{\tabularxcolumn}[1]{>{\centering}m{#1}} %(colonne m{} centrée, au lieu de p par défault) 
\newcommand{\tnl}{\tabularnewline}

\newcommand{\bmul}[1]{\begin{multicols}{#1}}
\newcommand{\emul}{\end{multicols}}

\newcommand{\trait}{\noindent \rule{\linewidth}{0.2mm}}
\newcommand{\hs}[1]{\hspace{#1}}
\newcommand{\vs}[1]{\vspace{#1}}

\newcommand{\N}{\mathbb{N}}
\newcommand{\Z}{\mathbb{Z}}
\newcommand{\R}{\mathbb{R}}
\newcommand{\C}{\mathbb{C}}
\newcommand{\Dcal}{\mathcal{D}}
\newcommand{\Ccal}{\mathcal{C}}
\newcommand{\mc}{\mathcal}

\newcommand{\vect}[1]{\overrightarrow{#1}}
\newcommand{\ds}{\displaystyle}
\newcommand{\eq}{\quad \Leftrightarrow \quad}
\newcommand{\vecti}{\vec{\imath}}
\newcommand{\vectj}{\vec{\jmath}}
\newcommand{\Oij}{(O;\vec{\imath}, \vec{\jmath})}
\newcommand{\OIJ}{(O;I,J)}


\newcommand{\reponse}[1][1]{%
\multido{}{#1}{\makebox[\linewidth]{\rule[0pt]{0pt}{20pt}\dotfill}
}}

\newcommand{\titre}[5] 
% #1: titre #2: haut gauche #3: bas gauche #4: haut droite #5: bas droite
{
\noindent #2 \hfill #4 \\
#3 \hfill #5

\vspace{-1.6cm}

\begin{center}\rule{6cm}{0.5mm}\end{center}
\vspace{0.2cm}
\begin{center}{\large{\textbf{#1}}}\end{center}
\begin{center}\rule{6cm}{0.5mm}\end{center}
}



\begin{document}
\pagestyle{empty}
\titre{Interrogation: Calcul numérique}{Nom :}{Prénom :}{Classe}{Date}



\vspace*{1.5cm}
\exo{3} Calculer les expressions suivantes en détaillant les étapes de vos calculs :

\bmul{2}




$S = (\dfrac{2}{5} \div \dfrac{12}{15}) \div (\dfrac{5}{7} + \dfrac{3}{21})$ \\
\reponse[10]\\


\columnbreak

$F = \dfrac{\dfrac{4}{7} - 1}{1 - \dfrac{11}{14}}$ \\
\reponse[10]\\


\emul

\exo{3} \textit{(Brevet 2006)}\\

$ A =\dfrac{13}{4}-\dfrac{5}{4} \times \dfrac{7}{3}$  \hspace*{1cm} $B =\dfrac{0,3 \times 10^{2} \times 5 \times 10^{-3}}{4 \times 10^{-4}}$\\

\initq \q Calculer A et donner le résultat sous la forme d'une fraction irréductible.\\

\q Donner l'écriture décimale de l'expression B.\\
\reponse[11]



\newpage
\vspace*{0.25cm}


\exo{4} \textit{(Métropole juin 2012)}\\
Cet exercice est un questionnaire à choix multiples.\\
Pour chaque question, quatre réponses sont proposées mais une seule est exacte. Pour chacune des questions, entourer la bonne réponse, aucune justification n'est demandée.\\

\vspace*{0.25cm}

\renewcommand{\arraystretch}
{3.2}

\begin{tabular}{|p{0.75cm}|p{4.5cm}|c|c|c|c|}
\hline 
\textbf{N} & \textbf{Question} & \textbf{Réponse A} & \textbf{Réponse B} & \textbf{Réponse C} & \textbf{Réponse D} \\ 
\hline 
\textbf{1}& Que vaut $ \dfrac{5^{n}}{5^{m}} $ ?  & $5^{n-m}$ & $1^{n-m}$ & $5^{n\div m}$ & $1^{n \div m}$ \\ 
\hline 
\textbf{2}& $\dfrac{5}{3}-\dfrac{6}{5}$ est égal à : & $\dfrac{11}{2}$ & $\dfrac{7}{15}$  & $-\dfrac{1}{8}$  & $0,46$ \\ 
\hline 
\textbf{3} & A quelle autre expression le nombre $\dfrac{7}{3}-\dfrac{4}{3} \div \dfrac{5}{2}$ est-il égal ? & $\dfrac{3}{3} \div \dfrac{5}{2}$ & $\dfrac{7}{3}-\dfrac{3}{4} \times \dfrac{2}{5}$& $\dfrac{27}{15}$ & $-1$ \\ 
\hline 
\textbf{4} & $7^{-2} \times 7^{3}-7=$ & $0$ & $7^{0}$ & $-1$ & $7^{-5}$ \\ 
\hline 
\textbf{5} & $2 \times 10^{-3} \times 10^{5}$ est égal à : & $2 \times 10^{-15}$& $2 \times 10^{2}$ & 0,2 & 0,02 \\ 
\hline 
\textbf{6} & $\dfrac{(10^{-3})^{2} \times 10^{4}}{10^{-5}}=$ & $10^{-7}$ & $10^{-15}$ & $10^{3}$ & $10^{4}$ \\ 
\hline 
\textbf{7} & $\dfrac{1}{3}+\dfrac{1}{9} $ est égal à : & $\dfrac{2}{12}$ & 0,44& $\dfrac{4}{9}$& $\dfrac{1}{12}$ \\ 
\hline 
\textbf{8} & $\dfrac{3,1 \times 10^{7} \times 20 \times 10^{-2}}{2 \times 10^{3}}=$   & 310 & $3,1 \times 10^{2}$ & $31 \times 10^{8}$ & $31 \times 10^{2}$ \\ 
\hline 
\end{tabular} 


\vspace*{1.5cm}


\exo{} BONUS

\noindent Quelle est le résultat de la somme de 2 et de l'inverse de la somme de 2 et de l'inverse de la somme de 2 et 2 ? (Expliquez votre résultat )\\
\reponse[7]

\end{document}
