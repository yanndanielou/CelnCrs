\documentclass[a4paper,11pt]{article}
\usepackage{amsmath,amsthm,amsfonts,amssymb,amscd,amstext,vmargin,graphics,graphicx,tabularx,multicol} 
\usepackage[francais]{babel}
\usepackage[utf8]{inputenc}  
\usepackage[T1]{fontenc} 
\usepackage{pstricks-add,tikz,tkz-tab,variations}
\usepackage[autolanguage,np]{numprint} 
\usepackage{color}

\setmarginsrb{1.5cm}{0.5cm}{1cm}{0.5cm}{0cm}{0cm}{0cm}{0cm} %Gauche, haut, droite, haut
\newcounter{numexo}
\newcommand{\exo}[1]{\stepcounter{numexo}\noindent{\bf Exercice~\thenumexo} : \marginpar{\hfill /#1}}
\reversemarginpar


\newcounter{enumtabi}
\newcounter{enumtaba}
\newcommand{\q}{\stepcounter{enumtabi} \theenumtabi.  }
\newcommand{\qa}{\stepcounter{enumtaba} (\alph{enumtaba}) }
\newcommand{\initq}{\setcounter{enumtabi}{0}}
\newcommand{\initqa}{\setcounter{enumtaba}{0}}

\newcommand{\be}{\begin{enumerate}}
\newcommand{\ee}{\end{enumerate}}
\newcommand{\bi}{\begin{itemize}}
\newcommand{\ei}{\end{itemize}}
\newcommand{\bp}{\begin{pspicture*}}
\newcommand{\ep}{\end{pspicture*}}
\newcommand{\bt}{\begin{tabular}}
\newcommand{\et}{\end{tabular}}
\renewcommand{\tabularxcolumn}[1]{>{\centering}m{#1}} %(colonne m{} centrée, au lieu de p par défault) 
\newcommand{\tnl}{\tabularnewline}

\newcommand{\bmul}[1]{\begin{multicols}{#1}}
\newcommand{\emul}{\end{multicols}}

\newcommand{\trait}{\noindent \rule{\linewidth}{0.2mm}}
\newcommand{\hs}[1]{\hspace{#1}}
\newcommand{\vs}[1]{\vspace{#1}}

\newcommand{\N}{\mathbb{N}}
\newcommand{\Z}{\mathbb{Z}}
\newcommand{\R}{\mathbb{R}}
\newcommand{\C}{\mathbb{C}}
\newcommand{\Dcal}{\mathcal{D}}
\newcommand{\Ccal}{\mathcal{C}}
\newcommand{\mc}{\mathcal}

\newcommand{\vect}[1]{\overrightarrow{#1}}
\newcommand{\ds}{\displaystyle}
\newcommand{\eq}{\quad \Leftrightarrow \quad}
\newcommand{\vecti}{\vec{\imath}}
\newcommand{\vectj}{\vec{\jmath}}
\newcommand{\Oij}{(O;\vec{\imath}, \vec{\jmath})}
\newcommand{\OIJ}{(O;I,J)}


\newcommand{\reponse}[1][1]{%
\multido{}{#1}{\makebox[\linewidth]{\rule[0pt]{0pt}{20pt}\dotfill}
}}

\newcommand{\titre}[5] 
% #1: titre #2: haut gauche #3: bas gauche #4: haut droite #5: bas droite
{
\noindent #2 \hfill #4 \\
#3 \hfill #5

\vspace{-1.6cm}

\begin{center}\rule{6cm}{0.5mm}\end{center}
\vspace{0.2cm}
\begin{center}{\large{\textbf{#1}}}\end{center}
\begin{center}\rule{6cm}{0.5mm}\end{center}
}



\begin{document}
\pagestyle{empty}
\titre{Correction de l'interrogation 2 }{3 ème}{}{}


\exo{2,5} 

\q Compléter les définitions suivantes :

- Un nombre est dit \textbf{premier} si \textcolor{red}{il possède exactement deux diviseurs, 1 et lui-même.} \\

- Deux nombres sont \textbf{premiers entre eux} si \textcolor{red}{leur seul diviseur commun est 1.}\\

\q Parmi les 20 premiers nombres entiers naturel, quels sont ceux qui sont premiers ?\\
\textcolor{red}{2 ; 3 ; 5 ; 7 ; 11 ; 13 ; 17 et 19}\\

\vspace*{0.6cm}

\exo{2,5}

\initq \q  Écrire la liste des diviseurs de 75 et celle de 46.\\
\textcolor{red}{$D_{75} = \{1 ; 3 ; 5 ; 15 ; 25 ; 75\}$}\\
\textcolor{red}{$D_{46} = \{1 ; 2 ; 23 ; 46\}$}\\

\q  Montrer que 75 et 46 sont premiers entre eux.\\
\textcolor{red}{Le seul diviseur en commun à 46 et 75 est 1. Ces nombres sont donc premiers entre eux.}\\

\vspace*{0.6cm}

\exo{1}
Je suis un nombre entier naturel compris entre 1 509 et 1 534. Je suis divisible par 2 et par 3, mais pas par 4 ni par 9. Qui suis-je ?\\
\textcolor{red}{Ce nombre est forcément un nombre pair et la somme de ses nombres doit être un multiple de 3. Je suis 1518.} \\

\vspace*{0.5cm}

\exo{2} 

\initq \q Décomposer 630 et 924 en produit de facteurs premiers.\\
\textcolor{red}{$630 = 63 \times 10 = 7 \times 9 \times 2 \times 5 = 2 \times 3^{2} \times 5 \times 7$}\\
\textcolor{red}{$924 = 2 \times 462 = 2 \times 2 \times 231 = 2 \times 2 \times 3 \times 77 = 2 \times 2 \times 3 \times 7 \times 11 = 2^{2} \times 3 \times 7 \times 11$}\\

\q En déduire le plus grand diviseur commun de 630 et 924.\\
\textcolor{red}{ Les facteurs communs aux deux nombres sont 2,3 et 7. Donc PGCD(924 ; 630) =$2 \times 3 \times 7$ = 42  }\\

\q Écrire sous forme \textbf{irréductible} la fraction  $\frac{630}{924}$ en donnant le détail de vos calculs.\\
\textcolor{red}{$\frac{630}{924} = \frac{630 \div 42}{924 \div 42} = \frac{15}{22}$}\\

\vspace*{0.6cm}

\exo{3}

Un philatéliste possède 1 631 timbres français et 932 timbres étrangers. Il souhaite vendre toute sa collection en réalisant des lots identiques, c'est à dire comportant le même nombre de timbres et la même répartition de timbres français et étrangers.\\

\initq \q Calculer le nombre maximum de lots qu'il pourra réaliser. (Utiliser dans votre réponse la notion de décomposition en facteurs premiers.)\\
\textcolor{red}{$1 631 = 7 \times 233$} et \textcolor{red}{$932 = 2 \times 466 = 2 \times 2 \times 233 $} \\
\textcolor{red}{Le diviseur commun de 1 631 et 932 est 233. Il pourra donc réaliser 233 lots identiques.}\\ 

\q Combien y aura-t-il, dans ce cas, de timbres français et étrangers par lot ?\\
\textcolor{red}{$1 631 \div 233 = 7$} \textcolor{red}{et $932 \div 233 = 4$}\\
\textcolor{red}{Il y aura donc 7 timbres français et 4 timbres étrangers.}




















\end{document}
