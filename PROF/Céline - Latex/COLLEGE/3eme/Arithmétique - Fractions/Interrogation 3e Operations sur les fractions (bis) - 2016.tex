\documentclass[a4paper,11pt]{article}
\usepackage{amsmath,amsthm,amsfonts,amssymb,amscd,amstext,vmargin,graphics,graphicx,tabularx,multicol} 
\usepackage[francais]{babel}
\usepackage[utf8]{inputenc}  
\usepackage[T1]{fontenc} 
\usepackage{pstricks-add,tikz,tkz-tab,variations}
\usepackage[autolanguage,np]{numprint} 

\setmarginsrb{1.5cm}{0.5cm}{1cm}{0.5cm}{0cm}{0cm}{0cm}{0cm} %Gauche, haut, droite, haut
\newcounter{numexo}
\newcommand{\exo}[1]{\stepcounter{numexo}\noindent{\bf Exercice~\thenumexo} : \marginpar{\hfill /#1}}
\reversemarginpar


\newcounter{enumtabi}
\newcounter{enumtaba}
\newcommand{\q}{\stepcounter{enumtabi} \theenumtabi.  }
\newcommand{\qa}{\stepcounter{enumtaba} (\alph{enumtaba}) }
\newcommand{\initq}{\setcounter{enumtabi}{0}}
\newcommand{\initqa}{\setcounter{enumtaba}{0}}

\newcommand{\be}{\begin{enumerate}}
\newcommand{\ee}{\end{enumerate}}
\newcommand{\bi}{\begin{itemize}}
\newcommand{\ei}{\end{itemize}}
\newcommand{\bp}{\begin{pspicture*}}
\newcommand{\ep}{\end{pspicture*}}
\newcommand{\bt}{\begin{tabular}}
\newcommand{\et}{\end{tabular}}
\renewcommand{\tabularxcolumn}[1]{>{\centering}m{#1}} %(colonne m{} centrée, au lieu de p par défault) 
\newcommand{\tnl}{\tabularnewline}

\newcommand{\bmul}[1]{\begin{multicols}{#1}}
\newcommand{\emul}{\end{multicols}}

\newcommand{\trait}{\noindent \rule{\linewidth}{0.2mm}}
\newcommand{\hs}[1]{\hspace{#1}}
\newcommand{\vs}[1]{\vspace{#1}}

\newcommand{\N}{\mathbb{N}}
\newcommand{\Z}{\mathbb{Z}}
\newcommand{\R}{\mathbb{R}}
\newcommand{\C}{\mathbb{C}}
\newcommand{\Dcal}{\mathcal{D}}
\newcommand{\Ccal}{\mathcal{C}}
\newcommand{\mc}{\mathcal}

\newcommand{\vect}[1]{\overrightarrow{#1}}
\newcommand{\ds}{\displaystyle}
\newcommand{\eq}{\quad \Leftrightarrow \quad}
\newcommand{\vecti}{\vec{\imath}}
\newcommand{\vectj}{\vec{\jmath}}
\newcommand{\Oij}{(O;\vec{\imath}, \vec{\jmath})}
\newcommand{\OIJ}{(O;I,J)}


\newcommand{\reponse}[1][1]{%
\multido{}{#1}{\makebox[\linewidth]{\rule[0pt]{0pt}{20pt}\dotfill}
}}

\newcommand{\titre}[5] 
% #1: titre #2: haut gauche #3: bas gauche #4: haut droite #5: bas droite
{
\noindent #2 \hfill #4 \\
#3 \hfill #5

\vspace{-1.6cm}

\begin{center}\rule{6cm}{0.5mm}\end{center}
\vspace{0.2cm}
\begin{center}{\large{\textbf{#1}}}\end{center}
\begin{center}\rule{6cm}{0.5mm}\end{center}
}



\begin{document}
\pagestyle{empty}
\titre{Interrogation: Opérations sur les fractions }{Nom :}{Prénom :}{Classe}{Date}


\exo{1} Cours \\

Compléter les propriétés suivantes :

\bi
\item Pour multiplier deux nombres en écriture fractionnaire,, il suffit\reponse[2]  ; \\


\item Soient a, b, f des nombres relatifs (b non nul),  $\dfrac{a}{b} + \dfrac{f}{b} = $ . . . . . . . . . . \\

\ei

\exo{4,5}

\bmul{3}

$B = \dfrac{-3}{5} - \dfrac{-11}{5}$\\
\reponse[4]\\

$J = \dfrac{-3}{5} \times \dfrac{12}{-9}$\\
\reponse[4]\\

\columnbreak

$N = \dfrac{2}{3} - \dfrac{4}{7}$\\
\reponse[4]\\

$ U = \dfrac{\dfrac{2}{3}}{\dfrac{5}{18}}$\\
\reponse[4]\\


\columnbreak

$P = 2 - \dfrac{19}{6}$\\
\reponse[4]\\

$H = \dfrac{\dfrac{4}{-7}}{16}$\\

\noindent \reponse[4]\\

\emul


\exo{4,5}

\bmul{3}

$E =\dfrac{19}{6} - \dfrac{4}{3} \times \dfrac{5}{2}$ \\
\reponse[8]\\


\columnbreak

$S = (\dfrac{2}{5} \div \dfrac{8}{15}) \div (\dfrac{5}{7} + \dfrac{3}{21})$ \\
\reponse[8]\\


\columnbreak

$F = \dfrac{\dfrac{4}{7} - 2}{2 - \dfrac{11}{14}}$ \\
\reponse[8]


\emul

\exo{} BONUS

\noindent Quelle est le résultat de la somme de 3 et de l'inverse de la somme de 3 et de l'inverse de la somme de 3 et 3 ? (Expliquez votre résultat au dos de la feuille)


\end{document}
