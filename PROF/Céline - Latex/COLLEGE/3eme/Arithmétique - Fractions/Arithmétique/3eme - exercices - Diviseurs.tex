\documentclass[a4paper,11pt]{article}
\usepackage{amsmath,amsthm,amsfonts,amssymb,amscd,amstext,vmargin,graphics,graphicx,tabularx,multicol} 
\usepackage[francais]{babel}
\usepackage[utf8]{inputenc}  
\usepackage[T1]{fontenc} 
\usepackage{pstricks-add,tikz,tkz-tab,variations}
\usepackage[autolanguage,np]{numprint} 

\setmarginsrb{1.5cm}{0.5cm}{1cm}{0.5cm}{0cm}{0cm}{0cm}{0cm} %Gauche, haut, droite, haut
\newcounter{numexo}
\newcommand{\exo}[1]{\stepcounter{numexo}\noindent{\bf Exercice~\thenumexo} : \marginpar{\hfill /#1}}
\reversemarginpar


\newcounter{enumtabi}
\newcounter{enumtaba}
\newcommand{\q}{\stepcounter{enumtabi} \theenumtabi.  }
\newcommand{\qa}{\stepcounter{enumtaba} (\alph{enumtaba}) }
\newcommand{\initq}{\setcounter{enumtabi}{0}}
\newcommand{\initqa}{\setcounter{enumtaba}{0}}

\newcommand{\be}{\begin{enumerate}}
\newcommand{\ee}{\end{enumerate}}
\newcommand{\bi}{\begin{itemize}}
\newcommand{\ei}{\end{itemize}}
\newcommand{\bp}{\begin{pspicture*}}
\newcommand{\ep}{\end{pspicture*}}
\newcommand{\bt}{\begin{tabular}}
\newcommand{\et}{\end{tabular}}
\renewcommand{\tabularxcolumn}[1]{>{\centering}m{#1}} %(colonne m{} centrée, au lieu de p par défault) 
\newcommand{\tnl}{\tabularnewline}

\newcommand{\bmul}[1]{\begin{multicols}{#1}}
\newcommand{\emul}{\end{multicols}}

\newcommand{\trait}{\noindent \rule{\linewidth}{0.2mm}}
\newcommand{\hs}[1]{\hspace{#1}}
\newcommand{\vs}[1]{\vspace{#1}}

\newcommand{\N}{\mathbb{N}}
\newcommand{\Z}{\mathbb{Z}}
\newcommand{\R}{\mathbb{R}}
\newcommand{\C}{\mathbb{C}}
\newcommand{\Dcal}{\mathcal{D}}
\newcommand{\Ccal}{\mathcal{C}}
\newcommand{\mc}{\mathcal}

\newcommand{\vect}[1]{\overrightarrow{#1}}
\newcommand{\ds}{\displaystyle}
\newcommand{\eq}{\quad \Leftrightarrow \quad}
\newcommand{\vecti}{\vec{\imath}}
\newcommand{\vectj}{\vec{\jmath}}
\newcommand{\Oij}{(O;\vec{\imath}, \vec{\jmath})}
\newcommand{\OIJ}{(O;I,J)}


\newcommand{\reponse}[1][1]{%
\multido{}{#1}{\makebox[\linewidth]{\rule[0pt]{0pt}{20pt}\dotfill}
}}

\newcommand{\titre}[5] 
% #1: titre #2: haut gauche #3: bas gauche #4: haut droite #5: bas droite
{
\noindent #2 \hfill #4 \\
#3 \hfill #5

\vspace{-1.6cm}

\begin{center}\rule{6cm}{0.5mm}\end{center}
\vspace{0.2cm}
\begin{center}{\large{\textbf{#1}}}\end{center}
\begin{center}\rule{6cm}{0.5mm}\end{center}
}



\begin{document}
\pagestyle{empty}


\begin{center}
\textbf{ {\large Exercices  : Arithmétique(1)}}
\end{center}

\vspace*{0.5cm}

\exo{} 
 Trouver un nombre à quatre chiffres : 
\bi
\item divisible par 2 : 
\item  divisible par 9 : :
\item divisible par 3  et 4    :
\item divisible par 5  et 9: \\

\ei




\exo{} 


Cette activité met en \oe{uvre} un algorithme appelé "le crible d'Erathostène" permettant de trouver tous les nombres premiers inférieurs à 100.\\

\includegraphics[scale=1]{Erathostene2.eps} \\

\q \qa Expliquer pourquoi le nombre 1 n'est pas premier puis le barrer dans la grille.\\
 \qa Le nombre 2 ne possède aucun diviseur autre que 1 et lui-même. 2 est donc un nombre premier. Entourer le nombre 2.\\
\qa Barrer tous les multiples de 2, qui ne sont donc pas des nombres premiers.\\


\q \initqa  \qa Entourer le plus petit nombre non barré et barrer tous ses multiples.\\
\qa Poursuivre de la même façon jusqu'à ce que le plus petit nombre non barré soit supérieur à 10. \\
Tous les nombres non barrés dans la liste, sont les nombres qui n'ont pas d'autre diviseur que 1 ou eux-mêmes. \textbf{On obtient tous les nombres premiers inférieur à 100.}\\

\q Écrire tous les nombres premiers inférieur à 40 : \\







\vspace*{0.5cm}

\exo{} \\

\initq
\q Quels sont les diviseurs de 21 ?\\
\q Quels sont les diviseurs de 63 ?\\
\q Quels sont les diviseurs en commun à 21 et 63 ?\\

\vspace*{0.5cm}

\exo{} \\

\initq
\q Quels sont les diviseurs de 350 ?\\
\q Quels sont les diviseurs de 210 ?\\
\q Quels sont les diviseurs en commun à 350 et 210 ?\\

\vspace*{0.5cm}

\exo{} 

Avec 168 disquettes bleues et 210 disquettes jaunes, un vendeur remarque qu'il remplit des boîtes identiques en utilisant toutes ses boîtes.\\

\initq \q Combien de boîtes au maximum peut-il constituer dans ces conditions ?\\

\q Dans ce cas, combien chacune des boîtes contient-elle de disquettes bleues et de disques jaunes ?\\





\end{document}
