\documentclass[a4paper,11pt]{article}
\usepackage{amsmath,amsthm,amsfonts,amssymb,amscd,amstext,vmargin,graphics,graphicx,tabularx,multicol} \usepackage[french]{babel}
\usepackage[utf8]{inputenc}  
\usepackage[T1]{fontenc} 
\usepackage[T1]{fontenc}
\usepackage{amsmath,amssymb}
\usepackage{pstricks-add,tikz,tkz-tab,variations}
\usepackage[autolanguage,np]{numprint} 
\usepackage{color}
\usepackage{ulem}
\usepackage{textcomp} 

\setmarginsrb{1.5cm}{0.5cm}{1cm}{0.5cm}{0cm}{0cm}{0cm}{0cm} %Gauche, haut, droite, haut
\newcounter{numexo}
\newcommand{\exo}[1]{\stepcounter{numexo}\noindent{\bf Exercice~\thenumexo} : \marginpar{\hfill /#1}}
\reversemarginpar


\newcounter{enumtabi}
\newcounter{enumtaba}
\newcommand{\q}{\stepcounter{enumtabi} \theenumtabi.  }
\newcommand{\qa}{\stepcounter{enumtaba} (\alph{enumtaba}) }
\newcommand{\initq}{\setcounter{enumtabi}{0}}
\newcommand{\initqa}{\setcounter{enumtaba}{0}}

\newcommand{\be}{\begin{enumerate}}
\newcommand{\ee}{\end{enumerate}}
\newcommand{\bi}{\begin{itemize}}
\newcommand{\ei}{\end{itemize}}
\newcommand{\bp}{\begin{pspicture*}}
\newcommand{\ep}{\end{pspicture*}}
\newcommand{\bt}{\begin{tabular}}
\newcommand{\et}{\end{tabular}}
\renewcommand{\tabularxcolumn}[1]{>{\centering}m{#1}} %(colonne m{} centrée, au lieu de p par défault) 
\newcommand{\tnl}{\tabularnewline}

\newcommand{\trait}{\noindent \rule{\linewidth}{0.2mm}}
\newcommand{\hs}[1]{\hspace{#1}}
\newcommand{\vs}[1]{\vspace{#1}}

\newcommand{\N}{\mathbb{N}}
\newcommand{\Z}{\mathbb{Z}}
\newcommand{\R}{\mathbb{R}}
\newcommand{\C}{\mathbb{C}}
\newcommand{\Dcal}{\mathcal{D}}
\newcommand{\Ccal}{\mathcal{C}}
\newcommand{\mc}{\mathcal}

\newcommand{\vect}[1]{\overrightarrow{#1}}
\newcommand{\ds}{\displaystyle}
\newcommand{\eq}{\quad \Leftrightarrow \quad}
\newcommand{\vecti}{\vec{\imath}}
\newcommand{\vectj}{\vec{\jmath}}
\newcommand{\Oij}{(O;\vec{\imath}, \vec{\jmath})}
\newcommand{\OIJ}{(O;I,J)}

\newcommand{\bmul}[1]{\begin{multicols}{#1}}
\newcommand{\emul}{\end{multicols}}


\newcommand{\reponse}[1][1]{%
\multido{}{#1}{\makebox[\linewidth]{\rule[0pt]{0pt}{20pt}\dotfill}
}}

\newcommand{\titre}[5] 
% #1: titre #2: haut gauche #3: bas gauche #4: haut droite #5: bas droite
{
\noindent #2 \hfill #4 \\
#3 \hfill #5

\vspace{-1.6cm}

\begin{center}\rule{6cm}{0.5mm}\end{center}
\vspace{0.2cm}
\begin{center}{\large{\textbf{#1}}}\end{center}
\begin{center}\rule{6cm}{0.5mm}\end{center}
}



\begin{document}
\pagestyle{empty}
\titre{Puissances et écritures scientifiques}{Nom}{Prénom}{Date}{Classe}
\vspace*{0.5cm}

\exo \\
Calculer les expressions suivantes en détaillant vos étapes de calculs.

\bmul{3}

$P = 20  - 3 \times 3^{2}$\\

$C = (3 \times 7)^{2} + 4 $\\

\columnbreak

$L =(7+2^{3}) \times 10$\\

$S = 3 \times (4 + 5 )^{2}$\\


\columnbreak

$B= 3 \times 7^{2} +4$\\

$G = \dfrac{131,2 - 2 \times 4^{3}}{7^{2} - 3^{2}}$\\

\emul

\vspace*{0.4cm}

\exo \\
Recopier le texte suivant en réécrivant les nombres en gras en écriture décimale.\\

"Le Soleil est une étoile de diamètre $0,14 \times 10^{7}$ km et vieille de $46 \times 10^{8}$ années. La température en son centre s'élève à $150 \times 10^{5} $ °C." \\

\vspace*{0.4cm}

\exo \\
Donner l'écriture scientifique des nombres suivants : 

\bmul{4}

6 510 000

\columnbreak

0,0007 23

\columnbreak

$46 \times 10^{4} \times 10^{-9}$

\columnbreak

$(8 \times 10^{6})^{3}$

\emul

\vspace*{0.4cm}

\exo \\
Voici les diamètres de deux types de bactéries et de deux virus.
\bi
\item Bactérie typique : $0,2 \times 10^{-7}$ m ; 
\item Nano bactérie : $50 \times 10^{-9} $ m ;
\item Virus de la varicelle : $1 750 \times 10^{-10}$ m ;
\item Virus de la gastroentérite : $0,017 \times 10^{-6}$ m.\\
\ei

Donner la notation scientifique de chaque diamètre, puis ranger ces diamètres dans l'ordre croissant.\\

\vspace*{0.4cm}


\exo \\
Donner l'écriture décimale puis la notation scientifique de A et de B.\\

$A = \dfrac{23-4 \times 10^{-3}}{8 \times 10^{2}}$ \hspace*{1.5cm} $B =\dfrac{6 \times 10^{-3} \times 36 \times 10^{-2}}{18 \times 10^{-4}}$\\

\vspace*{0.4cm}

\exo \\

\q L'égalité $10^{7} + 10^{-7} = 1$ est-elle vraie ?\\

\q L'égalité $\dfrac{10^{17} + 3}{10^{17}}=1$ est-elle vraie ?\\

\vspace*{0.4cm}

\exo \\

\initq \q Un carré a pour périmètre $2^{21}$ cm. Exprimer sous la forme d'une puissance de 2 :\\
\qa la longueur du côté de ce carré ;\\
\qa l'aire de ce carré.\\

\q Quelle est la longueur du côté d'un carré d'aire $9^{14}$ $cm^{2}$ ?\\

\q Une arête d'un cube mesure $5 \times 10^{-8}$ m.\\
\initqa \qa Quelle est l'aire d'une face ?\\
\qa Quelle est le volume de ce cube ?\\

\q Le volume d'un cube est $7^{12}$ $dm^{3}$. Quelle est la longueur d'une arête ?\\





\end{document}
