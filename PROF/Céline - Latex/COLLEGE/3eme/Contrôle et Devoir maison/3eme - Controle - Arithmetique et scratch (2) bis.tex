\documentclass[a4paper,12pt]{article}
\usepackage{amsmath,amsthm,amsfonts,amssymb,amscd,amstext,vmargin,graphics,graphicx,tabularx,multicol} 
\usepackage[francais]{babel}
\usepackage[utf8]{inputenc}  
\usepackage[T1]{fontenc} 
\usepackage{pstricks-add,tikz,tkz-tab,variations}
\usepackage[autolanguage,np]{numprint} 

\setmarginsrb{2cm}{1cm}{2cm}{1cm}{0cm}{0cm}{0cm}{0cm} %Gauche, haut, droite, haut
\newcounter{numexo}
\newcommand{\exo}[1]{\stepcounter{numexo}\noindent{\bf Exercice~\thenumexo} : \marginpar{\hfill /#1}}
\reversemarginpar


\newcounter{enumtabi}
\newcounter{enumtaba}
\newcommand{\q}{\stepcounter{enumtabi} \theenumtabi)  }
\newcommand{\qa}{\stepcounter{enumtaba} (\alph{enumtaba}) }
\newcommand{\initq}{\setcounter{enumtabi}{0}}
\newcommand{\initqa}{\setcounter{enumtaba}{0}}

\newcommand{\be}{\begin{enumerate}}
\newcommand{\ee}{\end{enumerate}}
\newcommand{\bi}{\begin{itemize}}
\newcommand{\ei}{\end{itemize}}
\newcommand{\bp}{\begin{pspicture*}}
\newcommand{\ep}{\end{pspicture*}}
\newcommand{\bt}{\begin{tabular}}
\newcommand{\et}{\end{tabular}}
\renewcommand{\tabularxcolumn}[1]{>{\centering}m{#1}} %(colonne m{} centrée, au lieu de p par défault) 
\newcommand{\tnl}{\tabularnewline}

\newcommand{\bmul}[1]{\begin{multicols}{#1}}
\newcommand{\emul}{\end{multicols}}

\newcommand{\trait}{\noindent \rule{\linewidth}{0.2mm}}
\newcommand{\hs}[1]{\hspace{#1}}
\newcommand{\vs}[1]{\vspace{#1}}

\newcommand{\N}{\mathbb{N}}
\newcommand{\Z}{\mathbb{Z}}
\newcommand{\R}{\mathbb{R}}
\newcommand{\C}{\mathbb{C}}
\newcommand{\Dcal}{\mathcal{D}}
\newcommand{\Ccal}{\mathcal{C}}
\newcommand{\mc}{\mathcal}

\newcommand{\vect}[1]{\overrightarrow{#1}}
\newcommand{\ds}{\displaystyle}
\newcommand{\eq}{\quad \Leftrightarrow \quad}
\newcommand{\vecti}{\vec{\imath}}
\newcommand{\vectj}{\vec{\jmath}}
\newcommand{\Oij}{(O;\vec{\imath}, \vec{\jmath})}
\newcommand{\OIJ}{(O;I,J)}


\newcommand{\reponse}[1][1]{%
\multido{}{#1}{\makebox[\linewidth]{\rule[0pt]{0pt}{20pt}\dotfill}
}}

\newcommand{\titre}[5] 
% #1: titre #2: haut gauche #3: bas gauche #4: haut droite #5: bas droite
{
\noindent #2 \hfill #4 \\
#3 \hfill #5

\vspace{-1.6cm}

\begin{center}\rule{6cm}{0.5mm}\end{center}
\vspace{0.2cm}
\begin{center}{\large{\textbf{#1}}}\end{center}
\begin{center}\rule{6cm}{0.5mm}\end{center}
}



\begin{document}
\pagestyle{empty}

\titre{Contrôle - Arithmétique}{Nom :}{Prénom :}{Classe}{Date}

\vspace*{0.25cm}

\exo{2} L'ensemble des écrits de Victor Hugo a été republié après sa mort en 1 433 volumes. La bibliothécaire classe ces volumes à raison de 7 volumes par étagères.\\

 Combien d'étagères faut-il pour exposer toute l'\oe{uvre} de Victor Hugo?\\ Combien de volumes l'étagère incomplète contiendra-t-elle? \textit{Justifier votre réponse.}\\

\vspace*{0.25cm}

\exo{4}

\q Lequel de ces nombres n'est pas premier ? Justifier votre réponse.\\
\hspace*{2.5cm}19 \hspace*{0.25cm};\hspace*{0.25cm} 23 \hspace*{0.25cm};\hspace*{0.25cm} 93 \hspace*{0.25cm};\hspace*{0.25cm} 43 \hspace*{0.25cm};\hspace*{0.25cm} 61 \hspace*{0.25cm};\hspace*{0.25cm} 89 \hspace*{0.25cm} et \hspace*{0.25cm} 103.\\

\q On donne les nombres suivants :\hfill 36 \hfill ; \hfill 58 180 \hfill; \hfill 27 900 \hfill; \hfill 63 604 \hfill; \hfill 42 324 \hfill; \hfill 34 410.\\
\qa Citer les nombres qui sont divisibles par 5. \textit{(Aucune justification n'est attendue.)}\\
\qa Citer ceux qui sont à la fois divisible par 3 et par 4. \textit{(Aucune justification n'est attendue.)}\\


\q Trouver un nombre à quatre chiffres à la fois divisible par 2;  divisible par 3; divisible par 5 et non divisible par 9. \textit{(Aucune justification n'est attendue.)}\\



\vspace*{0.25cm}


\exo{5} "Les nombres amicaux"\\
\initq \q Citer tous les diviseurs de 220 et 284.\\
\q Deux nombres sont amicaux si  les sommes de leurs diviseurs sont égales. Montrer que 220 et 284 sont amicaux.?\\


\vspace*{0.25cm}

\exo{3}Un carreleur doit poser le carrelage dans une pièce rectangulaire mesurant 6, 48 m de large sur 13, 50 m de long.
Il souhaite poser des carreaux de carrelage carré et ne faire aucune découpe.\\
\initq \q Peut-il poser des carreaux de 27 cm de côté ? \textit{Justifier votre réponse.}\\
\q Peut-il poser des carreaux de 50 cm de côté ? \textit{Justifier votre réponse.}\\




\vspace*{0.25cm}

\exo{6} Un restaurant vend des barquettes composées de nems et de samossas.\\ Le cuisinier a préparé 2 106 nems et 1 188 samossas.\\
Dans chaque barquette, le nombre de nems doit être le même et le nombre de samossas doit être le même.\\
Tous les nems et tous les samossas doivent être utilisés.\\

\noindent \initq \q Le cuisinier peut-il faire 45 barquettes en utilisant tous les samosas et tous les nems ?\\
 \q \initqa \qa Avec ces conditions, quel nombre de barquettes \textbf{au maximum} pourra-t-il réaliser?\\
\qa Dans ce cas, combien y aura-t-il de nems et de samossas dans chaque barquette ?\\


\vspace*{0.25cm}





\exo{} BONUS\\
Le nombre de marches d'un escalier est compris entre 40 et 80.\\
- Si on compte ces marches deux par deux, il en reste une.\\
- Si on compte ces marches trois par trois, il en reste deux.\\
- Si on compte ces marches cinq par cinq, il en reste quatre.\\

Quel est le nombre de marches de cet escalier ?\\







\end{document}
