\documentclass[a4paper,11pt]{article}
\usepackage{amsmath,amsthm,amsfonts,amssymb,amscd,amstext,vmargin,graphics,graphicx,tabularx,multicol} 
\usepackage[francais]{babel}
\usepackage[utf8]{inputenc}  
\usepackage[T1]{fontenc} 
\usepackage{pstricks-add,tikz,tkz-tab,variations}
\usepackage[autolanguage,np]{numprint} 

\setmarginsrb{1.5cm}{0.5cm}{1cm}{0.5cm}{0cm}{0cm}{0cm}{0cm} %Gauche, haut, droite, haut
\newcounter{numexo}
\newcommand{\exo}[1]{\stepcounter{numexo}\noindent{\bf Exercice~\thenumexo} : \marginpar{\hfill /#1}}
\reversemarginpar


\newcounter{enumtabi}
\newcounter{enumtaba}
\newcommand{\q}{\stepcounter{enumtabi} \theenumtabi.  }
\newcommand{\qa}{\stepcounter{enumtaba} (\alph{enumtaba}) }
\newcommand{\initq}{\setcounter{enumtabi}{0}}
\newcommand{\initqa}{\setcounter{enumtaba}{0}}

\newcommand{\be}{\begin{enumerate}}
\newcommand{\ee}{\end{enumerate}}
\newcommand{\bi}{\begin{itemize}}
\newcommand{\ei}{\end{itemize}}
\newcommand{\bp}{\begin{pspicture*}}
\newcommand{\ep}{\end{pspicture*}}
\newcommand{\bt}{\begin{tabular}}
\newcommand{\et}{\end{tabular}}
\renewcommand{\tabularxcolumn}[1]{>{\centering}m{#1}} %(colonne m{} centrée, au lieu de p par défault) 
\newcommand{\tnl}{\tabularnewline}

\newcommand{\bmul}[1]{\begin{multicols}{#1}}
\newcommand{\emul}{\end{multicols}}

\newcommand{\trait}{\noindent \rule{\linewidth}{0.2mm}}
\newcommand{\hs}[1]{\hspace{#1}}
\newcommand{\vs}[1]{\vspace{#1}}

\newcommand{\N}{\mathbb{N}}
\newcommand{\Z}{\mathbb{Z}}
\newcommand{\R}{\mathbb{R}}
\newcommand{\C}{\mathbb{C}}
\newcommand{\Dcal}{\mathcal{D}}
\newcommand{\Ccal}{\mathcal{C}}
\newcommand{\mc}{\mathcal}

\newcommand{\vect}[1]{\overrightarrow{#1}}
\newcommand{\ds}{\displaystyle}
\newcommand{\eq}{\quad \Leftrightarrow \quad}
\newcommand{\vecti}{\vec{\imath}}
\newcommand{\vectj}{\vec{\jmath}}
\newcommand{\Oij}{(O;\vec{\imath}, \vec{\jmath})}
\newcommand{\OIJ}{(O;I,J)}


\newcommand{\reponse}[1][1]{%
\multido{}{#1}{\makebox[\linewidth]{\rule[0pt]{0pt}{20pt}\dotfill}
}}

\newcommand{\titre}[5] 
% #1: titre #2: haut gauche #3: bas gauche #4: haut droite #5: bas droite
{
\noindent #2 \hfill #4 \\
#3 \hfill #5

\vspace{-1.6cm}

\begin{center}\rule{6cm}{0.5mm}\end{center}
\vspace{0.2cm}
\begin{center}{\large{\textbf{#1}}}\end{center}
\begin{center}\rule{6cm}{0.5mm}\end{center}
}



\begin{document}
\pagestyle{empty}
\titre{Division décimale  }{Nom :}{Prénom :}{Classe}{Date}



\exo{} \q Poser et effectuer la division décimale suivante : $159 \div 60$ \\

\vspace*{5.5cm}


\q Poser, effectuer les divisions décimales suivantes et donner les résultats \textit{arrondis au millième près} :\\
 $31,7 \div 12$ \hspace*{2.5cm} $3,59 \div 2,1$\\

\vspace*{6cm}


\exo{4} Répondre aux problèmes ci-dessous en détaillant vos réponses.\\

\initqa
\qa Gérard a payé 28,56 euros pour 12 pieds de tomates. Quel est le prix d'un pied de tomate ?\\
\reponse[5]\\

\qa Bastien a économisé tout son argent de poche de cette année. Ses parents lui ont donné la même somme d'argent tous les mois. Cela représente une somme de 480 euros.\\
 Combien d'argent ses parents lui donnait tous les mois ? \\
 \reponse[5]\\
 
\newpage 

\vspace*{0.5cm}

 \qa 6 amis se partagent au goûter une bouteille de 1,5L de soda. Combien de litre de soda boiront-ils chacun ?\\
\reponse[5]\\

\vspace*{0.5cm}

\exo{4}
Mélissa dispose de 20 euros. Elle achète 3 cahiers identiques et 8 stylos de couleur. \\
	Chaque stylo coûte  1,25 euros . La caissière lui rend 2,80 euros.\\
	 Quel est le prix d’un cahier ?\\
\reponse[12]\\


\vspace*{0.35cm}

\exo{} BONUS

Sur une planète, où poussent des fleurs immenses, un amoureux en cueille une
dont la corolle a 258 839 pétales !\\

Il commence à l’effeuiller et il dit « Elle m'aime » en enlevant le premier pétale, « un peu » en enlevant
le second, « beaucoup » en enlevant le troisième, puis « passionnément », « à la folie », « pas du
tout ». \\
Et il recommence : « Elle m'aime », « un peu », « beaucoup »...\\
\textbf{Que va-t-il dire en effeuillant le dernier pétale ? Justifie ta réponse.}\\
\reponse[7]\\







\end{document}
