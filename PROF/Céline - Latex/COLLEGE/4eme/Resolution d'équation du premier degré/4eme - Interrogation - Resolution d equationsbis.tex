\documentclass[a4paper,11pt]{article}
\usepackage{amsmath,amsthm,amsfonts,amssymb,amscd,amstext,vmargin,graphics,graphicx,tabularx,multicol} \usepackage[french]{babel}
\usepackage[utf8]{inputenc}  
\usepackage[T1]{fontenc} 
\usepackage[T1]{fontenc}
\usepackage{amsmath,amssymb}
\usepackage{pstricks-add,tikz,tkz-tab,variations}
\usepackage[autolanguage,np]{numprint} 
\usepackage{color}
\usepackage{ulem}

\setmarginsrb{1.5cm}{0.5cm}{1cm}{0.5cm}{0cm}{0cm}{0cm}{0cm} %Gauche, haut, droite, haut
\newcounter{numexo}
\newcommand{\exo}[1]{\stepcounter{numexo}\noindent{\bf Exercice~\thenumexo} : \marginpar{\hfill /#1}}
\reversemarginpar


\newcounter{enumtabi}
\newcounter{enumtaba}
\newcommand{\q}{\stepcounter{enumtabi} \theenumtabi.  }
\newcommand{\qa}{\stepcounter{enumtaba} (\alph{enumtaba}) }
\newcommand{\initq}{\setcounter{enumtabi}{0}}
\newcommand{\initqa}{\setcounter{enumtaba}{0}}

\newcommand{\be}{\begin{enumerate}}
\newcommand{\ee}{\end{enumerate}}
\newcommand{\bi}{\begin{itemize}}
\newcommand{\ei}{\end{itemize}}
\newcommand{\bp}{\begin{pspicture*}}
\newcommand{\ep}{\end{pspicture*}}
\newcommand{\bt}{\begin{tabular}}
\newcommand{\et}{\end{tabular}}
\renewcommand{\tabularxcolumn}[1]{>{\centering}m{#1}} %(colonne m{} centrée, au lieu de p par défault) 
\newcommand{\tnl}{\tabularnewline}

\newcommand{\trait}{\noindent \rule{\linewidth}{0.2mm}}
\newcommand{\hs}[1]{\hspace{#1}}
\newcommand{\vs}[1]{\vspace{#1}}

\newcommand{\N}{\mathbb{N}}
\newcommand{\Z}{\mathbb{Z}}
\newcommand{\R}{\mathbb{R}}
\newcommand{\C}{\mathbb{C}}
\newcommand{\Dcal}{\mathcal{D}}
\newcommand{\Ccal}{\mathcal{C}}
\newcommand{\mc}{\mathcal}

\newcommand{\vect}[1]{\overrightarrow{#1}}
\newcommand{\ds}{\displaystyle}
\newcommand{\eq}{\quad \Leftrightarrow \quad}
\newcommand{\vecti}{\vec{\imath}}
\newcommand{\vectj}{\vec{\jmath}}
\newcommand{\Oij}{(O;\vec{\imath}, \vec{\jmath})}
\newcommand{\OIJ}{(O;I,J)}

\newcommand{\bmul}[1]{\begin{multicols}{#1}}
\newcommand{\emul}{\end{multicols}}


\newcommand{\reponse}[1][1]{%
\multido{}{#1}{\makebox[\linewidth]{\rule[0pt]{0pt}{20pt}\dotfill}
}}

\newcommand{\titre}[5] 
% #1: titre #2: haut gauche #3: bas gauche #4: haut droite #5: bas droite
{
\noindent #2 \hfill #4 \\
#3 \hfill #5

\vspace{-1.6cm}

\begin{center}\rule{6cm}{0.5mm}\end{center}
\vspace{0.2cm}
\begin{center}{\large{\textbf{#1}}}\end{center}
\begin{center}\rule{6cm}{0.5mm}\end{center}
}



\begin{document}
\pagestyle{empty}
\titre{Interrogation : Résolution d'équations}{Nom}{Prénom}{Date}{Classe}


\exo{2} 

\q Soit l'équation $-5x + 3 =  19 - x$.  Est-ce que 8 est solution de cette équation ? Justifier votre réponse par des calculs.\\
\reponse[5]\\

\q Soit l'équation $4(x-5)=3x-27$. Est-ce que -7 est solution de cette équation ? Justifier votre réponse par des calculs.\\
\reponse[5]\\

\vspace*{0.5cm}

\exo{8} Résoudre les équations suivantes en détaillant la résolution :

\bmul{3}

$39 + x = 24$\\
\reponse[4]\\

$ -8x = 56$\\
\reponse[4]\\

\columnbreak

$  x -17 = -23$\\
\reponse[4]\\

$  \dfrac{5}{7}x=15 $\\
\reponse[4]\\


\columnbreak

$ 6x = 66$\\
\reponse[4]\\

$  -3-x=7 $\\
\reponse[4]\\

\emul

\bmul{2}

$ 6x + 7 = -11 $\\
\reponse[5]\\


\columnbreak

$ -11 - 5x = - 76$\\
\reponse[5]\\



\emul

\newpage

\vspace*{0.4cm}


\bmul{2}



$2-2x = 5x + 3$\\
\reponse[9]\\

\columnbreak



$4x - 19 = 7x - 10 $\\
\reponse[9]\\

\emul

\exo{} BONUS\\

ÉNIGME 1 : 

Justine a 8 ans et sa grand-mère a 50 ans. \\
Dans combien d'années, l'âge de sa grand-mère sera le triple de celui de Justine ?

\vspace*{0.5cm}

ÉNIGME 2 :

 Une brique pèse 1 kg plus la moitié de son poids.      Combien pèse-t-elle ?
\vspace*{0.5cm}

ÉNIGME 2 :

 Résoudre l'équation $-2(x+7) = 9-(x +4)$
\vspace*{0.5cm}

\noindent \reponse[15]


\end{document}
