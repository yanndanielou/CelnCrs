\documentclass[a4paper,11pt]{article}
\usepackage{amsmath,amsthm,amsfonts,amssymb,amscd,amstext,vmargin,graphics,graphicx,tabularx,multicol} 
\usepackage[francais]{babel}
\usepackage[utf8]{inputenc}  
\usepackage[T1]{fontenc} 
\usepackage{pstricks-add,tikz,tkz-tab,variations}
\usepackage[autolanguage,np]{numprint} 

\setmarginsrb{1.5cm}{0.5cm}{1cm}{0.5cm}{0cm}{0cm}{0cm}{0cm} %Gauche, haut, droite, haut
\newcounter{numexo}
\newcommand{\exo}[1]{\stepcounter{numexo}\noindent{\bf Exercice~\thenumexo} : \marginpar{\hfill /#1}}
\reversemarginpar


\newcounter{enumtabi}
\newcounter{enumtaba}
\newcommand{\q}{\stepcounter{enumtabi} \theenumtabi.  }
\newcommand{\qa}{\stepcounter{enumtaba} (\alph{enumtaba}) }
\newcommand{\initq}{\setcounter{enumtabi}{0}}
\newcommand{\initqa}{\setcounter{enumtaba}{0}}

\newcommand{\be}{\begin{enumerate}}
\newcommand{\ee}{\end{enumerate}}
\newcommand{\bi}{\begin{itemize}}
\newcommand{\ei}{\end{itemize}}
\newcommand{\bp}{\begin{pspicture*}}
\newcommand{\ep}{\end{pspicture*}}
\newcommand{\bt}{\begin{tabular}}
\newcommand{\et}{\end{tabular}}
\renewcommand{\tabularxcolumn}[1]{>{\centering}m{#1}} %(colonne m{} centrée, au lieu de p par défault) 
\newcommand{\tnl}{\tabularnewline}

\newcommand{\bmul}[1]{\begin{multicols}{#1}}
\newcommand{\emul}{\end{multicols}}

\newcommand{\trait}{\noindent \rule{\linewidth}{0.2mm}}
\newcommand{\hs}[1]{\hspace{#1}}
\newcommand{\vs}[1]{\vspace{#1}}

\newcommand{\N}{\mathbb{N}}
\newcommand{\Z}{\mathbb{Z}}
\newcommand{\R}{\mathbb{R}}
\newcommand{\C}{\mathbb{C}}
\newcommand{\Dcal}{\mathcal{D}}
\newcommand{\Ccal}{\mathcal{C}}
\newcommand{\mc}{\mathcal}

\newcommand{\vect}[1]{\overrightarrow{#1}}
\newcommand{\ds}{\displaystyle}
\newcommand{\eq}{\quad \Leftrightarrow \quad}
\newcommand{\vecti}{\vec{\imath}}
\newcommand{\vectj}{\vec{\jmath}}
\newcommand{\Oij}{(O;\vec{\imath}, \vec{\jmath})}
\newcommand{\OIJ}{(O;I,J)}


\newcommand{\reponse}[1][1]{%
\multido{}{#1}{\makebox[\linewidth]{\rule[0pt]{0pt}{20pt}\dotfill}
}}

\newcommand{\titre}[5] 
% #1: titre #2: haut gauche #3: bas gauche #4: haut droite #5: bas droite
{
\noindent #2 \hfill #4 \\
#3 \hfill #5

\vspace{-1.6cm}

\begin{center}\rule{6cm}{0.5mm}\end{center}
\vspace{0.2cm}
\begin{center}{\large{\textbf{#1}}}\end{center}
\begin{center}\rule{6cm}{0.5mm}\end{center}
}



\begin{document}
\pagestyle{empty}
\titre{Interrogation: Calcul littéral et Théorème de Pythagore }{Nom :}{Prénom :}{Classe}{Date}


\exo{2} Réduire si possible les expressions suivantes :\\

\bmul{2}

$M = 5z \times 3z $\\
\reponse[2]\\

$T = - 5x^{2} + 3 + 8x^{2} - 9 + 2x $\\
\reponse[2]\\


\columnbreak


$E = -7x - (4x - x) $\\
\reponse[2]\\

$L =(- 7) \times  2t^{2}  - 3 \times t^{2} + 9 \times t $\\
\reponse[2]\\

\emul




\exo{3} Développer et réduire les expressions suivantes : \\

\bmul{2}

$G = -11x(x -3)$\\
\reponse[4]\\

\columnbreak

$V = (x-2)(9 - 3x)$\\
\reponse[4]\\

\emul

$A = (4-x)(x+2) + 6 (-x + 3) - (3x^{2} + 7x - 13)$\\
\reponse[6]\\

\exo{2,5} 
On considère le triangle IMO rectangle en O tel que MO = 9 cm, OI = 12 cm.\\

Calculer la longueur MI.


\noindent \reponse[6]\\

\newpage

\exo{2,5} Nathan prétend que son potager est rectangulaire. Émilie veut le vérifier. Elle mesure la longueur des deux côtés consécutifs du potager et obtient 3,3 m pour l'un et 5,6 m pour l'autre. La diagonale mesure 6,6 m. Émilie affirme a Nathan que son potager n'est pas. \\

Qui de Émilie ou Nathan a raison ? \textbf{Démontrer} votre réponse rigoureusement.\\

\noindent \reponse[5]\\






\end{document}
