\documentclass[a4paper,11pt]{article}
\usepackage{amsmath,amsthm,amsfonts,amssymb,amscd,amstext,vmargin,graphics,graphicx,tabularx,multicol} 
\usepackage[francais]{babel}
\usepackage[utf8]{inputenc}  
\usepackage[T1]{fontenc} 
\usepackage{pstricks-add,tikz,tkz-tab,variations}
\usepackage[autolanguage,np]{numprint} 
\usepackage{calc}

\setmarginsrb{1.5cm}{0.5cm}{1cm}{0.5cm}{0cm}{0cm}{0cm}{0cm} %Gauche, haut, droite, haut
\newcounter{numexo}
\newcommand{\exo}[1]{\stepcounter{numexo}\noindent{\bf Exercice~\thenumexo} : }
\reversemarginpar

\newcommand{\bmul}[1]{\begin{multicols}{#1}}
\newcommand{\emul}{\end{multicols}}

\newcounter{enumtabi}
\newcounter{enumtaba}
\newcommand{\q}{\stepcounter{enumtabi} \theenumtabi.  }
\newcommand{\qa}{\stepcounter{enumtaba} (\alph{enumtaba}) }
\newcommand{\initq}{\setcounter{enumtabi}{0}}
\newcommand{\initqa}{\setcounter{enumtaba}{0}}

\newcommand{\be}{\begin{enumerate}}
\newcommand{\ee}{\end{enumerate}}
\newcommand{\bi}{\begin{itemize}}
\newcommand{\ei}{\end{itemize}}
\newcommand{\bp}{\begin{pspicture*}}
\newcommand{\ep}{\end{pspicture*}}
\newcommand{\bt}{\begin{tabular}}
\newcommand{\et}{\end{tabular}}
\renewcommand{\tabularxcolumn}[1]{>{\centering}m{#1}} %(colonne m{} centrée, au lieu de p par défault) 
\newcommand{\tnl}{\tabularnewline}

\newcommand{\trait}{\noindent \rule{\linewidth}{0.2mm}}
\newcommand{\hs}[1]{\hspace{#1}}
\newcommand{\vs}[1]{\vspace{#1}}

\newcommand{\N}{\mathbb{N}}
\newcommand{\Z}{\mathbb{Z}}
\newcommand{\R}{\mathbb{R}}
\newcommand{\C}{\mathbb{C}}
\newcommand{\Dcal}{\mathcal{D}}
\newcommand{\Ccal}{\mathcal{C}}
\newcommand{\mc}{\mathcal}

\newcommand{\vect}[1]{\overrightarrow{#1}}
\newcommand{\ds}{\displaystyle}
\newcommand{\eq}{\quad \Leftrightarrow \quad}
\newcommand{\vecti}{\vec{\imath}}
\newcommand{\vectj}{\vec{\jmath}}
\newcommand{\Oij}{(O;\vec{\imath}, \vec{\jmath})}
\newcommand{\OIJ}{(O;I,J)}


\newcommand{\reponse}[1][1]{%
\multido{}{#1}{\makebox[\linewidth]{\rule[0pt]{0pt}{20pt}\dotfill}
}}

\newcommand{\titre}[5] 
% #1: titre #2: haut gauche #3: bas gauche #4: haut droite #5: bas droite
{
\noindent #2 \hfill #4 \\
#3 \hfill #5

\vspace{-1.6cm}

\begin{center}\rule{6cm}{0.5mm}\end{center}
\vspace{0.2cm}
\begin{center}{\large{\textbf{#1}}}\end{center}
\begin{center}\rule{6cm}{0.5mm}\end{center}
}



\begin{document}
\pagestyle{empty}
\titre{Séance d'AP : Les pourcentages}{}{}{4ème}{}

\vspace*{0.2cm}


\setlength{\fboxrule}{2pt}
\begin{flushleft}
\framebox{\begin{minipage}{\linewidth}

\vspace*{0.5cm}

\underline{\textbf{{\large Cours} : Augmentation et réduction}}\\

\textbf{(a) Le prix d'un manteau de 160 euros est augmenté de 20 \% .\textit{ Quel est le nouveau prix ?}}\\

On calcule d'abord, le montant de l'augmentation :\hspace*{1cm} $\dfrac{20}{100} \times 160 = 0,2 \times 160 =32$ \\
On calcule ensuite le prix après augmentation :\hspace*{1.5cm} 160 + 32 = 192\\
Le nouveau prix est de 192 euros.\\

\textbf{(b) Le prix d'un DVD est de 17 euros. \textit{Quel est le nouveau prix après 15 \%  de réduction ?}}\\

On calcule d'abord, le montant de la réduction : \hspace*{1cm} $\dfrac{15}{100} \times 17 = 0,15 \times 17 =2,55$ \\
On calcule ensuite le prix après réduction : \hspace*{1.5cm} 17 - 2,55 = 14,45\\
Le nouveau prix est de 14,45 euros.\\



\end{minipage}}
\end{flushleft}


\vspace*{0.5cm}

\exo \\ \textit{Des prix.}\\

\initqa \qa Julie obtient une réduction de 15 $\%$ sur un vélo valant 158 euros . Quel est le montant de la réduction obtenue par Julie ?\\



\qa Patrick a obtenu une réduction de 27 euros sur une console de jeu qui valait 225 euros. Quel pourcentage de réduction a-t-il obtenu ?\\


\qa Paul a obtenu une baisse de 45 euros sur un appareil photo, soit une baisse de 30 $\%$ du prix initial. Quel était le prix initial de l'appareil photo ?\\

\vspace*{0.4cm}

\exo \\ Une montre coûtait 175 euros en 2006. Son prix est augmenté de 3 $\%$
en 2007, puis de 4 $\%$ en 2008.\\
\initq \q Calculer le prix de cette montre en 2007, puis en 2008.\\
\q Calculer le pourcentage d'augmentation sur les deux années.\\

\end{document}
