\documentclass[a4paper,11pt]{article}
\usepackage{amsmath,amsthm,amsfonts,amssymb,amscd,amstext,vmargin,graphics,graphicx,tabularx,multicol} 
\usepackage[francais]{babel}
\usepackage[utf8]{inputenc}  
\usepackage[T1]{fontenc} 
\usepackage{pstricks-add,tikz,tkz-tab,variations}
\usepackage[autolanguage,np]{numprint} 

\setmarginsrb{1.5cm}{0.5cm}{1cm}{0.5cm}{0cm}{0cm}{0cm}{0cm} %Gauche, haut, droite, haut
\newcounter{numexo}
\newcommand{\exo}[1]{\stepcounter{numexo}\noindent{\bf Exercice~\thenumexo} : \marginpar{\hfill /#1}}
\reversemarginpar


\newcounter{enumtabi}
\newcounter{enumtaba}
\newcommand{\q}{\stepcounter{enumtabi} \theenumtabi.  }
\newcommand{\qa}{\stepcounter{enumtaba} (\alph{enumtaba}) }
\newcommand{\initq}{\setcounter{enumtabi}{0}}
\newcommand{\initqa}{\setcounter{enumtaba}{0}}

\newcommand{\be}{\begin{enumerate}}
\newcommand{\ee}{\end{enumerate}}
\newcommand{\bi}{\begin{itemize}}
\newcommand{\ei}{\end{itemize}}
\newcommand{\bp}{\begin{pspicture*}}
\newcommand{\ep}{\end{pspicture*}}
\newcommand{\bt}{\begin{tabular}}
\newcommand{\et}{\end{tabular}}
\renewcommand{\tabularxcolumn}[1]{>{\centering}m{#1}} %(colonne m{} centrée, au lieu de p par défault) 
\newcommand{\tnl}{\tabularnewline}

\newcommand{\trait}{\noindent \rule{\linewidth}{0.2mm}}
\newcommand{\hs}[1]{\hspace{#1}}
\newcommand{\vs}[1]{\vspace{#1}}

\newcommand{\N}{\mathbb{N}}
\newcommand{\Z}{\mathbb{Z}}
\newcommand{\R}{\mathbb{R}}
\newcommand{\C}{\mathbb{C}}
\newcommand{\Dcal}{\mathcal{D}}
\newcommand{\Ccal}{\mathcal{C}}
\newcommand{\mc}{\mathcal}

\newcommand{\vect}[1]{\overrightarrow{#1}}
\newcommand{\ds}{\displaystyle}
\newcommand{\eq}{\quad \Leftrightarrow \quad}
\newcommand{\vecti}{\vec{\imath}}
\newcommand{\vectj}{\vec{\jmath}}
\newcommand{\Oij}{(O;\vec{\imath}, \vec{\jmath})}
\newcommand{\OIJ}{(O;I,J)}


\newcommand{\bmul}[1]{\begin{multicols}{#1}}
\newcommand{\emul}{\end{multicols}}

\newcommand{\reponse}[1][1]{%
\multido{}{#1}{\makebox[\linewidth]{\rule[0pt]{0pt}{20pt}\dotfill}
}}

\newcommand{\titre}[5] 
% #1: titre #2: haut gauche #3: bas gauche #4: haut droite #5: bas droite
{
\noindent #2 \hfill #4 \\
#3 \hfill #5

\vspace{-1.6cm}

\begin{center}\rule{6cm}{0.5mm}\end{center}
\vspace{0.2cm}
\begin{center}{\large{\textbf{#1}}}\end{center}
\begin{center}\rule{6cm}{0.5mm}\end{center}
}



\begin{document}
\pagestyle{empty}
\titre{Interrogation : Calcul littéral }{Nom :}{Prénom :}{Classe}{Date}

\exo{4,5} Réduire si possible les expressions suivantes :\\

\bmul{3}

$A = 5x + 3x =$\\

$E = -7x + 4x =$\\

$G = 3y(- 7x) =$


\columnbreak

$D = 5y + 3y^{2} =$\\

$M = 5z \times 3z =$\\

$T = - 5x^{2} + 3 + 8x^{2} - 9 = $

\columnbreak

$I = - 2x^{2} + 6x^{2} =$\\

$C = -8x \times 7=$\\

$L = 2t^{2} \times (- 7) =$

\emul

\vspace*{0.5cm}

\exo{3}Supprimer les parenthèses en utilisant les propriétés vues en cours et réduire les expressions suivantes :\\


\bmul{2}


$S = 3(2x^{2}) - 7(- 4x) + 4(- 2x^{2}) + 5(- 2x)$\\
\reponse[2]\\




\columnbreak

$L = (3y^{2} - 4y + 5) - (- 4y^{2} + 2y - 3)$\\
\reponse[2]\\


\emul

	
$K = - 3e -9d^{2}+ 5 + (- 6e -3) + 2 \times (-3d) \times d$\\
\reponse[2]\\



\exo{2,5} Développer et réduire chacune des expressions suivantes:\\


\bmul{2}

$B= 3( -2y + 9)$\\
\reponse[2]\\

$F= -8x(x -11)$\\
\reponse[2]\\

\columnbreak

$R = 11a(-5c+8b)$\\
\reponse[2]\\

$O= -3(6z-7) -(3z + 11 -9z) $\\
\reponse[2]\\

\emul





\exo{} Bonus

Wendy a cinq ans de plus que Marion. Samia a le double de l'âge de Wendy. On note $x$ l'âge de Marion.\\

- Exprimer en fonction de $x$ :\\

\noindent \qa l'âge de Wendy : \\
\qa l'âge de Samia :\\
\qa la somme des âges de Marion, Wendy et Samia :\\


- Marion a 14 ans.
Calculer la somme des âges de Marion, Wendy et Samia.\\
\reponse[3]\\


\end{document}
