\documentclass[a4paper,11pt]{article}
\usepackage{amsmath,amsthm,amsfonts,amssymb,amscd,amstext,vmargin,graphics,graphicx,tabularx,multicol} 
\usepackage[francais]{babel}
\usepackage[utf8]{inputenc}  
\usepackage[T1]{fontenc} 
\usepackage{pstricks-add,tikz,tkz-tab,variations}
\usepackage[autolanguage,np]{numprint} 

\setmarginsrb{1.5cm}{0.5cm}{1cm}{0.5cm}{0cm}{0cm}{0cm}{0cm} %Gauche, haut, droite, haut
\newcounter{numexo}
\newcommand{\exo}[1]{\stepcounter{numexo}\noindent{\bf Exercice~\thenumexo} : \marginpar{\hfill /#1}}
\reversemarginpar


\newcounter{enumtabi}
\newcounter{enumtaba}
\newcommand{\q}{\stepcounter{enumtabi} \theenumtabi.  }
\newcommand{\qa}{\stepcounter{enumtaba} (\alph{enumtaba}) }
\newcommand{\initq}{\setcounter{enumtabi}{0}}
\newcommand{\initqa}{\setcounter{enumtaba}{0}}

\newcommand{\be}{\begin{enumerate}}
\newcommand{\ee}{\end{enumerate}}
\newcommand{\bi}{\begin{itemize}}
\newcommand{\ei}{\end{itemize}}
\newcommand{\bp}{\begin{pspicture*}}
\newcommand{\ep}{\end{pspicture*}}
\newcommand{\bt}{\begin{tabular}}
\newcommand{\et}{\end{tabular}}
\renewcommand{\tabularxcolumn}[1]{>{\centering}m{#1}} %(colonne m{} centrée, au lieu de p par défault) 
\newcommand{\tnl}{\tabularnewline}

\newcommand{\trait}{\noindent \rule{\linewidth}{0.2mm}}
\newcommand{\hs}[1]{\hspace{#1}}
\newcommand{\vs}[1]{\vspace{#1}}

\newcommand{\N}{\mathbb{N}}
\newcommand{\Z}{\mathbb{Z}}
\newcommand{\R}{\mathbb{R}}
\newcommand{\C}{\mathbb{C}}
\newcommand{\Dcal}{\mathcal{D}}
\newcommand{\Ccal}{\mathcal{C}}
\newcommand{\mc}{\mathcal}

\newcommand{\vect}[1]{\overrightarrow{#1}}
\newcommand{\ds}{\displaystyle}
\newcommand{\eq}{\quad \Leftrightarrow \quad}
\newcommand{\vecti}{\vec{\imath}}
\newcommand{\vectj}{\vec{\jmath}}
\newcommand{\Oij}{(O;\vec{\imath}, \vec{\jmath})}
\newcommand{\OIJ}{(O;I,J)}


\newcommand{\bmul}[1]{\begin{multicols}{#1}}
\newcommand{\emul}{\end{multicols}}

\newcommand{\reponse}[1][1]{%
\multido{}{#1}{\makebox[\linewidth]{\rule[0pt]{0pt}{20pt}\dotfill}
}}

\newcommand{\titre}[5] 
% #1: titre #2: haut gauche #3: bas gauche #4: haut droite #5: bas droite
{
\noindent #2 \hfill #4 \\
#3 \hfill #5

\vspace{-1.6cm}

\begin{center}\rule{6cm}{0.5mm}\end{center}
\vspace{0.2cm}
\begin{center}{\large{\textbf{#1}}}\end{center}
\begin{center}\rule{6cm}{0.5mm}\end{center}
}



\begin{document}
\pagestyle{empty}
\titre{Interrogation : Pourcentages}{Nom :}{Prénom :}{Classe}{Date}



\exo{3}

Dans une ville, un musée a enregistré 25 425 entrées payantes.\\

\qa 16 $\%$ des visiteurs étaient des touristes étrangers. Combien de touristes étrangers ont visité ce musée ?\\
\reponse[3]\\

\qa 80 $\%$ des  touristes étrangers étaient américains. Combien de touristes américains ont visité ce musée ?\\
\reponse[3]\\

\qa 8 136 visiteurs ont bénéficié d'un tarif réduit. Quel pourcentage du total des entrées payantes cela représente-t-il ?\\
\reponse[3]\\

\exo{3}  Dans le jardin de Céleste, il y a deux volière.\\
Dans l'une, il y a 50 oiseaux dont 24 $\%$ sont des hirondelles.\\
Dans l'autre, il y a 60 oiseaux dont 35 $\%$  sont des hirondelles.\\
Céleste ouvre la porte des 2 volières et tous les oiseaux s'envolent.\\

$\rightarrow $ Quel est le pourcentage d'hirondelles parmi les oiseaux qui se sont envolés?\\
\reponse[10]\\

\newpage
\vspace*{0.5cm}
\exo{4} 

\initqa \qa Un scooter coûte 950 euros. Son prix augmente de 5 $\%$.\\
Quel est le nouveau prix ?\\
\reponse[5]\\

\qa Un pantalon était vendu 80 euros.
Le commerçant fait une remise de 15 $\%$.\\
Quel est le nouveau prix du pantalon après la remise ?\\
\reponse[5]\\


\qa Gilles profite d'une promotion pour un voyage en Egypte : 650 euros au lieu de 800 euros.\\
Quel est le pourcentage de réduction dont il a bénéficié ?\\
\reponse[7]\\





\end{document}
