\documentclass[a4paper,11pt]{article}
\usepackage{amsmath,amsthm,amsfonts,amssymb,amscd,amstext,vmargin,graphics,graphicx,tabularx,multicol} 
\usepackage[francais]{babel}
\usepackage[utf8]{inputenc}  
\usepackage[T1]{fontenc} 
\usepackage{pstricks-add,tikz,tkz-tab,variations}
\usepackage[autolanguage,np]{numprint} 
\usepackage{calc}

\setmarginsrb{1.5cm}{0.5cm}{1cm}{0.5cm}{0cm}{0cm}{0cm}{0cm} %Gauche, haut, droite, haut
\newcounter{numexo}
\newcommand{\exo}[1]{\stepcounter{numexo}\noindent{\bf Exercice~\thenumexo} : }
\reversemarginpar

\newcommand{\bmul}[1]{\begin{multicols}{#1}}
\newcommand{\emul}{\end{multicols}}

\newcounter{enumtabi}
\newcounter{enumtaba}
\newcommand{\q}{\stepcounter{enumtabi} \theenumtabi.  }
\newcommand{\qa}{\stepcounter{enumtaba} (\alph{enumtaba}) }
\newcommand{\initq}{\setcounter{enumtabi}{0}}
\newcommand{\initqa}{\setcounter{enumtaba}{0}}

\newcommand{\be}{\begin{enumerate}}
\newcommand{\ee}{\end{enumerate}}
\newcommand{\bi}{\begin{itemize}}
\newcommand{\ei}{\end{itemize}}
\newcommand{\bp}{\begin{pspicture*}}
\newcommand{\ep}{\end{pspicture*}}
\newcommand{\bt}{\begin{tabular}}
\newcommand{\et}{\end{tabular}}
\renewcommand{\tabularxcolumn}[1]{>{\centering}m{#1}} %(colonne m{} centrée, au lieu de p par défault) 
\newcommand{\tnl}{\tabularnewline}

\newcommand{\trait}{\noindent \rule{\linewidth}{0.2mm}}
\newcommand{\hs}[1]{\hspace{#1}}
\newcommand{\vs}[1]{\vspace{#1}}

\newcommand{\N}{\mathbb{N}}
\newcommand{\Z}{\mathbb{Z}}
\newcommand{\R}{\mathbb{R}}
\newcommand{\C}{\mathbb{C}}
\newcommand{\Dcal}{\mathcal{D}}
\newcommand{\Ccal}{\mathcal{C}}
\newcommand{\mc}{\mathcal}

\newcommand{\vect}[1]{\overrightarrow{#1}}
\newcommand{\ds}{\displaystyle}
\newcommand{\eq}{\quad \Leftrightarrow \quad}
\newcommand{\vecti}{\vec{\imath}}
\newcommand{\vectj}{\vec{\jmath}}
\newcommand{\Oij}{(O;\vec{\imath}, \vec{\jmath})}
\newcommand{\OIJ}{(O;I,J)}


\newcommand{\reponse}[1][1]{%
\multido{}{#1}{\makebox[\linewidth]{\rule[0pt]{0pt}{20pt}\dotfill}
}}

\newcommand{\titre}[5] 
% #1: titre #2: haut gauche #3: bas gauche #4: haut droite #5: bas droite
{
\noindent #2 \hfill #4 \\
#3 \hfill #5

\vspace{-1.6cm}

\begin{center}\rule{6cm}{0.5mm}\end{center}
\vspace{0.2cm}
\begin{center}{\large{\textbf{#1}}}\end{center}
\begin{center}\rule{6cm}{0.5mm}\end{center}
}



\begin{document}
\pagestyle{empty}
\titre{Exercices sur les nombres entiers }{}{}{4ème}{}

\vspace*{0.2cm}


\bmul{2}

\exo \\ Recopier et compléter avec les mots : \textit{chiffre}, \textit{nombre}.

\begin{flushleft}
\initqa \qa Dans une équipe de rugby, le . . . de joueur est 15.\\

\qa 7 est  le . . .des unités du . . . 57.\\

\qa Le . . . 148 est écrit avec trois . . .\\
\end{flushleft}

\vspace*{0.5cm}

\exo \\ Recopier et compléter avec les mots : \textit{unités}, \textit{dizaines}, \textit{centaines}, \textit{milliers}, \textit{...}\\

\initqa \noindent \qa 504 = 5 . . . 4 . . . \\
 \qa 2 320 = 2 . . . 3 . . . 2 . . .\\
\qa 2 005 087 = 2 . . . 5 . . . 8 . . . 7 . . . \\
\qa 3 009 000 205 = 3 . . . 9 . . . 2 . . . 5 . . .\\

\vspace*{0.5cm}


\exo \\ Recopier et compléter.
\begin{flushleft}
\initqa \qa (4 $\times$ 10 000) + (5 $\times$ 1 000) + (7 $\times$ 10) + 1 = ...\\

\qa (3 $\times$ 1 000 000) + (7 $\times$ 100 000) + (5 $\times$ 100)  = ...\\

\qa (2 $\times$ 100 000) + (3 $\times$ 1 000) + (1 $\times$ 100) + 7 = ...\\

\qa (9 $\times$ 10 000 000 000) + (7 $\times$ 1 000 000) + (2 $\times$ 10 000) + (6 $\times$ 100) = ...\\

\end{flushleft}



\columnbreak


\exo \\ Écrire chaque nombre de façon à le lire facilement et supprimer les zéros inutiles.

\initqa \qa 2 073 \hspace*{0.5cm} \qa 01234560  \hspace*{0.5cm} \qa 0056502430700\\

\vspace*{0.5cm}

\exo \\  Recopier et compléter par les nombres qui conviennent.\\

\initqa \noindent \qa 1 milliard = . . . millions \\
 \qa 1 milliard = . . . milliers\\
\qa 1 million = . . . centaines \\
 \qa 1 millier = . . . dizaines\\
 
 \vspace*{0.5cm}

\exo \\ Décomposer les nombres suivants comme sur l'exemple.\\
\textbf{Exemple :} 356 = (3  $\times$ 100) + (5 $\times$ 10) + 6\\

\initqa \qa 9 527 \hspace*{0.25cm} \qa 14 025 \hspace*{0.25cm} \qa 1 250 730
 
 
\emul

\vspace*{0.5cm}

\exo \\ Recopier et compléter.\\

\noindent \initqa \qa 1 235 = (1  $\times$ . . .) + (2  $\times$ . . .) + (3 $\times$ . . .) + 5\\
\qa 24 103 = (2 $\times$ . . . ) + (4  $\times$ . . .) + (1 $\times$ . . .) + (3 $\times$ . . .)\\
 \qa 3 758 000 = (3 $\times$ . . . ) + (7  $\times$ . . .) + (5 $\times$ . . .) + (8 $\times$ . . .)\\
 \qa 20 507 050 = (2 $\times$ . . . ) + (5  $\times$ . . .) + (7 $\times$ . . .) + (5 $\times$ . . .)\\
 
 \vspace*{0.5cm}
 
 
  
 
  
  
  \bmul{2}
  
  \exo \\ Recopier et compléter.
\begin{flushleft}
 \initqa \qa Dans 275, le chiffre des dizaines est ...\\
\qa Dans 8 345, le chiffre des ... est 8.\\
\qa Dans 135 437, le chiffre 3 est celui des ... et des ...\\
\end{flushleft}


  \vspace*{0.5cm}
  
   \exo \\ 70 532 spectateurs ont assisté à un concert à Bercy.\\
  Pour ce nombre, quel est le chiffre : \\
  \initqa \qa des dizaines ? \hspace*{0.25cm} \qa des milliers ? \\
   \qa des centaines ? \hspace*{0.25cm} \qa des dizaines de mille ?\\
  
\exo \\    Dans le nombre 7 156 940 238 :\\
\initqa \qa quel est le chiffre des unités de milliards ? \\
\qa quel est le nombre d'unités de millions ? \\
\qa quel est le chiffre des dizaines de mille? \\
\qa quel est le nombre de dizaines de mille? \\
 
 \columnbreak
 
 \exo \\ Dans le nombre 98 651 :\\
\initqa \qa quel est le nombre d'unités ?\\
\qa quel est le nombre de dizaines ?\\
\qa quel est le nombre de centaines ?\\
\qa quel est le nombre de milliers ?\\

 \exo \\ Dans chaque cas, écrire le nombre en chiffres.
\begin{flushleft}
  \initqa \qa 25 dizaines\\
 \qa 370 dizaines\\
 \qa 325 centaines\\
 \qa 21 dizaines de mile et 87 unités\\
 \qa 69 unités de millions et 146 dizaines\\
\end{flushleft} 

\vspace*{0.25cm}
 
 
  \exo \\ Écrire tous les nombres entiers possibles de trois chiffres en utilisant une seule fois chacun des chiffres 3 ; 4 et 0.\\
 

 

 
 \emul
 
 
 
 
 

 


\end{document}
