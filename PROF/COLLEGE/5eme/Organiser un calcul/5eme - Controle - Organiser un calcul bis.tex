\documentclass[a4paper,11pt]{article}
\usepackage{amsmath,amsthm,amsfonts,amssymb,amscd,amstext,vmargin,graphics,graphicx,tabularx,multicol} 
\usepackage[francais]{babel}
\usepackage[utf8]{inputenc}  
\usepackage[T1]{fontenc} 
\usepackage{pstricks-add,tikz,tkz-tab,variations}
\usepackage[autolanguage,np]{numprint} 

\setmarginsrb{1.5cm}{0.5cm}{1cm}{0.5cm}{0cm}{0cm}{0cm}{0cm} %Gauche, haut, droite, haut
\newcounter{numexo}
\newcommand{\exo}[1]{\stepcounter{numexo}\noindent{\bf Exercice~\thenumexo} : \marginpar{\hfill /#1}}
\reversemarginpar


\newcounter{enumtabi}
\newcounter{enumtaba}
\newcommand{\q}{\stepcounter{enumtabi} \theenumtabi.  }
\newcommand{\qa}{\stepcounter{enumtaba} (\alph{enumtaba}) }
\newcommand{\initq}{\setcounter{enumtabi}{0}}
\newcommand{\initqa}{\setcounter{enumtaba}{0}}

\newcommand{\be}{\begin{enumerate}}
\newcommand{\ee}{\end{enumerate}}
\newcommand{\bi}{\begin{itemize}}
\newcommand{\ei}{\end{itemize}}
\newcommand{\bp}{\begin{pspicture*}}
\newcommand{\ep}{\end{pspicture*}}
\newcommand{\bt}{\begin{tabular}}
\newcommand{\et}{\end{tabular}}
\renewcommand{\tabularxcolumn}[1]{>{\centering}m{#1}} %(colonne m{} centrée, au lieu de p par défault) 
\newcommand{\tnl}{\tabularnewline}

\newcommand{\trait}{\noindent \rule{\linewidth}{0.2mm}}
\newcommand{\hs}[1]{\hspace{#1}}
\newcommand{\vs}[1]{\vspace{#1}}

\newcommand{\N}{\mathbb{N}}
\newcommand{\Z}{\mathbb{Z}}
\newcommand{\R}{\mathbb{R}}
\newcommand{\C}{\mathbb{C}}
\newcommand{\Dcal}{\mathcal{D}}
\newcommand{\Ccal}{\mathcal{C}}
\newcommand{\mc}{\mathcal}

\newcommand{\vect}[1]{\overrightarrow{#1}}
\newcommand{\ds}{\displaystyle}
\newcommand{\eq}{\quad \Leftrightarrow \quad}
\newcommand{\vecti}{\vec{\imath}}
\newcommand{\vectj}{\vec{\jmath}}
\newcommand{\Oij}{(O;\vec{\imath}, \vec{\jmath})}
\newcommand{\OIJ}{(O;I,J)}


\newcommand{\bmul}[1]{\begin{multicols}{#1}}
\newcommand{\emul}{\end{multicols}}

\newcommand{\reponse}[1][1]{%
\multido{}{#1}{\makebox[\linewidth]{\rule[0pt]{0pt}{20pt}\dotfill}
}}

\newcommand{\titre}[5] 
% #1: titre #2: haut gauche #3: bas gauche #4: haut droite #5: bas droite
{
\noindent #2 \hfill #4 \\
#3 \hfill #5

\vspace{-1.6cm}

\begin{center}\rule{6cm}{0.5mm}\end{center}
\vspace{0.2cm}
\begin{center}{\large{\textbf{#1}}}\end{center}
\begin{center}\rule{6cm}{0.5mm}\end{center}
}



\begin{document}
\pagestyle{empty}
\titre{Contrôle - Organiser un calcul}{Nom :}{Prénom :}{Classe}{Date}



\exo{4} Questions de cours\\

\q \textbf{Traduire} les phrases suivantes par une expression numérique puis les \textbf{calculer}\\

\qa	La somme de 8 et 31. \\

\qa	La différence de 63 et du produit de 7 par 5. \\

\q Traduire l'expression numérique suivante par \textbf{une phrase} :  $L = (38 - 5 )\div 7$ \\


\vspace*{0.6cm}

\exo{3}

Chacune des expressions suivantes est fausse.\\
\textbf{Recopier} les expressions et placer, dans chaque cas, des parenthèses aux bons endroits pour rendre l'égalité vraie.\\

\initqa
\qa 2 $\times$ 5 + 2 = 14\\

\qa	1 + 3 + 2 $\times$ 6 = 31\\

\qa	1 + 2 $\times$ 5 + 3 $\times$ 10 - 4 = 33

\vspace*{0.6cm}

\exo{8}\\

Calculer les expressions suivantes en respectant les priorités (on détaillera toutes les étapes de calculs) :\\


\bmul{3}
 $D= 34-15+8-9$\\

 $M= 28 - 5 \times 2$\\

\columnbreak

 $G= 72 \div 9 \times 3$\\

 $V= (24-2-1) \div [4 \times (20 + 5)]$\\

\columnbreak

 $L = 57 + 42 \div 6$\\

 $S= 3 \times [18 -(4-1)\times2]$\\
\emul

\vspace*{0.6cm}


\exo{2.5} Camille a dépensé dans un magasin 77 euros. Elle a acheté cinq livres de poche de même prix et un CD à 17 euros. \\

\initq
\q	Écrire \textbf{une} expression numérique qui permet de calculer le prix d'un livre de poche.\\


\q	Calculer le prix d'un livre de poche \textbf{en indiquant les étapes de calcul}.\\




\vspace*{0.6cm}

\exo{2.5} Pour le tournoi de handball du collège, les professeurs d'EPS ont réparti les 96 élèves de $5^{eme}$ en équipes de 12. Pour l'échauffement, 24 ballons sont distribués équitablement entre les équipes.\\

\initq
\q Écrire \textbf{une} expression qui permet de calculer le nombre de ballons distribués par équipe.\\

\q Effectuer les calculs.
\vspace*{0.6cm}

\exo{} BONUS\\

Calculer : \hspace*{1cm} $O = 72,5 +(22,5-3) \times 3-[2 \times (17 \div 2 - 8)+1]$

\end{document}
