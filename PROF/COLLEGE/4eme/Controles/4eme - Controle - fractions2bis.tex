\documentclass[a4paper,11pt]{article}
\usepackage{amsmath,amsthm,amsfonts,amssymb,amscd,amstext,vmargin,graphics,graphicx,tabularx,multicol} 
\usepackage[francais]{babel}
\usepackage[utf8]{inputenc}  
\usepackage[T1]{fontenc} 
\usepackage{pstricks-add,tikz,tkz-tab,variations}
\usepackage[autolanguage,np]{numprint} 

\setmarginsrb{1.5cm}{0.5cm}{1cm}{0.5cm}{0cm}{0cm}{0cm}{0cm} %Gauche, haut, droite, haut
\newcounter{numexo}
\newcommand{\exo}[1]{\stepcounter{numexo}\noindent{\bf Exercice~\thenumexo} : \marginpar{\hfill /#1}}
\reversemarginpar


\newcounter{enumtabi}
\newcounter{enumtaba}
\newcommand{\q}{\stepcounter{enumtabi} \theenumtabi.  }
\newcommand{\qa}{\stepcounter{enumtaba} (\alph{enumtaba}) }
\newcommand{\initq}{\setcounter{enumtabi}{0}}
\newcommand{\initqa}{\setcounter{enumtaba}{0}}

\newcommand{\be}{\begin{enumerate}}
\newcommand{\ee}{\end{enumerate}}
\newcommand{\bi}{\begin{itemize}}
\newcommand{\ei}{\end{itemize}}
\newcommand{\bp}{\begin{pspicture*}}
\newcommand{\ep}{\end{pspicture*}}
\newcommand{\bt}{\begin{tabular}}
\newcommand{\et}{\end{tabular}}
\renewcommand{\tabularxcolumn}[1]{>{\centering}m{#1}} %(colonne m{} centrée, au lieu de p par défault) 
\newcommand{\tnl}{\tabularnewline}

\newcommand{\bmul}[1]{\begin{multicols}{#1}}
\newcommand{\emul}{\end{multicols}}

\newcommand{\trait}{\noindent \rule{\linewidth}{0.2mm}}
\newcommand{\hs}[1]{\hspace{#1}}
\newcommand{\vs}[1]{\vspace{#1}}

\newcommand{\N}{\mathbb{N}}
\newcommand{\Z}{\mathbb{Z}}
\newcommand{\R}{\mathbb{R}}
\newcommand{\C}{\mathbb{C}}
\newcommand{\Dcal}{\mathcal{D}}
\newcommand{\Ccal}{\mathcal{C}}
\newcommand{\mc}{\mathcal}

\newcommand{\vect}[1]{\overrightarrow{#1}}
\newcommand{\ds}{\displaystyle}
\newcommand{\eq}{\quad \Leftrightarrow \quad}
\newcommand{\vecti}{\vec{\imath}}
\newcommand{\vectj}{\vec{\jmath}}
\newcommand{\Oij}{(O;\vec{\imath}, \vec{\jmath})}
\newcommand{\OIJ}{(O;I,J)}


\newcommand{\reponse}[1][1]{%
\multido{}{#1}{\makebox[\linewidth]{\rule[0pt]{0pt}{20pt}\dotfill}
}}

\newcommand{\titre}[5] 
% #1: titre #2: haut gauche #3: bas gauche #4: haut droite #5: bas droite
{
\noindent #2 \hfill #4 \\
#3 \hfill #5

\vspace{-1.6cm}

\begin{center}\rule{6cm}{0.5mm}\end{center}
\vspace{0.2cm}
\begin{center}{\large{\textbf{#1}}}\end{center}
\begin{center}\rule{6cm}{0.5mm}\end{center}
}



\begin{document}
\pagestyle{empty}
\titre{Contrôle n\degre 3}{Nom :}{Prénom :}{Classe}{Date}




\exo{10} Calculer les expressions suivantes et donner la réponse \textbf{sous forme d'une fraction irréductible.
}
\bmul{3}

$B = \dfrac{-18}{7} - \dfrac{11}{7}+\dfrac{4}{7}$\\


$J = \dfrac{-3}{15} \times \dfrac{12}{-9} \times \dfrac{5}{6}$\\

\columnbreak

$K = \dfrac{3}{5} - \dfrac{5}{3}+\dfrac{1}{15}$\\



$E =\dfrac{1}{2}-\dfrac{6}{10} \times \dfrac{3}{2} - \dfrac{1}{7}$ \\


\columnbreak



$ U = \dfrac{2}{3} \div \dfrac{5}{18}$\\


$S = \dfrac{\dfrac{2}{5}-\dfrac{5}{2}}{\dfrac{5}{8}-1}$ \\
\emul

\vspace*{0.75cm}



\exo{4}
Eric doit parcourir 160 km en quatre jours.\\

Le premier jour, il parcourt $\dfrac{1}{4}$ du trajet. Le deuxième jour, il parcourt $\dfrac{2}{5}$ du trajet.\\

Le troisième jour, il parcourt $\dfrac{1}{10}$ du trajet. Le quatrième jour, il parcourt ce qu'il reste.\\

\textbf{
\q Quelle est la fraction de la distance parcourue le dernier jour?}\\

\textbf{\q Combien de kilomètres lui reste-t-il a parcourir le dernier jour ?\\}

\vspace*{0.75cm}

\exo{3}
Kenzo possède 84 cartes Yu-Gi-Ho. Il a vendu les $\dfrac{4}{7}$ de sa collection de cartes. \\
\textbf{Combien de cartes lui reste-t-il maintenant ?}\\

\vspace*{0.75cm}

\exo{3}
Un routier fait le plein avant de partir pour l'Espagne.\\
Il utilise un quart de son réservoir pour le premier déplacement, puis les deux tiers du reste pour le deuxième déplacement.\\

\initq \textbf{\q Quelle fraction du plein reste-t-il après le premier déplacement ?}\\

\textbf{\q Quelle fraction du plein utilise-t-il pour le deuxième déplacement ?}\\

\vspace*{0.75cm}

\exo{} BONUS

\noindent Quelle est le résultat de la somme de 3 et de l'inverse de la somme de 3 et de l'inverse de la somme de 3 et 3 ? (Expliquez votre résultat)\\


\end{document}
