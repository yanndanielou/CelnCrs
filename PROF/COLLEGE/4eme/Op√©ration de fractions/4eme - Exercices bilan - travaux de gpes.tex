\documentclass[a4paper,11pt]{article}
\usepackage{amsmath,amsthm,amsfonts,amssymb,amscd,amstext,vmargin,graphics,graphicx,tabularx,multicol} 
\usepackage[francais]{babel}
\usepackage[utf8]{inputenc}  
\usepackage[T1]{fontenc} 
\usepackage{pstricks-add,tikz,tkz-tab,variations}
\usepackage[autolanguage,np]{numprint} 

\setmarginsrb{1.5cm}{0.5cm}{1cm}{0.5cm}{0cm}{0cm}{0cm}{0cm} %Gauche, haut, droite, haut
\newcounter{numexo}
\newcommand{\exo}[1]{\stepcounter{numexo}\noindent{\bf Exercice~\thenumexo} : \marginpar{\hfill /#1}}
\reversemarginpar


\newcounter{enumtabi}
\newcounter{enumtaba}
\newcommand{\q}{\stepcounter{enumtabi} \theenumtabi.  }
\newcommand{\qa}{\stepcounter{enumtaba} (\alph{enumtaba}) }
\newcommand{\initq}{\setcounter{enumtabi}{0}}
\newcommand{\initqa}{\setcounter{enumtaba}{0}}

\newcommand{\be}{\begin{enumerate}}
\newcommand{\ee}{\end{enumerate}}
\newcommand{\bi}{\begin{itemize}}
\newcommand{\ei}{\end{itemize}}
\newcommand{\bp}{\begin{pspicture*}}
\newcommand{\ep}{\end{pspicture*}}
\newcommand{\bt}{\begin{tabular}}
\newcommand{\et}{\end{tabular}}
\renewcommand{\tabularxcolumn}[1]{>{\centering}m{#1}} %(colonne m{} centrée, au lieu de p par défault) 
\newcommand{\tnl}{\tabularnewline}

\newcommand{\bmul}[1]{\begin{multicols}{#1}}
\newcommand{\emul}{\end{multicols}}

\newcommand{\trait}{\noindent \rule{\linewidth}{0.2mm}}
\newcommand{\hs}[1]{\hspace{#1}}
\newcommand{\vs}[1]{\vspace{#1}}

\newcommand{\N}{\mathbb{N}}
\newcommand{\Z}{\mathbb{Z}}
\newcommand{\R}{\mathbb{R}}
\newcommand{\C}{\mathbb{C}}
\newcommand{\Dcal}{\mathcal{D}}
\newcommand{\Ccal}{\mathcal{C}}
\newcommand{\mc}{\mathcal}

\newcommand{\vect}[1]{\overrightarrow{#1}}
\newcommand{\ds}{\displaystyle}
\newcommand{\eq}{\quad \Leftrightarrow \quad}
\newcommand{\vecti}{\vec{\imath}}
\newcommand{\vectj}{\vec{\jmath}}
\newcommand{\Oij}{(O;\vec{\imath}, \vec{\jmath})}
\newcommand{\OIJ}{(O;I,J)}


\newcommand{\reponse}[1][1]{%
\multido{}{#1}{\makebox[\linewidth]{\rule[0pt]{0pt}{20pt}\dotfill}
}}

\newcommand{\titre}[5] 
% #1: titre #2: haut gauche #3: bas gauche #4: haut droite #5: bas droite
{
\noindent #2 \hfill #4 \\
#3 \hfill #5

\vspace{-1.6cm}

\begin{center}\rule{6cm}{0.5mm}\end{center}
\vspace{0.2cm}
\begin{center}{\large{\textbf{#1}}}\end{center}
\begin{center}\rule{6cm}{0.5mm}\end{center}
}



\begin{document}
\pagestyle{empty}
\titre{Exercices - Opérations de fractions }{Nom :}{Prénom :}{Classe}{Date}




\exo{4,5} Calculer les expressions suivantes et donner la réponse sous forme d'une \textbf{fraction irréductible}.

\bmul{3}

$N = \dfrac{2}{3} - \dfrac{4}{5}$\\



\columnbreak


$ \dfrac{-15}{7} \times \dfrac{10}{6} \times \dfrac{21}{-2}$



\columnbreak


$ U = \dfrac{\dfrac{2}{3}}{\dfrac{5}{18}}$\\




\emul

\vspace*{0.75cm}


\exo{3} Calculer l'expression suivante et donner la réponse sous forme d'une fraction irréductible.\\



$E =\dfrac{19}{18} - \dfrac{4}{6} \times \dfrac{5}{2}$ \\

\vspace*{1.5cm}


\exo{4} Maya  prépare un cocktail  pour son anniversaire : dans une carafe ayant une contenance d'un litre, elle verse $\dfrac{1}{3}$ L de jus d'orange, $\dfrac{1}{4}$ L de jus de mangue et pour finir décide d'ajouter encore $\dfrac{5}{12}$ L  de jus d'orange quand son copain Eric lui crie : "Stop, ça va déborder !!!"\\

A-t-il raison ? \textit{(Justifier votre réponse par des calculs)}.\\

\vspace*{1.5cm}

\exo{3}  Coraline, à qui tout cela a donné soif, boit alors les $\dfrac{2}{5}$ d'une bouteille de jus de fruits de 50 cL. Quelle quantité de jus de fruits a-t-elle bue? \\

\vspace*{1.5cm}

\exo{5} Avant le début de l'hiver, un écureuil a constitué une provision de noisette.\\
Il en a mangé les $\dfrac{2}{5}$ pendant le 1er mois d'hiver, puis un quart du reste le 2ème mois.\\

\q Quelle fraction de sa réserve de départ lui reste-t-il à la fin du 1er mois?\\

\q Quelle fraction de sa réserve de départ lui reste-t-il pour finir l'hiver ?\\

\q Le troisième mois, il mange les $\dfrac{4}{5}$ de ce qu'il lui reste. Lui reste-t-il des noisettes à la fin de l'hiver ?\\


\textit{Pour cette exercice, vous pouvez venir montrer vos réponses après chaque question !}\\

\vspace*{1.5cm}

\exo{3} Au goûter, Elise mange $\dfrac{1}{4}$ du paquet de gâteaux qu'elle vient d'ouvrir. De retour du collège, sa soeur Agathe mange les $\dfrac{2}{3}$ des gâteaux restants dans le paquet entamé par Elise.\\
Il reste alors 5 gâteaux.\\

Quel était le nombre initial de gâteaux dans le paquet ?\\


\end{document}
