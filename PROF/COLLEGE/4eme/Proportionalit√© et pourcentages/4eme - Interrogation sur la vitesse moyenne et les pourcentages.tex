\documentclass[a4paper,11pt]{article}
\usepackage{amsmath,amsthm,amsfonts,amssymb,amscd,amstext,vmargin,graphics,graphicx,tabularx,multicol} 
\usepackage[francais]{babel}
\usepackage[utf8]{inputenc}  
\usepackage[T1]{fontenc} 
\usepackage{pstricks-add,tikz,tkz-tab,variations}
\usepackage[autolanguage,np]{numprint} 

\setmarginsrb{1.5cm}{0.5cm}{1cm}{0.5cm}{0cm}{0cm}{0cm}{0cm} %Gauche, haut, droite, haut
\newcounter{numexo}
\newcommand{\exo}[1]{\stepcounter{numexo}\noindent{\bf Exercice~\thenumexo} : \marginpar{\hfill /#1}}
\reversemarginpar


\newcounter{enumtabi}
\newcounter{enumtaba}
\newcommand{\q}{\stepcounter{enumtabi} \theenumtabi.  }
\newcommand{\qa}{\stepcounter{enumtaba} (\alph{enumtaba}) }
\newcommand{\initq}{\setcounter{enumtabi}{0}}
\newcommand{\initqa}{\setcounter{enumtaba}{0}}

\newcommand{\be}{\begin{enumerate}}
\newcommand{\ee}{\end{enumerate}}
\newcommand{\bi}{\begin{itemize}}
\newcommand{\ei}{\end{itemize}}
\newcommand{\bp}{\begin{pspicture*}}
\newcommand{\ep}{\end{pspicture*}}
\newcommand{\bt}{\begin{tabular}}
\newcommand{\et}{\end{tabular}}
\renewcommand{\tabularxcolumn}[1]{>{\centering}m{#1}} %(colonne m{} centrée, au lieu de p par défault) 
\newcommand{\tnl}{\tabularnewline}

\newcommand{\trait}{\noindent \rule{\linewidth}{0.2mm}}
\newcommand{\hs}[1]{\hspace{#1}}
\newcommand{\vs}[1]{\vspace{#1}}

\newcommand{\N}{\mathbb{N}}
\newcommand{\Z}{\mathbb{Z}}
\newcommand{\R}{\mathbb{R}}
\newcommand{\C}{\mathbb{C}}
\newcommand{\Dcal}{\mathcal{D}}
\newcommand{\Ccal}{\mathcal{C}}
\newcommand{\mc}{\mathcal}

\newcommand{\vect}[1]{\overrightarrow{#1}}
\newcommand{\ds}{\displaystyle}
\newcommand{\eq}{\quad \Leftrightarrow \quad}
\newcommand{\vecti}{\vec{\imath}}
\newcommand{\vectj}{\vec{\jmath}}
\newcommand{\Oij}{(O;\vec{\imath}, \vec{\jmath})}
\newcommand{\OIJ}{(O;I,J)}


\newcommand{\bmul}[1]{\begin{multicols}{#1}}
\newcommand{\emul}{\end{multicols}}

\newcommand{\reponse}[1][1]{%
\multido{}{#1}{\makebox[\linewidth]{\rule[0pt]{0pt}{20pt}\dotfill}
}}

\newcommand{\titre}[5] 
% #1: titre #2: haut gauche #3: bas gauche #4: haut droite #5: bas droite
{
\noindent #2 \hfill #4 \\
#3 \hfill #5

\vspace{-1.6cm}

\begin{center}\rule{6cm}{0.5mm}\end{center}
\vspace{0.2cm}
\begin{center}{\large{\textbf{#1}}}\end{center}
\begin{center}\rule{6cm}{0.5mm}\end{center}
}



\begin{document}
\pagestyle{empty}
\titre{Interrogation : Vitesse moyenne et Pourcentages}{Nom :}{Prénom :}{Classe}{Date}



\exo{4} Exercices sur la vitesse moyenne\\

\q Un lévrier russe court à 85 km/h. Une gazelle springbok court à 24,2 m/s. Lequel de ces animaux court le plus vite?\\
\reponse[3]\\

\q \textit{De la Terre à la Lune, Trajet Direct en 97 heures 20 minutes} est un roman d'anticipation de Jules Verne paru en 1865. Dans ce roman, le Gun Club de Baltimore aux États-Unis tente d'envoyer un obus habité par trois hommes sur la Lune ! La distance entre la Terre et la Lune est de 384 500 km. Calculer quelle aurait été la vitesse moyenne de cet obus sur la distance Terre-Lune si le lancement avait eu lieu...\\
\reponse[3]\\



\q La plaque Coco est l'une des douze principales plaques lithosphériques à la surface de la Terre. On sait, grâce au GPS, qu'elle se déplace à la vitesse moyenne de 5 cm par an.
En combien d'années se déplace-t-elle de 18,75 cm ? Exprimer cette durée en année et mois.\\
\reponse[3]\\

\q Si je vois un éclair et si j'entends le tonnerre 8 secondes plus tard, sachant que la vitesse du son est de 340 m/s, puis-je savoir à quelle distance je me trouve de l'éclair ? Calculer cette distance.\\
\reponse[3]\\







\exo{2}\\
\initq 
\q "Dans les années 2 000, plus des trois quart des véhicules neufs immatriculés en France était des diesel. Cependant, depuis deux ans les choses bougent à la baisse. Seulement 6 français sur 10 achèteraient du diesel."\\
\initqa
\qa Traduire le texte avec des pourcentages : \\
"Dans les années 2 000, plus ........................................... des véhicules neufs immatriculés en France était des diesel. Cependant, depuis deux ans les choses bougent à la baisse. Seulement .......................................... des français achèteraient du diesel."\\

\q Calculer des pourcentages :\\

\bmul{2}

\initqa \qa 25 $\%$ de 64 km\\
\reponse[1]\\

\qa 45 $\%$ de 104 euros\\
\reponse[1]

\columnbreak

\qa 121 $\%$ de 210 L\\
\reponse[1]\\

\qa 59 $\%$ de 90 $m^{2}$\\
\reponse[1]

\emul


\exo{1}

Un magasin de vêtements annonce une baisse de 30 $\%$ sur le prix de tous ses articles.\\
Calculer le prix d'un pantalon et d'un chemisier qui coûtaient respectivement 48 euros et 32 euros avant la réduction.\\
\noindent\reponse[3]\\

\exo{1,5}

Monsieur Harère possède deux champs : l'aire du premier est égale à 2,5 ha et l'aire du second à 12 000 $m^{2}$.
78 $\%$ de la surface du premier et 40 $\%$ de la surface du deuxième sont recouverts de blés. ( \textit{1 ha = 10 000 $m^{2}$})

\initq \q Quelle est la surface totale de terre recouverte par les blés ?\\
\reponse[3]\\

\q Quel est le pourcentage de l'exploitation de ce Monsieur correspondant aux terres recouvertes par les blés?\\
\reponse[3]\\



\exo{1,5}

Voici les résultats au Brevet des collèges des deux collèges d'une même ville.
\bi	\item Collège Louis Lumière : 112 élèves se sont présentés et le pourcentage de réussite est 75 $\%$.
\item 	Collège Thomas Edison : 80 élèves se sont présentés et le pourcentage de réussite est 60 $\%$.
\ei
Quel est le pourcentage de réussite au brevet des élèves de cette ville ?\\


\noindent\reponse[3]\\










\end{document}
