\documentclass[a4paper,11pt]{article}
\usepackage{amsmath,amsthm,amsfonts,amssymb,amscd,amstext,vmargin,graphics,graphicx,tabularx,multicol} \usepackage[french]{babel}
\usepackage[utf8]{inputenc}  
\usepackage[T1]{fontenc} 
\usepackage[T1]{fontenc}
\usepackage{amsmath,amssymb}
\usepackage{pstricks-add,tikz,tkz-tab,variations}
\usepackage[autolanguage,np]{numprint} 
\usepackage{color}
\usepackage{ulem}

\setmarginsrb{1.5cm}{0.5cm}{1cm}{0.5cm}{0cm}{0cm}{0cm}{0cm} %Gauche, haut, droite, haut
\newcounter{numexo}
\newcommand{\exo}[1]{\stepcounter{numexo}\noindent{\bf Exercice~\thenumexo} : \marginpar{\hfill /#1}}
\reversemarginpar


\newcounter{enumtabi}
\newcounter{enumtaba}
\newcommand{\q}{\stepcounter{enumtabi} \theenumtabi.  }
\newcommand{\qa}{\stepcounter{enumtaba} (\alph{enumtaba}) }
\newcommand{\initq}{\setcounter{enumtabi}{0}}
\newcommand{\initqa}{\setcounter{enumtaba}{0}}

\newcommand{\be}{\begin{enumerate}}
\newcommand{\ee}{\end{enumerate}}
\newcommand{\bi}{\begin{itemize}}
\newcommand{\ei}{\end{itemize}}
\newcommand{\bp}{\begin{pspicture*}}
\newcommand{\ep}{\end{pspicture*}}
\newcommand{\bt}{\begin{tabular}}
\newcommand{\et}{\end{tabular}}
\renewcommand{\tabularxcolumn}[1]{>{\centering}m{#1}} %(colonne m{} centrée, au lieu de p par défault) 
\newcommand{\tnl}{\tabularnewline}

\newcommand{\trait}{\noindent \rule{\linewidth}{0.2mm}}
\newcommand{\hs}[1]{\hspace{#1}}
\newcommand{\vs}[1]{\vspace{#1}}

\newcommand{\N}{\mathbb{N}}
\newcommand{\Z}{\mathbb{Z}}
\newcommand{\R}{\mathbb{R}}
\newcommand{\C}{\mathbb{C}}
\newcommand{\Dcal}{\mathcal{D}}
\newcommand{\Ccal}{\mathcal{C}}
\newcommand{\mc}{\mathcal}

\newcommand{\vect}[1]{\overrightarrow{#1}}
\newcommand{\ds}{\displaystyle}
\newcommand{\eq}{\quad \Leftrightarrow \quad}
\newcommand{\vecti}{\vec{\imath}}
\newcommand{\vectj}{\vec{\jmath}}
\newcommand{\Oij}{(O;\vec{\imath}, \vec{\jmath})}
\newcommand{\OIJ}{(O;I,J)}

\newcommand{\bmul}[1]{\begin{multicols}{#1}}
\newcommand{\emul}{\end{multicols}}


\newcommand{\reponse}[1][1]{%
\multido{}{#1}{\makebox[\linewidth]{\rule[0pt]{0pt}{20pt}\dotfill}
}}

\newcommand{\titre}[5] 
% #1: titre #2: haut gauche #3: bas gauche #4: haut droite #5: bas droite
{
\noindent #2 \hfill #4 \\
#3 \hfill #5

\vspace{-1.6cm}

\begin{center}\rule{6cm}{0.5mm}\end{center}
\vspace{0.2cm}
\begin{center}{\large{\textbf{#1}}}\end{center}
\begin{center}\rule{6cm}{0.5mm}\end{center}
}



\begin{document}
\pagestyle{empty}
\titre{Interrogation : Résolution d'équations}{Nom}{Prénom}{Date}{Classe}


\exo{1,5} Cours :\\

\q Donner la définition d'une équation.\\
\reponse[2]\\

\q Soit l'équation $ 3x-11 = -5x + 5 $.\\
\qa Est-ce que -1 est solution de cette équation ?\\
\reponse[2]\\
\qa Est-ce que 2 est solution de cette équation ?\\
\reponse[2]\\


\exo{6,5} Résoudre les équations suivantes :

\bmul{3}

$7 + x = 2$\\
\reponse[3]\\

$ -7x = 49$\\
\reponse[3]\\

\columnbreak

$  x -23 = -17$\\
\reponse[3]\\

$ 3x - 7 = 12 $\\
\reponse[3]\\

\columnbreak

$ 3x = 33$\\
\reponse[3]\\

$ -11 - 5x = - 76$\\
\reponse[3]\\

\emul

\bmul{2}

$2-2x = 5x + 3$\\
\reponse[5]\\

\columnbreak

$4x - 19 = 7x - 10 $\\
\reponse[5]\\

\emul




\exo{2} Mise en équation.\\

\initq \q Pierre et Nathalie possèdent ensemble 144 timbres de collection. Si Nathalie donnait deux timbres à Pierre, alors celui-ci en aurait deux fois plus qu'elle. Combien chaque enfant a-t-il de timbres actuellement ?\\
\reponse[6]\\

\q Un père dispose de 1 600 euros pour ses trois enfants. Il veut que l'aîné ait 200 euros de plus que le second et
que le second ait 100 euros de plus que le dernier. Comment fait-il cette répartition ?\\
\reponse[6]








\end{document}
