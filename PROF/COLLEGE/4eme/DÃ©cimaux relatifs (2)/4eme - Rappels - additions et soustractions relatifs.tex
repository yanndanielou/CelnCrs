\documentclass[a4paper,11pt]{article}
\usepackage{amsmath,amsthm,amsfonts,amssymb,amscd,amstext,vmargin,graphics,graphicx,tabularx,multicol} 
\usepackage[francais]{babel}
\usepackage[utf8]{inputenc}  
\usepackage[T1]{fontenc} 
\usepackage{pstricks-add,tikz,tkz-tab,variations}
\usepackage[autolanguage,np]{numprint} 
\usepackage{calc}

\setmarginsrb{1.5cm}{0.5cm}{1cm}{0.5cm}{0cm}{0cm}{0cm}{0cm} %Gauche, haut, droite, haut
\newcounter{numexo}
\newcommand{\exo}[1]{\stepcounter{numexo}\noindent{\bf Exercice~\thenumexo} : }
\reversemarginpar

\newcommand{\bmul}[1]{\begin{multicols}{#1}}
\newcommand{\emul}{\end{multicols}}

\newcounter{enumtabi}
\newcounter{enumtaba}
\newcommand{\q}{\stepcounter{enumtabi} \theenumtabi.  }
\newcommand{\qa}{\stepcounter{enumtaba} (\alph{enumtaba}) }
\newcommand{\initq}{\setcounter{enumtabi}{0}}
\newcommand{\initqa}{\setcounter{enumtaba}{0}}

\newcommand{\be}{\begin{enumerate}}
\newcommand{\ee}{\end{enumerate}}
\newcommand{\bi}{\begin{itemize}}
\newcommand{\ei}{\end{itemize}}
\newcommand{\bp}{\begin{pspicture*}}
\newcommand{\ep}{\end{pspicture*}}
\newcommand{\bt}{\begin{tabular}}
\newcommand{\et}{\end{tabular}}
\renewcommand{\tabularxcolumn}[1]{>{\centering}m{#1}} %(colonne m{} centrée, au lieu de p par défault) 
\newcommand{\tnl}{\tabularnewline}

\newcommand{\trait}{\noindent \rule{\linewidth}{0.2mm}}
\newcommand{\hs}[1]{\hspace{#1}}
\newcommand{\vs}[1]{\vspace{#1}}

\newcommand{\N}{\mathbb{N}}
\newcommand{\Z}{\mathbb{Z}}
\newcommand{\R}{\mathbb{R}}
\newcommand{\C}{\mathbb{C}}
\newcommand{\Dcal}{\mathcal{D}}
\newcommand{\Ccal}{\mathcal{C}}
\newcommand{\mc}{\mathcal}

\newcommand{\vect}[1]{\overrightarrow{#1}}
\newcommand{\ds}{\displaystyle}
\newcommand{\eq}{\quad \Leftrightarrow \quad}
\newcommand{\vecti}{\vec{\imath}}
\newcommand{\vectj}{\vec{\jmath}}
\newcommand{\Oij}{(O;\vec{\imath}, \vec{\jmath})}
\newcommand{\OIJ}{(O;I,J)}


\newcommand{\reponse}[1][1]{%
\multido{}{#1}{\makebox[\linewidth]{\rule[0pt]{0pt}{20pt}\dotfill}
}}

\newcommand{\titre}[5] 
% #1: titre #2: haut gauche #3: bas gauche #4: haut droite #5: bas droite
{
\noindent #2 \hfill #4 \\
#3 \hfill #5

\vspace{-1.6cm}

\begin{center}\rule{6cm}{0.5mm}\end{center}
\vspace{0.2cm}
\begin{center}{\large{\textbf{#1}}}\end{center}
\begin{center}\rule{6cm}{0.5mm}\end{center}
}



\begin{document}
\pagestyle{empty}
\titre{Additions et soustractions de nombres relatifs}{}{}{4ème}{}


\setlength{\fboxrule}{2pt}
\begin{flushleft}
\framebox{\begin{minipage}{\linewidth}

\vspace*{0.2cm}

\underline{\textbf{{\large Rappels de cours}}}\\



 \underline{\textbf{1) Additionner deux nombres relatifs}}

\bi 

\item Pour additionner deux nombres de même signes : on garde le signe commun et on ajoute les deux parties numériques.\\

\textbf{Exemples : } $(+ 6,3) + (+ 5,7) = + (6,3 + 5,7) = + 12$ \hspace*{0.5cm} ou \hspace*{0.5cm} $(-9,1) + (-4) = - (9,1 + 4) = - 13,1$\\



\item Pour additionner deux nombres de  signes contraires : on garde le signe du nombre qui a la plus grande partie numérique puis on soustrait les deux nombres.\\

\textbf{Exemples : } $(-16) + (+9) = - (16-9) = - 7$ \hspace*{0.5cm} ou \hspace*{0.5cm} $(+8,4) + (-12,9) = - (12,9 -8,4) = - 4,5$\\


\ei

 \underline{\textbf{2) Soustraire deux nombres relatifs}}

Pour soustraire un nombre relatif, on ajoute son opposé.\\

\textbf{Exemples : } $(+6)-(+18) = (+6) + (-18) = -12$ \hspace*{0.5cm} ou \hspace*{0.5cm} $ (-14) - ( -20) = (-14) + (+20) = 6$\\


\underline{\textbf{3) Calculer une somme algébrique}}

Une somme algébrique est une suite d'additions et de soustractions dans laquelle on peut changer l'ordre des termes, on regroupe ensuite les termes positifs pour calculer plus simplement l'expression.\\


\textbf{Exemples : }\\   

$E = -7 + (-2)- (-16)-(+5)$ \hspace*{1cm} $\leftarrow$ On commence par transformer toutes les soustractions en additions \hspace*{9cm}(cf \textbf{2)})\\

$E = -7 + (-2)+ (+16)+(-5)$ \hspace*{1cm} $\leftarrow$  On regroupe ensuite les positifs ensembles et les négatifs ensemble\\

$E= (+16) + (-7) + (-2) + (-5)$ \hspace*{1cm} $\leftarrow$  On calcule séparément les nombres positifs puis les nombres négatifs\\

$E= (+16) + (-14) $\\

\fbox{$E= 2 $}








\vspace*{0.2cm}
\end{minipage}}
\end{flushleft}

\vspace*{0.2cm}


\exo \\ Donner le résultat des opérations suivantes.\\
 \qa $(-7) + (-5) = $\hspace*{1cm} \qa $(+3) + (+4,5) =$ \hspace*{1cm} \qa $- 6,1+ (-3,9) =$   \hspace*{1cm} \qa $-13+ (-5) =$\\


\exo \\ Donner le résultat des opérations suivantes.\\
\initqa \qa $(-3) + (+4,5) =$ \hspace*{1cm}  \qa $ -16 + (+3,9) =$ \hspace*{1cm} \qa $(+11) + (-12) =$ \hspace*{1cm}  \qa $-10 +14 =$\\

\exo \\ Donner le résultat des opérations suivantes après avoir transformé chaque soustraction en addition.\\
\initqa 
\qa$17 - (-3)=$ \hspace*{1cm} \qa$-5 - (+11)=$ \hspace*{1cm} \qa$7 - (-4)=$ \hspace*{1cm} \qa$9 + (-3)=$ \hspace*{1cm} \qa$ +7, 2 + (+1) = $\\

\exo \\ En utilisant la méthode vue juste au dessus, calculer les expressions algébriques suivantes.\\
\initqa 
\qa$ A = -5 - (+3) - (+2) - (-13)$  \hspace*{0.4cm} \qa$B = -5 + (-7) +10 -4 + (+17)$  \hspace*{0.4cm} \qa$C = 11,57 + (-7) - (-9) -11,57$\\



\end{document}
