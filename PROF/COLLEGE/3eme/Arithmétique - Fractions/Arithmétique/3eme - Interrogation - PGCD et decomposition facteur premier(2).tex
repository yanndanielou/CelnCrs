\documentclass[a4paper,11pt]{article}
\usepackage{amsmath,amsthm,amsfonts,amssymb,amscd,amstext,vmargin,graphics,graphicx,tabularx,multicol} 
\usepackage[francais]{babel}
\usepackage[utf8]{inputenc}  
\usepackage[T1]{fontenc} 
\usepackage{pstricks-add,tikz,tkz-tab,variations}
\usepackage[autolanguage,np]{numprint} 

\setmarginsrb{1.5cm}{0.5cm}{1cm}{0.5cm}{0cm}{0cm}{0cm}{0cm} %Gauche, haut, droite, haut
\newcounter{numexo}
\newcommand{\exo}[1]{\stepcounter{numexo}\noindent{\bf Exercice~\thenumexo} : \marginpar{\hfill /#1}}
\reversemarginpar


\newcounter{enumtabi}
\newcounter{enumtaba}
\newcommand{\q}{\stepcounter{enumtabi} \theenumtabi.  }
\newcommand{\qa}{\stepcounter{enumtaba} (\alph{enumtaba}) }
\newcommand{\initq}{\setcounter{enumtabi}{0}}
\newcommand{\initqa}{\setcounter{enumtaba}{0}}

\newcommand{\be}{\begin{enumerate}}
\newcommand{\ee}{\end{enumerate}}
\newcommand{\bi}{\begin{itemize}}
\newcommand{\ei}{\end{itemize}}
\newcommand{\bp}{\begin{pspicture*}}
\newcommand{\ep}{\end{pspicture*}}
\newcommand{\bt}{\begin{tabular}}
\newcommand{\et}{\end{tabular}}
\renewcommand{\tabularxcolumn}[1]{>{\centering}m{#1}} %(colonne m{} centrée, au lieu de p par défault) 
\newcommand{\tnl}{\tabularnewline}

\newcommand{\bmul}[1]{\begin{multicols}{#1}}
\newcommand{\emul}{\end{multicols}}

\newcommand{\trait}{\noindent \rule{\linewidth}{0.2mm}}
\newcommand{\hs}[1]{\hspace{#1}}
\newcommand{\vs}[1]{\vspace{#1}}

\newcommand{\N}{\mathbb{N}}
\newcommand{\Z}{\mathbb{Z}}
\newcommand{\R}{\mathbb{R}}
\newcommand{\C}{\mathbb{C}}
\newcommand{\Dcal}{\mathcal{D}}
\newcommand{\Ccal}{\mathcal{C}}
\newcommand{\mc}{\mathcal}

\newcommand{\vect}[1]{\overrightarrow{#1}}
\newcommand{\ds}{\displaystyle}
\newcommand{\eq}{\quad \Leftrightarrow \quad}
\newcommand{\vecti}{\vec{\imath}}
\newcommand{\vectj}{\vec{\jmath}}
\newcommand{\Oij}{(O;\vec{\imath}, \vec{\jmath})}
\newcommand{\OIJ}{(O;I,J)}


\newcommand{\reponse}[1][1]{%
\multido{}{#1}{\makebox[\linewidth]{\rule[0pt]{0pt}{20pt}\dotfill}
}}

\newcommand{\titre}[5] 
% #1: titre #2: haut gauche #3: bas gauche #4: haut droite #5: bas droite
{
\noindent #2 \hfill #4 \\
#3 \hfill #5

\vspace{-1.6cm}

\begin{center}\rule{6cm}{0.5mm}\end{center}
\vspace{0.2cm}
\begin{center}{\large{\textbf{#1}}}\end{center}
\begin{center}\rule{6cm}{0.5mm}\end{center}
}



\begin{document}
\pagestyle{empty}

\titre{Interrogation: Arithmétique (2)}{Nom :}{Prénom :}{Classe}{Date}



\begin{flushleft}
\begin{tabular}{|m{9.5cm}|m{1.25cm}|m{1.25cm}|m{1.25cm}|m{1.25cm}|m{1.25cm}|}
\hline 
\textbf{Compétences} & \begin{center}
\textbf{N.E.}
\end{center} & \begin{center}
\textbf{M.I.}
\end{center} & \begin{center}
\textbf{M.F.}
\end{center}  & \begin{center}
\textbf{M.S.}
\end{center} & \begin{center}
\textbf{T.B.M.}
\end{center} \\ 
\hline 
Je dois savoir écrire une décomposition en facteurs premiers dans des cas simples &  &  & & &\\
\hline 
Je dois savoir déterminer si deux nombres entiers sont premiers entre eux, notion de PGCD &  &  & & &\\
\hline
Je dois savoir simplifier une fraction pour la rendre irréductible irréductible &  &  & & &\\
\hline

\end{tabular} 
\end{flushleft}

\textit{N.E = Non évalué ; M.I. = Maîtrise insuffisante ; M.F. = Maîtrise fragile ; M.S. = Maîtrise satisfaisante ; T.B.M. = Très bonne maîtrise}\\


\vspace*{0.3cm}

\exo{1.5} On donne $1176 = 2^{3} \times 3 \times 7^{2}$. A l'aide de la décomposition en produit de facteurs premiers, citer les diviseurs de  1 176 parmi les nombres ci-dessous :\\

$2^{3} \times 3$ \hspace*{1cm} $2^{2} \times 3^{2}$ \hspace*{1cm} $2^{2} \times 7^{2}$ \hspace*{1cm} $21$ \hspace*{1cm} $49$\hspace*{1cm} $16$\hspace*{1cm} $9$ \hspace*{1cm} $42$\\
\reponse[3]\\


\exo{1.5} Calculer le PGCD de 108 et 175.\\
\reponse[10]\\

\vspace*{0.5cm}

\exo{3} On considère la fraction $S=\dfrac{882}{1134}$.\\

\initq 
\q La fraction ci-dessus est-elle irréductible ? Justifier votre réponse.\\
\reponse[4]\\

\newpage

\q Écrire sous forme irréductible la fraction $\dfrac{882}{1134}$. On indiquera le détail des calculs. \\
\reponse[4]\\

\vspace*{0.3cm}


\exo{4}

Pour son anniversaire, Ninon a acheté 648 carambars et 504 malabars. Elle veut faire des sachets pour ses amis.  Tous  les sachets  doivent  avoir  la  même composition  et  elle  doit  utiliser  tous  les  carambars  et  les 
malabars.\\

\initq \q Peut-elle faire 22 sachets? Si oui, quelle sera la composition de chaque sachet ?\\
\reponse[4]\\

\q Quel est le nombre de sachets maximum qu'elle pourra réaliser ? \\
\reponse[10]\\

\q Quelle sera alors la composition de chaque sachet ?\\
\reponse[3]\\


\exo{} BONUS\\
Le nombre de marches d'un escalier est compris entre 40 et 80.\\
- Si on compte ces marches deux par deux, il en reste une.\\
- Si on compte ces marches trois par trois, il en reste deux.\\
- Si on compte ces marches cinq par cinq, il en reste quatre.\\

Quel est le nombre de marches de cet escalier ?\\
\reponse[3]\\

\end{document}
