\documentclass[a4paper,11pt]{article}
\usepackage{amsmath,amsthm,amsfonts,amssymb,amscd,amstext,vmargin,graphics,graphicx,tabularx,multicol} 
\usepackage[francais]{babel}
\usepackage[utf8]{inputenc}  
\usepackage[T1]{fontenc} 
\usepackage{pstricks-add,tikz,tkz-tab,variations}
\usepackage[autolanguage,np]{numprint} 

\setmarginsrb{1.5cm}{0.5cm}{1cm}{0.5cm}{0cm}{0cm}{0cm}{0cm} %Gauche, haut, droite, haut
\newcounter{numexo}
\newcommand{\exo}[1]{\stepcounter{numexo}\noindent{\bf Exercice~\thenumexo} : \marginpar{\hfill /#1}}
\reversemarginpar


\newcounter{enumtabi}
\newcounter{enumtaba}
\newcommand{\q}{\stepcounter{enumtabi} \theenumtabi.  }
\newcommand{\qa}{\stepcounter{enumtaba} (\alph{enumtaba}) }
\newcommand{\initq}{\setcounter{enumtabi}{0}}
\newcommand{\initqa}{\setcounter{enumtaba}{0}}

\newcommand{\be}{\begin{enumerate}}
\newcommand{\ee}{\end{enumerate}}
\newcommand{\bi}{\begin{itemize}}
\newcommand{\ei}{\end{itemize}}
\newcommand{\bp}{\begin{pspicture*}}
\newcommand{\ep}{\end{pspicture*}}
\newcommand{\bt}{\begin{tabular}}
\newcommand{\et}{\end{tabular}}
\renewcommand{\tabularxcolumn}[1]{>{\centering}m{#1}} %(colonne m{} centrée, au lieu de p par défault) 
\newcommand{\tnl}{\tabularnewline}

\newcommand{\bmul}[1]{\begin{multicols}{#1}}
\newcommand{\emul}{\end{multicols}}

\newcommand{\trait}{\noindent \rule{\linewidth}{0.2mm}}
\newcommand{\hs}[1]{\hspace{#1}}
\newcommand{\vs}[1]{\vspace{#1}}

\newcommand{\N}{\mathbb{N}}
\newcommand{\Z}{\mathbb{Z}}
\newcommand{\R}{\mathbb{R}}
\newcommand{\C}{\mathbb{C}}
\newcommand{\Dcal}{\mathcal{D}}
\newcommand{\Ccal}{\mathcal{C}}
\newcommand{\mc}{\mathcal}

\newcommand{\vect}[1]{\overrightarrow{#1}}
\newcommand{\ds}{\displaystyle}
\newcommand{\eq}{\quad \Leftrightarrow \quad}
\newcommand{\vecti}{\vec{\imath}}
\newcommand{\vectj}{\vec{\jmath}}
\newcommand{\Oij}{(O;\vec{\imath}, \vec{\jmath})}
\newcommand{\OIJ}{(O;I,J)}


\newcommand{\reponse}[1][1]{%
\multido{}{#1}{\makebox[\linewidth]{\rule[0pt]{0pt}{20pt}\dotfill}
}}

\newcommand{\titre}[5] 
% #1: titre #2: haut gauche #3: bas gauche #4: haut droite #5: bas droite
{
\noindent #2 \hfill #4 \\
#3 \hfill #5

\vspace{-1.6cm}

\begin{center}\rule{6cm}{0.5mm}\end{center}
\vspace{0.2cm}
\begin{center}{\large{\textbf{#1}}}\end{center}
\begin{center}\rule{6cm}{0.5mm}\end{center}
}



\begin{document}
\pagestyle{empty}

\titre{Interrogation: Arithmétique (1)}{Nom :}{Prénom :}{Classe}{Date}

\vspace*{0.5cm}

\exo{2.5} Questions de cours


\q Donner la définition d'un nombre premier.\\
\reponse[3]\\

\q Citer tous les nombres premiers inférieurs à 20.\\
\reponse[1]\\


\q Trouver un nombre à quatre chiffres à la fois divisible par 2; 
 divisible par 3; divisible par 5 et non divisible par 9. (Aucune justification n'est attendue.)\\
\reponse[2]\\




\exo{2}

L'ensemble des écrits de Victor Hugo a été republié après sa mort en 153 volumes. La bibliothécaire classe ces volumes à raison de 8 volumes par étagères.\\

 \initq \q Combien d'étagères faut-il pour exposer toute l'\oe{uvre} de Victor Hugo? Justifier votre réponse.\\
\reponse[3]\\

\q Combien de volumes l'étagère incomplète contiendra-t-elle? Justifier votre réponse.\\
\reponse[2]\\

\vspace*{0.5cm}


\exo{2.5}
\initq \q Citer tous les diviseurs de 120 et 72.\\
\reponse[10]\\

\q Quels sont tous les diviseurs communs à 120 et 72?\\
\reponse[2]\\

\q 120 et 72 sont-ils premiers entre eux ? Justifier votre réponse.\\
\reponse[3]\\

\exo{3} 
Le centurion est fier de son armée. Pour le défilé à Rome, il demande à ses soldats de se ranger par lignes de cinq, mais il reste quatre soldats.\\
Il leur demande alors de se ranger par lignes de six, mais il reste cinq soldats.\\
Il leur demande de se ranger par lignes de huit, mais il reste sept soldats.\\

\noindent $\rightarrow$ \textbf{Combien cette armée comporte-t-elle de soldats sachant qu'elle compte moins de deux-cents hommes ? Justifier votre réponse.}\\
\reponse[16]\\


\vspace*{0.5cm}

\exo{} Bonus\\
Un nombre est \textbf{"gentil"} s'il est multiple des dix premiers nombres entiers.\\
\initq \noindent \q Expliquer pourquoi 10 080 est \textbf{"gentil"}. \\
\q Trouver le plus petit nombre \textbf{"gentil"}, en expliquant votre recherche.\\ 
\reponse[6]




\end{document}
