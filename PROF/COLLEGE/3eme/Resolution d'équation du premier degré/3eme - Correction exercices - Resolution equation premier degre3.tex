\documentclass[a4paper,11pt]{article}
\usepackage{amsmath,amsthm,amsfonts,amssymb,amscd,amstext,vmargin,graphics,graphicx,tabularx,multicol} 
\usepackage[francais]{babel}
\usepackage[utf8]{inputenc}  
\usepackage[T1]{fontenc} 
\usepackage{pstricks-add,tikz,tkz-tab,variations}
\usepackage[autolanguage,np]{numprint} 
\usepackage{calc}
\usepackage{mathrsfs}

\usepackage{cancel}

\setmarginsrb{1.5cm}{0.5cm}{1cm}{0.5cm}{0cm}{0cm}{0cm}{0cm} %Gauche, haut, droite, haut
\newcounter{numexo}
\newcommand{\exo}[1]{\stepcounter{numexo}\noindent{\bf Exercice~\thenumexo} : }
\reversemarginpar

\newcommand{\bmul}[1]{\begin{multicols}{#1}}
\newcommand{\emul}{\end{multicols}}

\newcounter{enumtabi}
\newcounter{enumtaba}
\newcommand{\q}{\stepcounter{enumtabi} \theenumtabi.  }
\newcommand{\qa}{\stepcounter{enumtaba} (\alph{enumtaba}) }
\newcommand{\initq}{\setcounter{enumtabi}{0}}
\newcommand{\initqa}{\setcounter{enumtaba}{0}}

\newcommand{\be}{\begin{enumerate}}
\newcommand{\ee}{\end{enumerate}}
\newcommand{\bi}{\begin{itemize}}
\newcommand{\ei}{\end{itemize}}
\newcommand{\bp}{\begin{pspicture*}}
\newcommand{\ep}{\end{pspicture*}}
\newcommand{\bt}{\begin{tabular}}
\newcommand{\et}{\end{tabular}}
\renewcommand{\tabularxcolumn}[1]{>{\centering}m{#1}} %(colonne m{} centrée, au lieu de p par défault) 
\newcommand{\tnl}{\tabularnewline}

\newcommand{\trait}{\noindent \rule{\linewidth}{0.2mm}}
\newcommand{\hs}[1]{\hspace{#1}}
\newcommand{\vs}[1]{\vspace{#1}}

\newcommand{\N}{\mathbb{N}}
\newcommand{\Z}{\mathbb{Z}}
\newcommand{\R}{\mathbb{R}}
\newcommand{\C}{\mathbb{C}}
\newcommand{\Dcal}{\mathcal{D}}
\newcommand{\Ccal}{\mathcal{C}}
\newcommand{\mc}{\mathcal}

\newcommand{\vect}[1]{\overrightarrow{#1}}
\newcommand{\ds}{\displaystyle}
\newcommand{\eq}{\quad \Leftrightarrow \quad}
\newcommand{\vecti}{\vec{\imath}}
\newcommand{\vectj}{\vec{\jmath}}
\newcommand{\Oij}{(O;\vec{\imath}, \vec{\jmath})}
\newcommand{\OIJ}{(O;I,J)}


\newcommand{\reponse}[1][1]{%
\multido{}{#1}{\makebox[\linewidth]{\rule[0pt]{0pt}{20pt}\dotfill}
}}

\newcommand{\titre}[5] 
% #1: titre #2: haut gauche #3: bas gauche #4: haut droite #5: bas droite
{
\noindent #2 \hfill #4 \\
#3 \hfill #5

\vspace{-1.6cm}

\begin{center}\rule{6cm}{0.5mm}\end{center}
\vspace{0.2cm}
\begin{center}{\large{\textbf{#1}}}\end{center}
\begin{center}\rule{6cm}{0.5mm}\end{center}
}



\begin{document}
\pagestyle{empty}
\titre{Séance d'exercices: Résolution d'équation du premier degré}{}{}{3ème}{}

\vspace*{0.2cm}


{\large \textbf{\underline{PARTIE A :}}  Résolution d'équation}\\

\vspace*{0.25cm}


\textbf{Exercice 3 :}
Résoudre les équations suivantes.


\bmul{3}
\initqa 



\textbf{b)} $$4x-7 = 3x+8$$

\color{red}
$$ 4x -7-3x = 3x+8 -3x$$

$$x-7  =  8$$

$$x-7+ 7 =  8+7$$

$$\fbox{x =15}$$
\color{black}






\columnbreak

\textbf{f)} $$7(2x+5) - 3x = 5 + 8x $$

\color{red}
$$ 14x +35 -3x= 5+8x$$

$$11x+35-8x= 5+8x-8x$$

$$3x+35= 5$$

$$3x+35-35= 5-35$$

$$3x = -30$$

$$\dfrac{3x}{3}= \dfrac{-30}{3}$$

$$\fbox{x =-10}$$
\color{black}


\columnbreak


\textbf{i)} $$( x - 1)( x + 3) = ( x + 5)( x - 4) $$
\color{red}
 $$x^{2} +3x-x-3 = x^{2}-4x+5x-20 $$

 $$x^{2} +2x-3 = x^{2}+x-20 $$
 
  $$x^{2} +2x-3 -x^{2} = x^{2}+x-20 -x^{2}$$
  
   $$ 2x-3 = x-20 $$
   
    $$ 2x-3-x = x-20-x $$
    
     $$ x-3 =-20 $$
   
 $$ x-3+3 =-20+3$$
 
 
$$\fbox{x =-17}$$

\emul







 





\end{document}
