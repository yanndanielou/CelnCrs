\documentclass[a4paper,11pt]{article}
\usepackage{amsmath,amsthm,amsfonts,amssymb,amscd,amstext,vmargin,graphics,graphicx,tabularx,multicol} \usepackage[french]{babel}
\usepackage[utf8]{inputenc}  
\usepackage[T1]{fontenc} 
\usepackage[T1]{fontenc}
\usepackage{amsmath,amssymb}
\usepackage{pstricks-add,tikz,tkz-tab,variations}
\usepackage[autolanguage,np]{numprint} 
\usepackage{color}
\usepackage{ulem}

\setmarginsrb{1.5cm}{0.5cm}{1cm}{0.5cm}{0cm}{0cm}{0cm}{0cm} %Gauche, haut, droite, haut
\newcounter{numexo}
\newcommand{\exo}[1]{\stepcounter{numexo}\noindent{\bf Exercice~\thenumexo} : \marginpar{\hfill /#1}}
\reversemarginpar


\newcounter{enumtabi}
\newcounter{enumtaba}
\newcommand{\q}{\stepcounter{enumtabi} \theenumtabi.  }
\newcommand{\qa}{\stepcounter{enumtaba} (\alph{enumtaba}) }
\newcommand{\initq}{\setcounter{enumtabi}{0}}
\newcommand{\initqa}{\setcounter{enumtaba}{0}}

\newcommand{\be}{\begin{enumerate}}
\newcommand{\ee}{\end{enumerate}}
\newcommand{\bi}{\begin{itemize}}
\newcommand{\ei}{\end{itemize}}
\newcommand{\bp}{\begin{pspicture*}}
\newcommand{\ep}{\end{pspicture*}}
\newcommand{\bt}{\begin{tabular}}
\newcommand{\et}{\end{tabular}}
\renewcommand{\tabularxcolumn}[1]{>{\centering}m{#1}} %(colonne m{} centrée, au lieu de p par défault) 
\newcommand{\tnl}{\tabularnewline}

\newcommand{\trait}{\noindent \rule{\linewidth}{0.2mm}}
\newcommand{\hs}[1]{\hspace{#1}}
\newcommand{\vs}[1]{\vspace{#1}}

\newcommand{\N}{\mathbb{N}}
\newcommand{\Z}{\mathbb{Z}}
\newcommand{\R}{\mathbb{R}}
\newcommand{\C}{\mathbb{C}}
\newcommand{\Dcal}{\mathcal{D}}
\newcommand{\Ccal}{\mathcal{C}}
\newcommand{\mc}{\mathcal}

\newcommand{\vect}[1]{\overrightarrow{#1}}
\newcommand{\ds}{\displaystyle}
\newcommand{\eq}{\quad \Leftrightarrow \quad}
\newcommand{\vecti}{\vec{\imath}}
\newcommand{\vectj}{\vec{\jmath}}
\newcommand{\Oij}{(O;\vec{\imath}, \vec{\jmath})}
\newcommand{\OIJ}{(O;I,J)}

\newcommand{\bmul}[1]{\begin{multicols}{#1}}
\newcommand{\emul}{\end{multicols}}


\newcommand{\reponse}[1][1]{%
\multido{}{#1}{\makebox[\linewidth]{\rule[0pt]{0pt}{20pt}\dotfill}
}}

\newcommand{\titre}[5] 
% #1: titre #2: haut gauche #3: bas gauche #4: haut droite #5: bas droite
{
\noindent #2 \hfill #4 \\
#3 \hfill #5

\vspace{-1.6cm}

\begin{center}\rule{6cm}{0.5mm}\end{center}
\vspace{0.2cm}
\begin{center}{\large{\textbf{#1}}}\end{center}
\begin{center}\rule{6cm}{0.5mm}\end{center}
}



\begin{document}
\pagestyle{empty}
\titre{Interrogation : Identités remarquables}{Nom}{Prénom}{Date}{Classe}

\begin{flushleft}
\begin{tabular}{|m{9.5cm}|m{1.25cm}|m{1.25cm}|m{1.25cm}|m{1.25cm}|m{1.25cm}|}
\hline 
\textbf{Compétences} & \begin{center}
\textbf{N.E.}
\end{center} & \begin{center}
\textbf{M.I.}
\end{center} & \begin{center}
\textbf{M.F.}
\end{center}  & \begin{center}
\textbf{M.S.}
\end{center} & \begin{center}
\textbf{T.B.M.}
\end{center} \\ 
\hline 
Je dois connaitre les identités remarquables &  &  & & &\\
\hline 
Je dois savoir développer en utilisant une identité remarquable sur des exemples numériques ou littéraux &  &  & & &\\
\hline
Je dois savoir factoriser en utilisant une identité remarquable sur des exemples numériques ou littéraux &  &  & & &\\ 
\hline


\end{tabular} 
\end{flushleft}

\textit{N.E = Non évalué ; M.I. = Maîtrise insuffisante ; M.F. = Maîtrise fragile ; M.S. = Maîtrise satisfaisante ; T.B.M. = Très bonne maîtrise}\\

\vspace*{0.5cm}


\exo{0.5} Compléter l'identité remarquable suivante :\\

\bi

\item $ (a+b)^{2}$ = . . . . . . . . . . . . . . . . . . . . . . . . \\

\ei

\vspace*{1cm}


\exo{4.5} Développer et réduire les identités remarquables suivantes :

\bmul{3}

$L = (x-2)^{2}$\\
\reponse[2]\\

\columnbreak

$E = (8x+7)^{2}$\\
\reponse[2]\\

\columnbreak

$A = (100x - 12)(100x+12)$\\
\reponse[2]\\

\emul

$M = (3x -9)^{2} - 5(x+1)^{2}$\\
\reponse[10]\\


\vspace*{1cm}


\exo{5} Factoriser les identités remarquables suivantes :

\bmul{3}

$R = 36x^{2} - 81 $\\
\reponse[2]\\

\columnbreak

$P = 25x^{2} - 40 x + 16$\\
\reponse[2]\\

\columnbreak

$S = 49  + 9x^{2} + 42x$\\
\reponse[2]\\

\emul

\newpage
\vspace*{0.5cm}

$G = (-7x+9)^{2} - (21+4x)^{2}$\\
\reponse[11]\\

\vspace*{0.5cm}

\exo{} BONUS\\
Factoriser \textbf{au maximum} les expressions suivantes :\\

 \hspace*{1cm} $ K = 4x^{3}-12x^{2}+x $ \hspace*{0.75cm} $Q= (4x-1)(13-x)-16x^{2}+8x-1$\\
\reponse[11]



\end{document}
