\documentclass[a4paper,11pt]{article}
\usepackage{amsmath,amsthm,amsfonts,amssymb,amscd,amstext,vmargin,graphics,graphicx,tabularx,multicol} 
\usepackage[francais]{babel}
\usepackage[utf8]{inputenc}  
\usepackage[T1]{fontenc} 
\usepackage{pstricks-add,tikz,tkz-tab,variations}
\usepackage[autolanguage,np]{numprint} 
\usepackage{color}
\usepackage{ulem}

\setmarginsrb{1.5cm}{0.5cm}{1cm}{0.5cm}{0cm}{0cm}{0cm}{0cm} %Gauche, haut, droite, haut
\newcounter{numexo}
\newcommand{\exo}[1]{\stepcounter{numexo}\noindent{\bf Exercice~\thenumexo} : \marginpar{\hfill /#1}}
\reversemarginpar


\newcounter{enumtabi}
\newcounter{enumtaba}
\newcommand{\q}{\stepcounter{enumtabi} \theenumtabi.  }
\newcommand{\qa}{\stepcounter{enumtaba} (\alph{enumtaba}) }
\newcommand{\initq}{\setcounter{enumtabi}{0}}
\newcommand{\initqa}{\setcounter{enumtaba}{0}}

\newcommand{\be}{\begin{enumerate}}
\newcommand{\ee}{\end{enumerate}}
\newcommand{\bi}{\begin{itemize}}
\newcommand{\ei}{\end{itemize}}
\newcommand{\bp}{\begin{pspicture*}}
\newcommand{\ep}{\end{pspicture*}}
\newcommand{\bt}{\begin{tabular}}
\newcommand{\et}{\end{tabular}}
\renewcommand{\tabularxcolumn}[1]{>{\centering}m{#1}} %(colonne m{} centrée, au lieu de p par défault) 
\newcommand{\tnl}{\tabularnewline}

\newcommand{\bmul}[1]{\begin{multicols}{#1}}
\newcommand{\emul}{\end{multicols}}

\newcommand{\trait}{\noindent \rule{\linewidth}{0.2mm}}
\newcommand{\hs}[1]{\hspace{#1}}
\newcommand{\vs}[1]{\vspace{#1}}

\newcommand{\N}{\mathbb{N}}
\newcommand{\Z}{\mathbb{Z}}
\newcommand{\R}{\mathbb{R}}
\newcommand{\C}{\mathbb{C}}
\newcommand{\Dcal}{\mathcal{D}}
\newcommand{\Ccal}{\mathcal{C}}
\newcommand{\mc}{\mathcal}

\newcommand{\vect}[1]{\overrightarrow{#1}}
\newcommand{\ds}{\displaystyle}
\newcommand{\eq}{\quad \Leftrightarrow \quad}
\newcommand{\vecti}{\vec{\imath}}
\newcommand{\vectj}{\vec{\jmath}}
\newcommand{\Oij}{(O;\vec{\imath}, \vec{\jmath})}
\newcommand{\OIJ}{(O;I,J)}


\newcommand{\reponse}[1][1]{%
\multido{}{#1}{\makebox[\linewidth]{\rule[0pt]{0pt}{20pt}\dotfill}
}}

\newcommand{\titre}[5] 
% #1: titre #2: haut gauche #3: bas gauche #4: haut droite #5: bas droite
{
\noindent #2 \hfill #4 \\
#3 \hfill #5

\vspace{-1.6cm}

\begin{center}\rule{6cm}{0.5mm}\end{center}
\vspace{0.2cm}
\begin{center}{\large{\textbf{#1}}}\end{center}
\begin{center}\rule{6cm}{0.5mm}\end{center}
}



\begin{document}
\pagestyle{empty}
\titre{Interrogation: Les nombres entiers }{Nom :}{Prénom :}{Classe}{Date}

\vspace*{0.3cm}

\exo{0.75} Compléter  les phrases suivantes :\\

\bi 
\item Un . . . . . . . . . . . . est composé de plusieurs chiffres.\\

\item 39 est un . . . . . . . . . composé de deux . . . . . . . . . . .\\

\ei

\vspace*{0.3cm}

\exo{2}\\ 

\initq \q Quels sont les nombres suivants :\\

DXXIV =  \hspace*{3cm}   MDCCLVII =\\



\q Écrire les nombres suivants en chiffres romains  :\\

1 991 = \hspace*{4cm}  3 109 =\\



\vspace*{0.3cm}





\exo{1.5}\\

\initq
\q Dans le nombre 450 019 237, quel est le chiffre des unités de milliards ?  . . . . . . . . . . . . . . . . . . . . . . . . .\\

\q Dans le nombre 99 500 346, quel est le chiffre des centaines de mille ?  . . . . . . . . . . . . . . . . . . . . . . . . .\\

\q Dans le nombre 36 041, quel est le nombre de dizaines ? . . . . . . . . . . . . . . . . . . . . . . . . .  \\

\vspace*{0.3cm}

\exo{1} Ajouter des zéros pour que 5 soit le chiffre \textit{des dizaines de mille} de chaque nombre :\\

 \hspace*{2cm} 7503		\hspace*{3cm}		251039	\\

\vspace*{0.3cm}

\exo{1} Dire si les nombres suivants sont égaux (=) ou différents ($ \ne$ )\\

	  405 .........  0405      \hspace*{3cm}                 10 571 .........  1 571










\end{document}


