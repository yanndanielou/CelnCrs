\documentclass[12pt,a4paper]{article}

%******************************* Paquetages ******************************

%Conventions fran\c{c}aises
\usepackage[francais]{babel}
\usepackage[latin1]{inputenc}
\usepackage[T1]{fontenc}
\usepackage{listing}
%Pour la g\'eom\'etrie.
\usepackage{pstricks-add}
\usepackage{fp}
%\droite gradu\'ee{Nb de divisions de la longueur unit\'e}{longueur axe}
\newcommand{\droite}[2]{
  \psline{->}(0,0)(#2,0)
\FPdiv{\division}{1}{#1}%Variable pour la division de la longueur Unit\'e.
\FPmul{\produit}{#1}{#2}%Variable pour les graduations sur toute la longueur de l'axe.
\FPclip{\produit}{\produit}%Supression des z\'eros inutiles.
  \multido{\r=0+\division}{\produit}%Boucle PsTricks
  {
  \psline(\r,-0.075)(\r,0.075)
  }
}

%\droite gradu\'ee avec nombres{Nb de divisions de la longueur unit\'e}{longueur axe}
\newcommand{\droitenombres}[2]{
  \psline{->}(0,0)(#2,0)
\FPdiv{\division}{1}{#1}
\FPmul{\produit}{#1}{#2}
\FPclip{\produit}{\produit}
  \multido{\r=0+\division}{\produit}
  {
  \psline(\r,-0.075)(\r,0.075)
  }
\multido{\i=0+1}{#2}{
\psline[linewidth=0.03](\i,-0.15)(\i,0.15)
\rput(\i,0.35){\i}}
}

%\droite gradu\'ee avec nombres{Nb de divisions de la longueur unit\'e}{longueur axe}{nb d\'epart}
\newcommand{\droitenombresorigine}[3]{
  \psline{->}(0,0)(#2,0)
\FPdiv{\division}{1}{#1}
\FPmul{\produit}{#1}{#2}
\FPclip{\produitentier}{\produit}
  \multido{\r=0+\division}{\produitentier}
  {
  \psline(\r,-0.075)(\r,0.075)
  }

\FPset{\nbchoisi}{#3}
\multido{\i=0+1}{#2}{
\psline[linewidth=0.03](\i,-0.15)(\i,0.15)
\rput(\i,0.35){\nbchoisi}
\FPadd{\nbchoisi}{1}{\nbchoisi}
\FPclip{\nbchoisi}{\nbchoisi}
}
}


%\droite gradu\'ee avec nombres{Nb de divisions de la longueur unit\'e}{longueur axe}{nb d\'epart}{pas}
\newcommand{\droitenombresoriginepas}[4]{
  \psline{->}(0,0)(#2,0)
\FPdiv{\division}{1}{#1}
\FPmul{\produit}{#1}{#2}
\FPclip{\produitentier}{\produit}
  \multido{\r=0+\division}{\produitentier}
  {
  \psline(\r,-0.075)(\r,0.075)
  }
\FPset{\nbchoisi}{#3}
\multido{\i=0+1}{#2}{
\psline[linewidth=0.03](\i,-0.15)(\i,0.15)
\rput(\i,0.35){\nbchoisi}
\FPadd{\nbchoisi}{#4}{\nbchoisi}
\FPclip{\nbchoisi}{\nbchoisi}
}
}


\begin{document}

\begin{pspicture}(-1,-1)(8,1)
 \droite{5}{8}
\end{pspicture}


\begin{pspicture}(-1,-1)(8,1)
 \droitenombres{4}{6}
\end{pspicture}


\begin{pspicture}(-1,-1)(8,1)
 \droitenombresorigine{10}{8}{-5}
\end{pspicture}


\begin{pspicture}(-1,-1)(8,1)
 \droitenombresoriginepas{2}{8}{-5}{0.2}
\end{pspicture}


\noindent
Exercice 1:\\
D\'eterminer les abscisses des points $A$, $B$, $C$ et $D$.\\
\begin{pspicture}(-1,-1)(8,2)
\rput(1.2,1){A}
\psline{->}(1.2,0.75)(1.2,0.075)
\rput(2.8,1){B}
\psline{->}(2.8,0.75)(2.8,0.075)
\rput(4.5,1){C}
\psline{->}(4.5,0.75)(4.5,0.075)
\rput(6.7,1){D}
\psline{->}(6.7,0.75)(6.7,0.075)
\droitenombresorigine{10}{8}{-5}
\end{pspicture}

Exercice 2:\\
D\'eterminer les abscisses des points $A$, $B$, $C$ et $D$.\\
\begin{pspicture}(-1,-1)(8,2)
\rput(1.2,1){A}
\psline{->}(1.2,0.75)(1.2,0.075)
\rput(2.8,1){B}
\psline{->}(2.8,0.75)(2.8,0.075)
\rput(4.4,1){C}
\psline{->}(4.4,0.75)(4.4,0.075)
\rput(6.2,1){D}
\psline{->}(6.2,0.75)(6.2,0.075)
\droitenombresorigine{5}{8}{0}
\end{pspicture}

\end{document}
