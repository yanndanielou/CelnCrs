\documentclass[a4paper,11pt]{article}
\usepackage{amsmath,amsthm,amsfonts,amssymb,amscd,amstext,vmargin,graphics,graphicx,tabularx,multicol} 
\usepackage[francais]{babel}
\usepackage[utf8]{inputenc}  
\usepackage[T1]{fontenc} 
\usepackage{pstricks-add,tikz,tkz-tab,variations}
\usepackage[autolanguage,np]{numprint} 

\setmarginsrb{1.5cm}{0.5cm}{1cm}{0.5cm}{0cm}{0cm}{0cm}{0cm} %Gauche, haut, droite, haut
\newcounter{numexo}
\newcommand{\exo}[1]{\stepcounter{numexo}\noindent{\bf Exercice~\thenumexo} : \marginpar{\hfill /#1}}
\reversemarginpar


\newcounter{enumtabi}
\newcounter{enumtaba}
\newcommand{\q}{\stepcounter{enumtabi} \theenumtabi.  }
\newcommand{\qa}{\stepcounter{enumtaba} (\alph{enumtaba}) }
\newcommand{\initq}{\setcounter{enumtabi}{0}}
\newcommand{\initqa}{\setcounter{enumtaba}{0}}

\newcommand{\be}{\begin{enumerate}}
\newcommand{\ee}{\end{enumerate}}
\newcommand{\bi}{\begin{itemize}}
\newcommand{\ei}{\end{itemize}}
\newcommand{\bp}{\begin{pspicture*}}
\newcommand{\ep}{\end{pspicture*}}
\newcommand{\bt}{\begin{tabular}}
\newcommand{\et}{\end{tabular}}
\renewcommand{\tabularxcolumn}[1]{>{\centering}m{#1}} %(colonne m{} centrée, au lieu de p par défault) 
\newcommand{\tnl}{\tabularnewline}

\newcommand{\bmul}[1]{\begin{multicols}{#1}}
\newcommand{\emul}{\end{multicols}}

\newcommand{\trait}{\noindent \rule{\linewidth}{0.2mm}}
\newcommand{\hs}[1]{\hspace{#1}}
\newcommand{\vs}[1]{\vspace{#1}}

\newcommand{\N}{\mathbb{N}}
\newcommand{\Z}{\mathbb{Z}}
\newcommand{\R}{\mathbb{R}}
\newcommand{\C}{\mathbb{C}}
\newcommand{\Dcal}{\mathcal{D}}
\newcommand{\Ccal}{\mathcal{C}}
\newcommand{\mc}{\mathcal}

\newcommand{\vect}[1]{\overrightarrow{#1}}
\newcommand{\ds}{\displaystyle}
\newcommand{\eq}{\quad \Leftrightarrow \quad}
\newcommand{\vecti}{\vec{\imath}}
\newcommand{\vectj}{\vec{\jmath}}
\newcommand{\Oij}{(O;\vec{\imath}, \vec{\jmath})}
\newcommand{\OIJ}{(O;I,J)}


\newcommand{\reponse}[1][1]{%
\multido{}{#1}{\makebox[\linewidth]{\rule[0pt]{0pt}{20pt}\dotfill}
}}

\newcommand{\titre}[5] 
% #1: titre #2: haut gauche #3: bas gauche #4: haut droite #5: bas droite
{
\noindent #2 \hfill #4 \\
#3 \hfill #5

\vspace{-1.6cm}

\begin{center}\rule{6cm}{0.5mm}\end{center}
\vspace{0.2cm}
\begin{center}{\large{\textbf{#1}}}\end{center}
\begin{center}\rule{6cm}{0.5mm}\end{center}
}



\begin{document}
\pagestyle{empty}
\titre{Interrogation: Comparer, intercaler et encadrer des nombres }{Nom :}{Prénom :}{Classe}{Date}

\vspace*{0.5cm}
\begin{flushleft}
\begin{tabular}{|m{9.5cm}|m{1.25cm}|m{1.25cm}|m{1.25cm}|m{1.25cm}|m{1.25cm}|}
\hline 
\textbf{Compétences} & \begin{center}
\textbf{N.E.}
\end{center} & \begin{center}
\textbf{M.I.}
\end{center} & \begin{center}
\textbf{M.F.}
\end{center}  & \begin{center}
\textbf{M.S.}
\end{center} & \begin{center}
\textbf{T.B.M.}
\end{center} \\ 
\hline 
Je dois savoir ranger des nombres dans le sens croissant ou dans le sens décroissant &  &  & & &\\
\hline 
Je dois savoir encadrer un nombre, intercaler un nombre entre deux autres &  &  & & &\\
\hline
Je dois savoir donner une valeur approchée (par excès ou par défaut) à l'unité, au dixième, au centième près &  &  & & &\\ 
\hline

\end{tabular} 
\end{flushleft}

\textit{N.E = Non évalué ; M.I. = Maîtrise insuffisante ; M.F. = Maîtrise fragile ; M.S. = Maîtrise satisfaisante ; T.B.M. = Très bonne maîtrise}\\


\vspace*{0.5cm}



\exo{1,5} Compléter les définitions du cours : \\


\q Ranger des nombres du plus petit au plus grand, c'est les ranger dans \reponse[2]\\

\q Encadrer un nombre, c'est \reponse[3]\\


\exo{3} 

\initq \qa Ranger dans l'ordre décroissant les nombres suivants : \\

$ 5,4$ \hspace*{0.3cm};\hspace*{0.3cm} $ \dfrac{542}{100} + \dfrac{3}{1 000}$ \hspace*{0.3cm} ;\hspace*{0.3cm} $ \dfrac{53}{10} + \dfrac{9}{100}$\hspace*{0.3cm} ; \hspace*{0.3cm}$ 538$ centièmes\hspace*{0.3cm} et\hspace*{0.3cm} $ \dfrac{5470}{1 000}$\\

\reponse[1]\\


\qa Compléter avec le nombre \textbf{entier} qui suit ou celui qui précède :\\

\bmul{3}

$12,4 < . . .$

\columnbreak

$ . . . < 6,19 $

\columnbreak

$ . . . < \dfrac{2 345 }{100}$\\

\emul

\exo{2,5} 

\initq \q Intercaler un nombre entre 3,1 et 3,2 : \\
\reponse[1]\\

\q Encadrer les nombres suivants par deux entiers consécutifs :

\bmul{2}

. . . . . . . < 74,586 < . . . . . . .

\columnbreak

. . . . . . . < $ \dfrac{8523}{100}$ < . . . . . . .


\emul

\newpage

\vspace*{0.5cm}

\exo{3} $\pi$ est un nombre qui a fasciné tant de savants depuis l'antiquité. \\
$\pi$ est un nombre irrationnel (c'est à dire qu'il s'écrit avec un nombre infini de décimales sans suite logique). \\
Le 2 Août 2010, 5 000 milliards de décimales de $\pi$ ont été découverts par deux japonnais Alexander J. Yee et Shigeru en 90 jours.\\
 Et 1 an plus tard après 371 jours de travail, ces même chercheurs ont battu leur record et ont découvert jusqu'à 10 000 milliards de décimales de $\pi$. En voici une toute petite approximation :

\begin{center}
{\large $\pi \approx 3,141 592 653 589 793 238 462 643 383 279 502 884 197 169 399 375 $}
\end{center}



\initq \q \initqa \qa Encadrer le nombre $\pi$ \textbf{au millième} près.\\
\reponse[4]\\

\qa Donner la valeur approchée \textbf{au millième} près de $\pi$ par défaut.\\
\reponse[4]\\


\q Donner la valeur approchée \textbf{au dixième} près de $\pi$ par excès.\\
\reponse[4]\\

\q BONUS. Donner un arrondi de $\pi$ au centième près. Expliquer votre réponse.\\
\reponse[4]\\




\end{document}
