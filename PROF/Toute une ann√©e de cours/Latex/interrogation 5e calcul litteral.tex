\documentclass[a4paper,11pt]{article}
\usepackage{amsmath,amsthm,amsfonts,amssymb,amscd,amstext,vmargin,graphics,graphicx,tabularx,multicol} 
\usepackage[francais]{babel}
\usepackage[utf8]{inputenc}  
\usepackage[T1]{fontenc} 
\usepackage{pstricks-add,tikz,tkz-tab,variations}
\usepackage[autolanguage,np]{numprint} 

\setmarginsrb{1.5cm}{0.5cm}{1cm}{0.5cm}{0cm}{0cm}{0cm}{0cm} %Gauche, haut, droite, haut
\newcounter{numexo}
\newcommand{\exo}[1]{\stepcounter{numexo}\noindent{\bf Exercice~\thenumexo} : \marginpar{\hfill /#1}}
\reversemarginpar


\newcounter{enumtabi}
\newcounter{enumtaba}
\newcommand{\q}{\stepcounter{enumtabi} \theenumtabi.  }
\newcommand{\qa}{\stepcounter{enumtaba} (\alph{enumtaba}) }
\newcommand{\initq}{\setcounter{enumtabi}{0}}
\newcommand{\initqa}{\setcounter{enumtaba}{0}}

\newcommand{\be}{\begin{enumerate}}
\newcommand{\ee}{\end{enumerate}}
\newcommand{\bi}{\begin{itemize}}
\newcommand{\ei}{\end{itemize}}
\newcommand{\bp}{\begin{pspicture*}}
\newcommand{\ep}{\end{pspicture*}}
\newcommand{\bt}{\begin{tabular}}
\newcommand{\et}{\end{tabular}}
\renewcommand{\tabularxcolumn}[1]{>{\centering}m{#1}} %(colonne m{} centrée, au lieu de p par défault) 
\newcommand{\tnl}{\tabularnewline}

\newcommand{\trait}{\noindent \rule{\linewidth}{0.2mm}}
\newcommand{\hs}[1]{\hspace{#1}}
\newcommand{\vs}[1]{\vspace{#1}}

\newcommand{\N}{\mathbb{N}}
\newcommand{\Z}{\mathbb{Z}}
\newcommand{\R}{\mathbb{R}}
\newcommand{\C}{\mathbb{C}}
\newcommand{\Dcal}{\mathcal{D}}
\newcommand{\Ccal}{\mathcal{C}}
\newcommand{\mc}{\mathcal}

\newcommand{\vect}[1]{\overrightarrow{#1}}
\newcommand{\ds}{\displaystyle}
\newcommand{\eq}{\quad \Leftrightarrow \quad}
\newcommand{\vecti}{\vec{\imath}}
\newcommand{\vectj}{\vec{\jmath}}
\newcommand{\Oij}{(O;\vec{\imath}, \vec{\jmath})}
\newcommand{\OIJ}{(O;I,J)}


\newcommand{\reponse}[1][1]{%
\multido{}{#1}{\makebox[\linewidth]{\rule[0pt]{0pt}{20pt}\dotfill}
}}

\newcommand{\titre}[5] 
% #1: titre #2: haut gauche #3: bas gauche #4: haut droite #5: bas droite
{
\noindent #2 \hfill #4 \\
#3 \hfill #5

\vspace{-1.6cm}

\begin{center}\rule{6cm}{0.5mm}\end{center}
\vspace{0.2cm}
\begin{center}{\large{\textbf{#1}}}\end{center}
\begin{center}\rule{6cm}{0.5mm}\end{center}
}



\begin{document}
\pagestyle{empty}
\titre{Interrogation: Calcul littéral}{Nom :}{Prénom :}{Classe}{Date}

\exo{4} Questions de cours\\


    Compléter :
    \begin{multicols}{4}
      \begin{enumerate}
      \item $\dfrac{5}{8}=\dfrac{\ldots}{56}$
      \item $\dfrac{36}{\ldots}=\dfrac{6}{5}$
      \item $\dfrac{\ldots}{8}=\dfrac{70}{80}$
      \item $\dfrac{9}{\ldots}=\dfrac{18}{14}$
      \item $\dfrac{\ldots}{25}=\dfrac{1}{5}$
      \item $\dfrac{72}{\ldots}=\dfrac{9}{4}$
      \item $\dfrac{5}{\ldots}=\dfrac{45}{63}$
      \item $\dfrac{40}{\ldots}=\dfrac{10}{9}$
      \end{enumerate}
    \end{multicols}


  
    Compléter :
    \begin{multicols}{4}
      \begin{enumerate}
      \item $\dfrac{1}{6}=\dfrac{2}{\ldots}$
      \item $\dfrac{\ldots}{8}=\dfrac{24}{48}$
      \item $\dfrac{\ldots}{7}=\dfrac{64}{56}$
      \item $\dfrac{2}{7}=\dfrac{\ldots}{35}$
      \item $\dfrac{18}{\ldots}=\dfrac{9}{2}$
      \item $\dfrac{45}{36}=\dfrac{\ldots}{4}$
      \item $\dfrac{\ldots}{18}=\dfrac{1}{9}$
      \item $\dfrac{28}{42}=\dfrac{\ldots}{6}$
      \end{enumerate}
    \end{multicols}


   
    Compléter :
    \begin{multicols}{4}
      \begin{enumerate}
      \item $\dfrac{\ldots}{70}=\dfrac{2}{7}$
      \item $\dfrac{45}{\ldots}=\dfrac{5}{2}$
      \item $\dfrac{\ldots}{24}=\dfrac{8}{3}$
      \item $\dfrac{12}{24}=\dfrac{\ldots}{4}$
      \item $\dfrac{42}{49}=\dfrac{\ldots}{7}$
      \item $\dfrac{7}{\ldots}=\dfrac{63}{72}$
      \item $\dfrac{\ldots}{90}=\dfrac{10}{9}$
      \item $\dfrac{\ldots}{48}=\dfrac{10}{6}$
      \end{enumerate}
    \end{multicols}


  
    \parbox{0.4\linewidth}{
      \begin{enumerate}
      \item Donner les coordonnées des points B, C, D, F, G et N.
      \item Placer dans le repère les points P, S, T, U, V et X de coordonnées
        respectives \hbox{$(-3{,}5~;~0)$}, \hbox{$(-4~;~-2{,}5)$},
        \hbox{$(4{,}5~;~-4{,}5)$}, \hbox{$(0~;~-3)$}, \hbox{$(-0{,}5~;~1)$} et
        \hbox{$(3{,}5~;~4{,}5)$}.
      \item Placer dans le repère le point Z d'abscisse -3 et d'ordonnée -1,5
      \end{enumerate}}\hfill
    \parbox{0.55\linewidth}{
      \psset{unit=0.8cm}
      \begin{pspicture}(-4.95,-4.95)(4.95,4.95)
        \psgrid[subgriddiv=2, subgridcolor=lightgray,
        gridlabels=8pt](0,0)(-5,-5)(5,5)
        \psline[linewidth=1.2pt]{->}(-5,0)(5,0)
        \psline[linewidth=1.2pt]{->}(0,-5)(0,5)
        \pstGeonode[PointSymbol=x,PosAngle={-135,45,135,45,-45,45,},PointNameSep=0.4](-4.0, -1.5){B}(0, 4.0){C}(-2.5, 3.0){D}(0.5, 0){F}(1.0, -3.5){G}(4.5, 3.0){N}
      \end{pspicture}}


    \parbox{0.4\linewidth}{
      \begin{enumerate}
      \item Donner les coordonnées des points A, B, D, E, G et I.
      \item Placer dans le repère les points J, M, N, P, U et W de coordonnées
        respectives \hbox{$(-2~;~-1{,}5)$}, \hbox{$(4~;~0)$},
        \hbox{$(0~;~4{,}5)$}, \hbox{$(4{,}5~;~4)$}, \hbox{$(-1{,}5~;~2)$} et
        \hbox{$(3{,}5~;~-3{,}5)$}.
      \item Placer dans le repère le point X d'abscisse -2 et d'ordonnée 3
      \end{enumerate}}\hfill
    \parbox{0.55\linewidth}{
      \psset{unit=0.8cm}
      \begin{pspicture}(-4.95,-4.95)(4.95,4.95)
        \psgrid[subgriddiv=2, subgridcolor=lightgray,
        gridlabels=8pt](0,0)(-5,-5)(5,5)
        \psline[linewidth=1.2pt]{->}(-5,0)(5,0)
        \psline[linewidth=1.2pt]{->}(0,-5)(0,5)
        \pstGeonode[PointSymbol=x,PosAngle={45,45,-135,45,-45,135,},PointNameSep=0.4](0.5, 2.5){A}(0, 0.5){B}(-2.5, -2.5){D}(1.5, 0){E}(4.0, -2.0){G}(-4.5, 0.5){I}
      \end{pspicture}}

    
    Construire la symétrique de chacune des figures par rapport au point O en
    utilisant le quadrillage :\par
    \psset{unit=.9cm}
    \begin{pspicture*}(-3,-3)(3,3)
      \psgrid[subgriddiv=2,gridlabels=0pt]
      \pstGeonode[PointSymbol=none,PointName=none](2.5,1.0){a}(-2.5,2.5){b}(-1.5,-1.0){c}(-1.5,-1.5){d}(1.0,-3.0){e}
      \pstGeonode[PointSymbol=x, linecolor=Black, dotsize=6pt](0.0,0.0){O}
      \pspolygon[linewidth=1pt](a)(b)(c)(d)(e)
    \end{pspicture*}
    \hfill
    \begin{pspicture*}(-3,-3)(3,3)
      \psgrid[subgriddiv=2,gridlabels=0pt]
      \pstGeonode[PointSymbol=none,PointName=none](2.0,1.0){a}(-0.5,3.0){b}(-2.5,-1.0){c}(-2.0,-3.0){d}(2.5,-2.5){e}
      \pstGeonode[PointSymbol=x, linecolor=Black, dotsize=6pt](0.0,0.0){O}
      \pspolygon[linewidth=1pt](a)(b)(c)(d)(e)
    \end{pspicture*}
    \hfill
    \begin{pspicture*}(-3,-3)(3,3)
      \psgrid[subgriddiv=2,gridlabels=0pt]
      \pstGeonode[PointSymbol=none,PointName=none](1.5,1.5){a}(0.5,3.0){b}(-3.0,-2.0){c}(-3.0,-3.0){d}(3.0,-2.5){e}
      \pstGeonode[PointSymbol=x, linecolor=Black, dotsize=6pt](0.0,0.0){O}
      \pspolygon[linewidth=1pt](a)(b)(c)(d)(e)
    \end{pspicture*}

    \exercice
    Construire la symétrique de chacune des figures par rapport au point O en
    utilisant le quadrillage :\par
    \psset{unit=.9cm}
    \begin{pspicture*}(-3,-3)(3,3)
      \psgrid[subgriddiv=2,gridlabels=0pt]
      \pstGeonode[PointSymbol=none,PointName=none](3.0,1.0){a}(-0.5,2.0){b}(-2.0,-1.5){c}(1.0,-3.0){d}(2.5,-1.0){e}
      \pstGeonode[PointSymbol=x, linecolor=Black, dotsize=6pt](0.0,0.0){O}
      \pspolygon[linewidth=1pt](a)(b)(c)(d)(e)
    \end{pspicture*}
    \hfill
    \begin{pspicture*}(-3,-3)(3,3)
      \psgrid[subgriddiv=2,gridlabels=0pt]
      \pstGeonode[PointSymbol=none,PointName=none](1.0,1.5){a}(-1.5,1.5){b}(-2.0,-0.5){c}(-1.0,-2.0){d}(2.5,-1.5){e}
      \pstGeonode[PointSymbol=x, linecolor=Black, dotsize=6pt](0.0,0.0){O}
      \pspolygon[linewidth=1pt](a)(b)(c)(d)(e)
    \end{pspicture*}
    \hfill
    \begin{pspicture*}(-3,-3)(3,3)
      \psgrid[subgriddiv=2,gridlabels=0pt]
      \pstGeonode[PointSymbol=none,PointName=none](2.5,2.5){a}(0.5,3.0){b}(-1.5,-0.5){c}(0.5,-2.5){d}(3.0,-2.0){e}
      \pstGeonode[PointSymbol=x, linecolor=Black, dotsize=6pt](0.5,0.0){O}
      \pspolygon[linewidth=1pt](a)(b)(c)(d)(e)
    \end{pspicture*}
    
\q Qu'est-ce qu'une expression littérale ?\\
\reponse[2]

\q Que signifie "développer une expression" ?\\
\reponse[2]

\q Simplifier les écritures suivantes :\\
\begin{multicols}{2}

$A=3\times x$\\
\reponse[1]

$B=x\times x$\\
\reponse[1]

$C=3\times x+3\times4$\\
\reponse[1]


\columnbreak


$E=6\times(3+y)$\\
\reponse[1]

$F=6\times(3+y)$\\
\reponse[1]


$G=6\times(3+y)$\\
\reponse[1]
\end{multicols}

\exo{4} Calculer les expressions suivantes en indiquant toutes les étapes de calculs.\\

\begin{multicols}{2}
\noindent
$A=6\times(x+4)$ pour $x=3$\\
\reponse[3]
\noindent
$C=t^{2}-8$ pour $t=4$\\
\reponse[3]




\columnbreak
\noindent
$B=(5+x)\times7$ pour $x=10$\\
\reponse[3]
\noindent
$B=(5+x)\times7$ pour $x=10$\\
\reponse[3]



\end{multicols}
\end{document}
