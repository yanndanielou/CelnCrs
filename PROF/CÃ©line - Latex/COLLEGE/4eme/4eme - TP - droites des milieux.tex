\documentclass[a4paper,11pt]{article}
\usepackage{amsmath,amsthm,amsfonts,amssymb,amscd,amstext,vmargin,graphics,graphicx,tabularx,multicol} 
\usepackage[francais]{babel}
\usepackage[utf8]{inputenc}  
\usepackage[T1]{fontenc} 
\usepackage{pstricks-add,tikz,tkz-tab,variations}
\usepackage[autolanguage,np]{numprint} 

\setmarginsrb{1.5cm}{0.5cm}{1cm}{0.5cm}{0cm}{0cm}{0cm}{0cm} %Gauche, haut, droite, haut
\newcounter{numexo}
\newcommand{\exo}[1]{\stepcounter{numexo}\noindent{\bf Exercice~\thenumexo} : \marginpar{\hfill /#1}}
\reversemarginpar


\newcounter{enumtabi}
\newcounter{enumtaba}
\newcommand{\q}{\stepcounter{enumtabi} \theenumtabi.  }
\newcommand{\qa}{\stepcounter{enumtaba} (\alph{enumtaba}) }
\newcommand{\initq}{\setcounter{enumtabi}{0}}
\newcommand{\initqa}{\setcounter{enumtaba}{0}}

\newcommand{\be}{\begin{enumerate}}
\newcommand{\ee}{\end{enumerate}}
\newcommand{\bi}{\begin{itemize}}
\newcommand{\ei}{\end{itemize}}
\newcommand{\bp}{\begin{pspicture*}}
\newcommand{\ep}{\end{pspicture*}}
\newcommand{\bt}{\begin{tabular}}
\newcommand{\et}{\end{tabular}}
\renewcommand{\tabularxcolumn}[1]{>{\centering}m{#1}} %(colonne m{} centrée, au lieu de p par défault) 
\newcommand{\tnl}{\tabularnewline}

\newcommand{\trait}{\noindent \rule{\linewidth}{0.2mm}}
\newcommand{\hs}[1]{\hspace{#1}}
\newcommand{\vs}[1]{\vspace{#1}}

\newcommand{\N}{\mathbb{N}}
\newcommand{\Z}{\mathbb{Z}}
\newcommand{\R}{\mathbb{R}}
\newcommand{\C}{\mathbb{C}}
\newcommand{\Dcal}{\mathcal{D}}
\newcommand{\Ccal}{\mathcal{C}}
\newcommand{\mc}{\mathcal}

\newcommand{\vect}[1]{\overrightarrow{#1}}
\newcommand{\ds}{\displaystyle}
\newcommand{\eq}{\quad \Leftrightarrow \quad}
\newcommand{\vecti}{\vec{\imath}}
\newcommand{\vectj}{\vec{\jmath}}
\newcommand{\Oij}{(O;\vec{\imath}, \vec{\jmath})}
\newcommand{\OIJ}{(O;I,J)}


\newcommand{\reponse}[1][1]{%
\multido{}{#1}{\makebox[\linewidth]{\rule[0pt]{0pt}{20pt}\dotfill}
}}

\newcommand{\titre}[5] 
% #1: titre #2: haut gauche #3: bas gauche #4: haut droite #5: bas droite
{
\noindent #2 \hfill #4 \\
#3 \hfill #5

\vspace{-1.6cm}

\begin{center}\rule{6cm}{0.5mm}\end{center}
\vspace{0.2cm}
\begin{center}{\large{\textbf{#1}}}\end{center}
\begin{center}\rule{6cm}{0.5mm}\end{center}
}



\begin{document}
\pagestyle{empty}
\titre{TP : Droite des milieux}{Nom :}{Prénom :}{Classe}{Date}


\vspace*{1cm}


\exo{5}\textbf{Droite passant par le milieu de deux côtés d'un triangle}\\

\q Construire un triangle ABC quelconque.\\

\q Placer D le milieu du segment [AB] et E le milieu de segment [AC].\\

\q Tracer la droite (DE). Puis déplacer les points A, B et C.\\

\q  Conjectures :\\

\qa Comment semblent être les droites (DE) et (BC) ? (Sur le logiciel, utiliser l'outil "Relation entre deux objets".)\\
\reponse[1]\\

\qa Mesurer la longueur des segments [DE] et [BC]. Que peut-on dire des segments [DE] et [BC] ?\\
\reponse[1]

\begin{center}
\textbf{Évaluation  durant le TP :}
\end{center}

\begin{center}
\begin{tabular}{|c|c|}
\hline 
Triangle ABC  & \hspace*{1cm}/1 \\ 
\hline 
Points D et E & \hspace*{1cm}/1 \\ 
\hline 
Droite & \hspace*{1cm}/1 \\ 
\hline 

\end{tabular} 
\end{center}

\vspace*{0.6cm}

\exo{15} \textbf{Droite passant par le milieu d'un côté et parallèle à un autre côté}\\

\vspace*{0.5cm}

\textbf{ PARTIE 1 : Conjecturer}\\

\initq \q Construire un triangle ABC.\\

\q \initqa
\qa Placer le point D, le milieu du segment [AB].\\

\qa Tracer la droite parallèle à (BC) passant par D.\\

\q Placer E, le point d'intersection de cette droite avec le segment [AC].\\

\q Mesurer la longueur des segments [EA] et [EC].\\
\reponse[1]

\q Déplacer les points de la figure Que peut-on conjecturer sur le point E ?\\
\reponse[2]\\


\begin{center}
\textbf{Évaluation  durant le TP :}
\end{center}

\begin{center}
\begin{tabular}{|c|c|}
\hline 
Triangle ABC  & \hspace*{1cm}/1 \\ 
\hline 
Points D et E & \hspace*{1cm}/1 \\ 
\hline 
Droite & \hspace*{1cm}/1 \\ 
\hline 

\end{tabular} 
\end{center}

\newpage

\textbf{ PARTIE 2 : Démontrer}\\

\initq \q Placer F, le milieu de [BC].\\

Répondre aux questions suivantes à l'aide de vos connaissances en rédigeant des démonstrations. \\

\q 
\initqa 
\qa Démontrer que les droite (DF) et (EC) sont parallèles.\\
\reponse[5]\\

\qa Démontrer que le quadrilatère DFCE est un parallélogramme\\
\reponse[5]\\

\q 
\initqa \qa Démontrer que DF = EC\\
\reponse[4]\\

\qa $DF= \dfrac{AC}{2}$\\
\reponse[4]\\

\qa En déduire que le point E est le milieu du segment [AC].\\
\reponse[4]\\


\vspace*{2cm}
\begin{flushright}
\textbf{Note : ......./ 20 }
\end{flushright}

 


\end{document}
