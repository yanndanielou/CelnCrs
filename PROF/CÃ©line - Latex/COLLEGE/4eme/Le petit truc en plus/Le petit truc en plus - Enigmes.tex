\documentclass[a4paper,11pt]{article}
\usepackage{amsmath,amsthm,amsfonts,amssymb,amscd,amstext,vmargin,graphics,graphicx,tabularx,multicol} 
\usepackage[francais]{babel}
\usepackage[utf8]{inputenc}  
\usepackage[T1]{fontenc} 
\usepackage{pstricks-add,tikz,tkz-tab,variations}
\usepackage[autolanguage,np]{numprint} 

\setmarginsrb{1.5cm}{0.5cm}{1cm}{0.5cm}{0cm}{0cm}{0cm}{0cm} %Gauche, haut, droite, haut
\newcounter{numexo}
\newcommand{\exo}[1]{\stepcounter{numexo}\noindent{\bf Exercice~\thenumexo} : \marginpar{\hfill /#1}}
\reversemarginpar


\newcounter{enumtabi}
\newcounter{enumtaba}
\newcommand{\q}{\stepcounter{enumtabi} \theenumtabi.  }
\newcommand{\qa}{\stepcounter{enumtaba} (\alph{enumtaba}) }
\newcommand{\initq}{\setcounter{enumtabi}{0}}
\newcommand{\initqa}{\setcounter{enumtaba}{0}}

\newcommand{\be}{\begin{enumerate}}
\newcommand{\ee}{\end{enumerate}}
\newcommand{\bi}{\begin{itemize}}
\newcommand{\ei}{\end{itemize}}
\newcommand{\bp}{\begin{pspicture*}}
\newcommand{\ep}{\end{pspicture*}}
\newcommand{\bt}{\begin{tabular}}
\newcommand{\et}{\end{tabular}}
\renewcommand{\tabularxcolumn}[1]{>{\centering}m{#1}} %(colonne m{} centrée, au lieu de p par défault) 
\newcommand{\tnl}{\tabularnewline}

\newcommand{\trait}{\noindent \rule{\linewidth}{0.2mm}}
\newcommand{\hs}[1]{\hspace{#1}}
\newcommand{\vs}[1]{\vspace{#1}}

\newcommand{\N}{\mathbb{N}}
\newcommand{\Z}{\mathbb{Z}}
\newcommand{\R}{\mathbb{R}}
\newcommand{\C}{\mathbb{C}}
\newcommand{\Dcal}{\mathcal{D}}
\newcommand{\Ccal}{\mathcal{C}}
\newcommand{\mc}{\mathcal}

\newcommand{\vect}[1]{\overrightarrow{#1}}
\newcommand{\ds}{\displaystyle}
\newcommand{\eq}{\quad \Leftrightarrow \quad}
\newcommand{\vecti}{\vec{\imath}}
\newcommand{\vectj}{\vec{\jmath}}
\newcommand{\Oij}{(O;\vec{\imath}, \vec{\jmath})}
\newcommand{\OIJ}{(O;I,J)}


\newcommand{\reponse}[1][1]{%
\multido{}{#1}{\makebox[\linewidth]{\rule[0pt]{0pt}{20pt}\dotfill}
}}

\newcommand{\titre}[5] 
% #1: titre #2: haut gauche #3: bas gauche #4: haut droite #5: bas droite
{
\noindent #2 \hfill #4 \\
#3 \hfill #5

\vspace{-1.6cm}

\begin{center}\rule{6cm}{0.5mm}\end{center}
\vspace{0.2cm}
\begin{center}{\large{\textbf{#1}}}\end{center}
\begin{center}\rule{6cm}{0.5mm}\end{center}
}



\begin{document}
\pagestyle{empty}
\titre{Le petit truc en plus}{Nom :}{Prénom :}{Classe}{Date}


\vspace*{1cm}

\textbf{ÉNIGME 1 :}\\

Pour assurer la sécurité du musée, l'accès à la salle des trésors nécessite non pas un mais deux codes de 5 chiffres.\\
Étonnamment, les gardiens ont le même moyen mnémotechnique pour se souvenir de ces deux suites de chiffres : la somme du deuxième et du cinquième chiffre donne le quatrième.\\
Le dernier chiffre est plus grand que le premier et leur produit donne le deuxième.\\
La somme du premier et du deuxième chiffre donne le troisième.\\
Il n'y a pas les chiffres 0, 5 et 7.\\

Quels sont les deux codes qui permettent d'ouvrir la salle ?\\




\vspace*{1cm}

\textbf{ÉNIGME 2:}\\

En pénétrant dans le manoir de la comtesse de la Beaujoire, le cambrioleur savait qu'il se trouverait nez à nez avec ses chats et ses perroquets.\\
A peine est-il entré que dans la pénombre, il aperçoit trente yeux qui l'observent.\\
S'habituant à l'obscurité, il réussit à compter quarante-quatre pattes.\\

Combien la comtesse a-t-elle de chats et de perroquets ?\\


\vspace*{1cm}

\textbf{ÉNIGME 3:}\\

La police a enfin réussi à retrouver les traces de Boomer. Deux jours après le jour qui est le lendemain de la veille d'hier, la police monte les escaliers et frappe à sa porte.\\

Quand la police a-t-elle été chez Boomer ?\\

\vspace*{1cm}


\textit{\textbf{Vos réponses devront être toutes justifiées}}\\

\begin{flushright}
\fbox{{\LARGE \textbf{Note} : . . . . . .}  }
\end{flushright}


\end{document}
