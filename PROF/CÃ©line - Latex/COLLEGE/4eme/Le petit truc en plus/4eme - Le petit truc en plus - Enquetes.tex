\documentclass[a4paper,11pt]{article}
\usepackage{amsmath,amsthm,amsfonts,amssymb,amscd,amstext,vmargin,graphics,graphicx,tabularx,multicol} 
\usepackage[francais]{babel}
\usepackage[utf8]{inputenc}  
\usepackage[T1]{fontenc} 
\usepackage{pstricks-add,tikz,tkz-tab,variations}
\usepackage[autolanguage,np]{numprint} 

\setmarginsrb{1.5cm}{0.5cm}{1cm}{0.5cm}{0cm}{0cm}{0cm}{0cm} %Gauche, haut, droite, haut
\newcounter{numexo}
\newcommand{\exo}[1]{\stepcounter{numexo}\noindent{\bf Exercice~\thenumexo} : \marginpar{\hfill /#1}}
\reversemarginpar


\newcounter{enumtabi}
\newcounter{enumtaba}
\newcommand{\q}{\stepcounter{enumtabi} \theenumtabi.  }
\newcommand{\qa}{\stepcounter{enumtaba} (\alph{enumtaba}) }
\newcommand{\initq}{\setcounter{enumtabi}{0}}
\newcommand{\initqa}{\setcounter{enumtaba}{0}}

\newcommand{\be}{\begin{enumerate}}
\newcommand{\ee}{\end{enumerate}}
\newcommand{\bi}{\begin{itemize}}
\newcommand{\ei}{\end{itemize}}
\newcommand{\bp}{\begin{pspicture*}}
\newcommand{\ep}{\end{pspicture*}}
\newcommand{\bt}{\begin{tabular}}
\newcommand{\et}{\end{tabular}}
\renewcommand{\tabularxcolumn}[1]{>{\centering}m{#1}} %(colonne m{} centrée, au lieu de p par défault) 
\newcommand{\tnl}{\tabularnewline}

\newcommand{\trait}{\noindent \rule{\linewidth}{0.2mm}}
\newcommand{\hs}[1]{\hspace{#1}}
\newcommand{\vs}[1]{\vspace{#1}}

\newcommand{\N}{\mathbb{N}}
\newcommand{\Z}{\mathbb{Z}}
\newcommand{\R}{\mathbb{R}}
\newcommand{\C}{\mathbb{C}}
\newcommand{\Dcal}{\mathcal{D}}
\newcommand{\Ccal}{\mathcal{C}}
\newcommand{\mc}{\mathcal}

\newcommand{\vect}[1]{\overrightarrow{#1}}
\newcommand{\ds}{\displaystyle}
\newcommand{\eq}{\quad \Leftrightarrow \quad}
\newcommand{\vecti}{\vec{\imath}}
\newcommand{\vectj}{\vec{\jmath}}
\newcommand{\Oij}{(O;\vec{\imath}, \vec{\jmath})}
\newcommand{\OIJ}{(O;I,J)}


\newcommand{\reponse}[1][1]{%
\multido{}{#1}{\makebox[\linewidth]{\rule[0pt]{0pt}{20pt}\dotfill}
}}

\newcommand{\titre}[5] 
% #1: titre #2: haut gauche #3: bas gauche #4: haut droite #5: bas droite
{
\noindent #2 \hfill #4 \\
#3 \hfill #5

\vspace{-1.6cm}

\begin{center}\rule{6cm}{0.5mm}\end{center}
\vspace{0.2cm}
\begin{center}{\large{\textbf{#1}}}\end{center}
\begin{center}\rule{6cm}{0.5mm}\end{center}
}



\begin{document}
\pagestyle{empty}
\titre{{\Large Le petit truc en plus : Enquête !}}{Nom :}{Prénom :}{4ème}


\vspace*{0.5cm}

\textit{{\large A rendre avant le 23 septembre !}}

\vspace*{0.5cm}


\begin{flushright}
\fbox{{\LARGE \textbf{Note} : . . . . /10}  }
\end{flushright}

\textbf{A vous de mener l'enquête !}\\

On se propose d'étudier le temps qu'un élève de 4ème passe devant les écrans. \\

\bi
\item Dans un premier temps, il faut que vous choisissiez un sujet. \textit{(Surlignez-le) }\\

\underline{Proposition 1 :} Le temps passé devant son téléphone portable\\


\underline{Proposition 2 :} Le temps passé sur les réseaux sociaux\\

\underline{Proposition 3 :} Le temps passé devant les jeux vidéos\\


\underline{Proposition 4 :} Le temps passé devant la télévision (divertissements, séries, films...)\\

\item Ensuite, une fois votre sujet choisi, vous devez récolter des données. Pour cela, interrogez au collège un minimum de 30 camarades (\textit{plus c'est mieux}), et notez leur réponse.

\setlength{\fboxrule}{1pt}
\begin{flushleft}
\framebox{\begin{minipage}{\linewidth}

\vspace*{0.2cm}

LES RÉPONSES :

\vspace*{5cm}

\end{minipage}}
\end{flushleft}

\vspace*{0.7cm}


\item Complétez ensuite le tableau suivant avec vos valeurs.

\vspace*{0.7cm}

\renewcommand{\arraystretch}{2.5}


\begin{flushleft}
\begin{tabular}{|c|c|c|c|c|c|}
\hline 
\textbf{Intervalle de temps (en ...)} & \textbf{[\hspace*{0.2cm}... , ...\hspace*{0.2cm}[ } & \textbf{[\hspace*{0.2cm}... , ...\hspace*{0.2cm}[ }  & \textbf{[\hspace*{0.2cm}... , ...\hspace*{0.2cm}[ }& \textbf{[\hspace*{0.2cm}... , ...\hspace*{0.2cm}[ }& \textbf{[\hspace*{0.2cm}... , ...\hspace*{0.2cm}[ } \\
\hline 
\textbf{Effectifs} &  &  &  &  &     \\ 
\hline 
\textbf{Effectifs cumulés croissants} &  &  &  &  &    \\ 
\hline 
\textbf{Fréquences}\textbf{ (\%)} &  &  &  &  &     \\ 
\hline 
\end{tabular} 
\end{flushleft}


\vspace*{0.7cm}

\item  Représentez l'histogramme des effectifs ou des fréquences de cette étude statistique. \textit{(sur une grande feuille blanche)}



\ei

\end{document}
