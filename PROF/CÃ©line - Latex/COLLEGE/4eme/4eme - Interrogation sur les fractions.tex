\documentclass[a4paper,11pt]{article}
\usepackage{amsmath,amsthm,amsfonts,amssymb,amscd,amstext,vmargin,graphics,graphicx,tabularx,multicol} 
\usepackage[francais]{babel}
\usepackage[utf8]{inputenc}  
\usepackage[T1]{fontenc} 
\usepackage{pstricks-add,tikz,tkz-tab,variations}
\usepackage[autolanguage,np]{numprint} 

\setmarginsrb{1.5cm}{0.5cm}{1cm}{0.5cm}{0cm}{0cm}{0cm}{0cm} %Gauche, haut, droite, haut
\newcounter{numexo}
\newcommand{\exo}[1]{\stepcounter{numexo}\noindent{\bf Exercice~\thenumexo} : \marginpar{\hfill /#1}}
\reversemarginpar


\newcounter{enumtabi}
\newcounter{enumtaba}
\newcommand{\q}{\stepcounter{enumtabi} \theenumtabi.  }
\newcommand{\qa}{\stepcounter{enumtaba} (\alph{enumtaba}) }
\newcommand{\initq}{\setcounter{enumtabi}{0}}
\newcommand{\initqa}{\setcounter{enumtaba}{0}}

\newcommand{\be}{\begin{enumerate}}
\newcommand{\ee}{\end{enumerate}}
\newcommand{\bi}{\begin{itemize}}
\newcommand{\ei}{\end{itemize}}
\newcommand{\bp}{\begin{pspicture*}}
\newcommand{\ep}{\end{pspicture*}}
\newcommand{\bt}{\begin{tabular}}
\newcommand{\et}{\end{tabular}}
\renewcommand{\tabularxcolumn}[1]{>{\centering}m{#1}} %(colonne m{} centrée, au lieu de p par défault) 
\newcommand{\tnl}{\tabularnewline}

\newcommand{\trait}{\noindent \rule{\linewidth}{0.2mm}}
\newcommand{\hs}[1]{\hspace{#1}}
\newcommand{\vs}[1]{\vspace{#1}}

\newcommand{\N}{\mathbb{N}}
\newcommand{\Z}{\mathbb{Z}}
\newcommand{\R}{\mathbb{R}}
\newcommand{\C}{\mathbb{C}}
\newcommand{\Dcal}{\mathcal{D}}
\newcommand{\Ccal}{\mathcal{C}}
\newcommand{\mc}{\mathcal}

\newcommand{\vect}[1]{\overrightarrow{#1}}
\newcommand{\ds}{\displaystyle}
\newcommand{\eq}{\quad \Leftrightarrow \quad}
\newcommand{\vecti}{\vec{\imath}}
\newcommand{\vectj}{\vec{\jmath}}
\newcommand{\Oij}{(O;\vec{\imath}, \vec{\jmath})}
\newcommand{\OIJ}{(O;I,J)}


\newcommand{\bmul}[1]{\begin{multicols}{#1}}
\newcommand{\emul}{\end{multicols}}

\newcommand{\reponse}[1][1]{%
\multido{}{#1}{\makebox[\linewidth]{\rule[0pt]{0pt}{20pt}\dotfill}
}}

\newcommand{\titre}[5] 
% #1: titre #2: haut gauche #3: bas gauche #4: haut droite #5: bas droite
{
\noindent #2 \hfill #4 \\
#3 \hfill #5

\vspace{-1.6cm}

\begin{center}\rule{6cm}{0.5mm}\end{center}
\vspace{0.2cm}
\begin{center}{\large{\textbf{#1}}}\end{center}
\begin{center}\rule{6cm}{0.5mm}\end{center}
}



\begin{document}
\pagestyle{empty}
\titre{Interrogation : Écritures fractionnaires}{Nom :}{Prénom :}{Classe}{Date}

\exo{2}\\

\q Les fractions $\dfrac{-2}{3}$  et $\dfrac{11}{-16,5}$ sont-elles égales ? \textbf{Justifier votre réponse avec une propriété du cours.}\\
\reponse[3]\\

\q Les fractions $\dfrac{-5}{-13}$  et $\dfrac{7}{18}$ sont-elles égales ? \textbf{Justifier votre réponse avec une propriété du cours.}\\
\reponse[3]\\


\exo{2} Simplifier les fractions suivantes au maximum :\\

\bmul{4}

$A = \dfrac{6}{12}$\\
\reponse[2]


\columnbreak

$O = \dfrac{-15}{20}$\\
\reponse[2]

\columnbreak

$M = \dfrac{-54}{27}$\\
\reponse[2]

\columnbreak

$S = \dfrac{-32}{-24}$\\
\reponse[2]


\emul


\exo{4,5}


\bmul{3}

$G =  \dfrac{5}{2}  +  \dfrac{4}{2}$\\
\reponse[4]\\

$J =  \dfrac{2}{18}  -  \dfrac{4}{9} $\\
\reponse[4]\\

\columnbreak

$H =  \dfrac{-5}{4}  -  \dfrac{3}{4} $\\
\reponse[4]\\

$K =  \dfrac{3}{5}  +  \dfrac{2}{45}$ \\
\reponse[4]\\

\columnbreak

$I =  \dfrac{1}{3}  -  \dfrac{-1}{3} $\\
\reponse[4]\\

$L =  \dfrac{-3}{5} + \dfrac{7}{6} $\\
\reponse[4]\\

\emul


\exo{1,5}\\

D'après une étude menée en 2006, $\dfrac{3}{50}$ de l'énergie consommée dans le monde proviendrait du nucléaire, $\dfrac{4}{5}$ serait produite à partir de combustibles fossiles (pétrole, charbon ou gaz) et le reste proviendrait d'énergie renouvelables.\\

Calculer la proportion de l'énergie consommée dans le monde qui provient des énergies renouvelables.




	









\end{document}
