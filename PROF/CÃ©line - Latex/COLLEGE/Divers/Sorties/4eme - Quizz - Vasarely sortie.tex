\documentclass[a4paper,11pt]{article}
\usepackage{amsmath,amsthm,amsfonts,amssymb,amscd,amstext,vmargin,graphics,graphicx,tabularx,multicol} 
\usepackage[francais]{babel}
\usepackage[utf8]{inputenc}  
\usepackage[T1]{fontenc} 
\usepackage{pstricks-add,tikz,tkz-tab,variations}
\usepackage[autolanguage,np]{numprint} 
\usepackage{calc}
\usepackage{pifont}
\usepackage{lscape}

\setmarginsrb{1.5cm}{0.5cm}{1cm}{0.5cm}{0cm}{0cm}{0cm}{0cm} %Gauche, haut, droite, haut
\newcounter{numexo}
\newcommand{\exo}[1]{\stepcounter{numexo}\noindent{\bf Exercice~\thenumexo} : }
\reversemarginpar

\newcommand{\bmul}[1]{\begin{multicols}{#1}}
\newcommand{\emul}{\end{multicols}}

\newcounter{enumtabi}
\newcounter{enumtaba}
\newcommand{\q}{\stepcounter{enumtabi} \theenumtabi.  }
\newcommand{\qa}{\stepcounter{enumtaba} (\alph{enumtaba}) }
\newcommand{\initq}{\setcounter{enumtabi}{0}}
\newcommand{\initqa}{\setcounter{enumtaba}{0}}

\newcommand{\be}{\begin{enumerate}}
\newcommand{\ee}{\end{enumerate}}
\newcommand{\bi}{\begin{itemize}}
\newcommand{\ei}{\end{itemize}}
\newcommand{\bp}{\begin{pspicture*}}
\newcommand{\ep}{\end{pspicture*}}
\newcommand{\bt}{\begin{tabular}}
\newcommand{\et}{\end{tabular}}
\renewcommand{\tabularxcolumn}[1]{>{\centering}m{#1}} %(colonne m{} centrée, au lieu de p par défault) 
\newcommand{\tnl}{\tabularnewline}

\newcommand{\trait}{\noindent \rule{\linewidth}{0.2mm}}
\newcommand{\hs}[1]{\hspace{#1}}
\newcommand{\vs}[1]{\vspace{#1}}

\newcommand{\N}{\mathbb{N}}
\newcommand{\Z}{\mathbb{Z}}
\newcommand{\R}{\mathbb{R}}
\newcommand{\C}{\mathbb{C}}
\newcommand{\Dcal}{\mathcal{D}}
\newcommand{\Ccal}{\mathcal{C}}
\newcommand{\mc}{\mathcal}

\newcommand{\vect}[1]{\overrightarrow{#1}}
\newcommand{\ds}{\displaystyle}
\newcommand{\eq}{\quad \Leftrightarrow \quad}
\newcommand{\vecti}{\vec{\imath}}
\newcommand{\vectj}{\vec{\jmath}}
\newcommand{\Oij}{(O;\vec{\imath}, \vec{\jmath})}
\newcommand{\OIJ}{(O;I,J)}


\newcommand{\reponse}[1][1]{%
\multido{}{#1}{\makebox[\linewidth]{\rule[0pt]{0pt}{20pt}\dotfill}
}}

\newcommand{\titre}[5] 
% #1: titre #2: haut gauche #3: bas gauche #4: haut droite #5: bas droite
{
\noindent #2 \hfill #4 \\
#3 \hfill #5

\vspace{-1.6cm}

\begin{center}\rule{6cm}{0.5mm}\end{center}
\vspace*{0.1cm}
\begin{center}{\large{\textbf{#1}}}\end{center}
\begin{center}\rule{6cm}{0.5mm}\end{center}
}



\begin{document}




\titre{{\large Quizz : Centre Pompidou, la fontaine Stravinsky et Vasarely}}{}{}{4ème}{}

\vspace*{0.7cm}

\textbf{{\large \underline{Questions :}}}\\

\q Vous avez travaillé ce matin autour de la fontaine Stravinsky, savez-vous quel est l'autre nom qu'on lui donne? \\
\textit{Fontaines des Automates.}\\

\q Ce monument évoque l' \oe{uvre} de Igor Stravinsky, un compositeur russe du XXème siècle. \\
De combien de sculptures la fontaine est-elle composée ?\\
\textit{16 sculptures.}\\


\q Pouvez-vous me citer parmi les sculptures une sculpture qui fait référence à la musique ?\\
\textit{La clef de sol.}\\

\q Si vous avez fait attention, vous avez pu remarquer qu'il y avait des sculptures de couleurs et des sculptures monochromes. \\
Combien y-a-t-il  de sculptures monochromes ?\\
\textit{Il y en a 7.}\\

\q On va parler maintenant du grand bâtiment où nous allons aller voir l'exposition cette après-midi, le centre Pompidou.\\
Savez-vous qui a eu l'idée de le construire et en quelle année ?\\
\textit{Le président Georges Pompidou, en 1969.}\\

\q Un concours d'idées est alors lancé, et sont invités à participer 681 architectes du mondes entier de 49 pays différents. 3 projets sont alors retenu par le jury.\\
Quelles sont les nationalités  des architectes des projets retenu ?\\
\textit{2 italiens et un anglais.}\\

BONUS : Quel était le président du jury ? \textit{Jean Prouvé}\\

\q Quels sont les architectes retenu pour la conduite du projet et donc à l'origine du centre Pompidou ?\\
\textit{Renzo Piano et Richard Rogers.}\\



\q Au début de sa construction à quoi avait été comparé le bâtiment ?\\
\textit{A une raffinerie de pétrole.}\\

\q A quelle date a été inauguré le centre Pompidou ?\\
\textit{Le 31 janvier 1977.}\\

\q Quel était le président à l'époque ?\\
\textit{Valery Giscard d'Estaing}.\\

\q La particularité du bâtiment est que les tuyaux sont apparents et ont été peints de plusieurs couleurs.\\
Combien de couleurs sont présentes sur le bâtiment et quelles sont-elles ?\\
\textit{Jaune, vert, bleu, rouge.}\\

\newpage

\vspace*{0.4cm}

\q Ces couleurs ont une signification bien particulière, pouvez-vous me dire à quoi font référence 2 de ces couleurs ?\\
\textit{bleu = circulation de l'air, jaune = circulation de l'électricité, vert = circulation de l'eau et rouge = circulation de personnes.}\\

\q Tout à l'heure, nous irons assisté à une exposition d'un artiste hongrois Victor Vasarely. 
Quelles sont ses dates de naissance et de mort ?\\
\textit{1906 / 1997.}\\


\q C'est un artiste très connu aujourd'hui  mais à l'époque il se destinait à une toute autre carrière. Pouvez-vous me dire vers dans domaine souhaitait-il travaillé?\\
\textit{la médecine.}\\

\q  Il est aujourd'hui reconnu comme le père fondateur d'un grand mouvement artistique. Quel est-il? \\
\textit{L'Op Art.}\\

\q Vasarely va travailler pendant sa carrière aussi pour de nombreuses entreprises. En 1972 c'est la firme Renault fait appel à lui.\\
Savez-vous pourquoi?\\
\textit{La firme Renault qui souhaite se développer à l'international, confie à Vasarely le soin de dépoussiérer son vieux logo.}\\

\q Dès 1973, les travaux de construction de la Fondation de Vasarely débutent. Savez-vous dans quel ville ?\\
\textit{Aix-en-Provence.}\\



\end{document}
