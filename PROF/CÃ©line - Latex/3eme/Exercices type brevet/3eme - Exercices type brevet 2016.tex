\documentclass[a4paper,11pt]{article}
\usepackage{amsmath,amsthm,amsfonts,amssymb,amscd,amstext,vmargin,graphics,graphicx,tabularx,multicol} \usepackage[french]{babel}
\usepackage[utf8]{inputenc}  
\usepackage[T1]{fontenc} 
\usepackage[T1]{fontenc}
\usepackage{amsmath,amssymb}
\usepackage{pstricks-add,tikz,tkz-tab,variations}
\usepackage[autolanguage,np]{numprint} 

\setmarginsrb{1.5cm}{0.5cm}{1cm}{0.5cm}{0cm}{0cm}{0cm}{0cm} %Gauche, haut, droite, haut
\newcounter{numexo}
\newcommand{\exo}[1]{\stepcounter{numexo}\noindent{\bf Exercice~\thenumexo} : \marginpar{\hfill /#1}}
\reversemarginpar
\newcommand{\initexo}{\setcounter{numexo}{0}}



\newcounter{enumtabi}
\newcounter{enumtaba}
\newcommand{\q}{\stepcounter{enumtabi} \theenumtabi.  }
\newcommand{\qa}{\stepcounter{enumtaba} (\alph{enumtaba}) }
\newcommand{\initq}{\setcounter{enumtabi}{0}}
\newcommand{\initqa}{\setcounter{enumtaba}{0}}

\newcommand{\be}{\begin{enumerate}}
\newcommand{\ee}{\end{enumerate}}
\newcommand{\bi}{\begin{itemize}}
\newcommand{\ei}{\end{itemize}}
\newcommand{\bp}{\begin{pspicture*}}
\newcommand{\ep}{\end{pspicture*}}
\newcommand{\bt}{\begin{tabular}}
\newcommand{\et}{\end{tabular}}
\renewcommand{\tabularxcolumn}[1]{>{\centering}m{#1}} %(colonne m{} centrée, au lieu de p par défault) 
\newcommand{\tnl}{\tabularnewline}

\newcommand{\trait}{\noindent \rule{\linewidth}{0.2mm}}
\newcommand{\hs}[1]{\hspace{#1}}
\newcommand{\vs}[1]{\vspace{#1}}

\newcommand{\N}{\mathbb{N}}
\newcommand{\Z}{\mathbb{Z}}
\newcommand{\R}{\mathbb{R}}
\newcommand{\C}{\mathbb{C}}
\newcommand{\Dcal}{\mathcal{D}}
\newcommand{\Ccal}{\mathcal{C}}
\newcommand{\mc}{\mathcal}

\newcommand{\vect}[1]{\overrightarrow{#1}}
\newcommand{\ds}{\displaystyle}
\newcommand{\eq}{\quad \Leftrightarrow \quad}
\newcommand{\vecti}{\vec{\imath}}
\newcommand{\vectj}{\vec{\jmath}}
\newcommand{\Oij}{(O;\vec{\imath}, \vec{\jmath})}
\newcommand{\OIJ}{(O;I,J)}

\newcommand{\bmul}[1]{\begin{multicols}{#1}}
\newcommand{\emul}{\end{multicols}}


\newcommand{\reponse}[1][1]{%
\multido{}{#1}{\makebox[\linewidth]{\rule[0pt]{0pt}{20pt}\dotfill}
}}

\newcommand{\titre}[5] 
% #1: titre #2: haut gauche #3: bas gauche #4: haut droite #5: bas droite
{
\noindent #2 \hfill #4 \\
#3 \hfill #5

\vspace{-1.6cm}

\begin{center}\rule{6cm}{0.5mm}\end{center}
\vspace{0.2cm}
\begin{center}{\large{\textbf{#1}}}\end{center}
\begin{center}\rule{6cm}{0.5mm}\end{center}
}



\begin{document}
\pagestyle{empty}
\titre{Exercices type brevet }{Nom :}{Prénom :}{Classe}{Date}



\exo{3}

Une grossiste en fleurs a reçu un lot de 5 815 tulipes et 3 489 roses. Elle veut réaliser des bouquets tous identiques, composés de roses et de tulipes, en utilisant toutes les fleurs.

\q Quel nombre maximal de bouquets peut-elle composer ?

\q Une rose est vendue 1,80 euros, ; une tulipe est vendue 0,90 euros.\\
Combien sera vendu l'un de ces bouquets ?\\



\exo{2}

En allant au collège, Pauline dit à Maxime : « J'ai plus de 400 CD mais moins de 450. En les groupant par 2, ou par 3, ou par 4, ou par 5, c'est toujours la même chose : il en reste un tout seul. »\\

Combien de CD Pauline possède-t-elle ? \textbf{(Justifier votre réponse)}


\vspace*{1cm}

\titre{Exercices type brevet }{Nom :}{Prénom :}{Classe}{Date}


\initexo
\exo{3}
 
Une grossiste en fleurs a reçu un lot de 1756 tulipes et 1317 roses. Elle veut réaliser des bouquets tous identiques, composés de roses et de tulipes, en utilisant toutes les fleurs.

\initq \q Quel nombre maximal de bouquets peut-elle composer ?

\q Une rose est vendue 1,80 euros, ; une tulipe est vendue 0,90 euros.\\
Combien sera vendu l'un de ces bouquets ?\\



\exo{2}

En allant au collège, Pauline dit à Maxime : « J'ai plus de 400 CD mais moins de 450. En les groupant par 2, ou par 3, ou par 4, ou par 5, c'est toujours la même chose : il en reste un tout seul. »\\

Combien de CD Pauline possède-t-elle ? \textbf{(Justifier votre réponse)}


\vspace*{1cm}

\titre{Exercices type brevet }{Nom :}{Prénom :}{Classe}{Date}


\initexo
\exo{3}
 
Une grossiste en fleurs a reçu un lot de 1756 tulipes et 1317 roses. Elle veut réaliser des bouquets tous identiques, composés de roses et de tulipes, en utilisant toutes les fleurs.

\initq \q Quel nombre maximal de bouquets peut-elle composer ?

\q Une rose est vendue 1,80 euros, ; une tulipe est vendue 0,90 euros.\\
Combien sera vendu l'un de ces bouquets ?\\



\exo{2}

En allant au collège, Pauline dit à Maxime : « J'ai plus de 400 CD mais moins de 450. En les groupant par 2, ou par 3, ou par 4, ou par 5, c'est toujours la même chose : il en reste un tout seul. »\\

Combien de CD Pauline possède-t-elle ? \textbf{(Justifier votre réponse)}




\end{document}
