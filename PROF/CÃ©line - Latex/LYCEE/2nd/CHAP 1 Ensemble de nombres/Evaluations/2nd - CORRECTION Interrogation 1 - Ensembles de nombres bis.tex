\documentclass[a4paper,11pt]{article}
\usepackage{amsmath,amsthm,amsfonts,amssymb,amscd,amstext,vmargin,graphics,graphicx,tabularx,multicol} 
\usepackage[francais]{babel}
\usepackage[utf8]{inputenc}  
\usepackage[T1]{fontenc} 
\usepackage{pstricks-add,tikz,tkz-tab,variations}
\usepackage[autolanguage,np]{numprint} 

\setmarginsrb{1.5cm}{0.5cm}{1cm}{0.5cm}{0cm}{0cm}{0cm}{0cm} %Gauche, haut, droite, haut
\newcounter{numexo}
\newcommand{\exo}[1]{\stepcounter{numexo}\noindent{\bf Exercice~\thenumexo} : \marginpar{\hfill /#1}}
\reversemarginpar


\newcounter{enumtabi}
\newcounter{enumtaba}
\newcommand{\q}{\textbf{\stepcounter{enumtabi} \theenumtabi)}  }
\newcommand{\qa}{\textbf{\stepcounter{enumtaba} (\alph{enumtaba})} }
\newcommand{\initq}{\setcounter{enumtabi}{0}}
\newcommand{\initqa}{\setcounter{enumtaba}{0}}

\newcommand{\be}{\begin{enumerate}}
\newcommand{\ee}{\end{enumerate}}
\newcommand{\bi}{\begin{itemize}}
\newcommand{\ei}{\end{itemize}}
\newcommand{\bp}{\begin{pspicture*}}
\newcommand{\ep}{\end{pspicture*}}
\newcommand{\bt}{\begin{tabular}}
\newcommand{\et}{\end{tabular}}
\renewcommand{\tabularxcolumn}[1]{>{\centering}m{#1}} %(colonne m{} centrée, au lieu de p par défault) 
\newcommand{\tnl}{\tabularnewline}

\newcommand{\bmul}[1]{\begin{multicols}{#1}}
\newcommand{\emul}{\end{multicols}}

\newcommand{\trait}{\noindent \rule{\linewidth}{0.2mm}}
\newcommand{\hs}[1]{\hspace{#1}}
\newcommand{\vs}[1]{\vspace{#1}}

\newcommand{\N}{\mathbb{N}}
\newcommand{\Z}{\mathbb{Z}}
\newcommand{\R}{\mathbb{R}}
\newcommand{\C}{\mathbb{C}}
\newcommand{\Dcal}{\mathcal{D}}
\newcommand{\Ccal}{\mathcal{C}}
\newcommand{\mc}{\mathcal}

\newcommand{\vect}[1]{\overrightarrow{#1}}
\newcommand{\ds}{\displaystyle}
\newcommand{\eq}{\quad \Leftrightarrow \quad}
\newcommand{\vecti}{\vec{\imath}}
\newcommand{\vectj}{\vec{\jmath}}
\newcommand{\Oij}{(O;\vec{\imath}, \vec{\jmath})}
\newcommand{\OIJ}{(O;I,J)}


\newcommand{\reponse}[1][1]{%
\multido{}{#1}{\makebox[\linewidth]{\rule[0pt]{0pt}{20pt}\dotfill}
}}

\newcommand{\titre}[5] 
% #1: titre #2: haut gauche #3: bas gauche #4: haut droite #5: bas droite
{
\noindent #2 \hfill #4 \\
#3 \hfill #5

\vspace{-1.6cm}

\begin{center}\rule{6cm}{0.5mm}\end{center}
\vspace{0.2cm}
\begin{center}{\large{\textbf{#1}}}\end{center}
\begin{center}\rule{6cm}{0.5mm}\end{center}
}



\begin{document}
\pagestyle{empty}
\titre{Correction Interrogation 1 : Ensembles de nombres }{Nom :}{Prénom :}{\textbf{2nd 8}}{Date:}


\exo{3} QUESTIONS DE COURS\\ 

\initq \q Donner la définition de l'ensemble des nombres décimaux.\\
\color{purple}
Voir dans votre leçon !\\
\color{black}

\q Compléter les phrases suivantes :\\

\initqa \qa $\Z$ est \textcolor{purple}{l'ensemble des entiers relatifs}\\

\qa \textcolor{purple}{$\mathbb{Q}^+$} est l'ensemble des nombres rationnels positifs.\\

\qa $\mathbb{D}^*$ est \textcolor{purple}{l'ensemble des décimaux non nuls}\\

\qa \textcolor{purple}{$\mathbb{N}^-$} est l'ensemble des nombres entiers naturels négatifs\\

\q Représenter le diagramme de Venn avec tous les ensembles de nombres vus en classe.\\
\color{purple}
Voir dans votre leçon !\\
\color{black}

\exo{2} Citer : \\
\initqa \qa un nombre appartenant à $\mathbb{D}$ mais pas à $\mathbb{Z}$ \hspace*{0.5cm} 	\qa un nombre appartenant à $\mathbb{R}$ mais pas à $\mathbb{Q}$.\\

\color{purple}
\initqa \qa $2,15 \in \mathbb{D} $ et $2,15 \notin \mathbb{Z}$ \hspace*{0.5cm}
\qa $\dfrac{\pi}{3} \in \mathbb{R} $ et $\dfrac{\pi}{3} \notin \mathbb{Q}$\\


\color{black}
\vspace*{0.25cm}

\exo{4}
Compléter en utilisant le symbole qui convient parmi $\in$, $\notin$, $\subset$ ou $\not\subset$ les phrases suivantes :\\

\hspace*{0.5cm}$\mathbb{N}$ \textcolor{purple}{$\subset$} $\Z$\hspace*{0.35cm}; \hspace*{0.35cm}  $-2,6767\underline{67}$ \textcolor{purple}{$\in$} $\mathbb{Q}$ \hspace*{0.35cm}; \hspace*{0.35cm} $\dfrac{3}{4}$ \textcolor{purple}{$\in$} $\mathbb{D	}$ \hspace*{0.35cm}; \hspace*{0.35cm} $\dfrac{7}{9}$ \textcolor{purple}{$\notin$} $\mathbb{D	}$ \\

\hspace*{0.5cm}$\sqrt{36}$ \textcolor{purple}{$\in$} $\Z$ \hspace*{0.35cm}; \hspace*{0.35cm}  $\{-1;1\}$ \textcolor{purple}{$\not\subset$} $\N$ \hspace*{0.35cm}; \hspace*{0.35cm} $\{2 ; \dfrac{10}{2}\}$ \textcolor{purple}{$\subset$} $\N$ \hspace*{0.35cm}; \hspace*{0.35cm} $\mathbb{Q}$ \textcolor{purple}{$\not\subset$} $\mathbb{D}$ \\

\end{document}
