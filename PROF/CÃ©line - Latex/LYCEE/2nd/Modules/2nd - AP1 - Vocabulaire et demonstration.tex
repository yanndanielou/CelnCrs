\documentclass[a4paper,11pt]{article}
\usepackage{amsmath,amsthm,amsfonts,amssymb,amscd,amstext,vmargin,graphics,graphicx,tabularx,multicol} 
\usepackage[francais]{babel}
\usepackage[utf8]{inputenc}  
\usepackage[T1]{fontenc} 
\usepackage{pstricks-add,tikz,tkz-tab,variations}
\usepackage[autolanguage,np]{numprint} 
\usepackage{calc}
\usepackage{mathrsfs}

\setmarginsrb{1.5cm}{0.5cm}{1cm}{0.5cm}{0cm}{0cm}{0cm}{0cm} %Gauche, haut, droite, haut
\newcounter{numexo}
\newcommand{\exo}[1]{\stepcounter{numexo}\noindent{\bf Exercice~\thenumexo} : }
\reversemarginpar

\newcommand{\bmul}[1]{\begin{multicols}{#1}}
\newcommand{\emul}{\end{multicols}}

\renewcommand{\thesection}{\Roman{section}.}
	\renewcommand{\thesubsection}{\hspace{.5cm}\arabic{subsection}.}
	\renewcommand{\thesubsubsection}{\hspace{1cm}\alph{subsubsection})}

\newcounter{enumtabi}
\newcounter{enumtaba}
\newcommand{\q}{\stepcounter{enumtabi} \theenumtabi)  }
\newcommand{\qa}{\stepcounter{enumtaba} (\alph{enumtaba}) }
\newcommand{\initq}{\setcounter{enumtabi}{0}}
\newcommand{\initqa}{\setcounter{enumtaba}{0}}

\newcommand{\be}{\begin{enumerate}}
\newcommand{\ee}{\end{enumerate}}
\newcommand{\bi}{\begin{itemize}}
\newcommand{\ei}{\end{itemize}}
\newcommand{\bp}{\begin{pspicture*}}
\newcommand{\ep}{\end{pspicture*}}
\newcommand{\bt}{\begin{tabular}}
\newcommand{\et}{\end{tabular}}
\renewcommand{\tabularxcolumn}[1]{>{\centering}m{#1}} %(colonne m{} centrée, au lieu de p par défault) 
\newcommand{\tnl}{\tabularnewline}

\newcommand{\trait}{\noindent \rule{\linewidth}{0.2mm}}
\newcommand{\hs}[1]{\hspace{#1}}
\newcommand{\vs}[1]{\vspace{#1}}

\newcommand{\N}{\mathbb{N}}
\newcommand{\Z}{\mathbb{Z}}
\newcommand{\R}{\mathbb{R}}
\newcommand{\C}{\mathbb{C}}
\newcommand{\Dcal}{\mathcal{D}}
\newcommand{\Ccal}{\mathcal{C}}
\newcommand{\mc}{\mathcal}

\newcommand{\vect}[1]{\overrightarrow{#1}}
\newcommand{\ds}{\displaystyle}
\newcommand{\eq}{\quad \Leftrightarrow \quad}
\newcommand{\vecti}{\vec{\imath}}
\newcommand{\vectj}{\vec{\jmath}}
\newcommand{\Oij}{(O;\vec{\imath}, \vec{\jmath})}
\newcommand{\OIJ}{(O;I,J)}


\newcommand{\reponse}[1][1]{%
\multido{}{#1}{\makebox[\linewidth]{\rule[0pt]{0pt}{20pt}\dotfill}
}}

\newcommand{\titre}[5] 
% #1: titre #2: haut gauche #3: bas gauche #4: haut droite #5: bas droite
{
\noindent #2 \hfill #4 \\
#3 \hfill #5

\vspace{-1.6cm}

\begin{center}\rule{6cm}{0.5mm}\end{center}
\vspace{0.2cm}
\begin{center}{\Large{\textbf{#1}}}\end{center}
\begin{center}\rule{6cm}{0.5mm}\end{center}
}



\begin{document}
\pagestyle{empty}

\titre{Séance d'AP 1 : Outils pour la démonstration}{}{}{2nd}{}

\vspace*{0.2cm}

\section{Vocabulaire}



\textbf{\textcolor{purple}{\underline{PROPRIETE}}}\\
Une propriété mathématique est une affirmation qui est toujours vraie. Elle ne comporte aucune exception.\\

\underline{Exemple :}\\
\textit{L'énoncé : " Si M est un point de la médiatrice du segment [AB] alors M est équidistant de A et de B" est toujours vrai : c'est donc une propriété.}\\
\vspace*{0.5cm}

\textbf{\textcolor{violet}{\underline{HYPOTHESE - CONCLUSION}}}\\
On note P la phrase : "M est un point de la médiatrice du segment [AB] "\\
On note Q la phrase : " M est équidistant de A et B".\\
On dit que P implique Q et on note alors $P \Rightarrow Q$.\\
P est appelé l'hypothèse ; Q est appelé la conclusion.\\
\vspace*{0.5cm}

\textbf{\textcolor{purple}{\underline{RECIPROQUE}}}\\
L'énoncé réciproque d'une propriété s'obtient en inversant conclusion et hypothèse. Si l'énoncé réciproque est vrai, il est appelé propriété réciproque.\\

\underline{Exemples :}
\bi
\item \textit{L'énoncé réciproque de la propriété ci-dessus est : " Si M est équidistant de A et de B alors M appartient à la médiatrice du segment [AB] " . Cet énoncé est vrai, c’est donc une propriété réciproque .}
\item \textit{L'énoncé " si un quadrilatère est un losange alors ses diagonales sont perpendiculaires " est un énoncé vrai : c’est donc une propriété. L'énoncé réciproque est faux.}\\
\ei
\vspace*{0.5cm}

\textbf{\textcolor{violet}{\underline{EQUIVALENCE}}}\\
Quand l'énoncé réciproque d'une propriété est vrai, on peut regrouper propriété et propriété réciproque en un seul énoncé utilisant l'expression si et seulement si.\\
 On dit que l'on a une équivalence et on note : $P \Leftrightarrow Q$.\\
 
 \underline{Exemple :}\\
\textit{Un point appartient à la médiatrice d’un segment si et seulement si il est équidistant des deux extrémités de ce segment.}\\
 
\section{Démontrer}

\textbf{\textcolor{violet}{\underline{VERIFIER}}}\\
Vérifier une affirmation sur quelques exemples n’est pas démontrer.\\

\underline{Exemple :}\\
\textit{Quelqu'un affirme : "Pour toutes les valeurs de $x$, le nombre $x^{2}-x + 41$ est un nombre premier".}\\
\textit{On peut vérifier cette affirmation en remplaçant x par 0, 1, 2,.....}\\
\textit{Le fait de trouver toujours un nombre premier n'est pas une démonstration.}\\
\textit{D'ailleurs, l'affirmation est fausse ( pour $x = 41$; $41^{2}-41+41$ = 1681 = $41^{2}$).}\\
\newpage

\textbf{\textcolor{violet}{\underline{VOIR}}}\\
Voir sur une figure n'est pas démontrer.\\

\underline{Exemple :}\\
\textit{Construire un triangle ABC isocèle en A, de base 8 cm et de hauteur AH mesurant 7 cm. \\
On voit très bien que le triangle est équilatéral, mais cela ne le démontre pas. D'ailleurs, il ne l'est pas. ( Utiliser Pythagore
pour montrer que $AB = \sqrt{65}\approx 8,06$ )}\\

\vspace*{0.5cm}


\textbf{\textcolor{violet}{\underline{CONJECTURER}}}\\
Conjecturer c'est formuler (supposer, deviner, imaginer, émettre, . . . ) des hypothèses.\\

\underline{Exemples :}
\bi
\item \textit{Voir sur une figure et vérifier sont des étapes permettant de conjecturer. Mais après la conjecture, il faut démontrer ce que l'on a supposé . Une conjecture peut être vraie ou fausse.}
\item  \textit{Il y a des conjecture célèbres : Goldbach (1690-1764)" tout nombre pair supérieur à deux est somme de deux nombres premiers ". Cette conjecture n'est toujours pas prouvée mais il n'y a toujours pas
de contre-exemple.}\\
\ei

\vspace*{0.5cm}


\textbf{\textcolor{purple}{\underline{PROUVER - MONTRER - DEMONTRER}}}\\
C'est réaliser un raisonnement qui est rédigé à partir des données du problème, grâce aux outils de la démonstration( définitions ou propriétés).\\
En déduire que c'est utiliser impérativement le résultat de la question précédente dans un nouveau raisonnement.\\


\section{Les différents raisonnements}

\subsection{Prouver que quelque chose est vrai.}

\textbf{\textcolor{purple}{\underline{DEDUCTION}}}\\
Effectuer un raisonnement déductif, c'est à partir des données du problème, aboutir à la conclusion souhaitée en utilisant les outils du cours ( définitions, propriétés, formules. . . ).\\

\underline{Exemple :} \textit{Voir l'exercice 3}




\subsection{Prouver que quelque chose est faux.}

\textbf{\textcolor{purple}{\underline{CONTRE-EXEMPLE}}}\\
Raisonner par contre-exemple, c'est trouver un exemple qui met en échec la conclusion de la proposition tout en respectant les hypothèses de celle-ci.\\

\underline{Exemple :}\\
\textit{Quelqu'un affirme : " Si les diagonales d’un quadrilatère sont égales, alors c'est un rectangle."\\
On peut dire que c'est faux, et pour le démontrer, il suffit de dessiner un quadrilatère dont les diagonales sont de même longueur, mais qui ne soit pas un rectangle.}\\


\vspace*{0.5cm}

\textbf{\textcolor{purple}{\underline{RAISONNEMENT PAR L'ABSURDE}}}\\
Raisonner par l'absurde c'est prendre pour hypothèse la négation du résultat à démontrer, puis effectuer un raisonnement déductif qui amène à une contradiction avec une donnée du problème ou avec une propriété
connue.\\

\underline{Exemple :}\\
\textit{Démontrer que le triangle dont les côtés ont pour longueur 5, 6 et 7 cm n’est pas un triangle rectangle.}\\

\textit{On suppose que ce triangle est rectangle. D'après la propriété de Pythagore, on devrait avoir "$7^{2} = 6^{2} + 5^{2}$".\\
Or $7^{2}=49$ et $6^{2} + 5^{2}= 36+ 25 = 61$. L'égalité est fausse, la supposition faite est donc fausse, d'où le triangle n'est pas rectangle.
} \\


\newpage


\vspace*{0.5cm}

\begin{large}
\textbf{EXERCICE 1}\\
\end{large}
Pour chaque ligne du tableau, dire si la proposition P implique la proposition Q, si la proposition Q implique la proposition P ou s'il y a équivalence.\\

\renewcommand{\arraystretch}{2}

\begin{tabular}{|p{5cm}|p{5cm}|c|c|c|}
\hline 
\textbf{Proposition $P$} & \textbf{Proposition $Q$} & $\mathbf{P \Rightarrow Q}$ & $\mathbf{Q \Rightarrow P}$ & $\mathbf{P \Leftrightarrow Q}$ \\ 
\hline 
M est une point de la médiatrice
de [AB] & M équidistant de A et de B & • & • & • \\ 
\hline 
Je réside en France & Je réside en Europe & • & • & • \\ 
\hline 
Je suis majeur(e) & J’ai 19 ans & • & • & • \\ 
\hline 
CDEF est un parallélogramme & CDEF est un carré & • & • & • \\ 
\hline 
$x = 3$ & $x^{2} = 9$ & • & • & • \\ 
\hline 
MNP est rectangle en M & $MP^{2} +MN^{2} = NP^{2}$ & • & • & • \\ 
\hline 
$x\ge -2$ & $x\ge -1$ & • & • & • \\ 
\hline 
$a+b=5$& a = 2 et b = 3 & • & • & • \\ 
\hline 
$4x-(x-5)=7$ & $x=\dfrac{2}{3}$ & • & • & • \\ 
\hline 
$(ax+b)(cx+d)=0$ &  $ax+b=0$ ou $cx+d=0$ & • & • &\\ 
\hline 
\end{tabular} 


\vspace*{1cm}


\begin{large}
\textbf{EXERCICE 2}\\
\end{large}
Toutes les affirmations suivantes sont fausses. Pour chacune, donner un contre exemple.\\

\noindent \q Si $x^{2}$ > 4, alors $x > 2$.\\
\q Pour tout couple de réels $(x; y)$, on a $(x + y)^{3} = x^{3} + y^{3}$.\\
\q Si $x^{2}$ = 9 alors $x = 3$.\\
\q Pour tout couple de réels positifs $(a; b)$, on a $\sqrt{a} + \sqrt{b} = \sqrt{a+b}$.\\
\q Pour tout réel $p$, le réel $−10p$ est négatif.\\
\q Tous les réels ont un inverse.\\
\q Tous les multiples de 5 sont des multiples de 10.\\
\q Si $x(x − 3) = 0$, alors $x = 3$.\\
\q Si $x < 1$, alors $x < 0$.\\
\q Si $x < 2$, alors $x^{2} < 4$.\\
\q Pour tout $x$, $−x$ est un nombre négatif.\\
\q Pour tout entier n, si n est divisible par 3, il est divisible par 6.\\
\q Si $1 \le x \le 3$ alors $x \in ]1;3[$.\\
\q Si $x \in [1 ; 5[$, alors $1 \le x \le 5$.\\
\q Si $x \in [0 ; 10]$, alors $x$ est un entier naturel.\\

\vspace*{1cm}

\begin{large}
\textbf{EXERCICE 3}\\
\end{large}
Pour aller plus loin.\\
 Ecrire la démonstration de la propriété suivante : "La somme de deux nombres impairs est un nombre pair."\\

\end{document}
