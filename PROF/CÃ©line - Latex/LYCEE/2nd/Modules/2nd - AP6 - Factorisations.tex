\documentclass[a4paper,12pt]{article}
\usepackage{amsmath,amsthm,amsfonts,amssymb,amscd,amstext,vmargin,graphics,graphicx,tabularx,multicol} 
\usepackage[francais]{babel}
\usepackage[utf8]{inputenc}  
\usepackage[T1]{fontenc} 
\usepackage{pstricks-add,tikz,tkz-tab,variations}
\usepackage[autolanguage,np]{numprint} 
\usepackage{calc}
\usepackage{mathrsfs}

\setmarginsrb{1.5cm}{0.5cm}{1cm}{0.5cm}{0cm}{0cm}{0cm}{0cm} %Gauche, haut, droite, haut
\newcounter{numexo}
\newcommand{\exo}[1]{\stepcounter{numexo}\noindent{\bf Exercice~\thenumexo} : }
\reversemarginpar

\newcommand{\bmul}[1]{\begin{multicols}{#1}}
\newcommand{\emul}{\end{multicols}}

\renewcommand{\thesection}{\Roman{section}.}
	\renewcommand{\thesubsection}{\hspace{.5cm}\arabic{subsection}.}
	\renewcommand{\thesubsubsection}{\hspace{1cm}\alph{subsubsection})}

\newcounter{enumtabi}
\newcounter{enumtaba}
\newcommand{\q}{\stepcounter{enumtabi} \theenumtabi)  }
\newcommand{\qa}{\stepcounter{enumtaba} (\alph{enumtaba}) }
\newcommand{\initq}{\setcounter{enumtabi}{0}}
\newcommand{\initqa}{\setcounter{enumtaba}{0}}

\newcommand{\be}{\begin{enumerate}}
\newcommand{\ee}{\end{enumerate}}
\newcommand{\bi}{\begin{itemize}}
\newcommand{\ei}{\end{itemize}}
\newcommand{\bp}{\begin{pspicture*}}
\newcommand{\ep}{\end{pspicture*}}
\newcommand{\bt}{\begin{tabular}}
\newcommand{\et}{\end{tabular}}
\renewcommand{\tabularxcolumn}[1]{>{\centering}m{#1}} %(colonne m{} centrée, au lieu de p par défault) 
\newcommand{\tnl}{\tabularnewline}

\newcommand{\trait}{\noindent \rule{\linewidth}{0.2mm}}
\newcommand{\hs}[1]{\hspace{#1}}
\newcommand{\vs}[1]{\vspace{#1}}

\newcommand{\N}{\mathbb{N}}
\newcommand{\Z}{\mathbb{Z}}
\newcommand{\R}{\mathbb{R}}
\newcommand{\C}{\mathbb{C}}
\newcommand{\Dcal}{\mathcal{D}}
\newcommand{\Ccal}{\mathcal{C}}
\newcommand{\mc}{\mathcal}

\newcommand{\vect}[1]{\overrightarrow{#1}}
\newcommand{\ds}{\displaystyle}
\newcommand{\eq}{\quad \Leftrightarrow \quad}
\newcommand{\vecti}{\vec{\imath}}
\newcommand{\vectj}{\vec{\jmath}}
\newcommand{\Oij}{(O;\vec{\imath}, \vec{\jmath})}
\newcommand{\OIJ}{(O;I,J)}


\newcommand{\reponse}[1][1]{%
\multido{}{#1}{\makebox[\linewidth]{\rule[0pt]{0pt}{20pt}\dotfill}
}}

\newcommand{\titre}[5] 
% #1: titre #2: haut gauche #3: bas gauche #4: haut droite #5: bas droite
{
\noindent #2 \hfill #4 \\
#3 \hfill #5

\vspace{-1.6cm}

\begin{center}\rule{6cm}{0.5mm}\end{center}
\vspace{0.2cm}
\begin{center}{\Large{\textbf{#1}}}\end{center}
\begin{center}\rule{6cm}{0.5mm}\end{center}
}



\begin{document}
\pagestyle{empty}

\titre{Séance d'AP 6 : Factorisations}{}{}{2nd}{}
\vspace*{0.75cm}

\textbf{RAPPELS}\\

\textbf{\textcolor{green}{\underline{Factorisation :}}} Soient $a$, $b$ et $k$ trois réels, $ka+kb=k(a+b)$ et $ka-kb=k(a-b)$   \\

\underline{Exemples :}\\
\hspace*{2cm}  $A=5x^2+x$  \hfill    $B=16x^x-24$ \hfill  $C=12x^2+3x$ \hfill $S=(x-1)(7-6x)-(x+2)(x-1)$ \hspace*{5cm}\\




\textbf{\textcolor{green}{\underline{Identités remarquables :}}}
Soient $a$ et $b$ deux réels, $a^2+2ab+b^2=(a+b)^2$   \\
\hspace*{10.2cm} $a^2-2ab+b^2=(a-b)^2$ \\
\hspace*{10.2cm}  $a^2-b^2=(a+b)(a-b)$ \\

\underline{Exemples :}\\
\hspace*{1cm} $D=100x^2-40x +4$  \hfill   $G=81+108x+36x^2$  \hfill  $M=64x^2-25$ \hfill  $V=64x^2-(x-2)^2$ \hspace*{1cm}\\

\vspace*{1.5cm}
\exo\\ Factoriser les expressions littérales suivantes.
\bmul{2}
\initqa \qa  $(2x-1)(x-5)+(3x+7)(x-5)$\\

\qa $(-3x+4)(3x-8)-(3x+4)(7x+2)$\\

\columnbreak

\qa  $(2x+3)^2+(x-2)(2x+3)$\\

\qa $(2x-7)-(5x-1)(2x-7)$\\





\emul

\vspace*{1cm}
\exo\\ Factoriser les expressions littérales suivantes.
\bmul{4}
\initqa \qa  $x^2-49$\\

\qa $81-4x^2$


\columnbreak

\qa  $9x^2- 12x+4$\\

\qa $1+20x+100x^2$

\columnbreak

\qa    $16x^2-36$\\ 

\qa $(x+1)^2-25$

\columnbreak

\qa  $81-(2x+1)^2$\\

\qa $(x-1)^2-(5x-6)^2$



\emul



\vspace*{1cm}
\exo\\ 

\initq \q Démontrer que, pour tout réel $x$ : \\
$(2 x - 3)(x +5)=2x^2+ 7 x -15 $\\

\q Soit la fonction $f$ définie sur $\R$ par : $f (x )=x (10 - x)$.\\
Montrer que : $f ( x )=25 -( x - 5)^2$

   
\end{document}
