\documentclass[a4paper,11pt]{article}
\usepackage{amsmath,amsthm,amsfonts,amssymb,amscd,amstext,vmargin,graphics,graphicx,tabularx,multicol} 
\usepackage[francais]{babel}
\usepackage[utf8]{inputenc}  
\usepackage[T1]{fontenc} 
\usepackage{pstricks-add,tikz,tkz-tab,variations}
\usepackage[autolanguage,np]{numprint} 

\setmarginsrb{1.5cm}{0.5cm}{1cm}{0.5cm}{0cm}{0cm}{0cm}{0cm} %Gauche, haut, droite, haut
\newcounter{numexo}
\newcommand{\exo}[1]{\stepcounter{numexo}\noindent{\bf Exercice~\thenumexo} : \marginpar{\hfill /#1}}
\reversemarginpar


\newcounter{enumtabi}
\newcounter{enumtaba}
\newcommand{\q}{\stepcounter{enumtabi} \theenumtabi.  }
\newcommand{\qa}{\stepcounter{enumtaba} (\alph{enumtaba}) }
\newcommand{\initq}{\setcounter{enumtabi}{0}}
\newcommand{\initqa}{\setcounter{enumtaba}{0}}

\newcommand{\be}{\begin{enumerate}}
\newcommand{\ee}{\end{enumerate}}
\newcommand{\bi}{\begin{itemize}}
\newcommand{\ei}{\end{itemize}}
\newcommand{\bp}{\begin{pspicture*}}
\newcommand{\ep}{\end{pspicture*}}
\newcommand{\bt}{\begin{tabular}}
\newcommand{\et}{\end{tabular}}
\renewcommand{\tabularxcolumn}[1]{>{\centering}m{#1}} %(colonne m{} centrée, au lieu de p par défault) 
\newcommand{\tnl}{\tabularnewline}

\newcommand{\trait}{\noindent \rule{\linewidth}{0.2mm}}
\newcommand{\hs}[1]{\hspace{#1}}
\newcommand{\vs}[1]{\vspace{#1}}

\newcommand{\N}{\mathbb{N}}
\newcommand{\Z}{\mathbb{Z}}
\newcommand{\R}{\mathbb{R}}
\newcommand{\C}{\mathbb{C}}
\newcommand{\Dcal}{\mathcal{D}}
\newcommand{\Ccal}{\mathcal{C}}
\newcommand{\mc}{\mathcal}

\newcommand{\vect}[1]{\overrightarrow{#1}}
\newcommand{\ds}{\displaystyle}
\newcommand{\eq}{\quad \Leftrightarrow \quad}
\newcommand{\vecti}{\vec{\imath}}
\newcommand{\vectj}{\vec{\jmath}}
\newcommand{\Oij}{(O;\vec{\imath}, \vec{\jmath})}
\newcommand{\OIJ}{(O;I,J)}


\newcommand{\bmul}[1]{\begin{multicols}{#1}}
\newcommand{\emul}{\end{multicols}}

\newcommand{\reponse}[1][1]{%
\multido{}{#1}{\makebox[\linewidth]{\rule[0pt]{0pt}{20pt}\dotfill}
}}

\newcommand{\titre}[5] 
% #1: titre #2: haut gauche #3: bas gauche #4: haut droite #5: bas droite
{
\noindent #2 \hfill #4 \\
#3 \hfill #5

\vspace{-1.6cm}

\begin{center}\rule{6cm}{0.5mm}\end{center}
\vspace{0.2cm}
\begin{center}{\large{\textbf{#1}}}\end{center}
\begin{center}\rule{6cm}{0.5mm}\end{center}
}



\begin{document}
\pagestyle{empty}
\titre{Interrogation : Écritures fractionnaires}{Nom :}{Prénom :}{Classe}{Date}


\exo{1} Cours :\\

\q Citer la propriété qui concerne le produit en croix.\\
\reponse[2]\\

\exo{3}\\

\initq \q Les quotients  $\dfrac{2}{3}$  et $\dfrac{11}{16,5}$ sont-ils égaux? \textbf{Justifier votre réponse.}\\
\reponse[3]\\

\q Les fractions $\dfrac{5}{13}$  et $\dfrac{7}{18}$ sont-elles égales ? \textbf{Justifier votre réponse.}\\
\reponse[3]\\


\exo{2} Simplifier les fractions suivantes \textbf{au maximum} :\\

\bmul{4}

$A = \dfrac{6}{12}$\\
\reponse[2]\\


\columnbreak

$O = \dfrac{15}{20}$\\
\reponse[2]\\

\columnbreak

$M = \dfrac{54}{27}$\\
\reponse[2]\\

\columnbreak

$S = \dfrac{32}{24}$\\
\reponse[2]\\


\emul


\exo{2} Trouver, parmi les quotients suivants, celui qui n'est pas égal aux autres. \textbf{Justifier votre réponse.}\\

$\dfrac{12}{16}$ \hspace{0.2cm} ; \hspace{0.2cm} $\dfrac{1,5}{2}$ \hspace{0.2cm} ; \hspace{0.2cm} $\dfrac{6}{8}$ \hspace{0.2cm} ; \hspace{0.2cm} $\dfrac{15}{20}$ \hspace{0.2cm} ; \hspace{0.2cm} $\dfrac{18}{30}$ \hspace{0.2cm} et \hspace{0.2cm} $\dfrac{21}{28}$ \hspace{0.2cm} \\

\noindent \reponse[3]\\


\exo{2} Calculer dans chaque cas le nombre x pour lequel l'égalité est vraie.

\bmul{2}

\initq \q \hspace*{0.2cm} $\dfrac{x}{6} = \dfrac{9}{2}$\\
\reponse[4]\\

\columnbreak


\q \hspace*{0.2cm} $\dfrac{10}{7} = \dfrac{8}{x}$\\
\reponse[4]\\


\emul

\end{document}
