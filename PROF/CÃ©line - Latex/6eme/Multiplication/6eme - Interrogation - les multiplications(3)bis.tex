\documentclass[a4paper,11pt]{article}
\usepackage{amsmath,amsthm,amsfonts,amssymb,amscd,amstext,vmargin,graphics,graphicx,tabularx,multicol} 
\usepackage[francais]{babel}
\usepackage[utf8]{inputenc}  
\usepackage[T1]{fontenc} 
\usepackage{pstricks-add,tikz,tkz-tab,variations}
\usepackage[autolanguage,np]{numprint} 

\setmarginsrb{1.5cm}{0.5cm}{1cm}{0.5cm}{0cm}{0cm}{0cm}{0cm} %Gauche, haut, droite, haut
\newcounter{numexo}
\newcommand{\exo}[1]{\stepcounter{numexo}\noindent{\bf Exercice~\thenumexo} : \marginpar{\hfill /#1}}
\reversemarginpar


\newcounter{enumtabi}
\newcounter{enumtaba}
\newcommand{\q}{\stepcounter{enumtabi} \theenumtabi.  }
\newcommand{\qa}{\stepcounter{enumtaba} (\alph{enumtaba}) }
\newcommand{\initq}{\setcounter{enumtabi}{0}}
\newcommand{\initqa}{\setcounter{enumtaba}{0}}

\newcommand{\be}{\begin{enumerate}}
\newcommand{\ee}{\end{enumerate}}
\newcommand{\bi}{\begin{itemize}}
\newcommand{\ei}{\end{itemize}}
\newcommand{\bp}{\begin{pspicture*}}
\newcommand{\ep}{\end{pspicture*}}
\newcommand{\bt}{\begin{tabular}}
\newcommand{\et}{\end{tabular}}
\renewcommand{\tabularxcolumn}[1]{>{\centering}m{#1}} %(colonne m{} centrée, au lieu de p par défault) 
\newcommand{\tnl}{\tabularnewline}

\newcommand{\bmul}[1]{\begin{multicols}{#1}}
\newcommand{\emul}{\end{multicols}}

\newcommand{\trait}{\noindent \rule{\linewidth}{0.2mm}}
\newcommand{\hs}[1]{\hspace{#1}}
\newcommand{\vs}[1]{\vspace{#1}}

\newcommand{\N}{\mathbb{N}}
\newcommand{\Z}{\mathbb{Z}}
\newcommand{\R}{\mathbb{R}}
\newcommand{\C}{\mathbb{C}}
\newcommand{\Dcal}{\mathcal{D}}
\newcommand{\Ccal}{\mathcal{C}}
\newcommand{\mc}{\mathcal}

\newcommand{\vect}[1]{\overrightarrow{#1}}
\newcommand{\ds}{\displaystyle}
\newcommand{\eq}{\quad \Leftrightarrow \quad}
\newcommand{\vecti}{\vec{\imath}}
\newcommand{\vectj}{\vec{\jmath}}
\newcommand{\Oij}{(O;\vec{\imath}, \vec{\jmath})}
\newcommand{\OIJ}{(O;I,J)}


\newcommand{\reponse}[1][1]{%
\multido{}{#1}{\makebox[\linewidth]{\rule[0pt]{0pt}{20pt}\dotfill}
}}

\newcommand{\titre}[5] 
% #1: titre #2: haut gauche #3: bas gauche #4: haut droite #5: bas droite
{
\noindent #2 \hfill #4 \\
#3 \hfill #5

\vspace{-1.6cm}

\begin{center}\rule{6cm}{0.5mm}\end{center}
\vspace{0.2cm}
\begin{center}{\large{\textbf{#1}}}\end{center}
\begin{center}\rule{6cm}{0.5mm}\end{center}
}



\begin{document}
\pagestyle{empty}
\titre{Interrogation: Multiplications }{Nom :}{Prénom :}{Classe}{Date}


\vspace*{0.5cm}
\begin{flushleft}
\begin{tabular}{|m{9.5cm}|m{1.25cm}|m{1.25cm}|m{1.25cm}|m{1.25cm}|m{1.25cm}|}
\hline 
\textbf{Compétences} & \begin{center}
\textbf{N.E.}
\end{center} & \begin{center}
\textbf{M.I.}
\end{center} & \begin{center}
\textbf{M.F.}
\end{center}  & \begin{center}
\textbf{M.S.}
\end{center} & \begin{center}
\textbf{T.B.M.}
\end{center} \\ 
\hline 
Je dois savoir multiplier des nombres décimaux (calcul mental ou posé) & & &  & &\\
\hline
Je dois savoir multiplier par 10, 100, 1000 etc& & &  & & \\ 
\hline
Je dois savoir multiplier par 0,1; 0,01 ; 0,001 etc & & &  & & \\ 
\hline 


\end{tabular}  
\end{flushleft}

\textit{N.E = Non évalué ; M.I. = Maîtrise insuffisante ; M.F. = Maîtrise fragile ; M.S. = Maîtrise satisfaisante ; T.B.M. = Très bonne maîtrise}\\

\exo{2} Poser et calculer les opérations suivantes :\\

$869 \times 34$ \hspace*{6cm} $7,6 \times 84,2$\\

\vspace*{7cm}

\vspace*{0.5cm}

\exo{1.5}   Compléter par le nombre manquant :\\

\bmul{3}



$1,7 \times \hspace*{0.3cm}. . .\hspace*{0.3cm} =$ 0,17 \\

$13,6 \times \hspace*{0.3cm}. . .\hspace*{0.3cm} =$ 1,36 \\


\columnbreak



$0,16 \times \hspace*{0.3cm}. . .\hspace*{0.3cm} =$ 1600 \\

$\hspace*{0.3cm}. . .\hspace*{0.3cm} \times 100 =$ 52,5\\

\columnbreak


$543  \times \hspace*{0.3cm}. . .\hspace*{0.3cm} =$ 0,543 \\

$\hspace*{0.3cm}. . .\hspace*{0.3cm} \times 1 =$ 27,3 \\

\emul

\exo{2.5} Calculer astucieusement en détaillant les étapes de calculs.\\

\bmul{2}

$K = 50 \times 12,39 \times 2 \times 10$\\
\reponse[5]\\

\columnbreak

$K = 2,5 \times 50 \times 4 \times 0,0256 \times 2 $\\
\reponse[5]\\

\emul



\exo{4}\\
Une famille composée des deux parents, du fils aîné Pierre, âgé de 16 ans, des deux jumelles Mathilde et Noémie, âgées de 8 ans et du petit dernier Gabin âgé de 4 ans décide de passer 4 jours à la montagne.\\
La location du logement est déjà réglée. Pierre et son père feront du snowboard alors que les jumelles, Gabin et leur mère feront du ski. Ils doivent louer le matériel.\\

\initq \textbf{\q	Calculer la somme totale dépensée pour toute la famille pour le matériel de ski et de snowboard pendant les 4 jours, sachant que :}\\

Pour une journée de location :	\hspace*{1cm}- une paire de skis coûte 10,50 euros ;\\
\hspace*{6.8cm}- un snowboard coûte 14,60 euros.\\


\noindent \reponse[9]\\

\textbf{\q	Calculer la somme totale dépensée pour les forfaits permettant d'accéder aux pistes sachant que :}\\

Pour une semaine : 	\hspace*{1cm}	- le forfait « adulte » et « plus de 13 ans » coûte 180 euros ;\\
\hspace*{5.1cm}- le forfait pour les enfants de 5 à 13 ans coûte 148 euros;\\
\hspace*{5cm}   	- le forfait est gratuit pour les enfants de moins de 5 ans.\\

\noindent \reponse[10]\\

\exo{}BONUS\\
Pour assurer la sécurité du musée, l'accès à la salle des trésors nécessite non pas un mais deux codes de 5 chiffres.\\
Étonnamment, les gardiens ont le même moyen mnémotechnique pour se souvenir de ces deux suites de chiffres : la somme du deuxième et du cinquième chiffre donne le quatrième.\\
Le dernier chiffre est plus grand que le premier et leur produit donne le deuxième.\\
La somme du premier et du deuxième chiffre donne le troisième.\\
Il n'y a pas les chiffres 0, 5 et 7.
\textbf{Quels sont les deux codes qui permettent d'ouvrir la salle ?}\\
\reponse[1]


\end{document}
