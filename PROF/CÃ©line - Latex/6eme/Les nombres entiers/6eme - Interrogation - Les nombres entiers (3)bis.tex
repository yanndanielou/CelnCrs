\documentclass[a4paper,11pt]{article}
\usepackage{amsmath,amsthm,amsfonts,amssymb,amscd,amstext,vmargin,graphics,graphicx,tabularx,multicol} 
\usepackage[francais]{babel}
\usepackage[utf8]{inputenc}  
\usepackage[T1]{fontenc} 
\usepackage{pstricks-add,tikz,tkz-tab,variations}
\usepackage[autolanguage,np]{numprint} 

\setmarginsrb{1.5cm}{0.5cm}{1cm}{0.5cm}{0cm}{0cm}{0cm}{0cm} %Gauche, haut, droite, haut
\newcounter{numexo}
\newcommand{\exo}[1]{\stepcounter{numexo}\noindent{\bf Exercice~\thenumexo} : \marginpar{\hfill /#1}}
\reversemarginpar


\newcounter{enumtabi}
\newcounter{enumtaba}
\newcommand{\q}{\stepcounter{enumtabi} \theenumtabi.  }
\newcommand{\qa}{\stepcounter{enumtaba} (\alph{enumtaba}) }
\newcommand{\initq}{\setcounter{enumtabi}{0}}
\newcommand{\initqa}{\setcounter{enumtaba}{0}}

\newcommand{\be}{\begin{enumerate}}
\newcommand{\ee}{\end{enumerate}}
\newcommand{\bi}{\begin{itemize}}
\newcommand{\ei}{\end{itemize}}
\newcommand{\bp}{\begin{pspicture*}}
\newcommand{\ep}{\end{pspicture*}}
\newcommand{\bt}{\begin{tabular}}
\newcommand{\et}{\end{tabular}}
\renewcommand{\tabularxcolumn}[1]{>{\centering}m{#1}} %(colonne m{} centrée, au lieu de p par défault) 
\newcommand{\tnl}{\tabularnewline}

\newcommand{\bmul}[1]{\begin{multicols}{#1}}
\newcommand{\emul}{\end{multicols}}

\newcommand{\trait}{\noindent \rule{\linewidth}{0.2mm}}
\newcommand{\hs}[1]{\hspace{#1}}
\newcommand{\vs}[1]{\vspace{#1}}

\newcommand{\N}{\mathbb{N}}
\newcommand{\Z}{\mathbb{Z}}
\newcommand{\R}{\mathbb{R}}
\newcommand{\C}{\mathbb{C}}
\newcommand{\Dcal}{\mathcal{D}}
\newcommand{\Ccal}{\mathcal{C}}
\newcommand{\mc}{\mathcal}

\newcommand{\vect}[1]{\overrightarrow{#1}}
\newcommand{\ds}{\displaystyle}
\newcommand{\eq}{\quad \Leftrightarrow \quad}
\newcommand{\vecti}{\vec{\imath}}
\newcommand{\vectj}{\vec{\jmath}}
\newcommand{\Oij}{(O;\vec{\imath}, \vec{\jmath})}
\newcommand{\OIJ}{(O;I,J)}


\newcommand{\reponse}[1][1]{%
\multido{}{#1}{\makebox[\linewidth]{\rule[0pt]{0pt}{20pt}\dotfill}
}}

\newcommand{\titre}[5] 
% #1: titre #2: haut gauche #3: bas gauche #4: haut droite #5: bas droite
{
\noindent #2 \hfill #4 \\
#3 \hfill #5

\vspace{-1.6cm}

\begin{center}\rule{6cm}{0.5mm}\end{center}
\vspace{0.2cm}
\begin{center}{\large{\textbf{#1}}}\end{center}
\begin{center}\rule{6cm}{0.5mm}\end{center}
}



\begin{document}
\pagestyle{empty}
\titre{Interrogation: Les nombres entiers }{Nom :}{Prénom :}{Classe}{Date}




\vspace*{0.3cm}

\exo{1} Écrire les nombres suivants en chiffres.\\


\qa Sept cent neuf mille deux cents  : . . . . . . . . . . . . . . . . . . . . . . . . . . . . . . . .\\

\qa Seize millions cinq cent vingt-trois : . . . . . . . . . . . . . . . . . . . . . . . . . . . . . . . . . . .\\

\vspace*{0.5cm}

\exo{2} Compléter comme indiqué à la première ligne.\\

\renewcommand{\arraystretch}{2.5}
\begin{tabular}{|c|c|c|}
\hline 
\textbf{En chiffres} & \textbf{En lettres} & \textbf{Décomposition} \\ 
\hline 
\textit{2 154} & \textit{Deux mille cent cinquante-quatre }& \textit{$(2 \times 1 000)+(1 \times 100)+(5 \times 10)+4$} \\ 
\hline 
 &  & $(4 \times 1 000 000)+(7 \times 10 000)+(1 \times 100)+ (8 \times 10)$ \\ 
\hline 
4 009 003 700 &  &  \\ 
\hline 
\end{tabular} 


\vspace*{0.7cm}

\exo{3} Dans le nombre 10 654 931 278 :\\

\initqa 

\qa  quel est le chiffre des unités de milliards ?  ....................................................................................\\

\qa  quel est le nombre d'unités de millions ?  ....................................................................................\\

\qa quel est le chiffre des dizaines ? ....................................................................................\\

\qa quel est le nombre de centaines ?  ....................................................................................\\

\vspace*{0.6cm}

\exo{2} Réécriver les nombres en ajoutant des zéros (s'il en faut) pour que 5 soit le chiffre \textit{des dizaines de mille} de chaque nombre :\\

9785	 :....................................................................................\\

		7503	 :...................................................................................	\\

\vspace*{0.4cm}

\exo{1}Comparer les nombres suivants.\\

10 571 .........  010 571	       \hfill    21 405 .........  20 405              \hfill  100 010 .........  101 010     \hfill 20 007 586 .........  20 007 568\\ 


\vspace*{0.4cm}

\exo{1} Encadrer les nombres suivants par des entiers.\\

\initqa \qa . . . . . . . . . . . .  < 999 999 < . . . . . . . . . . . . .\\


\qa . . . . . . . . . . . .  < 7 124 400 < . . . . . . . . . . . . .





\end{document}


