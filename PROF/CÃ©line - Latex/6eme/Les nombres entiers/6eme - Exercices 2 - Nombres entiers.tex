\documentclass[a4paper,11pt]{article}
\usepackage{amsmath,amsthm,amsfonts,amssymb,amscd,amstext,vmargin,graphics,graphicx,tabularx,multicol} 
\usepackage[francais]{babel}
\usepackage[utf8]{inputenc}  
\usepackage[T1]{fontenc} 
\usepackage{pstricks-add,tikz,tkz-tab,variations}
\usepackage[autolanguage,np]{numprint} 
\usepackage{calc}

\setmarginsrb{1.5cm}{0.5cm}{1cm}{0.5cm}{0cm}{0cm}{0cm}{0cm} %Gauche, haut, droite, haut
\newcounter{numexo}
\newcommand{\exo}[1]{\stepcounter{numexo}\noindent{\bf Exercice~\thenumexo} : }
\reversemarginpar

\newcommand{\bmul}[1]{\begin{multicols}{#1}}
\newcommand{\emul}{\end{multicols}}

\newcounter{enumtabi}
\newcounter{enumtaba}
\newcommand{\q}{\stepcounter{enumtabi} \theenumtabi.  }
\newcommand{\qa}{\stepcounter{enumtaba} (\alph{enumtaba}) }
\newcommand{\initq}{\setcounter{enumtabi}{0}}
\newcommand{\initqa}{\setcounter{enumtaba}{0}}

\newcommand{\be}{\begin{enumerate}}
\newcommand{\ee}{\end{enumerate}}
\newcommand{\bi}{\begin{itemize}}
\newcommand{\ei}{\end{itemize}}
\newcommand{\bp}{\begin{pspicture*}}
\newcommand{\ep}{\end{pspicture*}}
\newcommand{\bt}{\begin{tabular}}
\newcommand{\et}{\end{tabular}}
\renewcommand{\tabularxcolumn}[1]{>{\centering}m{#1}} %(colonne m{} centrée, au lieu de p par défault) 
\newcommand{\tnl}{\tabularnewline}

\newcommand{\trait}{\noindent \rule{\linewidth}{0.2mm}}
\newcommand{\hs}[1]{\hspace{#1}}
\newcommand{\vs}[1]{\vspace{#1}}

\newcommand{\N}{\mathbb{N}}
\newcommand{\Z}{\mathbb{Z}}
\newcommand{\R}{\mathbb{R}}
\newcommand{\C}{\mathbb{C}}
\newcommand{\Dcal}{\mathcal{D}}
\newcommand{\Ccal}{\mathcal{C}}
\newcommand{\mc}{\mathcal}

\newcommand{\vect}[1]{\overrightarrow{#1}}
\newcommand{\ds}{\displaystyle}
\newcommand{\eq}{\quad \Leftrightarrow \quad}
\newcommand{\vecti}{\vec{\imath}}
\newcommand{\vectj}{\vec{\jmath}}
\newcommand{\Oij}{(O;\vec{\imath}, \vec{\jmath})}
\newcommand{\OIJ}{(O;I,J)}


\newcommand{\reponse}[1][1]{%
\multido{}{#1}{\makebox[\linewidth]{\rule[0pt]{0pt}{20pt}\dotfill}
}}

\newcommand{\titre}[5] 
% #1: titre #2: haut gauche #3: bas gauche #4: haut droite #5: bas droite
{
\noindent #2 \hfill #4 \\
#3 \hfill #5

\vspace{-1.6cm}

\begin{center}\rule{6cm}{0.5mm}\end{center}
\vspace{0.2cm}
\begin{center}{\large{\textbf{#1}}}\end{center}
\begin{center}\rule{6cm}{0.5mm}\end{center}
}



\begin{document}
\pagestyle{empty}
\titre{Exercices sur les nombres entiers (2)}{}{}{6ème}{}

\vspace*{0.2cm}


\exo \\ Trouver le nombre entier qui précède chacun de
ces nombres :\\

a. .............. < 1 000  \hspace*{0.5cm} b. .............. < 1 000 001 \hspace*{0.5cm} c. .............. < 1 000 000 \\

d. .............. < 2 \hspace*{0.5cm} e. .............. < 9 786 000 \hspace*{0.5cm}f. .............. < 740 000\\


\vspace*{0.5cm}

\exo \\ \textbf{Comparer} les nombres suivants.\\

34 895 ......... 234 113 \hspace*{0.5cm} 86 325 ......... 85 981 \hspace*{0.5cm} 202 598 ......... 22 598 \hspace*{0.5cm} 056 640 ......... 56 640 \\

 87 479.........87 489  \hspace*{0.5cm}641 200 199.........641 199 \hspace*{0.5cm} 200 19 450.........19 405\\

\vspace*{0.5cm}
 \exo \\ Ranger ces nombres \textbf{par ordre croissant}.\\
 
26 014 ; 26 140 ; 26 104 ; 26410 ; 26 401\\

. . . . . . . . . . . < . . . . . . . . . . . < . . . . . . . . . . . < . . . . . . . . . . . < . . . . . . . . . . .\\

\vspace*{0.5cm}
 \exo \\ Ranger ces nombres \textbf{par ordre décroissant}.\\
 
101 010 ; 1 000 101 ; 11 001 ; 100 110 ; 101 111\\

. . . . . . . . . . . > . . . . . . . . . . . > . . . . . . . . . . . > . . . . . . . . . . . > . . . . . . . . . . .\\

\vspace*{0.5cm}

 \exo \\ Encadrer les nombres suivants par des entiers.\\
 
a)  . . . . . . . . . . . < 50 < . . . . . . . . . . .\\
 
 b) . . . . . . . . . . . <  75 359 433 < . . . . . . . . . . .\\
 
c) . . . . . . . . . . . < 9 999 < . . . . . . . . . . .\\

d) . . . . . . . . . . . < 122 000 000 < . . . . . . . . . . .


\end{document}
