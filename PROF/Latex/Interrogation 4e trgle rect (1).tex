\documentclass[a4paper,11pt]{article}
\usepackage{amsmath,amsthm,amsfonts,amssymb,amscd,amstext,vmargin,graphics,graphicx,tabularx,multicol} 
\usepackage[francais]{babel}
\usepackage[utf8]{inputenc}  
\usepackage[T1]{fontenc} 
\usepackage{pstricks-add,tikz,tkz-tab,variations}
\usepackage[autolanguage,np]{numprint} 

\setmarginsrb{1.5cm}{0.5cm}{1cm}{0.5cm}{0cm}{0cm}{0cm}{0cm} %Gauche, haut, droite, haut
\newcounter{numexo}
\newcommand{\exo}[1]{\stepcounter{numexo}\noindent{\bf Exercice~\thenumexo} : \marginpar{\hfill /#1}}
\reversemarginpar


\newcounter{enumtabi}
\newcounter{enumtaba}
\newcommand{\q}{\stepcounter{enumtabi} \theenumtabi.  }
\newcommand{\qa}{\stepcounter{enumtaba} (\alph{enumtaba}) }
\newcommand{\initq}{\setcounter{enumtabi}{0}}
\newcommand{\initqa}{\setcounter{enumtaba}{0}}

\newcommand{\be}{\begin{enumerate}}
\newcommand{\ee}{\end{enumerate}}
\newcommand{\bi}{\begin{itemize}}
\newcommand{\ei}{\end{itemize}}
\newcommand{\bp}{\begin{pspicture*}}
\newcommand{\ep}{\end{pspicture*}}
\newcommand{\bt}{\begin{tabular}}
\newcommand{\et}{\end{tabular}}
\renewcommand{\tabularxcolumn}[1]{>{\centering}m{#1}} %(colonne m{} centrée, au lieu de p par défault) 
\newcommand{\tnl}{\tabularnewline}

\newcommand{\trait}{\noindent \rule{\linewidth}{0.2mm}}
\newcommand{\hs}[1]{\hspace{#1}}
\newcommand{\vs}[1]{\vspace{#1}}

\newcommand{\N}{\mathbb{N}}
\newcommand{\Z}{\mathbb{Z}}
\newcommand{\R}{\mathbb{R}}
\newcommand{\C}{\mathbb{C}}
\newcommand{\Dcal}{\mathcal{D}}
\newcommand{\Ccal}{\mathcal{C}}
\newcommand{\mc}{\mathcal}

\newcommand{\vect}[1]{\overrightarrow{#1}}
\newcommand{\ds}{\displaystyle}
\newcommand{\eq}{\quad \Leftrightarrow \quad}
\newcommand{\vecti}{\vec{\imath}}
\newcommand{\vectj}{\vec{\jmath}}
\newcommand{\Oij}{(O;\vec{\imath}, \vec{\jmath})}
\newcommand{\OIJ}{(O;I,J)}


\newcommand{\reponse}[1][1]{%
\multido{}{#1}{\makebox[\linewidth]{\rule[0pt]{0pt}{20pt}\dotfill}
}}

\newcommand{\titre}[5] 
% #1: titre #2: haut gauche #3: bas gauche #4: haut droite #5: bas droite
{
\noindent #2 \hfill #4 \\
#3 \hfill #5

\vspace{-1.6cm}

\begin{center}\rule{6cm}{0.5mm}\end{center}
\vspace{0.2cm}
\begin{center}{\large{\textbf{#1}}}\end{center}
\begin{center}\rule{6cm}{0.5mm}\end{center}
}

\begin{document}
\pagestyle{empty}
\titre{Interrogation : Triangle rectangle et cercle circonscrit}{Nom}{Prénom}{Classe}{Date}


\textbf{Tableau de compétences}

\bigskip

\exo{2} Questions de cours :\\

\q Enoncer le théorème 1 (concernant le cercle circonscrit dans un triangle rectangle) et faire un schéma :\\
\begin{multicols}{2}
\noindent \reponse[3]

\hfill
\columnbreak

Schéma:

\end{multicols}

\q Enoncer le théorème 3 ( qui détermine la nature d’un triangle inscrit dans un cercle ) :\\

\noindent \reponse[2]

\bigskip
\exo{1}
Construire le cercle circonscrit au triangle MNP rectangle en M.  \\
\psset{xunit=1.0cm,yunit=1.0cm,algebraic=true,dotstyle=o,dotsize=3pt 0,linewidth=0.8pt,arrowsize=3pt 2,arrowinset=0.25}
\begin{pspicture*}(0,3)(15,10.78)
\pspolygon[linecolor=white,fillcolor=white,fillstyle=solid,opacity=0.1](5.04,6.76)(7.96,8.56)(10.06,5.14)
\psline[linecolor=white](5.04,6.76)(7.96,8.56)
\psline[linecolor=white](7.96,8.56)(10.06,5.14)
\psline[linecolor=white](10.06,5.14)(5.04,6.76)
\psline(5.04,6.76)(10.06,5.14)
\psline(10.06,5.14)(7.96,8.56)
\psline(7.96,8.56)(5.04,6.76)
\begin{scriptsize}
\psdots[dotstyle=*,linecolor=blue](5.04,6.76)
\rput[bl](5.12,6.88){\blue{$M$}}
\psdots[dotstyle=*,linecolor=blue](7.96,8.56)
\rput[bl](8.04,8.68){\blue{$N$}}
\psdots[dotstyle=*,linecolor=blue](10.06,5.14)
\rput[bl](10.14,5.26){\blue{$P$}}
\end{scriptsize}
\end{pspicture*}



\exo{4} Dans chaque cas, déterminer à l'aide d'une démonstration la longueur du segment [GH].\\

\initq

\begin{multicols}{2}

\q 

\newrgbcolor{qqwuqq}{0 0.39 0}
\psset{xunit=1.0cm,yunit=1.0cm,algebraic=true,dotstyle=o,dotsize=3pt 0,linewidth=0.8pt,arrowsize=3pt 2,arrowinset=0.25}
\begin{pspicture*}(-3,0)(7.06,6.3)
\pspolygon[linecolor=qqwuqq,fillcolor=qqwuqq,fillstyle=solid,opacity=0.1](-2.76,4.7)(-2.34,4.7)(-2.34,5.12)(-2.76,5.12)
\psline(-2.76,3.44)(-2.76,5.12)
\psline(4.54,5.12)(-2.76,5.12)
\psline(-2.76,3.44)(4.54,5.12)
\psline(-2.76,5.12)(0.89,4.28)
\rput[tl](1.1,4.26){$AC \, = \, 7.49$}
\begin{scriptsize}
\psdots[dotstyle=*,linecolor=blue](-2.76,3.44)
\rput[bl](-2.68,3.56){\blue{$A$}}
\psdots[dotstyle=*,linecolor=blue](-2.76,5.12)
\rput[bl](-2.68,5.24){\blue{$G$}}
\psdots[dotstyle=*,linecolor=blue](4.54,5.12)
\rput[bl](4.62,5.24){\blue{$C$}}
\rput[bl](-2.58,4.78){\qqwuqq{$\alpha$}}
\psdots[dotstyle=*,linecolor=darkgray](0.89,4.28)
\rput[bl](0.98,4.4){\darkgray{$H$}}
\psdots[dotstyle=*,linecolor=darkgray](0.89,4.28)
\end{scriptsize}
\end{pspicture*}

%\hfill
\columnbreak

\reponse[4]

\end{multicols}

\newpage


\begin{multicols}{2}

\q 

\newrgbcolor{qqwuqq}{0 0.39 0}
\psset{xunit=1.0cm,yunit=1.0cm,algebraic=true,dotstyle=o,dotsize=3pt 0,linewidth=0.8pt,arrowsize=3pt 2,arrowinset=0.25}
\begin{pspicture*}(-4.3,-2.56)(7.06,6.3)
\pspolygon[linecolor=qqwuqq,fillcolor=qqwuqq,fillstyle=solid,opacity=0.1](-2.76,4.7)(-2.34,4.7)(-2.34,5.12)(-2.76,5.12)
\psline(-2.76,3.44)(-2.76,5.12)
\psline(4.54,5.12)(-2.76,5.12)
\psline(-2.76,3.44)(4.54,5.12)
\psline(-2.76,5.12)(0.89,4.28)
\rput[tl](1.1,4.26){$AC \, = \, 7.49$}
\begin{scriptsize}
\psdots[dotstyle=*,linecolor=blue](-2.76,3.44)
\rput[bl](-2.68,3.56){\blue{$A$}}
\psdots[dotstyle=*,linecolor=blue](-2.76,5.12)
\rput[bl](-2.68,5.24){\blue{$G$}}
\psdots[dotstyle=*,linecolor=blue](4.54,5.12)
\rput[bl](4.62,5.24){\blue{$C$}}
\rput[bl](-2.58,4.78){\qqwuqq{$\alpha$}}
\psdots[dotstyle=*,linecolor=darkgray](0.89,4.28)
\rput[bl](0.98,4.4){\darkgray{$H$}}
\psdots[dotstyle=*,linecolor=darkgray](0.89,4.28)
\end{scriptsize}
\end{pspicture*}


\hfill
\columnbreak

\reponse[3]

\end{multicols}


\exo{3}\\
\initq

\q Tracer deux droites (d) et (d') sécantes en I qui ne sont pas perpendiculaires.\\

\q Placer un point J sur la droite (d) tel que IJ = 6cm.\\

\q Tracer le cercle de diamètre [IJ], il coupe (d’) en K.\\

\q Prouver que le triangle IJK est rectangle en K.



\end{document}