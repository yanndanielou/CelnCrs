\documentclass[a4paper,11pt]{article}
\usepackage{amsmath,amsthm,amsfonts,amssymb,amscd,amstext,vmargin,graphics,graphicx,tabularx,multicol} 
\usepackage[francais]{babel}
\usepackage[utf8]{inputenc}  
\usepackage[T1]{fontenc} 
\usepackage{pstricks-add,tikz,tkz-tab,variations}
\usepackage[autolanguage,np]{numprint} 

\setmarginsrb{1.5cm}{0.5cm}{1cm}{0.5cm}{0cm}{0cm}{0cm}{0cm} %Gauche, haut, droite, haut
\newcounter{numexo}
\newcommand{\exo}[1]{\stepcounter{numexo}\noindent{\bf Exercice~\thenumexo} : \marginpar{\hfill /#1}}
\reversemarginpar


\newcounter{enumtabi}
\newcounter{enumtaba}
\newcommand{\q}{\stepcounter{enumtabi} \theenumtabi.  }
\newcommand{\qa}{\stepcounter{enumtaba} (\alph{enumtaba}) }
\newcommand{\initq}{\setcounter{enumtabi}{0}}
\newcommand{\initqa}{\setcounter{enumtaba}{0}}

\newcommand{\be}{\begin{enumerate}}
\newcommand{\ee}{\end{enumerate}}
\newcommand{\bi}{\begin{itemize}}
\newcommand{\ei}{\end{itemize}}
\newcommand{\bp}{\begin{pspicture*}}
\newcommand{\ep}{\end{pspicture*}}
\newcommand{\bt}{\begin{tabular}}
\newcommand{\et}{\end{tabular}}
\renewcommand{\tabularxcolumn}[1]{>{\centering}m{#1}} %(colonne m{} centrée, au lieu de p par défault) 
\newcommand{\tnl}{\tabularnewline}

\newcommand{\trait}{\noindent \rule{\linewidth}{0.2mm}}
\newcommand{\hs}[1]{\hspace{#1}}
\newcommand{\vs}[1]{\vspace{#1}}

\newcommand{\N}{\mathbb{N}}
\newcommand{\Z}{\mathbb{Z}}
\newcommand{\R}{\mathbb{R}}
\newcommand{\C}{\mathbb{C}}
\newcommand{\Dcal}{\mathcal{D}}
\newcommand{\Ccal}{\mathcal{C}}
\newcommand{\mc}{\mathcal}

\newcommand{\vect}[1]{\overrightarrow{#1}}
\newcommand{\ds}{\displaystyle}
\newcommand{\eq}{\quad \Leftrightarrow \quad}
\newcommand{\vecti}{\vec{\imath}}
\newcommand{\vectj}{\vec{\jmath}}
\newcommand{\Oij}{(O;\vec{\imath}, \vec{\jmath})}
\newcommand{\OIJ}{(O;I,J)}


\newcommand{\reponse}[1][1]{%
\multido{}{#1}{\makebox[\linewidth]{\rule[0pt]{0pt}{20pt}\dotfill}
}}

\newcommand{\titre}[5] 
% #1: titre #2: haut gauche #3: bas gauche #4: haut droite #5: bas droite
{
\noindent #2 \hfill #4 \\
#3 \hfill #5

\vspace{-1.6cm}

\begin{center}\rule{6cm}{0.5mm}\end{center}
\vspace{0.2cm}
\begin{center}{\large{\textbf{#1}}}\end{center}
\begin{center}\rule{6cm}{0.5mm}\end{center}
}



\begin{document}






\titre{Nombres relatifs et repérage}
\niveau{6\ieme}



    Compléter :
    \begin{multicols}{4}
      \begin{enumerate}
      \item $\dfrac{5}{8}=\dfrac{\ldots}{56}$
      \item $\dfrac{36}{\ldots}=\dfrac{6}{5}$
      \item $\dfrac{\ldots}{8}=\dfrac{70}{80}$
      \item $\dfrac{9}{\ldots}=\dfrac{18}{14}$
      \item $\dfrac{\ldots}{25}=\dfrac{1}{5}$
      \item $\dfrac{72}{\ldots}=\dfrac{9}{4}$
      \item $\dfrac{5}{\ldots}=\dfrac{45}{63}$
      \item $\dfrac{40}{\ldots}=\dfrac{10}{9}$
      \end{enumerate}
    \end{multicols}


    \exercice
    Compléter :
    \begin{multicols}{4}
      \begin{enumerate}
      \item $\dfrac{1}{6}=\dfrac{2}{\ldots}$
      \item $\dfrac{\ldots}{8}=\dfrac{24}{48}$
      \item $\dfrac{\ldots}{7}=\dfrac{64}{56}$
      \item $\dfrac{2}{7}=\dfrac{\ldots}{35}$
      \item $\dfrac{18}{\ldots}=\dfrac{9}{2}$
      \item $\dfrac{45}{36}=\dfrac{\ldots}{4}$
      \item $\dfrac{\ldots}{18}=\dfrac{1}{9}$
      \item $\dfrac{28}{42}=\dfrac{\ldots}{6}$
      \end{enumerate}
    \end{multicols}


    \exercice
    Compléter :
    \begin{multicols}{4}
      \begin{enumerate}
      \item $\dfrac{\ldots}{70}=\dfrac{2}{7}$
      \item $\dfrac{45}{\ldots}=\dfrac{5}{2}$
      \item $\dfrac{\ldots}{24}=\dfrac{8}{3}$
      \item $\dfrac{12}{24}=\dfrac{\ldots}{4}$
      \item $\dfrac{42}{49}=\dfrac{\ldots}{7}$
      \item $\dfrac{7}{\ldots}=\dfrac{63}{72}$
      \item $\dfrac{\ldots}{90}=\dfrac{10}{9}$
      \item $\dfrac{\ldots}{48}=\dfrac{10}{6}$
      \end{enumerate}
    \end{multicols}


    \exercice
    \parbox{0.4\linewidth}{
      \begin{enumerate}
      \item Donner les coordonnées des points B, C, D, F, G et N.
      \item Placer dans le repère les points P, S, T, U, V et X de coordonnées
        respectives \hbox{$(-3{,}5~;~0)$}, \hbox{$(-4~;~-2{,}5)$},
        \hbox{$(4{,}5~;~-4{,}5)$}, \hbox{$(0~;~-3)$}, \hbox{$(-0{,}5~;~1)$} et
        \hbox{$(3{,}5~;~4{,}5)$}.
      \item Placer dans le repère le point Z d'abscisse -3 et d'ordonnée -1,5
      \end{enumerate}}\hfill
    \parbox{0.55\linewidth}{
      \psset{unit=0.8cm}
      \begin{pspicture}(-4.95,-4.95)(4.95,4.95)
        \psgrid[subgriddiv=2, subgridcolor=lightgray,
        gridlabels=8pt](0,0)(-5,-5)(5,5)
        \psline[linewidth=1.2pt]{->}(-5,0)(5,0)
        \psline[linewidth=1.2pt]{->}(0,-5)(0,5)
        \pstGeonode[PointSymbol=x,PosAngle={-135,45,135,45,-45,45,},PointNameSep=0.4](-4.0, -1.5){B}(0, 4.0){C}(-2.5, 3.0){D}(0.5, 0){F}(1.0, -3.5){G}(4.5, 3.0){N}
      \end{pspicture}}

    \exercice
    \parbox{0.4\linewidth}{
      \begin{enumerate}
      \item Donner les coordonnées des points A, B, D, E, G et I.
      \item Placer dans le repère les points J, M, N, P, U et W de coordonnées
        respectives \hbox{$(-2~;~-1{,}5)$}, \hbox{$(4~;~0)$},
        \hbox{$(0~;~4{,}5)$}, \hbox{$(4{,}5~;~4)$}, \hbox{$(-1{,}5~;~2)$} et
        \hbox{$(3{,}5~;~-3{,}5)$}.
      \item Placer dans le repère le point X d'abscisse -2 et d'ordonnée 3
      \end{enumerate}}\hfill
    \parbox{0.55\linewidth}{
      \psset{unit=0.8cm}
      \begin{pspicture}(-4.95,-4.95)(4.95,4.95)
        \psgrid[subgriddiv=2, subgridcolor=lightgray,
        gridlabels=8pt](0,0)(-5,-5)(5,5)
        \psline[linewidth=1.2pt]{->}(-5,0)(5,0)
        \psline[linewidth=1.2pt]{->}(0,-5)(0,5)
        \pstGeonode[PointSymbol=x,PosAngle={45,45,-135,45,-45,135,},PointNameSep=0.4](0.5, 2.5){A}(0, 0.5){B}(-2.5, -2.5){D}(1.5, 0){E}(4.0, -2.0){G}(-4.5, 0.5){I}
      \end{pspicture}}

    \exercice
    Construire la symétrique de chacune des figures par rapport au point O en
    utilisant le quadrillage :\par
    \psset{unit=.9cm}
    \begin{pspicture*}(-3,-3)(3,3)
      \psgrid[subgriddiv=2,gridlabels=0pt]
      \pstGeonode[PointSymbol=none,PointName=none](2.5,1.0){a}(-2.5,2.5){b}(-1.5,-1.0){c}(-1.5,-1.5){d}(1.0,-3.0){e}
      \pstGeonode[PointSymbol=x, linecolor=Black, dotsize=6pt](0.0,0.0){O}
      \pspolygon[linewidth=1pt](a)(b)(c)(d)(e)
    \end{pspicture*}
    \hfill
    \begin{pspicture*}(-3,-3)(3,3)
      \psgrid[subgriddiv=2,gridlabels=0pt]
      \pstGeonode[PointSymbol=none,PointName=none](2.0,1.0){a}(-0.5,3.0){b}(-2.5,-1.0){c}(-2.0,-3.0){d}(2.5,-2.5){e}
      \pstGeonode[PointSymbol=x, linecolor=Black, dotsize=6pt](0.0,0.0){O}
      \pspolygon[linewidth=1pt](a)(b)(c)(d)(e)
    \end{pspicture*}
    \hfill
    \begin{pspicture*}(-3,-3)(3,3)
      \psgrid[subgriddiv=2,gridlabels=0pt]
      \pstGeonode[PointSymbol=none,PointName=none](1.5,1.5){a}(0.5,3.0){b}(-3.0,-2.0){c}(-3.0,-3.0){d}(3.0,-2.5){e}
      \pstGeonode[PointSymbol=x, linecolor=Black, dotsize=6pt](0.0,0.0){O}
      \pspolygon[linewidth=1pt](a)(b)(c)(d)(e)
    \end{pspicture*}

    \exercice
    Construire la symétrique de chacune des figures par rapport au point O en
    utilisant le quadrillage :\par
    \psset{unit=.9cm}
    \begin{pspicture*}(-3,-3)(3,3)
      \psgrid[subgriddiv=2,gridlabels=0pt]
      \pstGeonode[PointSymbol=none,PointName=none](3.0,1.0){a}(-0.5,2.0){b}(-2.0,-1.5){c}(1.0,-3.0){d}(2.5,-1.0){e}
      \pstGeonode[PointSymbol=x, linecolor=Black, dotsize=6pt](0.0,0.0){O}
      \pspolygon[linewidth=1pt](a)(b)(c)(d)(e)
    \end{pspicture*}
    \hfill
    \begin{pspicture*}(-3,-3)(3,3)
      \psgrid[subgriddiv=2,gridlabels=0pt]
      \pstGeonode[PointSymbol=none,PointName=none](1.0,1.5){a}(-1.5,1.5){b}(-2.0,-0.5){c}(-1.0,-2.0){d}(2.5,-1.5){e}
      \pstGeonode[PointSymbol=x, linecolor=Black, dotsize=6pt](0.0,0.0){O}
      \pspolygon[linewidth=1pt](a)(b)(c)(d)(e)
    \end{pspicture*}
    \hfill
    \begin{pspicture*}(-3,-3)(3,3)
      \psgrid[subgriddiv=2,gridlabels=0pt]
      \pstGeonode[PointSymbol=none,PointName=none](2.5,2.5){a}(0.5,3.0){b}(-1.5,-0.5){c}(0.5,-2.5){d}(3.0,-2.0){e}
      \pstGeonode[PointSymbol=x, linecolor=Black, dotsize=6pt](0.5,0.0){O}
      \pspolygon[linewidth=1pt](a)(b)(c)(d)(e)
    \end{pspicture*}
\section{Nombres relatifs}
\begin{definition}
Pour repérer des choses on a parfois besoin des nombres. On fixe une
origine et chaque chose est repérée par un nombre.\par
Parfois, il est nécessaire de repérer les objets de part et d'autre de
l'origine. On précise le côté choisi en notant que le nombre est
{\em positif} ($+$) ou {\em négatif} ($-$) : ce sont les nombres relatifs.
Le nombre 0 est particulier : il est à la fois positif et négatif.

\end{definition}
\begin{ex}
\parbox[t]{0.7\linewidth}{
La température : on fixe comme origine la température à laquelle l'eau gèle :
$0^{\circ}$C.
Lorsqu'on est plus chaud que cette origine, on parle de températures
{\em positives} (comme  $+16^{\circ}$C) et lorsqu'on est plus froid  que cette origine,
on parle de températures {\em négatives} (comme  $-8^{\circ}$C).}%
\hskip3em
\parbox{0.1\linewidth}{%
\setlength{\unitlength}{0.1cm}
\begin{picture}(6.91,41.21)
\shade\put(3.46,3.45){\arc{6.73504}{5.1720}{10.5306}}
\put(3.46,39.45){\arc{2.98}{3.1482}{6.2765}}
\path(1.95,39.46)(1.95,6.46)
\path(4.95,39.46)(4.95,6.46)
\shade\path(1.95,6)(1.95,33)(4.95,33)(4.95,6)(1.95,6)
\multiput(1.55,7)(0,1){33}{\line(1,0){3.8}}
{\tiny\multiputlist(7.95,7)(0,2){$-10$,$-8$,$-6$,$-4$,$-2$,$0$,$+2$,$+4$,
	$+6$,$+8$,$+10$,$+12$,$+14$,$+16$,$+18$,$+20$,$+22$}
\multiputlist(-1,8)(0,2){$-9$,$-7$,$-5$,$-3$,$-1$,$+1$,$+3$,$+5$,$+7$,
	$+9$,$+11$,$+13$,$+15$,$+17$,$+19$,$+21$}
}
\thicklines\put(1.55,17){\line(1,0){3.8}}
\end{picture}}
\end{ex}


\begin{ex}
Les étages d'un immeuble sont  repérés par rapport à un niveau 0 : le
rez-de-chaussée. Les étages au dessus sont les étages positifs et les
étages en dessous (cave, garages) sont les étages négatifs. On retrouve
cette notation sur les commandes d'ascenseur.\par

\setlength{\unitlength}{0.00068in} %d'après Xfig 
\begin{picture}(6286,4239)
\put(492,159.119){\arc{70.237}{6.2011}{9.5069}}
\put(772,159.119){\arc{70.237}{6.2011}{9.5069}}
\path(492,162)(912,162)(912,237)(772,312)(562,312)(387,237)
	(387,162)(492,162)
\put(1242,159.119){\arc{70.237}{6.2011}{9.5069}}
\put(1522,159.119){\arc{70.237}{6.2011}{9.5069}}
\path(1242,162)(1662,162)(1662,237)(1522,312)(1312,312)(1137,237)
	(1137,162)(1242,162)
\put(1932,164.053){\arc{80.105}{0.0513}{3.0903}}
\put(2252,164.053){\arc{80.105}{0.0513}{3.0903}}
\path(1932,162)(2412,162)(2412,237)(2252,312)(2012,312)(1812,237)
	(1812,162)(1932,162)
\put(2682,164.053){\arc{80.105}{0.0513}{3.0903}}
\put(3002,164.053){\arc{80.105}{0.0513}{3.0903}}
\path(2682,162)(3162,162)(3162,237)(3002,312)(2762,312)(2562,237)
	(2562,162)(2682,162)
\path(987,912)(12,912)
\path(312,462)(4212,462)
\path(312,912)(312,12)(4212,12)(4212,912)
\path(312,912)(12,1212)
\path(312,12)(12,387)(12,1212)
\path(312,462)(12,762)
\path(12,1212)(912,1212)
\path(4212,912)(3912,1212)
\path(4212,912)(3912,912)
\path(612,912)(612,462)
\path(912,912)(912,462)
\path(912,1062)(162,1062)
\path(162,1062)(162,612)
\path(612,912)(312,1212)
\path(912,912)(612,1212)
\path(1212,462)(1212,837)
\path(1512,837)(1512,462)
\path(1812,837)(1812,462)
\path(2112,837)(2112,462)
\path(2412,837)(2412,462)
\path(2712,837)(2712,462)
\path(3012,837)(3012,462)
\path(3312,837)(3312,462)
\path(3612,837)(3612,462)
\path(3912,837)(3912,462)
\blacken\path(4482,642)(4362,612)(4482,582)(4482,642)
\path(4362,612)(5412,612)
\blacken\path(4482,192)(4362,162)(4482,132)(4482,192)
\path(4362,162)(5412,162)
\blacken\path(4332,1092)(4212,1062)(4332,1032)(4332,1092)
\path(4212,1062)(5412,1062)
\blacken\path(4332,1317)(4212,1287)(4332,1257)(4332,1317)
\path(4212,1287)(5412,1287)
\blacken\path(4332,1767)(4212,1737)(4332,1707)(4332,1767)
\path(4212,1737)(5412,1737)
\blacken\path(4332,1992)(4212,1962)(4332,1932)(4332,1992)
\path(4212,1962)(5412,1962)
\blacken\path(4332,2217)(4212,2187)(4332,2157)(4332,2217)
\path(4212,2187)(5412,2187)
\blacken\path(4332,2442)(4212,2412)(4332,2382)(4332,2442)
\path(4212,2412)(5412,2412)
\blacken\path(4332,2637)(4212,2607)(4332,2577)(4332,2637)
\path(4212,2607)(5412,2607)
\blacken\path(4332,2892)(4212,2862)(4332,2832)(4332,2892)
\path(4212,2862)(5412,2862)
\blacken\path(4332,1542)(4212,1512)(4332,1482)(4332,1542)
\path(4212,1512)(5412,1512)
\put(5487,987){\makebox(0,0)[lb]{Niveau 0}}
\put(5487,87){\makebox(0,0)[lb]{Niveau $-2$}}
\put(5487,537){\makebox(0,0)[lb]{Niveau $-1$}}
\put(5472,1242){\makebox(0,0)[lb]{Niveau $+1$}}
\put(5442,1467){\makebox(0,0)[lb]{Niveau $+2$}}
\put(5442,1677){\makebox(0,0)[lb]{Niveau $+3$}}
\put(5442,1917){\makebox(0,0)[lb]{Niveau $+4$}}
\put(5457,2142){\makebox(0,0)[lb]{Niveau $+5$}}
\put(5442,2352){\makebox(0,0)[lb]{Niveau $+6$}}
\put(5442,2577){\makebox(0,0)[lb]{Niveau $+7$}}
\put(5427,2817){\makebox(0,0)[lb]{Niveau $+8$}}
\texture{555555 55000000 555555 55888888 88555555 55000000 555555 55808080 
	80555555 55000000 555555 55888888 88555555 55000000 555555 55888088 
	80555555 55000000 555555 55888888 88555555 55000000 555555 55808080 
	80555555 55000000 555555 55888888 88555555 55000000 555555 55888088 }
\shade\path(1164,2439)(1164,3875)(1528,4095)
	(1843,3942)(1843,3672)(1339,3672)
	(1339,2439)(1164,2439)
\path(1164,2439)(1164,3875)(1528,4095)
	(1843,3942)(1843,3672)(1339,3672)
	(1339,2439)(1164,2439)
\texture{aa555555 55bbbbbb bb555555 55fefefe fe555555 55bbbbbb bb555555 55eeefee 
	ef555555 55bbbbbb bb555555 55fefefe fe555555 55bbbbbb bb555555 55efefef 
	ef555555 55bbbbbb bb555555 55fefefe fe555555 55bbbbbb bb555555 55eeefee 
	ef555555 55bbbbbb bb555555 55fefefe fe555555 55bbbbbb bb555555 55efefef }
\shade\path(1164,2439)(1022,2439)(1022,3992)
	(1370,4212)(1528,4095)(1164,3875)(1164,2439)
\path(1164,2439)(1022,2439)(1022,3992)
	(1370,4212)(1528,4095)(1164,3875)(1164,2439)
\path(1164,3875)(1022,3992)
\path(1164,3875)(1022,3992)
\path(1954,887)(1954,1107)(2160,1107)(2160,887)
\path(2049,1057)(2049,887)(1985,887)(1985,1057)(2049,1057)
\path(2096,1057)(2096,887)(2049,887)(2049,1057)(2096,1057)
\path(2870,887)(2870,1107)(3091,1107)(3091,887)
\path(2981,1057)(2981,887)(2918,887)(2918,1057)(2981,1057)
\path(3028,1057)(3028,887)(2981,887)(2981,1057)(3028,1057)
\path(3837,987)(3837,1104)(3220,1104)(3220,987)(3837,987)
\path(3331,1104)(3331,987)
\path(3441,1104)(3441,987)
\path(3520,1104)(3520,987)
\path(3631,1104)(3631,987)
\path(3741,1104)(3741,987)
\path(3837,1215)(3837,1315)(3220,1315)(3220,1215)(3837,1215)
\path(3331,1315)(3331,1215)
\path(3441,1315)(3441,1215)
\path(3520,1315)(3520,1215)
\path(3631,1315)(3631,1215)
\path(3741,1315)(3741,1215)
\path(3837,1437)(3837,1557)(3220,1557)(3220,1437)(3837,1437)
\path(3331,1557)(3331,1437)
\path(3441,1557)(3441,1437)
\path(3520,1557)(3520,1437)
\path(3631,1557)(3631,1437)
\path(3741,1557)(3741,1437)
\path(3837,1662)(3837,1765)(3220,1765)(3220,1662)(3837,1662)
\path(3331,1765)(3331,1662)
\path(3441,1765)(3441,1662)
\path(3520,1765)(3520,1662)
\path(3631,1765)(3631,1662)
\path(3741,1765)(3741,1662)
\path(3837,1887)(3837,1987)(3220,1987)(3220,1887)(3837,1887)
\path(3331,1987)(3331,1887)
\path(3441,1987)(3441,1887)
\path(3520,1987)(3520,1887)
\path(3631,1987)(3631,1887)
\path(3741,1987)(3741,1887)
\path(3837,2115)(3837,2233)(3220,2233)(3220,2115)(3837,2115)
\path(3331,2233)(3331,2115)
\path(3441,2233)(3441,2115)
\path(3520,2233)(3520,2115)
\path(3631,2233)(3631,2115)
\path(3741,2233)(3741,2115)
\path(3837,2337)(3837,2458)(3220,2458)(3220,2337)(3837,2337)
\path(3331,2458)(3331,2337)
\path(3441,2458)(3441,2337)
\path(3520,2458)(3520,2337)
\path(3631,2458)(3631,2337)
\path(3741,2458)(3741,2337)
\path(3837,2562)(3837,2647)(3220,2647)(3220,2562)(3837,2562)
\path(3331,2647)(3331,2562)
\path(3441,2647)(3441,2562)
\path(3520,2647)(3520,2562)
\path(3631,2647)(3631,2562)
\path(3741,2647)(3741,2562)
\path(3837,2787)(3837,2904)(3220,2904)(3220,2787)(3837,2787)
\path(3331,2904)(3331,2787)
\path(3441,2904)(3441,2787)
\path(3520,2904)(3520,2787)
\path(3631,2904)(3631,2787)
\path(3741,2904)(3741,2787)
\shade\path(1022,887)(912,1057)(912,2439)(1022,2290)(1022,887)
\path(1022,887)(912,1057)(912,2439)
	(1022,2290)(1022,887)
\shade\path(3091,2222)(2981,2372)(2981,3267)(3091,3100)(3091,2222)
\path(3091,2222)(2981,2372)(2981,3267)(3091,3100)(3091,2222)
\shade\path(2254,2102)(2160,2290)(2160,2880)(2254,2677)(2254,2102)
\path(2254,2102)(2160,2290)(2160,2880)(2254,2677)(2254,2102)
\shade\path(1418,2290)(1339,2439)(1339,3672)(1418,3555)(1418,2290)
\path(1418,2290)(1339,2439)(1339,3672)(1418,3555)(1418,2290)
\texture{80222222 22555555 55808080 80555555 55222222 22555555 55880888 8555555 
	55222222 22555555 55808080 80555555 55222222 22555555 55080808 8555555 
	55222222 22555555 55808080 80555555 55222222 22555555 55880888 8555555 
	55222222 22555555 55808080 80555555 55222222 22555555 55080808 8555555 }
\shade\path(1418,2290)(1022,2290)(912,2439)(1339,2439)(1418,2290)
\path(1418,2290)(1022,2290)(912,2439)(1339,2439)(1418,2290)
\shade\path(1418,3555)(1339,3672)(1954,3672)(2049,3555)(1418,3555)
\path(1418,3555)(1339,3672)(1954,3672)(2049,3555)(1418,3555)
\shade\path(2254,2102)(2049,2102)(2049,2290)(2160,2290)(2254,2102)
\path(2254,2102)(2049,2102)(2049,2290)(2160,2290)(2254,2102)
\shade\path(2870,2677)(2254,2677)(2160,2880)(2728,2880)(2870,2677)
\path(2870,2677)(2254,2677)(2160,2880)(2728,2880)(2870,2677)
\shade\path(3091,2222)(2870,2222)(2870,2372)(2981,2372)(3091,2222)
\path(3091,2222)(2870,2222)(2870,2372)(2981,2372)(3091,2222)
\path(1954,3100)(1954,3167)(1528,3167)(1528,3100)(1954,3100)
\path(1954,2002)(1954,2052)(1528,2052)(1528,2002)(1954,2002)
\path(1954,1782)(1954,1832)(1528,1832)(1528,1782)(1954,1782)
\path(1954,2222)(1954,2290)(1528,2290)(1528,2222)(1954,2222)
\path(1954,2439)(1954,2492)(1528,2492)(1528,2439)(1954,2439)
\path(1954,2677)(1954,2709)(1528,2709)(1528,2677)(1954,2677)
\path(1954,2880)(1954,2947)(1528,2947)(1528,2880)(1954,2880)
\path(1954,3335)(1954,3370)(1528,3370)(1528,3335)(1954,3335)
\path(2460,2290)(2460,2610)(2364,2610)(2364,2290)(2460,2290)
\path(2775,2290)(2775,2610)(2664,2610)(2664,2290)(2775,2290)
\path(2775,1782)(2775,2102)(2664,2102)(2664,1782)(2775,1782)
\path(2775,1292)(2775,1615)(2664,1615)(2664,1292)(2775,1292)
\path(2460,1782)(2460,2102)(2364,2102)(2364,1782)(2460,1782)
\path(2460,1292)(2460,1615)(2364,1615)(2364,1292)(2460,1292)
\path(1954,1562)(1954,1615)(1528,1615)(1528,1562)(1954,1562)
\path(1843,1007)(1843,1345)(1481,1345)(1481,1007)(1843,1007)
\shade\path(3908,3087)(3087,3087)(2977,3254)(3750,3254)(3908,3087)
\path(3908,3087)(3087,3087)(2977,3254)(3750,3254)(3908,3087)
\path(1022,887)(1022,2290)(1418,2290)(1418,3555)(2049,3555)(2049,2102)
	(2254,2102)(2254,2677)(2870,2677)(2870,2222)(3091,2222)(3091,3100)
	(3912,3100)(3912,887)(1022,887)
\end{picture}
\end{ex}

\begin{ex}
\`A la banque, lorsqu'on a de l'argent sur un compte, 100\euro\ par exemple,
la banque note $+100$. Si on
dépense 150\euro\, on doit de l'argent à la banque (on doit 50\euro). La
banque note : $-50$.
\end{ex}
\begin{ex}
Pour situer à quelle altitude on se trouve, on prend comme origine
le niveau de la mer : c'est le niveau 0. Le sommet du Mont Blanc se
trouve à 4810 mètres au dessus de ce niveau, on note : $+4810$. Si un
sous-marin se trouve à 1200 mètres en dessous du niveau de la mer, on note :
$-1200$.\\
\setlength{\unitlength}{0.0006in} %d'après Xfig
\begin{picture}(7463,3000)(0,-10)
\put(1362,787){\ellipse{600}{200}}
\shade\path(2412,1212)(2187,987)(1962,387)
	(1662,87)(12,12)(12,1212)
	(2412,1212)(2412,1212)
\put(1362,787){\blacken\ellipse{600}{200}}
\blacken\path(1400,887)(1400,987)(1325,987)
	(1325,936)(1362,936)(1362,887)(1400,887)
\path(2412,1212)(2187,987)(1962,387)
	(1662,87)(12,12)(12,1212)
	(2412,1212)(2412,1212)
\path(1662,87)(1962,387)(2187,987)(2412,1212)
	(3012,1362)(3612,1362)(3912,1587)
	(4212,2187)(4362,2037)(4512,2412)
	(4587,2337)(4737,2637)(4962,1962)
	(5112,2337)(5187,2262)(5262,2412)
	(5637,1362)(5937,1662)(6312,1212)
\dashline{60.000}(4737,2637)(6912,2637)
\dashline{60.000}(2187,837)(6912,837)
\dashline{60.000}(2412,1200)(6987,1200)
\put(6987,762){\makebox(0,0)[lb]{$-1200$}}
\put(7062,1137){\makebox(0,0)[lb]{0}}
\put(6987,2562){\makebox(0,0)[lb]{$+4810$}}
\end{picture}


\end{ex}


\begin{remark}
Dans certains pays, comme la France, on repère l'année par rapport
à une année origine : celle qui a vu la naissance de Jésus.
On peut situer des évènements par rapport à cette origine : révolution
française en $+1789$, naissance de Jules César en $-101$.

Malheureusement, l'année origine n'est pas l'an 0 mais l'an 1. On ne peut
donc pas tout à fait dire que les années sont comme les nombres relatifs
puisque l'an 0 n'existe pas.
\end{remark}



\section{Droite graduée}

\begin{definition}
	Sur une droite, on repère chaque point par un nombre : son {\em abscisse}.
	Pour cela, on place un point d'abscisse 0. D'un côté, on place régulièrement
	les  nombres positifs et de l'autre les  nombres négatifs.
\begin{center}
	\setlength{\unitlength}{1cm}
	\begin{picture}(12,2)(-6,-1)
		\put(-6,0){\vector(1,0){12}}
		\multiput(-5,-0.1)(1,0){11}{\line(0,1){0.2}}
		\multiputlist(-5,-0.3)(1,0){$-5$,$-4$,$-3$,$-2$,$-1$,0,$+1$,
			$+2$,$+3$,$+4$,$+5$}
		\ptcentre(0,0.3){$\bullet$}\ptcentre(0,-1){$\bullet$}
		\put(0,0.3){\vector(1,0){5}}
		\put(0,-1){\vector(-1,0){5}}
		\ptne(0,0.3){nombres positifs}
		\ptno(0,-1){nombres négatifs}
	\end{picture}
\end{center}
\end{definition}
\begin{remark}
Pour indiquer de quel côté de l'origine se trouvent les nombres positifs, la
droite graduée est orientée : une flèche indique le côté des positifs.
\end{remark}

\begin{ex}
\begin{center}
\setlength{\unitlength}{5mm}
	\begin{picture}(12,2)(-6,0)
		\put(-6,0){\vector(1,0){12}}
		\multiput(-5,-0.1)(1,0){11}{\line(0,1){0.2}}
		{\tiny
		\multiputlist(-5,-0.3)(1,0){$-5$,$-4$,$-3$,$-2$,$-1$,0,$+1$,
			$+2$,$+3$,$+4$,$+5$}}
		\ptn(-3,0){A} \ptn(0,0){B} \ptn(4,0){C}
	\end{picture}
\end{center}
La droite est graduée de gauche à droite : les nombres positifs sont à droite
de l'origine.\par
L'abscisse de A est $-3$ (~on note : $\mathrm A(-3)$~);
l'abscisse de B est 0 et l'abscisse de C
est $+4$.
\end{ex}




\section{Repère du plan}
\begin{definition}
 Deux droites graduées perpendiculaires forment {\em un repère du plan}. Ces deux
droites permettent de repérer un point dans le plan par ses {\em coordonnées} :
{\em l'abscisse} (sur la droite horizontale) et {\em l'ordonnée}
(sur la droite verticale).
\begin{center}
\setlength{\unitlength}{7.5mm}
	\begin{picture}(12,10)(-6,-4)
		\put(-6,0){\vector(1,0){12}}
		\put(0,-4){\vector(0,1){10}}
		\multiput(-5,-0.1)(1,0){11}{\line(0,1){0.2}}
		\multiput(-0.1,-4)(0,1){10}{\line(1,0){0.2}}
		\multiputlist(-5,-0.3)(1,0){$-5$,$-4$,$-3$,$-2$,$-1$,,$+1$,
			$+2$,$+3$,$+4$,$+5$}
		\multiputlist(0.5,-4)(0,1){$-4$,$-3$,$-2$,$-1$,,$+1$,
			$+2$,$+3$,$+4$,$+5$}
		\ptse(0,0){0}
		\dashline{0.2}(3,0)(3,2)(0,2) \ptne(3,2){A}
		\dashline{0.2}(0,-3)(1,-3)(1,0) \ptse(1,-3){B}
		\dashline{0.2}(-4,0)(-4,5)(0,5) \ptno(-4,5){C}
		\put(-1,-2){\vector(1,2){1}}\pto(-1,-2){Origine du repère}
		\put(-2,-3.5){\vector(1,0){2}}\pto(-2,-3.5){Droite des ordonnées}
		\put(3.5,-2){\vector(0,1){2}}\pts(3.5,-2){Droite des abscisses}
	\end{picture}

\end{center}
Les coordonnées de A sont $(+3;+2)$. Son abscisse est $+3$, son ordonnée est $+2$.\\
Les coordonnées de B sont $(+1;-3)$. Son abscisse est $+1$, son ordonnée est $-3$.\\
Les coordonnées de C sont $(-4;+5)$. Son abscisse est $-4$, son ordonnée est $+5$.\\
\end{definition}


\end{document}
