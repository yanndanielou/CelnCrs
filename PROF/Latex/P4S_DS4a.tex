\documentclass[a4paper,11pt]{article}
\usepackage{amsmath,amsthm,amsfonts,amssymb,amscd,amstext,vmargin,graphics,graphicx,tabularx,multicol} 
\usepackage[francais]{babel}
\usepackage[utf8]{inputenc}  
\usepackage[T1]{fontenc} 
\usepackage{pstricks-add,tikz,tkz-tab,variations}
\usepackage[autolanguage,np]{numprint} 

\setmarginsrb{1.5cm}{0.5cm}{1cm}{0.5cm}{0cm}{0cm}{0cm}{0cm} %Gauche, haut, droite, haut
\newcounter{numexo}
\newcommand{\exo}[1]{\stepcounter{numexo}\noindent{\bf Exercice~\thenumexo} : \marginpar{\hfill /#1}}
\reversemarginpar


\newcounter{enumtabi}
\newcounter{enumtaba}
\newcommand{\q}{\stepcounter{enumtabi} \theenumtabi.  }
\newcommand{\qa}{\stepcounter{enumtaba} (\alph{enumtaba}) }
\newcommand{\initq}{\setcounter{enumtabi}{0}}
\newcommand{\initqa}{\setcounter{enumtaba}{0}}

\newcommand{\be}{\begin{enumerate}}
\newcommand{\ee}{\end{enumerate}}
\newcommand{\bi}{\begin{itemize}}
\newcommand{\ei}{\end{itemize}}
\newcommand{\bp}{\begin{pspicture*}}
\newcommand{\ep}{\end{pspicture*}}
\newcommand{\bt}{\begin{tabular}}
\newcommand{\et}{\end{tabular}}
\renewcommand{\tabularxcolumn}[1]{>{\centering}m{#1}} %(colonne m{} centrée, au lieu de p par défault) 
\newcommand{\tnl}{\tabularnewline}

\newcommand{\trait}{\noindent \rule{\linewidth}{0.2mm}}
\newcommand{\hs}[1]{\hspace{#1}}
\newcommand{\vs}[1]{\vspace{#1}}

\newcommand{\N}{\mathbb{N}}
\newcommand{\Z}{\mathbb{Z}}
\newcommand{\R}{\mathbb{R}}
\newcommand{\C}{\mathbb{C}}
\newcommand{\Dcal}{\mathcal{D}}
\newcommand{\Ccal}{\mathcal{C}}
\newcommand{\mc}{\mathcal}

\newcommand{\vect}[1]{\overrightarrow{#1}}
\newcommand{\ds}{\displaystyle}
\newcommand{\eq}{\quad \Leftrightarrow \quad}
\newcommand{\vecti}{\vec{\imath}}
\newcommand{\vectj}{\vec{\jmath}}
\newcommand{\Oij}{(O;\vec{\imath}, \vec{\jmath})}
\newcommand{\OIJ}{(O;I,J)}


\newcommand{\titre}[5] 
% #1: titre #2: haut gauche #3: bas gauche #4: haut droite #5: bas droite
{
\noindent #2 \hfill #4 \\
#3 \hfill #5

\vspace{-1.6cm}

\begin{center}\rule{6cm}{0.5mm}\end{center}
\vspace{0.2cm}
\begin{center}{\large{\textbf{#1}}}\end{center}
\begin{center}\rule{6cm}{0.5mm}\end{center}
}



\begin{document}
\pagestyle{empty}
\titre{DS 4a}{Nom :}{Prénom :}{P4S}{19/12/14}

\exo{6} $\mathcal{C}_f$ est la courbe représentative d'une fonction $f$ définie et dérivable sur $[-6\,;5]$. Par ailleurs,
on a tracé en pointillé les tangentes à $\mc C_f$ aux points d'abscisse $-3$ et $-4$. 

 \begin{center}
\psset{xunit=1cm,yunit=1cm}
\psset{algebraic=true}
\def\xmin{-7} \def\xmax{6} \def\ymin{-7} \def\ymax{4.5}
      \begin{pspicture}(\xmin,\ymin)(\xmax,\ymax)
\psgrid[subgriddiv=1,griddots=10,gridlabels=0](0,0)(\xmin,\ymin)(\xmax,4)
\psaxes[labels=all,yticksize=-.05 .05,xticksize=-.1 .1,Dx=1,Dy=1,labelFontSize=\scriptstyle]{->}(0,0)(\xmin,\ymin)(\xmax,\ymax)[$x$,0][$y$,0]
\uput[dl](0,0){$O$}
\psline{->}(0,0)(1,0)
\uput[d](.5,.05){$\vec\imath$}
\psline{->}(0,0)(0,1)
\uput[l](0,.5){$\vec\jmath$}
\psplot[plotpoints=500,arrows=*-]{-6}{-3}{-2*(x+3)*(x+5)}
\psplot[plotpoints=500]{-3}{-.5}{4*(x+3)*x/3}
\psplot[plotpoints=500]{-.5}{1.5}{-20*(x-1)^2/27}
\psplot[plotpoints=500]{1.5}{3}{88*x^2/243-176*x/81+61/27}
\psplot[plotpoints=500,arrows=-*]{3}{5}{(x-4)*(x-2)}
\psplot[linestyle=dashed]{-4}{-1.25}{-4*(x+3)}
\psplot[linestyle=dashed]{\xmin}{\xmax}{2}
\uput[ur](-3,0){$A$}
\uput[r](-2,-4){$B$}
\uput[135](-6,-5){$\mathcal{C}_f$}
\psdots[dotscale=1,dotstyle=+](-3,0)(-2,-4)
      \end{pspicture}
    \end{center}
\be 
\item Par lecture graphique : on a $f'(-4) = \qquad \qquad $ et $f'(-3)= \qquad \qquad $.

\item Déterminer les solutions de l'équation $f'(x)=0$. (\emph{justifier succinctement})
\ \\
\ \\
\ \\

\item Le maximum de la fonction $f$ sur $[-6\,;\,5]$ est $\qquad \qquad $. Il est atteint en $\qquad\qquad$ .

\item La fonction $f$ admet-elle d'autres maximums locaux ? Si oui préciser.
\ \\
\ \\
\ \\

\item Soit $g$ une fonction dérivable sur $[-6\,;5]$ telle que $g'(x)=f(x)$ pour $x \in [-6\,;5]$. Déterminer le tableau de variation de $g$. (\emph{justifier})
\ee

\newpage 

\ \\

\exo{8} Soit $f$ la fonction définie par $\ds f(x)= -x^3+2x^2+4x+4$.
\be \item Après avoir déterminé le domaine de définition de $f$ ainsi que son domaine de dérivabilité calculer $f'(x)$.
\item \'Etudier le signe du trinôme $-3x^2+4x+4$.
\item En déduire le tableau de variation complet (avec les extremums locaux) de $f$.
\item Déterminer l'équation de la tangente $\mc T_1$ à $\mc C_f$ au point d'abscisse 1.
\item[*5.] En quel(s) point(s) la tangente à $\mc C_f$ est-elle parallèle à $\mc T_1$ ?
\ee

\vs{0.5cm}

\exo{7} Soit $f$ la fonction définie par $\ds f(x)=\frac{x^2-x+4}{x-1}$. 

\be \item Déterminer le domaine de définition et de dérivabilité de $f$.
\item Montrer que pour tout $x$ dans le domaine on a $\ds f'(x)=\frac{x^2-2x-3}{(x-1)^2}$.
\item \'Etudier (\emph{soigneusement}) les variations de $f$.
\ee

\end{document}