\documentclass[a4paper,11pt]{article}
\usepackage{amsmath,amsthm,amsfonts,amssymb,amscd,amstext,vmargin,graphics,graphicx,tabularx,multicol} 
\usepackage[francais]{babel}
\usepackage[utf8]{inputenc}  
\usepackage[T1]{fontenc} 
\usepackage{pstricks-add,tikz,tkz-tab,variations}
\usepackage[autolanguage,np]{numprint} 

\setmarginsrb{1.5cm}{0.5cm}{1cm}{0.5cm}{0cm}{0cm}{0cm}{0cm} %Gauche, haut, droite, haut
\newcounter{numexo}
\newcommand{\exo}[1]{\stepcounter{numexo}\noindent{\bf Exercice~\thenumexo} : \marginpar{\hfill /#1}}
\reversemarginpar


\newcounter{enumtabi}
\newcounter{enumtaba}
\newcommand{\q}{\stepcounter{enumtabi} \theenumtabi.  }
\newcommand{\qa}{\stepcounter{enumtaba} (\alph{enumtaba}) }
\newcommand{\initq}{\setcounter{enumtabi}{0}}
\newcommand{\initqa}{\setcounter{enumtaba}{0}}

\newcommand{\be}{\begin{enumerate}}
\newcommand{\ee}{\end{enumerate}}
\newcommand{\bi}{\begin{itemize}}
\newcommand{\ei}{\end{itemize}}
\newcommand{\bp}{\begin{pspicture*}}
\newcommand{\ep}{\end{pspicture*}}
\newcommand{\bt}{\begin{tabular}}
\newcommand{\et}{\end{tabular}}
\renewcommand{\tabularxcolumn}[1]{>{\centering}m{#1}} %(colonne m{} centrée, au lieu de p par défault) 
\newcommand{\tnl}{\tabularnewline}

\newcommand{\trait}{\noindent \rule{\linewidth}{0.2mm}}
\newcommand{\hs}[1]{\hspace{#1}}
\newcommand{\vs}[1]{\vspace{#1}}

\newcommand{\N}{\mathbb{N}}
\newcommand{\Z}{\mathbb{Z}}
\newcommand{\R}{\mathbb{R}}
\newcommand{\C}{\mathbb{C}}
\newcommand{\Dcal}{\mathcal{D}}
\newcommand{\Ccal}{\mathcal{C}}
\newcommand{\mc}{\mathcal}

\newcommand{\vect}[1]{\overrightarrow{#1}}
\newcommand{\ds}{\displaystyle}
\newcommand{\eq}{\quad \Leftrightarrow \quad}
\newcommand{\vecti}{\vec{\imath}}
\newcommand{\vectj}{\vec{\jmath}}
\newcommand{\Oij}{(O;\vec{\imath}, \vec{\jmath})}
\newcommand{\OIJ}{(O;I,J)}


\newcommand{\reponse}[1][1]{%
\multido{}{#1}{\makebox[\linewidth]{\rule[0pt]{0pt}{20pt}\dotfill}
}}

\newcommand{\titre}[5] 
% #1: titre #2: haut gauche #3: bas gauche #4: haut droite #5: bas droite
{
\noindent #2 \hfill #4 \\
#3 \hfill #5

\vspace{-1.6cm}

\begin{center}\rule{6cm}{0.5mm}\end{center}
\vspace{0.2cm}
\begin{center}{\large{\textbf{#1}}}\end{center}
\begin{center}\rule{6cm}{0.5mm}\end{center}
}



\begin{document}
\pagestyle{empty}
\titre{Interrogation: Décimaux relatifs(1)}{Nom :}{Prénom :}{Classe}{Date}



\vspace*{1cm}

\begin{tabular}{|c|c|c|c|}
\hline 
Compétences & Acquis & En cours  & Non acquis \\ 
\hline 
- Savoir additionner et soustraire des entiers relatifs &  &  &  \\ 
\hline 
- Savoir multiplier deux nombres positifs  &  &  &  \\ 
\hline 
- Respecter les priorités de calculs  &  &  &  \\ 
\hline 
- Calculer le produit de plusieurs nombres relatifs simples &  &  &  \\ 
\hline 
- Organiser et effectuer à la main les séquences de calcul correspondantes &  &  &  \\ 
\hline 
\end{tabular} 

\vspace*{0.5cm}

\exo{2} Questions de cours\\

\q Compléter la propriété suivante.\\

Pour additionner deux nombres relatifs de \textbf{même signe} :	
\begin{itemize}
\item \reponse[1] ;
\item \reponse[1].
\end{itemize}

\q Quel est le signe d'un produit de 67 facteurs non nuls dont 31 sont positifs ? (\textbf{Justifier})\\
\reponse[2]\\

\medskip

\exo{6,5} Effectuer les calculs suivants \textbf{en indiquant toutes les étapes de calculs}.\\

\begin{multicols}{3}

$H=-5 + 13 $\\
\reponse[1]\\

$D = -8 \times 6 $\\
\reponse[1]\\

$K = 15 + 4,2 - 6 + 3,7 - 8,1$\\
\reponse[2]\\

\columnbreak

$M = 15 - ( -8)$\\
\reponse[1]\\

$P = -100 \times (-70,8) $\\
\reponse[1]\\

$U = ( 6 - 17 ) \times ( 4 - 3 )$\\
\reponse[2]\\

\columnbreak


$S = 13 \times (-3)$\\
\reponse[1]\\

$Z = 8 + 3 \times ( - 7 ) $\\
\reponse[1]\\

$R = 11-(8 - 12)$\\
\reponse[2]\\



\end{multicols}



\exo{1,5} \\

\initq
\q Déterminer le signe du produit suivant : $W= -1 \times (-2) \times 2,5 \times (-4) \times 7 \times (-0,5) $\\
\reponse[1]\\

\q Calculer l'expression suivante \textbf{en indiquant toutes les étapes de calculs}.\\

$W= -1 \times (-2) \times 2,5 \times (-4) \times 7 \times (-0,5) $\\
\reponse[4]





\end{document}
