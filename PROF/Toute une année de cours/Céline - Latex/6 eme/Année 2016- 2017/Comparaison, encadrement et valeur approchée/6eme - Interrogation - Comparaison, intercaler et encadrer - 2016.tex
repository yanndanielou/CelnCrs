\documentclass[a4paper,11pt]{article}
\usepackage{amsmath,amsthm,amsfonts,amssymb,amscd,amstext,vmargin,graphics,graphicx,tabularx,multicol} 
\usepackage[francais]{babel}
\usepackage[utf8]{inputenc}  
\usepackage[T1]{fontenc} 
\usepackage{pstricks-add,tikz,tkz-tab,variations}
\usepackage[autolanguage,np]{numprint} 

\setmarginsrb{1.5cm}{0.5cm}{1cm}{0.5cm}{0cm}{0cm}{0cm}{0cm} %Gauche, haut, droite, haut
\newcounter{numexo}
\newcommand{\exo}[1]{\stepcounter{numexo}\noindent{\bf Exercice~\thenumexo} : \marginpar{\hfill /#1}}
\reversemarginpar


\newcounter{enumtabi}
\newcounter{enumtaba}
\newcommand{\q}{\stepcounter{enumtabi} \theenumtabi.  }
\newcommand{\qa}{\stepcounter{enumtaba} (\alph{enumtaba}) }
\newcommand{\initq}{\setcounter{enumtabi}{0}}
\newcommand{\initqa}{\setcounter{enumtaba}{0}}

\newcommand{\be}{\begin{enumerate}}
\newcommand{\ee}{\end{enumerate}}
\newcommand{\bi}{\begin{itemize}}
\newcommand{\ei}{\end{itemize}}
\newcommand{\bp}{\begin{pspicture*}}
\newcommand{\ep}{\end{pspicture*}}
\newcommand{\bt}{\begin{tabular}}
\newcommand{\et}{\end{tabular}}
\renewcommand{\tabularxcolumn}[1]{>{\centering}m{#1}} %(colonne m{} centrée, au lieu de p par défault) 
\newcommand{\tnl}{\tabularnewline}

\newcommand{\bmul}[1]{\begin{multicols}{#1}}
\newcommand{\emul}{\end{multicols}}

\newcommand{\trait}{\noindent \rule{\linewidth}{0.2mm}}
\newcommand{\hs}[1]{\hspace{#1}}
\newcommand{\vs}[1]{\vspace{#1}}

\newcommand{\N}{\mathbb{N}}
\newcommand{\Z}{\mathbb{Z}}
\newcommand{\R}{\mathbb{R}}
\newcommand{\C}{\mathbb{C}}
\newcommand{\Dcal}{\mathcal{D}}
\newcommand{\Ccal}{\mathcal{C}}
\newcommand{\mc}{\mathcal}

\newcommand{\vect}[1]{\overrightarrow{#1}}
\newcommand{\ds}{\displaystyle}
\newcommand{\eq}{\quad \Leftrightarrow \quad}
\newcommand{\vecti}{\vec{\imath}}
\newcommand{\vectj}{\vec{\jmath}}
\newcommand{\Oij}{(O;\vec{\imath}, \vec{\jmath})}
\newcommand{\OIJ}{(O;I,J)}


\newcommand{\reponse}[1][1]{%
\multido{}{#1}{\makebox[\linewidth]{\rule[0pt]{0pt}{20pt}\dotfill}
}}

\newcommand{\titre}[5] 
% #1: titre #2: haut gauche #3: bas gauche #4: haut droite #5: bas droite
{
\noindent #2 \hfill #4 \\
#3 \hfill #5

\vspace{-1.6cm}

\begin{center}\rule{6cm}{0.5mm}\end{center}
\vspace{0.2cm}
\begin{center}{\large{\textbf{#1}}}\end{center}
\begin{center}\rule{6cm}{0.5mm}\end{center}
}



\begin{document}
\pagestyle{empty}
\titre{Interrogation: Comparer, intercaler et encadrer des nombres }{Nom :}{Prénom :}{Classe}{Date}


\exo{1,5} Compléter les définitions du cours : 

\q Comparer deux nombres, c'est \reponse[2]\\

\q Ranger des nombres du plus grand au plus petit, c'est les ranger dans . . . . . . . . . . . . . . . . . . . . . .\\

\q Encadrer un nombre, c'est \reponse[2]\\


\exo{3} 

\initq \q Ranger dans l'ordre décroissant les nombres suivants : \\

$ 5,4$ \hspace*{0.3cm};\hspace*{0.3cm} $ \dfrac{542}{100} + \dfrac{3}{1 000}$ \hspace*{0.3cm} ;\hspace*{0.3cm} $ \dfrac{53}{10} + \dfrac{9}{100}$\hspace*{0.3cm} ; \hspace*{0.3cm}$ 538$ centièmes\hspace*{0.3cm} et\hspace*{0.3cm} $ \dfrac{5470}{1 000}$\\

\reponse[1]\\


\q Compléter avec le nombre \textbf{entier} qui suit ou celui qui précède :\\

\bmul{3}

$12,6 < . . .$

\columnbreak

$6,09 > . . .$

\columnbreak

$ . . . < \dfrac{2 453 }{100}$

\emul

\exo{2,5} 

\initq \q Intercaler un nombre entre 3,1 et 3,2 : \\
\reponse[1]

\q Encadrer les nombres suivants par deux entiers consécutifs :\\

\bmul{2}

. . . . . . . < 74,586 < . . . . . . .

\columnbreak

. . . . . . . < $ \dfrac{8523}{100}$ < . . . . . . .


\emul


\exo{3} $\pi$ est un nombre qui a fasciné tant de savants depuis l'antiquité. \\
$\pi$ est un nombre irrationnel (c'est à dire qu'il s'écrit avec un nombre infini de décimales sans suite logique). \\
Le 2 Août 2010, 5 000 milliards de décimales de $\pi$ ont été découverts par deux japonnais Alexander J. Yee et Shigeru en 90 jours.\\
 Et 1 an plus tard après 371 jours de travail, ces même chercheurs ont battu leur record et ont découvert jusqu'à 10 000 milliards de décimales de $\pi$. En voici une toute petite approximation :

\begin{center}
$\pi \approx 3.141 592 653 589 793 238 462 643 383 279 502 884 197 169 399 375 $
\end{center}



\initq \q Encadrer le nombre $\pi$ au millième près.\\
\reponse[1]\\

\q Donner la valeur approchée au millième près de $\pi$ par défaut.\\
\reponse[1]\\


\q Encadrer le nombre $\pi$ au dixième près.\\
\reponse[1]\\

\q Donner la valeur approchée au dixième près de $\pi$ par excès.\\
\reponse[1]\\


\end{document}
