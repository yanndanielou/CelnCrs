\documentclass[a4paper,11pt]{article}
\usepackage{amsmath,amsthm,amsfonts,amssymb,amscd,amstext,vmargin,graphics,graphicx,tabularx,multicol} 
\usepackage[francais]{babel}
\usepackage[utf8]{inputenc}  
\usepackage[T1]{fontenc} 
\usepackage{pstricks-add,tikz,tkz-tab,variations}
\usepackage[autolanguage,np]{numprint} 

\setmarginsrb{1.5cm}{0.5cm}{1cm}{0.5cm}{0cm}{0cm}{0cm}{0cm} %Gauche, haut, droite, haut
\newcounter{numexo}
\newcommand{\exo}[1]{\stepcounter{numexo}\noindent{\bf Exercice~\thenumexo} : \marginpar{\hfill /#1}}
\reversemarginpar


\newcounter{enumtabi}
\newcounter{enumtaba}
\newcommand{\q}{\stepcounter{enumtabi} \theenumtabi.  }
\newcommand{\qa}{\stepcounter{enumtaba} (\alph{enumtaba}) }
\newcommand{\initq}{\setcounter{enumtabi}{0}}
\newcommand{\initqa}{\setcounter{enumtaba}{0}}

\newcommand{\be}{\begin{enumerate}}
\newcommand{\ee}{\end{enumerate}}
\newcommand{\bi}{\begin{itemize}}
\newcommand{\ei}{\end{itemize}}
\newcommand{\bp}{\begin{pspicture*}}
\newcommand{\ep}{\end{pspicture*}}
\newcommand{\bt}{\begin{tabular}}
\newcommand{\et}{\end{tabular}}
\renewcommand{\tabularxcolumn}[1]{>{\centering}m{#1}} %(colonne m{} centrée, au lieu de p par défault) 
\newcommand{\tnl}{\tabularnewline}

\newcommand{\trait}{\noindent \rule{\linewidth}{0.2mm}}
\newcommand{\hs}[1]{\hspace{#1}}
\newcommand{\vs}[1]{\vspace{#1}}

\newcommand{\N}{\mathbb{N}}
\newcommand{\Z}{\mathbb{Z}}
\newcommand{\R}{\mathbb{R}}
\newcommand{\C}{\mathbb{C}}
\newcommand{\Dcal}{\mathcal{D}}
\newcommand{\Ccal}{\mathcal{C}}
\newcommand{\mc}{\mathcal}

\newcommand{\vect}[1]{\overrightarrow{#1}}
\newcommand{\ds}{\displaystyle}
\newcommand{\eq}{\quad \Leftrightarrow \quad}
\newcommand{\vecti}{\vec{\imath}}
\newcommand{\vectj}{\vec{\jmath}}
\newcommand{\Oij}{(O;\vec{\imath}, \vec{\jmath})}
\newcommand{\OIJ}{(O;I,J)}


\newcommand{\reponse}[1][1]{%
\multido{}{#1}{\makebox[\linewidth]{\rule[0pt]{0pt}{20pt}\dotfill}
}}

\newcommand{\titre}[5] 
% #1: titre #2: haut gauche #3: bas gauche #4: haut droite #5: bas droite
{
\noindent #2 \hfill #4 \\
#3 \hfill #5

\vspace{-1.6cm}

\begin{center}\rule{6cm}{0.5mm}\end{center}
\vspace{0.2cm}
\begin{center}{\large{\textbf{#1}}}\end{center}
\begin{center}\rule{6cm}{0.5mm}\end{center}
}



\begin{document}
\pagestyle{empty}
\titre{TP : Premier pas en géométrie}{Nom :}{Prénom :}{Classe}{Date}


\vspace*{1cm}

\textbf{\begin{large}
Ouvrir le logiciel GeoGebra\\
\end{large}}
	
\noindent \q Placer deux points A et B \\
\q Tracer la droite (AB)\\
\q Placer un point C appartenant au segment [AB]\\
\q Placer un point D n’appartenant pas à la droite (AB)\\
\q Tracer la demi-droite [CD)\\

Appeler le professeur\\

\noindent \q Placer un point E tel que E $\in$ [BC]\\
\q Placer un point F sur la demi-droite [CD)\\
\q Tracer le segment [EF]\\
\q Placer un point G tel que G $\in$ [EF]\\
\q Tracer la demi-droite [GC)\\

Appeler le professeur\\

\noindent \q Placer un point H tel que H $\in$ [GC) et H $\notin$ [GC]\\
\q Activer la trace du point H\\
\q Sur une figure de quelle nature semble se déplacer le point H lorsque l’on déplace le point G?\\
\reponse[2]\\
\q Sur une figure de quelle nature semble se déplacer le point H lorsque l’on déplace le point F?\\
\reponse[2]\\
\q Activer la trace du point G\\
\q Sur une figure de quelle nature semble se déplacer le point G lorsque l’on déplace le point F?\\
\reponse[2]\\

Enregistrer le travail dans le dossier "Devoirs".\\


\vspace*{2cm}




\begin{flushright}
\textbf{Note : ......./ 10 }
\end{flushright}


\newpage

\pagestyle{empty}
\titre{TP : Premier pas en géométrie (2)}{Nom :}{Prénom :}{Classe}{Date}


\vspace*{1cm}

\textbf{\begin{large}
Ouvrir le logiciel GeoGebra\\
\end{large}}

\initq 
\noindent \q Construire un segment [AB]\\
\q Placer le milieu C du segment [AB]\\
\q	Tracer la médiatrice du segment [AB]\\
\q	Placer un point D sur cette médiatrice\\
\q	Placer un point E qui n'appartient ni au segment [AB] ni à sa médiatrice\\
\q	Afficher la longueur du segment [CD]\\

Appeler le professeur\\

\noindent \q	Déplacer le point D pour que CD=4\\
\q	Afficher la longueur du segment [AD]\\
\q	Compléter : AD = ..........\\
\q	En déduire la longueur du segment [BD]. Expliquer\\
\reponse[3]\\

Appeler le professeur\\

\noindent \q	Déplacer le point E tel que ED = EC.\\
\q	Tracer la médiatrice du segment [CD]\\
\q	Que constate-t-on? Expliquer\\
\reponse[3]\\
\q	Déplacer le point D tel que CD = 5\\
\q	Placer le point E sur la médiatrice du segment [CD]\\
\q	Que peut-on dire pour les segments [CE] et [DE] ? Expliquer\\
\reponse[3]\\

Enregistrer le travail dans le dossier "Devoirs".\\


\vspace*{2cm}




\begin{flushright}
\textbf{Note : ......./ 10 }
\end{flushright}






\end{document}
