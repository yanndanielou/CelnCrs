\documentclass[a4paper,11pt]{article}
\usepackage{amsmath,amsthm,amsfonts,amssymb,amscd,amstext,vmargin,graphics,graphicx,tabularx,multicol} 
\usepackage[francais]{babel}
\usepackage[utf8]{inputenc}  
\usepackage[T1]{fontenc} 
\usepackage{pstricks-add,tikz,tkz-tab,variations}
\usepackage[autolanguage,np]{numprint} 
\usepackage{color}

\setmarginsrb{1.5cm}{0.5cm}{1cm}{0.5cm}{0cm}{0cm}{0cm}{0cm} %Gauche, haut, droite, haut
\newcounter{numexo}
\newcommand{\exo}[1]{\stepcounter{numexo}\noindent{\bf Exercice~\thenumexo} : \marginpar{\hfill /#1}}
\reversemarginpar


\newcounter{enumtabi}
\newcounter{enumtaba}
\newcommand{\q}{\stepcounter{enumtabi} \theenumtabi.  }
\newcommand{\qa}{\stepcounter{enumtaba} (\alph{enumtaba}) }
\newcommand{\initq}{\setcounter{enumtabi}{0}}
\newcommand{\initqa}{\setcounter{enumtaba}{0}}

\newcommand{\be}{\begin{enumerate}}
\newcommand{\ee}{\end{enumerate}}
\newcommand{\bi}{\begin{itemize}}
\newcommand{\ei}{\end{itemize}}
\newcommand{\bp}{\begin{pspicture*}}
\newcommand{\ep}{\end{pspicture*}}
\newcommand{\bt}{\begin{tabular}}
\newcommand{\et}{\end{tabular}}
\renewcommand{\tabularxcolumn}[1]{>{\centering}m{#1}} %(colonne m{} centrée, au lieu de p par défault) 
\newcommand{\tnl}{\tabularnewline}

\newcommand{\trait}{\noindent \rule{\linewidth}{0.2mm}}
\newcommand{\hs}[1]{\hspace{#1}}
\newcommand{\vs}[1]{\vspace{#1}}

\newcommand{\N}{\mathbb{N}}
\newcommand{\Z}{\mathbb{Z}}
\newcommand{\R}{\mathbb{R}}
\newcommand{\C}{\mathbb{C}}
\newcommand{\Dcal}{\mathcal{D}}
\newcommand{\Ccal}{\mathcal{C}}
\newcommand{\mc}{\mathcal}

\newcommand{\vect}[1]{\overrightarrow{#1}}
\newcommand{\ds}{\displaystyle}
\newcommand{\eq}{\quad \Leftrightarrow \quad}
\newcommand{\vecti}{\vec{\imath}}
\newcommand{\vectj}{\vec{\jmath}}
\newcommand{\Oij}{(O;\vec{\imath}, \vec{\jmath})}
\newcommand{\OIJ}{(O;I,J)}


\newcommand{\bmul}[1]{\begin{multicols}{#1}}
\newcommand{\emul}{\end{multicols}}

\newcommand{\reponse}[1][1]{%
\multido{}{#1}{\makebox[\linewidth]{\rule[0pt]{0pt}{20pt}\dotfill}
}}

\newcommand{\titre}[5] 
% #1: titre #2: haut gauche #3: bas gauche #4: haut droite #5: bas droite
{
\noindent #2 \hfill #4 \\
#3 \hfill #5

\vspace{-1.6cm}

\begin{center}\rule{6cm}{0.5mm}\end{center}
\vspace{0.2cm}
\begin{center}{\large{\textbf{#1}}}\end{center}
\begin{center}\rule{6cm}{0.5mm}\end{center}
}



\begin{document}
\pagestyle{empty}
\titre{Correction du contrôle sur les 3 premiers chapitres}{5 ème}{}{}{}



\exo{5} \\

\q Relier chaque phrase à l'expression qui lui correspond (\textbf{sur le sujet})

\begin{multicols}{2}

\bi
\item La somme de 9 et du quotient de 7 par 2.\\

\item Le produit de 7 par la somme 9 et de 2.\\

\item Le quotient de 9 par la somme 7 et de 2.\\

\item La somme de 9 et du produit de 7 par 2.\\

\item Le quotient d'une somme par 2.
\ei

\hfill
\columnbreak
\hfill

\bi
\item $A= 9 + 7 \times 2$\\

\item $ B= 7 \times (9 +2)$\\

\item $ C=9 + \dfrac{7}{2}$\\

\item $D= (9+7) \div 2$\\

\item $E= 9 \div (7+2)$
\ei 

\end{multicols}

\q Calculer les expressions A, B, C, D et E.\\

\red
\bmul{3}

$A= 9 + 7 \times 2$\\

$A= 9 + 14$\\

\fbox{$A= 23$}\\

$D= (9+7) \div 2$\\

$D= 16 \div 2$\\

\fbox{$D= 8$}\\

\columnbreak

$ B= 7 \times (9 +2)$\\

$ B= 7 \times 11$\\

\fbox{$ B= 77$}\\

$E= 9 \div (7+2)$\\

$E= 9 \div 9$\\

\fbox{$E= 1$}\\

\columnbreak

$ C= 9 + \dfrac{7}{2}$\\

$ C= 9 + 3,5 $\\

\fbox{$ C= 12,5 $}\\

\emul


\vspace*{0.6cm}

\black \exo{2}(\textbf{Sur le sujet})

Chacune des expressions suivantes est fausse. Placer, dans chaque cas, des parenthèses aux bons endroits pour rendre l'égalité vraie.\\


\qa 2 $\times$ \textcolor{red}{(}5 + 2\textcolor{red}{)} = 14\\

\qa	1 + \textcolor{red}{(} 3 + 2 \textcolor{red}{)} $\times$ 6 = 31\\

\qa	\textcolor{red}{(} 1 + 2 \textcolor{red}{)} $\times$ 5 + 3 $\times$ \textcolor{red}{(} 10 - 4 \textcolor{red}{)} = 33

\vspace*{0.6cm}

\exo{4}\\

Calculer les expressions suivantes en respectant les priorités (on détaillera toutes les étapes de calculs) :\\

\initq

\bmul{3}
\q $D= 24-15+8$\\

\red $D= 9+8$\\

\fbox{$D= 17$}\\



\columnbreak

\black \q $M= 18 - 5 \times 2$\\

\red $M= 18 - 10$\\

\fbox{$M= 8$}\\


\columnbreak

\black \q $G= 81 \div 9 \times 3$\\

\red $G= 9 \times 3$\\

\fbox{$G= 27$}\\



\emul

\bmul{3}
\black \q $V= (24-2-1) \div (4 \times 25)$\\

\red $V= (21) \div (4 \times 25)$\\

$V= 21 \div 100$\\

\fbox{$V= 0,21$}\\

\columnbreak


\black \q $L = 57 + 30 \div 6$\\

\red $L = 57 + 5$\\

\fbox{$L = 62$}\\



\columnbreak




\black \q $S= 3 \times [18 -(4-1)\times2]$\\

\red $S= 3 \times [18 - 3 \times2]$\\

$S= 3 \times [18 - 6]$\\

$S= 3 \times 12$\\

\fbox{$S= 36$}\\



\emul

\vspace*{0.6cm}

\black \exo{2}\\

Pour le tournoi de handball du collège, les professeurs d'EPS ont réparti les 96 élèves de $5^{eme}$ en équipes de 12. Pour l'échauffement, 24 ballons sont distribués équitablement entre les équipes.\\

\initq
\q Écrire \textbf{une} expression qui permet de calculer le nombre de ballons distribués par équipe.\\

\red L'expression qui nous permettra de calculer le nombre de ballons distribués par équipe est la suivante : $ M = 24 \div ( 96 \div 12) $\\

\black \q Effectuer les calculs.\\

\red $ M = 24 \div ( 96 \div 12) $\\

$ M = 24 \div 8 $\\

\fbox{$M = 3$} Il y aura donc 3 ballons par équipes.\\

\vspace*{0.6cm}


\black \exo{3}\\

\initq
\q Peut-on construire un triangle dont les côtés mesurent 9 cm, 5,5 cm et 6,1 cm?(\textbf{Justifier votre réponse}) Si oui, construire ce triangle.\\

\red La plus grande longueur est 9 cm . La somme des deux autres vaut : 5,5 + 6,1 = 11,6cm. Ainsi, 9 < 5,5 + 6,1 donc on peut en déduire que le triangle sera constructible.\\

\black \q Des segments de longueurs 8,3 cm, 12,4 cm et 3,4 cm peuvent-ils être les côtés d'un triangle ? (\textbf{Justifier votre réponse}) Si oui, construire ce triangle.

\red La plus grande longueur est 12,4 cm . La somme des deux autres vaut : 8,3 + 3,4 = 11,7 cm. Ainsi, 12,4 > 8,3 + 3,4 donc on peut en déduire que le triangle ne sera pas constructible.\\


\vspace*{0.6cm}

\black Pour les exercices de géométrie, reprendre le cours sur comment tracer un triangle.

\end{document}
