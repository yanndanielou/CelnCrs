\documentclass[a4paper,11pt]{article}
\usepackage{amsmath,amsthm,amsfonts,amssymb,amscd,amstext,vmargin,graphics,graphicx,tabularx,multicol} \usepackage[french]{babel}
\usepackage[utf8]{inputenc}  
\usepackage[T1]{fontenc} 
\usepackage[T1]{fontenc}
\usepackage{amsmath,amssymb}
\usepackage{pstricks-add,tikz,tkz-tab,variations}
\usepackage[autolanguage,np]{numprint} 

\setmarginsrb{1.5cm}{0.5cm}{1cm}{0.5cm}{0cm}{0cm}{0cm}{0cm} %Gauche, haut, droite, haut
\newcounter{numexo}
\newcommand{\exo}[1]{\stepcounter{numexo}\noindent{\bf Exercice~\thenumexo} : \marginpar{\hfill /#1}}
\reversemarginpar


\newcounter{enumtabi}
\newcounter{enumtaba}
\newcommand{\q}{\stepcounter{enumtabi} \theenumtabi.  }
\newcommand{\qa}{\stepcounter{enumtaba} (\alph{enumtaba}) }
\newcommand{\initq}{\setcounter{enumtabi}{0}}
\newcommand{\initqa}{\setcounter{enumtaba}{0}}

\newcommand{\be}{\begin{enumerate}}
\newcommand{\ee}{\end{enumerate}}
\newcommand{\bi}{\begin{itemize}}
\newcommand{\ei}{\end{itemize}}
\newcommand{\bp}{\begin{pspicture*}}
\newcommand{\ep}{\end{pspicture*}}
\newcommand{\bt}{\begin{tabular}}
\newcommand{\et}{\end{tabular}}
\renewcommand{\tabularxcolumn}[1]{>{\centering}m{#1}} %(colonne m{} centrée, au lieu de p par défault) 
\newcommand{\tnl}{\tabularnewline}

\newcommand{\trait}{\noindent \rule{\linewidth}{0.2mm}}
\newcommand{\hs}[1]{\hspace{#1}}
\newcommand{\vs}[1]{\vspace{#1}}

\newcommand{\N}{\mathbb{N}}
\newcommand{\Z}{\mathbb{Z}}
\newcommand{\R}{\mathbb{R}}
\newcommand{\C}{\mathbb{C}}
\newcommand{\Dcal}{\mathcal{D}}
\newcommand{\Ccal}{\mathcal{C}}
\newcommand{\mc}{\mathcal}

\newcommand{\vect}[1]{\overrightarrow{#1}}
\newcommand{\ds}{\displaystyle}
\newcommand{\eq}{\quad \Leftrightarrow \quad}
\newcommand{\vecti}{\vec{\imath}}
\newcommand{\vectj}{\vec{\jmath}}
\newcommand{\Oij}{(O;\vec{\imath}, \vec{\jmath})}
\newcommand{\OIJ}{(O;I,J)}

\newcommand{\bmul}[1]{\begin{multicols}{#1}}
\newcommand{\emul}{\end{multicols}}


\newcommand{\reponse}[1][1]{%
\multido{}{#1}{\makebox[\linewidth]{\rule[0pt]{0pt}{20pt}\dotfill}
}}

\newcommand{\titre}[5] 
% #1: titre #2: haut gauche #3: bas gauche #4: haut droite #5: bas droite
{
\noindent #2 \hfill #4 \\
#3 \hfill #5

\vspace{-1.6cm}

\begin{center}\rule{6cm}{0.5mm}\end{center}
\vspace{0.2cm}
\begin{center}{\large{\textbf{#1}}}\end{center}
\begin{center}\rule{6cm}{0.5mm}\end{center}
}



\begin{document}
\pagestyle{empty}
\titre{Contrôle : Triangles et distributivité}{Nom :}{Prénom :}{Classe}{Date}





\vspace*{0.5cm}

\exo{2,5}  Cours\\

\noindent \q Donner la définition d'une hauteur dans un triangle.\\
\q Comment appelle-t-on le point de concours des médianes d'un triangle ? \\
\q Compléter \textbf{sur le sujet} les propriétés suivantes :\\

Pour tous les nombres que l'on note k, a et b : \hspace*{1cm}     $k \times (a +b)= ...................... + ............................$\\

Pour tous les nombres que l'on note k, a et b : \hspace*{1cm}     $............................= k \times a - k \times b$\\

\vspace*{0,5cm}

\exo{4,5} Soit le triangle ABC tel que AB = 6 cm, AC = 3,7 cm et BC = 4,4 cm.\\

\initq 
\q Ce triangle est-il constructible? Si oui, construire le triangle ABC.\\

\q Tracer les 3 hauteurs de ce triangle.\\

\q Que remarque-t-on?\\

\vspace*{0,5cm}

\exo{3} \\

\initq 
\q Construire un triangle SUR tel que UR = 5 cm , $\widehat{RUS}= 100$ et  $\widehat{SRU} = 30 $ puis placer le point A tel que U soit le milieu du segment [AR].\\

\q Que représente, de façon précise, la droite (SU) pour le triangle SAR ? \\

\vspace*{0,5cm}

\exo{4} Entourer \textbf{sur le sujet} en rouge les expressions factorisées et en bleu les expressions développées.\\

\bmul{3}

\qa $4 \times (2 + 3)$\\

\qa $1,7 \times 4 + 1,7 \times 6$

\columnbreak

\qa $k \times (a + b)$\\

\qa $(9-3) \times 7$

\columnbreak

\qa $7 \times 3 - 7 \times 2 + 7 \times 7$\\

\qa $k \times a - k \times b$

\emul

\vspace*{0,5cm}

\exo{2}
Une maman achète 5 pains au chocolat et 5 croissants. 	\\		Un pain au chocolat coûte 0,90 euros et un croissant, 0,70 euros  .\\

Calculer de \textbf{deux façons} différentes le montant de la dépense.	\\

\vspace*{0,5cm}

\exo{4}
Utiliser la distributivité pour calculer astucieusement les expressions suivantes. \textbf{(Détailler vos calculs)}\\

\bmul{4}

$A = 99 \times 42,5	$

\columnbreak

$C = 58 \times 7,4 - 58 \times  7,3$

\columnbreak

$B = 1,7 \times 1 001$		

\columnbreak

$D = 63 \times  4,1 + 37 \times  4,1	$

\emul


\exo{} Bonus \\

Sachant que  $45 \times 23 = 1 035$  et  $45 \times  52 = 2 340$, utiliser la distributivité pour calculer :\\

E = $45 \times 75$	\hspace*{1cm} $F = 29 \times 45 	$ 				(Détailler vos calculs)





\end{document}
