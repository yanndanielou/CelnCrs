\documentclass[a4paper,11pt]{article}
\usepackage{amsmath,amsthm,amsfonts,amssymb,amscd,amstext,vmargin,graphics,graphicx,tabularx,multicol} \usepackage[french]{babel}
\usepackage[utf8]{inputenc}  
\usepackage[T1]{fontenc} 
\usepackage[T1]{fontenc}
\usepackage{amsmath,amssymb}
\usepackage{pstricks-add,tikz,tkz-tab,variations}
\usepackage[autolanguage,np]{numprint} 

\setmarginsrb{1.5cm}{0.5cm}{1cm}{0.5cm}{0cm}{0cm}{0cm}{0cm} %Gauche, haut, droite, haut
\newcounter{numexo}
\newcommand{\exo}[1]{\stepcounter{numexo}\noindent{\bf Exercice~\thenumexo} : \marginpar{\hfill /#1}}
\reversemarginpar


\newcounter{enumtabi}
\newcounter{enumtaba}
\newcommand{\q}{\stepcounter{enumtabi} \theenumtabi.  }
\newcommand{\qa}{\stepcounter{enumtaba} (\alph{enumtaba}) }
\newcommand{\initq}{\setcounter{enumtabi}{0}}
\newcommand{\initqa}{\setcounter{enumtaba}{0}}

\newcommand{\be}{\begin{enumerate}}
\newcommand{\ee}{\end{enumerate}}
\newcommand{\bi}{\begin{itemize}}
\newcommand{\ei}{\end{itemize}}
\newcommand{\bp}{\begin{pspicture*}}
\newcommand{\ep}{\end{pspicture*}}
\newcommand{\bt}{\begin{tabular}}
\newcommand{\et}{\end{tabular}}
\renewcommand{\tabularxcolumn}[1]{>{\centering}m{#1}} %(colonne m{} centrée, au lieu de p par défault) 
\newcommand{\tnl}{\tabularnewline}

\newcommand{\trait}{\noindent \rule{\linewidth}{0.2mm}}
\newcommand{\hs}[1]{\hspace{#1}}
\newcommand{\vs}[1]{\vspace{#1}}

\newcommand{\N}{\mathbb{N}}
\newcommand{\Z}{\mathbb{Z}}
\newcommand{\R}{\mathbb{R}}
\newcommand{\C}{\mathbb{C}}
\newcommand{\Dcal}{\mathcal{D}}
\newcommand{\Ccal}{\mathcal{C}}
\newcommand{\mc}{\mathcal}

\newcommand{\vect}[1]{\overrightarrow{#1}}
\newcommand{\ds}{\displaystyle}
\newcommand{\eq}{\quad \Leftrightarrow \quad}
\newcommand{\vecti}{\vec{\imath}}
\newcommand{\vectj}{\vec{\jmath}}
\newcommand{\Oij}{(O;\vec{\imath}, \vec{\jmath})}
\newcommand{\OIJ}{(O;I,J)}

\newcommand{\bmul}[1]{\begin{multicols}{#1}}
\newcommand{\emul}{\end{multicols}}


\newcommand{\reponse}[1][1]{%
\multido{}{#1}{\makebox[\linewidth]{\rule[0pt]{0pt}{20pt}\dotfill}
}}

\newcommand{\titre}[5] 
% #1: titre #2: haut gauche #3: bas gauche #4: haut droite #5: bas droite
{
\noindent #2 \hfill #4 \\
#3 \hfill #5

\vspace{-1.6cm}

\begin{center}\rule{6cm}{0.5mm}\end{center}
\vspace{0.2cm}
\begin{center}{\large{\textbf{#1}}}\end{center}
\begin{center}\rule{6cm}{0.5mm}\end{center}
}



\begin{document}
\pagestyle{empty}
\titre{Contrôle : Calcul littéral}{Nom :}{Prénom :}{Classe}{Date}




\exo{1,5}  Simplifier chacune des écritures suivantes :
 
\bmul{3}
\begin{flushleft}
$8c \times 5a = ....................... $
\end{flushleft}

\columnbreak

\begin{flushleft}
	$  4 \times y \times y \times y = ....................... $ 
	\end{flushleft}	

\columnbreak

\begin{flushleft}
$7 \times x + 3 \times 8 = ....................... $                
\end{flushleft}

\emul

\vspace*{0.5cm}

\exo{1} Réduire les expressions suivantes :

\bmul{2}
$18n - 15n = .......................$

\columnbreak

$7a + 5z - 3a + z = .......................$
\emul

\vspace*{0.5cm}


\exo{2} Développer chaque expression puis en donner une écriture simplifiée.

\bmul{2}

$T = 5(4x - 1)$\\   
\reponse[3]  

\columnbreak

$R = x(2 + x)  $  \\
\reponse[3]     

\emul         

\exo{3,5} Factoriser chacune des expressions suivantes :

\bmul{2}
$A = 2x + 2 y $\\
\reponse[2]\\

$B = ab - a   $\\
\reponse[2]\\

\columnbreak

$V = 3x - x^{2}     $\\
\reponse[2]\\

$S = 3a + 12$\\
\reponse[2]\\

\emul

\exo{2} Soit l'égalité suivante : $x^{2} + y^{2}= 10x - 2y - 1$ 

\q Tester cette égalité pour $x=9$ et $y= 2$\\
\reponse[3]\\

\q Tester cette égalité pour $x=5$ et $y= 0$\\
\reponse[3]\\

\exo{} BONUS\\

\initq
\noindent \q Factoriser au maximum $Z = 100 yx^{2} - 25x$\\

\noindent \q Développer et réduire l'expression suivantes : $D = 3x(2-x) + x(4-x) - 10x + 4x^{2}$ 
 











\end{document}
