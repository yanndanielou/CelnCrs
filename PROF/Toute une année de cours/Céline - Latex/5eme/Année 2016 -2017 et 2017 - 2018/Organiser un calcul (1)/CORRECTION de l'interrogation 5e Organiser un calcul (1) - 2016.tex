\documentclass[a4paper,11pt]{article}
\usepackage{amsmath,amsthm,amsfonts,amssymb,amscd,amstext,vmargin,graphics,graphicx,tabularx,multicol} 
\usepackage[francais]{babel}
\usepackage[utf8]{inputenc}  
\usepackage[T1]{fontenc} 
\usepackage{pstricks-add,tikz,tkz-tab,variations}
\usepackage[autolanguage,np]{numprint} 
\usepackage{color}
\usepackage{ulem}

\setmarginsrb{1.5cm}{0.5cm}{1cm}{0.5cm}{0cm}{0cm}{0cm}{0cm} %Gauche, haut, droite, haut
\newcounter{numexo}
\newcommand{\exo}[1]{\stepcounter{numexo}\noindent{\bf Exercice~\thenumexo} : \marginpar{\hfill /#1}}
\reversemarginpar


\newcounter{enumtabi}
\newcounter{enumtaba}
\newcommand{\q}{\stepcounter{enumtabi} \theenumtabi.  }
\newcommand{\qa}{\stepcounter{enumtaba} (\alph{enumtaba}) }
\newcommand{\initq}{\setcounter{enumtabi}{0}}
\newcommand{\initqa}{\setcounter{enumtaba}{0}}

\newcommand{\be}{\begin{enumerate}}
\newcommand{\ee}{\end{enumerate}}
\newcommand{\bi}{\begin{itemize}}
\newcommand{\ei}{\end{itemize}}
\newcommand{\bp}{\begin{pspicture*}}
\newcommand{\ep}{\end{pspicture*}}
\newcommand{\bt}{\begin{tabular}}
\newcommand{\et}{\end{tabular}}
\renewcommand{\tabularxcolumn}[1]{>{\centering}m{#1}} %(colonne m{} centrée, au lieu de p par défault) 
\newcommand{\tnl}{\tabularnewline}

\newcommand{\trait}{\noindent \rule{\linewidth}{0.2mm}}
\newcommand{\hs}[1]{\hspace{#1}}
\newcommand{\vs}[1]{\vspace{#1}}

\newcommand{\N}{\mathbb{N}}
\newcommand{\Z}{\mathbb{Z}}
\newcommand{\R}{\mathbb{R}}
\newcommand{\C}{\mathbb{C}}
\newcommand{\Dcal}{\mathcal{D}}
\newcommand{\Ccal}{\mathcal{C}}
\newcommand{\mc}{\mathcal}

\newcommand{\vect}[1]{\overrightarrow{#1}}
\newcommand{\ds}{\displaystyle}
\newcommand{\eq}{\quad \Leftrightarrow \quad}
\newcommand{\vecti}{\vec{\imath}}
\newcommand{\vectj}{\vec{\jmath}}
\newcommand{\Oij}{(O;\vec{\imath}, \vec{\jmath})}
\newcommand{\OIJ}{(O;I,J)}


\newcommand{\reponse}[1][1]{%
\multido{}{#1}{\makebox[\linewidth]{\rule[0pt]{0pt}{20pt}\dotfill}
}}

\newcommand{\titre}[5] 
% #1: titre #2: haut gauche #3: bas gauche #4: haut droite #5: bas droite
{
\noindent #2 \hfill #4 \\
#3 \hfill #5

\vspace{-1.6cm}

\begin{center}\rule{6cm}{0.5mm}\end{center}
\vspace{0.2cm}
\begin{center}{\large{\textbf{#1}}}\end{center}
\begin{center}\rule{6cm}{0.5mm}\end{center}
}



\begin{document}
\pagestyle{empty}
\titre{Interrogation: Organiser un calcul(1)}{Nom :}{Prénom :}{Classe}{Date}



\vspace*{1cm}



\exo{4} Questions de cours\\

\q \textbf{Traduire} les phrases suivantes par une expression numérique puis \textbf{calculer} :\\

\qa	La somme de 5 et 12.\\

\textcolor{red}{$5 + 12 = 17$\\}

\qa	La différence de 54 et du produit de 8 par 4.\\

\textcolor{red}{$54 - 8 \times 4 = 54 - 32 $\\}

\textcolor{red}{$ \hspace*{1.8cm}= 22 $\\}

\q Traduire les expressions numériques suivantes par \textbf{des phrases} :\\

\initqa
\qa $(38 - 5 )\times 7$ \hspace*{0,3cm}
\textcolor{red}{Le produit de la différence de 38 et de 5 par 7 .\\}

\qa $100 - (36 \times 2)$ \hspace*{0,3cm}
\textcolor{red}{La différence de 100 et du produit de 36 par 2.\\}

\medskip

\exo{3} Camille a acheté pour 73 euros, cinq livres de poche de même prix et un CD à 13 euros. \\

\initq
\q	Écrire \textbf{une} expression numérique qui permet de calculer le prix d'un livre de poche.\\

\textcolor{red}{L'expression numérique qui permet de calculer le prix d'un livre de poche est $T = (73 - 13 ) : 5$\\}

\q	Calculer le prix d'un livre de poche \textbf{en indicant les étapes de calcul}.\\

\textcolor{red}{$ T = (73 - 13 ) : 5$\\}

\textcolor{red}{$ T = (60 ) : 5$\\}

\textcolor{red}{\fbox{$ T = 12$ }  \hspace*{0.4cm} Un livre de poche coûte 12 euros.}\\


\exo{3} Calculer les expressions suivantes \textbf{en détaillant les étapes de calcul}.\\

\begin{multicols}{2}
$V = 9 - [2 \times (12-8)] $\\

\textcolor{red}{$V = 9 - [2 \times4]$\\}

\textcolor{red}{$V = 9 - 8$\\}

\textcolor{red}{\fbox{$V = 1$\\}}


\columnbreak

$Q = (15-4) \times (7+2) $\\

\textcolor{red}{$Q = 11 \times 9$\\}

\textcolor{red}{\fbox{$Q = 99$\\}}


\end{multicols}


$M =(34-(25-6))\times 8 $\\

\textcolor{red}{$M =(34-19)\times 8 $\\}

\textcolor{red}{$M = 15 \times 8 $\\}

\textcolor{red}{\fbox{$M = 120$\\}}



\end{document}
